\documentclass[12pt]{article}
\usepackage[margin=1in]{geometry}

% Start of preamble
%==========================================================================================%
% Required to support mathematical unicode
\usepackage[warnunknown, fasterrors, mathletters]{ucs}
\usepackage[utf8x]{inputenc}

\usepackage[dvipsnames,table,xcdraw]{xcolor} % colors
\usepackage{hyperref} % links
\hypersetup{
	colorlinks=true,
	linkcolor=blue,
	filecolor=magenta,      
	urlcolor=cyan,
	pdfpagemode=FullScreen
}

% Standard mathematical typesetting packages
\usepackage{amsmath,amssymb,amscd,amsthm,amsxtra, pxfonts}
\usepackage{mathtools,mathrsfs,dsfont,xparse}

% Symbol and utility packages
\usepackage{cancel, textcomp}
\usepackage[mathscr]{euscript}
\usepackage[nointegrals]{wasysym}
\usepackage{apacite}

% Extras
\usepackage{physics}  % Lots of useful shortcuts and macros
\usepackage{tikz-cd}  % For drawing commutative diagrams easily
\usepackage{microtype}  % Minature font tweaks
%\usepackage{pgfplots} % plots

\usepackage{enumitem}
\usepackage{titling}

\usepackage{graphicx}

% Fancy theorems due to @intuitively on discord
\usepackage{mdframed}
\newmdtheoremenv[
backgroundcolor=NavyBlue!30,
linewidth=2pt,
linecolor=NavyBlue,
topline=false,
bottomline=false,
rightline=false,
innertopmargin=10pt,
innerbottommargin=10pt,
innerrightmargin=10pt,
innerleftmargin=10pt,
skipabove=\baselineskip,
skipbelow=\baselineskip
]{mytheorem}{Theorem}

\newenvironment{theorem}{\begin{mytheorem}}{\end{mytheorem}}

\newtheorem{corollary}{Corollary}
\newtheorem{lemma}{Lemma}

\newtheoremstyle{definitionstyle}
{\topsep}%
{\topsep}%
{}%
{}%
{\bfseries}%
{.}%
{.5em}%
{}%
\theoremstyle{definitionstyle}
\newmdtheoremenv[
backgroundcolor=Violet!30,
linewidth=2pt,
linecolor=Violet,
topline=false,
bottomline=false,
rightline=false,
innertopmargin=10pt,
innerbottommargin=10pt,
innerrightmargin=10pt,
innerleftmargin=10pt,
skipabove=\baselineskip,
skipbelow=\baselineskip,
]{mydef}{Definition}
\newenvironment{definition}{\begin{mydef}}{\end{mydef}}

\newtheorem*{remark}{Remark}

\newtheorem*{example}{Example}

% Common shortcuts
\def\mbb#1{\mathbb{#1}}
\def\mfk#1{\mathfrak{#1}}

\def\bN{\mbb{N}}
\def \C{\mbb{C}}
\def \R{\mbb{R}}
\def\bQ{\mbb{Q}}
\def\bZ{\mbb{Z}}
\def \cph{\varphi}
\renewcommand{\th}{\theta}
\def \ve{\varepsilon}
\newcommand{\mg}[1]{\| #1 \|}

% Often helpful macros
\newcommand{\floor}[1]{\left\lfloor#1\right\rfloor}
\newcommand{\ceil}[1]{\left\lceil#1\right\rceil}
\renewcommand{\qed}{\hfill\qedsymbol}
\renewcommand{\ip}[2]{\langle #1, #2 \rangle}
\newcommand{\seq}[2]{\qty(#1_#2)_{#2=1}^{\infty}}

% Sets
\DeclarePairedDelimiterX\set[1]\lbrace\rbrace{\def\given{\;\delimsize\vert\;}#1}

% End of preamble
%==========================================================================================%

% Start of commands specific to this file
%==========================================================================================%

\usepackage{braket}

%==========================================================================================%
% End of commands specific to this file

\title{Math 461 HW7}
\date{\today}
\author{Rohan Mukherjee}

\begin{document}
	\maketitle
	\begin{enumerate}[leftmargin=\labelsep]
		\item \begin{enumerate}
			\item We construct a bijection between the partitions of $n$ where the largest part occurs at least twice and partitions of $n-2$ with smallest part at least 2. The bijection is to take the transpose then remove the last 
			
			\item We use the same trick as before. We shall construct a bijection between partitions of $n$ where the largest part appears exactly 2 times and partitions of $n-2$ where the smallest part is at least 2. The bijection this time is to take a transpose then remove the bottom row, and the inverse being add 2 blocks in a new row on the bottom and take a transpose. The condition of having the smallest part at least 2 this operation indeed yields a young diagram. In any case, the generating function for the number of partitions of $n$ with smallest part at least 2 is just
			\begin{align*}
				\prod_{k=2}^\infty (1+x^k+x^{2k}+\cdots) = P(x) / (1+x+x^2+\cdots) = (1-x)P(x) = p(0) + \sum_{n=1}^\infty (p(n)-p(n-1))x^n
			\end{align*}
			So the number of partitions of $n$ in which the smallest part is at least 2 is just $p(n)-p(n-1)$ for $n \geq 1$ and $p(0)=1$ for $n = 0$. So the number of partitions of $n$ where the largest part appears exactly 2 times is just $p(n-2) - p(n-3)$ for $n \geq 3$. We conclude that the number of ways to have the largest part occur at least 3 times is just the number of ways to have it occur at least 2 times minus the number of ways to have it occur exactly 2 times, which is just 
			$p(n)-p(n-1) - (p(n-2) - p(n-3)) = p(n) - p(n-1) - p(n-2) + p(n-3)$. 
		\end{enumerate}
	
		\item Let $A$ be the ways Alice is sitting next to Bob, $C$ Charlie next to Danielle, and $E$ ed next to Francine. We use complimentary counting to find $|A|$, so put Alice and Bob in a block, pick an ordering of the block Alice and Bob and everyone else, then choose if Alice is to the left or right of Bob. This can be done in $5! \cdot 2$ ways, and the same logic applies to $C$ and $E$ to get they also equal $5! \cdot 2$. If we want Alice next to Bob and Charlie next to Danielle, i.e. $|A \cap C|$, we would now have 2 blocks of people (Alice and Bob and Charlie and Danielle), with 2 remaining people, which using the same logic as above gives a total of $4! \cdot 2^2$, since we need to choose if Alice is left/right of Bob and if Charlie is left/right of Danielle. Finally, using the same logic we would have $|A \cap C \cap E| = 3! \cdot 2^3$. Ignoring the ordering, the total number of seatings is just $6!$, so using complimentary counting and inclusion/exclusion gives us:
		$|(A^c \cap C^c \cap D^c)^c| = 6! - |A \cup C \cup D| = 6! - (|A| + |C| + |E| - |A \cap C| - |A \cap E| - |E \cap C| + |A \cap C \cap |E|) = 6! - (3 \cdot 5! \cdot 2 - 3 \cdot 4! \cdot 2^2 + 3! \cdot 2^3) = 240$.
		
		\item This is just a question of derangements. First, with no conditions, there are $7!$ ways to return their hats back to them. Now we need to find the number of ways less than 2 of them get their own hats back. This is just the number of ways that precisely 1 gets his hat back or precisely 0 gets his hat back. This is just ${7 \choose 0}D_7 + {7 \choose 1}D_6$. From the notes we have $D_5 = 44$, so $D_6 = 6 \cdot 44 + (-1)^6 = 265$, and $D_7 = 7 \cdot 265 + (-1)^7 = 1854$. So our final answer is just
		\begin{align*}
			7! - (1 \cdot 1854 + 7 \cdot 265) = 1131
		\end{align*}
		Which is around $1/7$th of the possibilities. This sort of says most ($6/7$) of the possibilities return either 1 or no hats. Also, in expectation you would expect 1 hat back, so this makes sense, but is tighter than I would've expected.
		
		\item This is just exclusion exclusion. Let $A$ denote the ways the first shelf gets exactly 6 books, $B$, the number of ways the second gets exactly books, $C$ the third, and $D$ the fourth. We want to find $|(A^c \cap B^c \cap C^c \cap D^c)|$, and this is of course just an inclusion-exclusion problem. To find $|A|$, put 6 books in the first shelf, then stars and bars the remaining books into the other shelves, which can be done in ${18 + 3 - 1 \choose 3 - 1} = {20 \choose 2}$ ways. The same logic can be applied to the other shelves, so $|A| = |B| = |C| = |D| = {20 \choose 2} = 190$. To find $|A \cap B|$, we put 6 books in $A$, 6 in $B$, then we stars and bars the remaining 12 books among the last 2 shelves to get ${12 + 2 -1 \choose 2 - 1} = {13 \choose 1} = 13$ ways. So all ${4 \choose 2}$ ways of picking 2 bookshelves with exactly 6 books in both will yield $13$ ways. We see that if we have exactly 6 books in the first 3 shelves, then we also have exactly 6 books in the last one, which would yield 1 way. This the ${4 \choose 3} = 4$ ways of picking $3$ shelves to have exactly 6 books all give one way. Finally, there is 1 way to have exactly 6 books in all the shelves. This gives us an answer of
		\begin{align*}
			4 \cdot {20 \choose 2} - {4 \choose 2} \cdot 13 + {4 \choose 3} \cdot 1 - 1 = 685
		\end{align*}
		
		\item Let $m = p_1^{\alpha_1} \cdots p_n^{\alpha_n}$, and $A_i = \Set{ \ell | 1 \leq i \leq m, p_i \mid \ell}$. Notice that for any integer $1 \leq x \leq m$, $\gcd(x, m) \neq 1$ iff $p_i \mid x$ for some $p_i$ (Both directions are clear). We see that $|A_i| = \frac{m}{p_i}$, that $|A_i \cap A_j| = \frac{m}{p_ip_j}$, and in general, $|A_{i_1} \cap \cdots \cap A_{i_\ell}| = \frac{m}{p_{i_1}  \cdots  \; p_{i_\ell}}$. By inclusion-exclusion, we have that
		\begin{align*}
			\qty|A_1 \cup \cdots \cup A_k| = \sum_{i} |A_i| - \sum_{i < j} |A_i \cap A_j| + \sum_{i < j < k} |A_i \cap A_j \cap A_k| + \cdots + (-1)^{n+1} \qty|A_1 \cap \cdots \cap A_k|
		\end{align*}
		We now claim that
		\begin{align*}
			-\prod_{i=1}^n (1-x_i) = -1+\sum_i x_i - \sum_{i < j} x_ix_j + \sum_{i < j < k} x_ix_jx_k + \cdots + (-1)^{n+1} x_1 \cdots x_n
		\end{align*}
		First because there is only one way to pick all 1s, and the coefficient is 1 but we flip it. Then notice that to get $x_i$, we need to pick $-x_i$ from the $i$th factor and 1 from all the other ones, and finally we flip the sign to get a coefficient of 1. To get $x_ix_j$, we need to choose $-x_i$ from the $i$th factor and $-x_j$ from the $j$th factor and 1 from all the other ones, and finally flip the sign, for a coefficient of $-1$. In general to get $x_{i_1} \cdots \; x_{i_k}$ we need to pick $-x_{i_1}$ from the $i_1$th factor and so on and 1s for every other factor, finally flipping the sign, for a coefficient of $(-1)^{k+1}$, which proves the above claim. We may now conclude that,
		\begin{align*}
			|A_1 \cup \cdots \cup A_k| = n\qty(\sum_i \frac{1}{p_i} - \sum_{i < j} \frac{1}{p_ip_j} + \cdots + (-1)^{n+1} \frac{1}{p_1 \cdots \; p_n}) = -n \qty(\prod_{i=1}^n \qty(1-\frac{1}{p_i})+1)
		\end{align*}
		But we wanted to find the number which have gcd 1 with $n$. This is just the complement of the above set, who's cardinality is
		\begin{align*}
			n - \qty(n\prod_{i=1}^n \qty(1-\frac{1}{p_i})+n) = n\prod_{i=1}^n \qty(1-\frac{1}{p_i})
		\end{align*}
		Which completes the proof.
		
		
	\end{enumerate}
\end{document}
