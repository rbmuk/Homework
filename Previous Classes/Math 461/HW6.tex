\documentclass[12pt]{article}
\usepackage[margin=1in]{geometry}

% Start of preamble
%==========================================================================================%
% Required to support mathematical unicode
\usepackage[warnunknown, fasterrors, mathletters]{ucs}
\usepackage[utf8x]{inputenc}

\usepackage[dvipsnames,table,xcdraw]{xcolor} % colors
\usepackage{hyperref} % links
\hypersetup{
	colorlinks=true,
	linkcolor=blue,
	filecolor=magenta,      
	urlcolor=cyan,
	pdfpagemode=FullScreen
}

% Standard mathematical typesetting packages
\usepackage{amsmath,amssymb,amscd,amsthm,amsxtra}
\usepackage{mathtools,mathrsfs,dsfont,xparse}

% Symbol and utility packages
\usepackage{cancel, textcomp}
\usepackage[mathscr]{euscript}
\usepackage[nointegrals]{wasysym}
\usepackage{apacite}

% Extras
\usepackage{physics}  % Lots of useful shortcuts and macros
\usepackage{tikz-cd}  % For drawing commutative diagrams easily
\usepackage{microtype}  % Minature font tweaks
%\usepackage{pgfplots} % plots

\usepackage{enumitem}
\usepackage{titling}

\usepackage{graphicx}

\usepackage{setspace}

% Fancy theorems due to @intuitively on discord
\usepackage{mdframed}
\newmdtheoremenv[
backgroundcolor=NavyBlue!30,
linewidth=2pt,
linecolor=NavyBlue,
topline=false,
bottomline=false,
rightline=false,
innertopmargin=10pt,
innerbottommargin=10pt,
innerrightmargin=10pt,
innerleftmargin=10pt,
skipabove=\baselineskip,
skipbelow=\baselineskip
]{mytheorem}{Theorem}

\newenvironment{theorem}{\begin{mytheorem}}{\end{mytheorem}}

\newtheorem{corollary}{Corollary}
\newtheorem{lemma}{Lemma}

\newtheoremstyle{definitionstyle}
{\topsep}%
{\topsep}%
{}%
{}%
{\bfseries}%
{.}%
{.5em}%
{}%
\theoremstyle{definitionstyle}
\newmdtheoremenv[
backgroundcolor=Violet!30,
linewidth=2pt,
linecolor=Violet,
topline=false,
bottomline=false,
rightline=false,
innertopmargin=10pt,
innerbottommargin=10pt,
innerrightmargin=10pt,
innerleftmargin=10pt,
skipabove=\baselineskip,
skipbelow=\baselineskip,
]{mydef}{Definition}
\newenvironment{definition}{\begin{mydef}}{\end{mydef}}

\newtheorem*{remark}{Remark}

\newtheorem*{example}{Example}

% Common shortcuts
\def\mbb#1{\mathbb{#1}}
\def\mfk#1{\mathfrak{#1}}

\def\bN{\mbb{N}}
\def \C{\mbb{C}}
\def \R{\mbb{R}}
\def\bQ{\mbb{Q}}
\def\bZ{\mbb{Z}}
\def \cph{\varphi}
\renewcommand{\th}{\theta}
\def \ve{\varepsilon}
\newcommand{\mg}[1]{\| #1 \|}

% Often helpful macros
\newcommand{\floor}[1]{\left\lfloor#1\right\rfloor}
\newcommand{\ceil}[1]{\left\lceil#1\right\rceil}
\renewcommand{\qed}{\hfill\qedsymbol}
\renewcommand{\ip}[2]{\langle #1, #2 \rangle}
\newcommand{\seq}[2]{\qty(#1_#2)_{#2=1}^{\infty}}

% Sets
\DeclarePairedDelimiterX\set[1]\lbrace\rbrace{\def\given{\;\delimsize\vert\;}#1}

% Sus...
\input pdfmsym

% End of preamble
%==========================================================================================%

% Start of commands specific to this file
%==========================================================================================%

%==========================================================================================%
% End of commands specific to this file

\title{Math 461 HW6}
\date{\today}
\author{Rohan Mukherjee}

\begin{document}
	\maketitle
	\begin{enumerate}[leftmargin=\labelsep]
		\item We see that
		\begin{align*}
			(a_0+a_1x+a_2x^2+ \ldots)(1-2x) = a_0 + \sum_{n=1}^\infty (a_n-2a_{n-1})x^n
		\end{align*}
		Since $a_n-2a_{n-1} = \begin{cases}
			1, \; a_n \text{ odd} \\
			0, \; \text{o.w.}
		\end{cases}$, this evaluates to
	\begin{align*}
		0 + \sum_{n \text{ odd}} x^n
	\end{align*}
	I noted on the previous homework that the generating function for $a_1, a_3, a_5, \ldots$ was $\frac12 (f(x)-f(-x))$, so this geometric series equals
	\begin{align*}
		\frac12 \qty(\frac{1}{1-x} - \frac{1}{1+x})
	\end{align*}
	We conclude that $2A(x) = \frac{1}{(1-2x)(1-x)} - \frac{1}{(1-2x)(1+x)}$. Repeated applications of the cover-up method shows that these fractions equal
	\begin{align*}
		2A(x) = \frac{4/3}{1-2x} - \frac{1}{1-x} - \frac{1/3}{1+x} = \sum_{n=0}^\infty \qty(\frac43 \cdot 2^n - 1 - \frac{(-1)^n}{3})x^n
	\end{align*}
	Thus $a_n = \frac16(2^{n+2}-3+(-1)^{n+1})$.
	
	\item I claim that the sum of $abc$ over all $(a, b, c) \in S$ where $S$ is the set of all positive integers such that $a+b+c=n$, which I shall denote $a_n$, has generating function
	\begin{align*}
		(0+1x+2x^2+\ldots)^3
	\end{align*}
	This is because when we expand this product, to get $x^n$ we would need to pick an $ax^a$ from the first factor, a $bx^b$ from the second, and a $cx^c$ from the third, giving a term of $abcx^{a+b+c}$, and sum over all possible $(a,b,c)$ so that $a+b+c=n$. Thus the coefficient is exactly what we are looking for. The above function now equals
	\begin{align*}
		\qty(\frac{x}{(1-x)^2})^3 = x^3 \cdot \frac{1}{(1-x)^6} = x^3 \cdot \sum_{n=0}^\infty {n+5 \choose 5} x^n = \sum_{n=0}^\infty {n +5 \choose 5} x^{n+3} = \sum_{n=3}^\infty {n+2 \choose 5}x^n
	\end{align*}
	We conclude that $a_n = {n+2 \choose 5}$, and in particular $a_{25} = {27 \choose 5} = 80,730$.
	
	\item First we find a generating function for the first row. Let $a_n$ be the number of ways to have $n$ identical coins, each lined up in a row, with an odd number of heads. We can evaluate $a_n$ by summing over the number of heads. Let $1 \leq k \leq n$ be odd and the number of heads in our row. There are ${n \choose k}$ ways to do this. Summing over $k$ yields
	\begin{align*}
		a_n = \sum_{k \text{ odd}} {n \choose k} = 2^{n-1}
	\end{align*}
	The generating function for $a_n$ is thus $\sum_{n=1}^\infty 2^{n-1} x^n$ (We note for completeness that $a_0 = 0$, since in that case we can't have an odd number of heads). Similarly, if we let $b_n$ be the number of ways we can have the second row with $n$ identical coins, we would have $b_0 = 1$ (since 0 is even), and $b_n = 2^{n-1}$ for $n \geq 1$ (there are also $2^{n-1}$ even sized subsets). If we let $c_n$ be the answer to the question, it's generating function is just the product of these two, since this asks what happens if we split $n$ coins over the two different rows:
	\begin{align*}
		A(x) = \qty(\sum_{n=1}^\infty 2^{n-1}x^n)\qty(1+\sum_{n=1}^\infty 2^{n-1}x^n) &= \sum_{n=1}^\infty 2^{n-1}x^n + \qty(\sum_{n=1}^\infty 2^{n-1}x^n)^2 = \frac{x}{1-2x} + \frac{x^2}{(1-2x)^2} \\
		&= \sum_{n=1}^\infty \qty(2^{n-1} + 2^{n-2}(n-1))x^n
	\end{align*}
	We conclude that $c_n = 2^{n-1} + 2^{n-2}(n-1)$ for $n > 0$ and $c_0 = 0$.
	
	\item This is extremely similar to the problem we had in class. We proceed first by fixing the number of parts then summing over it. Let $k$ be the number of parts. We shall find the generating function for the sequence $a_n$ which counts the number of compositions of $n$ with $k$ parts, who are all odd. This is just,
	\begin{align*}
		\prod_{i=1}^k (x+x^3+x^5+\ldots) = \prod_{i=1}^k \frac12\qty(\frac{1}{1-x} - \frac{1}{1+x}) = \qty(\frac{x}{1-x^2})^k
	\end{align*}
	Now we can just sum over the number of parts. This yields, noting we need at least 1 part,
	\begin{align*}
		\sum_{k=1}^\infty \qty(\frac{x}{1-x^2})^k = \frac{x/(1-x^2)}{1-x/(1-x^2)} = \frac{x}{1-x-x^2}
	\end{align*}
	The answer to our question is just the coefficient of $x^n$ in the above generating function. At this time the curious reader is reminded of the Fibonacci sequence. Indeed, if we let $F_n = F_{n-1} + F_{n-2}$, with $F_0 = 0$ and $F_1 = 1$, and if we denote $F(x)$ as the generating function for $F_n$, we have that
	\begin{align*}
		F(x)(1-x-x^2) = F_0 + x(F_1-F_0) + \sum_{n=2}^\infty (F_n-F_{n-1}-F_{n-2})x^n
	\end{align*} 
	So \begin{align*}
		F(x) = \frac{x}{1-x-x^2}
	\end{align*}
	We conclude that 
	\begin{align*}
		a_n = F_n = \frac1{\sqrt{5}}\qty(\qty(1+\frac{\sqrt{5}}2)^n + \qty(1-\frac{\sqrt{5}}2)^n)
	\end{align*}
	\item The generating function where no part occurs more than 3 times is just
	\begin{align*}
		(1+x+x^2+x^3)(1+x^2+x^4+x^6)(1+x^3+x^6+x^9) \cdots = \frac{1-x^4}{1-x} \cdot \frac{1-x^8}{1-x^2} \cdot \frac{1-x^{12}}{1-x^3} \cdots
	\end{align*}
	We can see that all terms on the top will cancel with terms on the bottom of the form $1-x^{4k}$ for integers $k$. This leaves
	\begin{align*}
		\prod_{\substack{i \geq 1 \\ i \not \equiv_4 0}} \frac{1}{1-x^i} = \prod_{\substack{i \geq 1 \\ i \not \equiv_4 0}} (1+x^i+x^{2i}+x^{3i} + \cdots)
	\end{align*}This is just the generating function for the number of partitions with no part a multiple of 4, which completes the proof.
	\end{enumerate}
\end{document}
