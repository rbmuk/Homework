\documentclass[12pt]{article}
\usepackage[margin=1in]{geometry}

% Start of preamble
%==========================================================================================%
% Required to support mathematical unicode
\usepackage[warnunknown, fasterrors, mathletters]{ucs}
\usepackage[utf8x]{inputenc}

% Always typeset math in display style
%\everymath{\displaystyle}

% Standard mathematical typesetting packages
\usepackage{amsmath,amssymb,amscd,amsthm,amsxtra, pxfonts}
\usepackage{mathtools,mathrsfs,dsfont,xparse}

% Symbol and utility packages
\usepackage{cancel, textcomp}
\usepackage[nointegrals]{wasysym}

% Extras
\usepackage{physics}  % Lots of useful shortcuts and macros
\usepackage{tikz-cd}  % For drawing commutative diagrams easily
\usepackage{color}  % Add some color to life
\usepackage{microtype}  % Minature font tweaks
%\usepackage{pgfplots} % plots

\usepackage{enumitem}
\usepackage{titling}

\usepackage{graphicx}

% Common shortcuts
\def\mbb#1{\mathbb{#1}}
\def\mfk#1{\mathfrak{#1}}

\def\bN{\mbb{N}}
\def \C{\mbb{C}}
\def \R{\mbb{R}}
\def\bQ{\mbb{Q}}
\def\bZ{\mbb{Z}}
\def \cph{\varphi}
\renewcommand{\th}{\theta}
\def \ve{\varepsilon}
\newcommand{\mg}[1]{\| #1 \|}

% Sometimes helpful macros
\newcommand{\floor}[1]{\left\lfloor#1\right\rfloor}
\newcommand{\ceil}[1]{\left\lceil#1\right\rceil}
\renewcommand{\qed}{\hfill\qedsymbol}

% Sets
\DeclarePairedDelimiterX\set[1]\lbrace\rbrace{\def\given{\;\delimsize\vert\;}#1}

% Some standard theorem definitions
\newtheorem{theorem}{Theorem}[section]
\newtheorem{corollary}{Corollary}[theorem]
\newtheorem{lemma}[theorem]{Lemma}

\theoremstyle{definition}
\newtheorem{definition}{Definition}[section]

\theoremstyle{remark}
\newtheorem*{remark}{Remark}

% End of preamble
%==========================================================================================%

\usepackage{braket}

% Start of commands specific to this file
%==========================================================================================%

\renewcommand{\ip}[2]{\langle #1, #2 \rangle}
\newcommand{\linf}[1]{\max_{1\leq i \leq #1}}
\newcommand{\seq}[2]{\qty(#1_#2)_{#2=1}^{\infty}}

%==========================================================================================%
% End of commands specific to this file

\title{Math 461 Last Homework}
\date{\today}
\author{Rohan Mukherjee}

\begin{document}
	\maketitle
	\begin{enumerate}[leftmargin=\labelsep]
		\item Define the following map: to each matrix with two rows with entries $1, \ldots, 2n$ with the entries increasing from left to right and top to bottom, for every $j$ in the top row, put an open parenthesis at spot $j$, and for every $k$ in the bottom row, put a closed parenthesis at spot $k$. Every string of parenthesis is valid. Every string of parenthesis is valid since we can only place closing parenthesis after each other, since the bottom row is increasing, and we must close all previous parenthesis, since going from top to bottom is increasing. For example, in this case
		\begin{align*}
			\begin{pmatrix} 1 & 2 & 3 \\ 4 & 5 & 6 \end{pmatrix}
		\end{align*}
		Since the 5 is in spot 2, and since going from top to bottom is increasing, and since the top row is increasing, we are guaranteed to have two left parenthesis before it, so we can certainly close it.
		
		The inverse map is as follows: given a string of $n$ left and $n$ right parenthesis that is correctly matched, if the indexes of the left parenthesis are $(1, a_2, \ldots, a_n)$ and $(b_1, \ldots, b_n)$, our matrix would just be
		\begin{align*}
			\begin{pmatrix} 1 & a_2 & \cdots & a_n \\ b_1 & b_2 & \cdots & b_n \end{pmatrix}
		\end{align*}
		The indexes of parenthesis are increasing so both the top and bottom row are increasing. Now, this increases from top to bottom because if it didn't, we would have a right parenthesis in spot $i$ with $< i$ left parenthesis before it, which can't be.
		
		It would look like this for $n=8$:
		\begin{align*}
			\begin{pmatrix} 1 & 3 & 5 & 7 & 9 & 10 & 13 & 15 \\ 2 & 4 & 6 & 8 & 11 & 12 & 14 & 16 \end{pmatrix} \leftrightarrow ()()()()(())()()
		\end{align*}
		
		\item First notice that a plane tree with $n+1$ leaves has exactly $n$ edges. Perform a DFS on this plane tree, and any time you go down an edge, walk right, and every time you go back up the edge, walk up. Every plane tree will admit a walk from $(0, 0)$ to $(n, n)$ where we stay below the diagonal, since each edge we go up we had to walk down it previously, so the number of right steps will be at least the number of up steps. Similarly, the inverse map would be: 
		
		(1) Place a root node
		
		(2) If you walk right, place a new child and go down to it, and if you walk up, go up one node
		
		\item We claim we can find a bijection between this and parenthesis around $12 \cdots n$. Put a clockwise ordering on the points. If $i$ is in the same set as $i+1$, add a parenthesis, otherwise close a parenthesis, and once we reach 1 again close one last parenthesis. Since we don't have any segments crossing each other, we are guaranteed that parenthesis will remain balanced--otherwise, we would have a situation as follows:
		\[\begin{tikzcd}
			2 & 3 \\
			1 & 4
			\arrow[no head, from=2-1, to=1-2]
			\arrow[no head, from=1-1, to=2-2]
		\end{tikzcd}\]
		Where arrows are in the same set, which would give us $(1(2(3(4))$, which is unbalanced. In general if we had more numbers in these two sets we would have a similar case but just more numbers in between instead of just 1, a contradiction. To go back, we put all numbers in the same parenthesis level in the same set. Examples are attached.
		
		
		\item We construct a bijection from sequences with $a_1 = 0$ and $0 \leq a_{i+1} \leq a_i + 1$ and plane trees with $i+1$ vertices. Place a root node. Construct a tree inductively: $a_1$ corresponds to putting the first node as a child of the root node, and then place $a_i$ at level $i$ (the first node we added is at level 0), a child of the rightmost node on level $i-1$ (such a node is guaranteed to exist, since $a_{i+1} \leq a_i + 1$, and each node is at level at least 0 since $0 \leq a_i$). To go back, perform a DFS, and when you see the $j$th new node, record what level of the tree it is at with the same scheme as above. That will produce a string where the next element could either be before it, or only one after it. From the above plane trees with $n+1$ vertices are in bijection with the Catalan numbers, so we are done. Examples are as follows.
		
		\item We shall show they are in bijection with the number of ways to put parenthesis around $12\cdots n$. Notice exactly one rectangle contains the furthest right element on the $i$th row for each $i$. We shall call this the $i$th diagonal element. Starting from the bottom, if the rectangle with the second diagonal element also tiles the first row, then don't open a parenthesis, otherwise, open a parenthesis. Repeat this process--the rectangle with the 3rd diagonal element either sits at the same level as the second, or doesn't, and now this is either inside parenthesis showing that these both sit above the first or don't. Once there are no more rectangles sitting above the one who opened the parenthesis, close the parenthesis. To go backwards, notice that the width of each rectangle will be uniquely specified the number of nested parenthesis it has next to it plus 1, and you can determine it's height from knowing exactly which rectangles it sits above. Since the number of ways to put parenthesis around $12 \cdots n$ is just $C_n$, this shows we are in bijection with the Catalan numbers. Examples.
		
		
	\end{enumerate}
\end{document}