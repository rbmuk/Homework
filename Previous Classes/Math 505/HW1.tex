\documentclass[12pt]{article}
\usepackage[margin=1in]{geometry}
\usepackage{setspace}
\onehalfspacing

% Start of preamble
%==========================================================================================%
% Required to support mathematical unicode
\usepackage[warnunknown, fasterrors, mathletters]{ucs}
\usepackage[utf8x]{inputenc}

\usepackage[dvipsnames,table,xcdraw]{xcolor} % colors
\usepackage{hyperref} % links
\hypersetup{
	colorlinks=true,
	linkcolor=blue,
	filecolor=magenta,      
	urlcolor=cyan,
	pdfpagemode=FullScreen
}

% Standard mathematical typesetting packages
\usepackage{amsmath,amssymb,amscd,amsthm,amsxtra, pxfonts}
\usepackage{mathtools,mathrsfs,xparse}

% Symbol and utility packages
\usepackage{cancel, textcomp}
\usepackage[mathscr]{euscript}
\usepackage[nointegrals]{wasysym}
\usepackage{apacite}

% Extras
\usepackage{physics}  % Lots of useful shortcuts and macros
\usepackage{tikz-cd}  % For drawing commutative diagrams easily
\usepackage{microtype}  % Minature font tweaks
%\usepackage{pgfplots} % plots

\usepackage{enumitem}
\usepackage{titling}

\usepackage{graphicx}

% Fancy theorems due to @intuitively on discord
\usepackage{mdframed}
\newmdtheoremenv[
backgroundcolor=NavyBlue!30,
linewidth=2pt,
linecolor=NavyBlue,
topline=false,
bottomline=false,
rightline=false,
innertopmargin=10pt,
innerbottommargin=10pt,
innerrightmargin=10pt,
innerleftmargin=10pt,
skipabove=\baselineskip,
skipbelow=\baselineskip
]{mytheorem}{Theorem}

\newenvironment{theorem}{\begin{mytheorem}}{\end{mytheorem}}

\newtheorem{corollary}{Corollary}
\newtheorem{lemma}{Lemma}

\newtheoremstyle{definitionstyle}
{\topsep}%
{\topsep}%
{}%
{}%
{\bfseries}%
{.}%
{.5em}%
{}%
\theoremstyle{definitionstyle}
\newmdtheoremenv[
backgroundcolor=Violet!30,
linewidth=2pt,
linecolor=Violet,
topline=false,
bottomline=false,
rightline=false,
innertopmargin=10pt,
innerbottommargin=10pt,
innerrightmargin=10pt,
innerleftmargin=10pt,
skipabove=\baselineskip,
skipbelow=\baselineskip,
]{mydef}{Definition}
\newenvironment{definition}{\begin{mydef}}{\end{mydef}}

\newtheorem*{remark}{Remark}

\newtheorem*{example}{Example}

% Common shortcuts
\def\mbb#1{\mathbb{#1}}
\def\mfk#1{\mathfrak{#1}}

\def\bN{\mbb{N}}
\def \C{\mbb{C}}
\def \R{\mbb{R}}
\def\bQ{\mbb{Q}}
\def\bZ{\mbb{Z}}
\def \cph{\varphi}
\renewcommand{\th}{\theta}
\def \ve{\varepsilon}
\newcommand{\mg}[1]{\| #1 \|}

% Often helpful macros
\newcommand{\floor}[1]{\left\lfloor#1\right\rfloor}
\newcommand{\ceil}[1]{\left\lceil#1\right\rceil}
\renewcommand{\qed}{\hfill\qedsymbol}
\renewcommand{\ip}[2]{\langle #1, #2 \rangle}
\newcommand{\seq}[2]{\qty(#1_#2)_{#2=1}^{\infty}}

% Sets
\DeclarePairedDelimiterX\set[1]\lbrace\rbrace{\def\given{\;\delimsize\vert\;}#1}

% End of preamble
%==========================================================================================%

% Start of commands specific to this file
%==========================================================================================%

\usepackage{braket}
\newcommand{\Z}{\mbb Z}
\newcommand{\gen}[1]{\left\langle #1 \right\rangle}
\newcommand{\nsg}{\trianglelefteq}
\newcommand{\F}{\mbb F}
\newcommand{\Aut}{\mathrm{Aut}}

%==========================================================================================%
% End of commands specific to this file

\title{Math 504 HW7}
\date{\today}
\author{Rohan Mukherjee}

\begin{document}
	\maketitle
	\begin{enumerate}[leftmargin=\labelsep]
		\item We begin by proving the following lemma:
		
		\begin{lemma}
			Let $p$ be a prime element of a UFD $A$. Then 
			\begin{align*}
				(p) = \Set{\begin{array}{l}
						\text{Polynomials with coefficients in $A$}  \\ \quad \quad \quad \quad \text{divisible by $p$}
					\end{array}}
			\end{align*}
			is a prime ideal of $A[x]$. 
		\end{lemma}
		\begin{proof}
			Let $f(x) = a_0 + a_1x + \cdots a_nx^n$ and $g(x) = b_0 + b_1x + \cdots + b_mx^m$ with $fg \in (p)$. Obviously $p \mid a_nb_m$, so WLOG $p \mid a_n$. Then,
			\begin{align*}
				fg = (a_0 + \cdots + a_nx^n)(b_0 + b_1x + \cdots + b_mx^m) = (a_0 + \cdots a_{n-1}x^{n-1})g + a_nx^ng
			\end{align*}
			Clearly showing that $(a_0 + \cdots a_{n-1}x^{n-1})g \in (p)$. Now repeat the procedure: if $p \mid a_{n-1}$, we can run the above again, and eventually either all the $a_i$'s are divisible by $p$ or there is a $k$ such that $a_k$ is not divisible by $p$, and $(a_0 + \cdots + a_kx^k)g \in (p)$. In this case, we have $p \mid a_kb_m$ but $p \not \mid a_k$, so $p \mid b_m$. Using the above argument again, we can get $(a_0 + \cdots + a_kx^k)(b_0 + \cdots + b_{m-1}x^{m-1}) \in (p)$, which shows $p \mid a_kb_{m-1}$ which forces $p \mid b_{m-1}$. Continuing this process \textit{ad infinitum} shows that $p \mid b_i$ for every $i$, i.e. that $g \in (p)$, which completes the proof.
		\end{proof}
		
		\begin{proof}[Sketch of second proof.]
			A simple exercise shows that if $A \subset R$ is an ideal, then $R[x]/A[x] \cong (R/A)[x]$. Also, if $S$ is an integral domain, then $S[x]$ is too (one could see this by looking at just the leading coefficient). So, if $p$ is prime in $R$, then $(p)$ is a prime ideal of $R$, so $R[x]/(p)[x] \cong (R/(p))[x]$ which is clearly an integral domain, completing the sketch.
		\end{proof}
	
		Let $p$ be a prime dividing the gcd of the coefficients of $fg$ and $m$ be the max power of $p$ appearing in the gcd of the coefficients. Then $fg \in (p^n)$ and in particular $fg \in (p)$, which by the lemma shows that (WLOG) $f \in (p)$. Then $\frac fp \in A[x]$, and,
		\begin{align*}
			\frac{f}{p} g \in (p^{n-1}) \subset (p)
		\end{align*}
		Then $\frac fp \in (p)$ or $g \in (p)$. Repeating this process will yield a $k \in \bN$ such that $\frac{f}{p^k} \in A[x]$ and $\frac{g}{p^{n-k}} \in A[x]$, which shows that $p^k \mid c(f)$ and $p^{n-k} \mid c(g)$, so $p^n \mid c(fg)$. Now, since $c(f)c(g) \mid a_ib_{n-i}$ for every $i$, we have $c(f)c(g) \mid c(fg)$. Since the above holds for every prime $p$ dividing $c(fg)$, we have $c(fg) \mid c(f)c(g)$, so $c(fg) = c(f)c(g)$, which completes the proof.
		
		\item Suppose that $f \in A[x]$ is reducible in $K[x]$ and write $f = gh$ for $g(x) = \frac{c_0}{a_0} + \cdots + \frac{c_n}{a_n}x^n$ and $h(x) = \frac{d_0}{b_0} + \cdots + \frac{d_m}{b_m}x^m \in K[x]$, and assume that $c(f) = 1$. Then,
		\begin{align*}
			\prod a_ib_i f = \prod a_i g \cdot \prod b_i h
		\end{align*}
		Now, $\prod a_i g, \prod b_i h \in A[x]$, so, since $\gcd(da, db) = d\gcd(a, b)$, we have that $\prod a_ib_i = \prod a_ib_i c(f) = c(\prod a_i b_i f) = c(\prod a_i g) c(\prod b_i h)$. Now let $p$ be a prime divisor and $\alpha$ its maximal power in the prime decomposition of $\prod a_ib_i$. Then $p \mid c(\prod a_i g) c(\prod b_i h)$ so $p \mid c(\prod a_i g)$ or $p \mid c(\prod b_i h)$, WLOG suppose its the first case. Repeat this process inductively until we find $k$ such that $c(\prod a_i g)/p^k \in A$ and $c(\prod b_i h)/p^{\alpha-k} \in A$. Then we see that $\prod a_ib_i / p^\alpha \cdot f = g/p^k \cdot h/p^{\alpha-k}$ where $g/p^k \in A[x]$ and $h/p^{\alpha-k} \in A[x]$. Since this holds for every prime divisor of $\prod a_ib_i$, we can repeat this to eventually get $f = kl$ for $k, l \in A[x]$, which shows that $f$ is reducible in $A[x]$.
		
		\item We shall first prove that all polynomials with content 1 can be factored as a product of irreducibles, and we shall do so by strong induction on the degree. Let $f(x) = ax + b$ be a degree 1 polynomial with $a \neq 0$ and content 1. If $f(x) = g(x)h(x)$, then WLOG $\deg g(x) = 1$ and $\deg h(x) = 0$ by some simple casework. Now, by the above we have that $1 = c(f) = c(g) \cdot c(h) = c(g) \cdot h$, so $h$ is a unit and $f$ is irreducible. Suppose that all polynomials with degree $\leq n$ can be factored as a product of irreducibles for some $n > 1$, and let $f(x)$ be degree $n+1$ with content 1. If $f(x)$ is irreducible we are done, otherwise we have $f(x) = g(x)h(x)$ where neither $g$ nor $h$ are units. Re-using the above if $g$ or $h$ had degree 0 then they would be units a contradiction, so each has degree $\geq 1$ and $\deg g, \deg h \leq n$. Then $g, h$ can be factored as a product of irreducibles and henceforth $f$ can do. Now, if $f$ is any polynomial in $A[x]$, then write $f = c(f)g$ for some $g$ with content 1. Now, since $A$ is a UFD $c(f)$ can be written as a product of irreducibles in $A$, which are also irreducible in $A[x]$ by degree considerations, and $g$ can be written as a product of irreducibles by the above so $f$ can too.
		
		Now, suppose that $f$ has the following  factorizations:
		\begin{align*}
			f = wp_1^{\alpha_1} \cdots p_n^{\alpha_n} = vq_1^{\beta_1} \cdots q_m^{\beta_m}
		\end{align*}
		where $w,v$ are units and the rest are irreducibles. Notice that these are also factorizations of $f$ into a product of irreducibles in the UFD $\mathrm{Frac}(A)[x]$ by the previous lemma. Thus, $m = n$ and there is a bijection $\cph: \Set{p_1, \ldots, p_n} \to \Set{q_1, \ldots, q_n}$ such that $\cph(p_i)$ is an associate of $q_j$. Now, suppose that $p(x) = \frac rs q(x)$ where $\frac rs$ is a unit in $\mathrm{Frac}(A)$. Then $sp(x) = rq(x)$, so $sc(p) = rc(q)$ meaning $\frac{s}{r} = c(p)^{-1}c(q) \in A$ since $c(p)$ is a unit, showing that $\frac rs$ is a unit in $A$ and so $p$ is an associate of $q$ in $A[x]$, verifying uniqueness on polynomials with content 1. In the more general case, we can write $f = c(f) \cdot g$ for a polynomial of content 1 $g$. This decomposition is clearly unique up to units (since $c(f)$ is unique up to a unit). Now, $g$ can be written uniquely as the product of irreducibles, and $c(f)$ can too since $A$ is a UFD. Also, given any factorization of $f$ as $w \cdot p_1^{\alpha_1} \cdots p_n^{\alpha_n}$ where $p_i$ are not necessarily content-1, we can just factor out the content from each and collect it towards the front yielding a decomposition of the above form, which is unique. 
		
		\item We start by proving the following lemma:
		
		\begin{lemma}
			Let $R$ be an integral domain and $n \geq 1$. Then the only divisors of $cx^n$ for $0 \neq a \in R$ in $R[x]$ are $dx^k$ for $0 \leq k \leq n$ and $d \mid c$.
		\end{lemma}
	
		\begin{proof}
			Let $f(x) = a_kx^k + \cdots + a_0$ be a divisor of $cx^n$. Then there is a $g(x) = b_{n-k}x^{n-k} + \cdots + b_0$ such that $f(x)g(x) = x^n$ since $\deg g = n - \deg f = n-k$. Now, $f(x)g(x) = a_kb_{n-k}x^n + \cdots + a_0b_0$. In particular, $a_kb_{n-k} = c$, so $a_k \mid c$. Now, suppose that there was an $i < k$ so that $a_i \neq 0$, and then find the minimum such $j$. Simultaneously find the minimum $l$ such that $b_l \neq 0$ (this could be $b_{n-k}$). Then the term with the smallest power in $f(x)g(x)$ is $a_jb_lx^{j+l}$. Since $j < k$ and $l \leq n-k$ we have $j+l < n$, which shows that this product has a term other than $cx^n$, a contradiction. Thus $f(x) = a_kx^k$ for $a_k \mid c$. It is obvious to see that this is indeed a divisor, so we are done.
		\end{proof}
	
		Now let bars denote passage to $A/(p)[x]$ (i.e., reducing the coefficients mod $(p)$). Suppose that $f$ has content 1. We shall show that $f$ is irreducible in $A[x]$. Indeed, let $f(x) = g(x)h(x)$ where neither $g(x)$ nor $h(x)$ are units. By the same content considerations as above, this shows that neither $g(x)$ nor $h(x)$ are constant, otherwise they would be units. By the lemma $\bar g(x)$ and $\bar h(x)$ are of the form $dx^k$ for some $k$. In particular, the constant term of both $g(x)$ and $h(x)$ is divisible by $p$. But then the product of their constant terms would be divisible by $p^2$, a contradiction, which shows that $f$ is irreducible in $A[x]$. In the more general case, write $f(x) = c(f)g(x)$ for a content-1 polynomial $g(x)$. $g(x)$ is now irreducible in $K[x]$ and $c(f)$, being in $A$, is a unit in $K$ and therefore $K[x]$, so $f(x)$ is an associate of a irreducible and hence itself irreducible, which completes the proof.
		
		\item Let $f(x) = x^{p-1} + \cdots + x + 1$, and notice that $f(x)(x-1) = x^p - 1$. Thus, $f(x+1)x = (x+1)^p - 1$, and,
		\begin{align*}
			f(x+1) = \frac{(x+1)^p - 1}{x} = \sum_{k=1}^p {p \choose k}x^{k-1} = \sum_{k=0}^{p-1} {p \choose k+1} x^k
		\end{align*}
		Notice that the constant term is ${p \choose 1} = p$, and the leading coefficient is ${p \choose p} = 1$. For $1 \leq k \leq p-1$, ${p \choose k}$ is divisible by $p$ since,
		\begin{align*}
			{p \choose k} = \frac{p \cdot (p-1) \cdots (p-k+1)}{k \cdot (k-1) \cdots 1}
		\end{align*}
		And no term on the bottom can cancel the $p$ on the top since $p$ is a prime greater than $k$, meaning its only divisors are $p$ and $1$ so to cancel it $p$ would have to appear on the bottom. We are now in the case to apply Eisenstein: $p^2$ does not divide the constant term, $p$ does not divide the leading coefficient, and $p$ divides every other coefficient, so $f(x+1)$ is irreducible. If $f(x) = g(x)h(x)$ where neither $g$ nor $h$ are constants, then $f(x+1) = g(x+1)h(x+1)$, where neither $g(x+1)$ nor $h(x+1)$ are constant, a contradiction. So $f(x)$ is irreducible too, and we are done.
	\end{enumerate}
\end{document}