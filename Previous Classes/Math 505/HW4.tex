\documentclass[12pt]{article}
\usepackage[margin=1in]{geometry}
\usepackage{setspace}
\onehalfspacing

% Start of preamble
%==========================================================================================%
% Required to support mathematical unicode
\usepackage[warnunknown, fasterrors, mathletters]{ucs}
\usepackage[utf8x]{inputenc}

\usepackage[dvipsnames,table,xcdraw]{xcolor} % colors
\usepackage{hyperref} % links
\hypersetup{
	colorlinks=true,
	linkcolor=blue,
	filecolor=magenta,      
	urlcolor=cyan,
	pdfpagemode=FullScreen
}

% Standard mathematical typesetting packages
\usepackage{amsmath,amssymb,amscd,amsthm,amsxtra, pxfonts}
\usepackage{mathtools,mathrsfs,xparse}

% Symbol and utility packages
\usepackage{cancel, textcomp}
\usepackage[mathscr]{euscript}
\usepackage[nointegrals]{wasysym}
\usepackage{apacite}

% Extras
\usepackage{physics}  % Lots of useful shortcuts and macros
\usepackage{tikz-cd}  % For drawing commutative diagrams easily
\usepackage{microtype}  % Minature font tweaks
%\usepackage{pgfplots} % plots

\usepackage{enumitem}
\usepackage{titling}

\usepackage{graphicx}

% Fancy theorems due to @intuitively on discord
\usepackage{mdframed}
\newmdtheoremenv[
backgroundcolor=NavyBlue!30,
linewidth=2pt,
linecolor=NavyBlue,
topline=false,
bottomline=false,
rightline=false,
innertopmargin=10pt,
innerbottommargin=10pt,
innerrightmargin=10pt,
innerleftmargin=10pt,
skipabove=\baselineskip,
skipbelow=\baselineskip
]{mytheorem}{Theorem}

\newenvironment{theorem}{\begin{mytheorem}}{\end{mytheorem}}

\newtheorem{corollary}{Corollary}
\newtheorem{lemma}{Lemma}

\newtheoremstyle{definitionstyle}
{\topsep}%
{\topsep}%
{}%
{}%
{\bfseries}%
{.}%
{.5em}%
{}%
\theoremstyle{definitionstyle}
\newmdtheoremenv[
backgroundcolor=Violet!30,
linewidth=2pt,
linecolor=Violet,
topline=false,
bottomline=false,
rightline=false,
innertopmargin=10pt,
innerbottommargin=10pt,
innerrightmargin=10pt,
innerleftmargin=10pt,
skipabove=\baselineskip,
skipbelow=\baselineskip,
]{mydef}{Definition}
\newenvironment{definition}{\begin{mydef}}{\end{mydef}}

\newtheorem*{remark}{Remark}

\newtheorem*{example}{Example}
\newtheorem*{claim}{Claim}

% Common shortcuts
\def\mbb#1{\mathbb{#1}}
\def\mfk#1{\mathfrak{#1}}

\def\bN{\mbb{N}}
\def \C{\mbb{C}}
\def \R{\mbb{R}}
\def\bQ{\mbb{Q}}
\def\bZ{\mbb{Z}}
\def \cph{\varphi}
\renewcommand{\th}{\theta}
\def \ve{\varepsilon}
\newcommand{\mg}[1]{\| #1 \|}

% Often helpful macros
\newcommand{\floor}[1]{\left\lfloor#1\right\rfloor}
\newcommand{\ceil}[1]{\left\lceil#1\right\rceil}
\renewcommand{\qed}{\hfill\qedsymbol}
\renewcommand{\ip}[2]{\langle #1, #2 \rangle}
\newcommand{\seq}[2]{\qty(#1_#2)_{#2=1}^{\infty}}

% Sets
\DeclarePairedDelimiterX\set[1]\lbrace\rbrace{\def\given{\;\delimsize\vert\;}#1}

% End of preamble
%==========================================================================================%

% Start of commands specific to this file
%==========================================================================================%

\usepackage{braket}
\newcommand{\Z}{\mbb Z}
\newcommand{\gen}[1]{\left\langle #1 \right\rangle}
\newcommand{\nsg}{\trianglelefteq}
\newcommand{\F}{\mbb F}
\newcommand{\Aut}{\mathrm{Aut}}
\newcommand{\sepdeg}[1]{[#1]_{\mathrm{sep}}}
\newcommand{\Q}{\mbb Q}
\newcommand{\Gal}{\mathrm{Gal}\qty}

%==========================================================================================%
% End of commands specific to this file

\title{Math 505 HW4}
\date{\today}
\author{Rohan Mukherjee}

\begin{document}
	\maketitle
		\subsection*{Problem 1.}
		
		\begin{enumerate}[label=(\arabic*)]
			\item We showed on HW1 (and HW3) that $f(x) = x^{p-1} + \cdots + 1$ is irreducible. Since $\Q$ has characteristic 0 any irreducible polynomial is separable, so we only need to show that $\Q(\rho) / \Q$ is normal. This follows immediately as,
			\begin{align*}
				f(x) = \prod_{n=1}^{p-1} (x-\rho^n)
			\end{align*}
			As was shown on HW1, so $\Q(\rho)$ is a splitting field and hence normal.
			
			\item We shall show that $\Gal(\Q(\rho)/\Q)$ is cyclic and has order $p-1$, thus is isomorphic to $(\Z/p)^\times$. First, notice that the claim is trivial for $p = 2$ since in that case the splitting field of $x+1$ is just $\Q$, so let $p$ be an odd prime. Let $\alpha$ be a generator for the cyclic group $(\Z/p)^\times$. We claim that $\Gal(\Q(\rho)/\Q)$ is generated by
			\begin{align*}
				\sigma&: \Q(\rho) \to \Q(\rho) \\
				\sigma&: \rho \mapsto \rho^\alpha
			\end{align*}
			Notice first that
			\begin{align*}
				\Gal(\Q(\rho)/\Q) = \Set{\sigma: \rho \mapsto \rho^n \mid n \in [p-1]}
			\end{align*}
			Since specifying where $\rho$ goes completely determines the automorphism, and we have $p-1$ choices to send $\rho$ to, being any root of $f(x) = x^{p-1} + \cdots + 1$. Let $n \in [p-1]$ be an integer. Since $\alpha$ is a generator of $(\Z/p)^\times$, we can find a $k$ such that $\alpha^k = n$. Now, $\sigma^k(\rho) = \rho^{\alpha^k} = \rho^n$, which completes the proof. The obvious isomorphism is just $\alpha \mapsto \sigma: \rho \mapsto \rho^\alpha$.
		\end{enumerate}
	
		\subsection*{Problem 2.}
		Let $f(x) = x^p-2$ and $\zeta$ be a primitive $p$th root of unity. We see immediately that $\Set{\sqrt[p]{2}, \sqrt[p]{2}\zeta, \;\ldots, \sqrt[p]{2}\zeta^{p-1}}$ are all the $p$ distinct roots of this irreducible polynomial, so in particular,
		\begin{align*}
			\Q_f = \Q\qty(\sqrt[p]{2}, \zeta)
		\end{align*}
		We have the following diagram of field extensions:
		\[\begin{tikzcd}
			& {\bQ\qty(\sqrt[p]{2}, \zeta)} \\
			{\bQ\qty(\sqrt[p]{2})} && {\bQ\qty(\zeta)} \\
			& \bQ
			\arrow[no head, from=1-2, to=2-1]
			\arrow[no head, from=1-2, to=2-3]
			\arrow["p", no head, from=3-2, to=2-1]
			\arrow["{p-1}"', no head, from=3-2, to=2-3]
		\end{tikzcd}\]
		Where the degrees marked are obvious (For example, $x^p-2$ is irreducible, and we calculated the bottom-right extension in the last problem). Since $g(x) = x^{p-1} + \cdots + 1$ is a polynomial with $\zeta$ as a root, it follows that $|\Q\qty(\sqrt[p]{2}, \zeta) : \Q(\sqrt[p]{2})| \leq p-1$. Since $p$ and $p-1$ are coprime, we have that $p(p-1) \mid |\Q\qty(\sqrt[p]{2}, \zeta) : \Q|$, and also that
		\begin{align*}
			|\Q\qty(\sqrt[p]{2}, \zeta) : \Q| = |\Q\qty(\sqrt[p]{2}, \zeta) : \Q(\sqrt[p]{2})| \cdot |\Q(\sqrt[p]{2}) : \Q| \leq (p-1)p
		\end{align*}
		Thus $|\Q\qty(\sqrt[p]{2}, \zeta) : \Q| = p(p-1)$. Once again let $\alpha$ be a generator for $(\Z/p)^\times$. Notice that,
		\begin{align*}
			\sigma&: \begin{cases} \sqrt[p]{2} \mapsto \sqrt[p]{2}\zeta \\ \zeta \mapsto \zeta \end{cases} \quad \text{has order $p$} \\
			\tau&: \begin{cases} \sqrt[p]{2} \mapsto \sqrt[p]{2} \\ \zeta \mapsto \zeta^\alpha \end{cases}\quad \text{has order $p-1$}
		\end{align*}
		Notice also that if $G$ is a group of order $p(p-1)$, then $n_p \equiv 1 \mod p$ and $n_p \mid p-1$ so the subgroup of order $p$ is normal, so if $H \leq G$ is the subgroup of order $p$, and if there is a subgroup $N$ of order $p-1$, then $G \cong H \rtimes N$. Notice that $\tau \sigma \tau^{-1}(\sqrt[p]{2}) = \tau \sigma(\sqrt[p]{2}) = \tau(\sqrt[p]{2} \zeta) = \sqrt[p]{2} \zeta^\alpha$, and $\tau \sigma \tau^{-1}(\zeta) = \zeta$. Thus $\tau \sigma \tau^{-1} = \sigma^\alpha$.
		
		Thus $\Gal(\Q\qty(\sqrt[p]{2}, \zeta) / \Q) \cong \gen{\sigma} \rtimes \gen{\tau} \cong \Z/p \rtimes (\Z/p)^\times$ with multiplication on the furthest right group given by $(a,b)(c,d) = (a+cb, bd)$. This is just the holomorph of $\Z/p$, i.e. $\Gal(\Q\qty(\sqrt[p]{2}, \zeta) / \Q) \cong \Z/p \rtimes \Aut(\Z/p) = \mathrm{Hol}(\Z/p)$.
		
		\subsection*{Problem 3.}
		\begin{enumerate}[label=(\arabic*)]
			\item We first prove the following lemma.
			\begin{lemma}
				For coprime $a, b \in \Z$, (at least) one of which is not a perfect square, $\sqrt{\frac ab} \not \in \Q$.
			\end{lemma}
			\begin{proof}
				Suppose that $d\sqrt{p} = c\sqrt{q}$ for $c, d \in \bZ$ coprime neither of which are 0. Squaring both sides shows that $d^2b=c^2a$. WLOG let $a$ be the non-perfect square, and $p$ a prime divisor whose power in the prime factorization of $a$ is not a multiple of 2. It follows that $p \mid d^2$ so $p \mid d$, so write $d = d'p$ and we see that $d'^2p^2b = c^2a$, equivalently $d^2pb = c^2\frac ap$. If $\frac ap$ is not divisible by $p$ then $p \mid c^2$ so $p \mid c$ a contradiction, otherwise keep canceling factors of $p$ from $a$ until this happens.
			\end{proof}
			We claim that $\Q(\sqrt{17}, \sqrt{239})$ is a degree 4 Galois extension over $\Q$. If not, we would have $\Q(\sqrt{17}, \sqrt{239}) = \Q(\sqrt{17}) = \Q(\sqrt{239})$. Then $\sqrt{239} = a\sqrt{17} + b$. Squaring both sides shows that $b = 0$ (otherwise $\sqrt{17}$ would be rational). This would say that $\sqrt{\frac{239}{17}} \in \Q$, which is false by the lemma. Notice that $|\Q(\sqrt{17}, \sqrt{239}) : \Q| \leq 4$ since $x^2-239$ is a degree 2 polynomial with coefficients in $\Q(\sqrt{17})$ with $\sqrt{239}$ as a root. Thus $\Q(\sqrt{17}, \sqrt{239}$ has degree strictly greater than 2, $\leq 4$, and divisible by 2, so it equals 4. The extension is Galois as $\Q$ has characteristic 0 and it is the splitting field of $(x^2-17)(x^2-239)$. We now compute $\Gal(\Q(\sqrt{17}, \sqrt{239})/\Q)$. $x^2-17$ is an irreducible polynomial with $\sqrt{17}$ as a root, thus $\sqrt{17}$ must get sent to either itself or the other root of this polynomial: $-\sqrt{17}$. $x^2-\sqrt{239}$ is an irreducible polynomial with coefficients in $\Q(\sqrt{17})$ (It is irreducible by our lemma above--this field does not contain $\sqrt{239}$), so $\sqrt{239}$ goes to $\pm \sqrt{239}$. Thus the Galois group is generated by $\sigma: \sqrt{17} \mapsto -\sqrt{17}$ and $\tau: \sqrt{239} \mapsto -\sqrt{239}$, which are both of order 2, so the Galois group is equal to $\Z/2 \times \Z/2$. The four conjugates of $\sqrt{17}+\sqrt{239}$ under the action of the Galois group are
			\begin{align*}
				\pm \sqrt{17} \pm \sqrt{239}
			\end{align*}
			In particular, the only element of the Galois group fixing $\sqrt{17}+\sqrt{239}$ is just $e$. Thus $\Q(\sqrt{17}+\sqrt{239}) = \Q(\sqrt{17}, \sqrt{239})^{\gen{e}} = \Q(\sqrt{17},\sqrt{239})$. Multiplying together
			\begin{align*}
				(x-(\sqrt{17}+\sqrt{239}))&(x-(-\sqrt{17}+\sqrt{239}))(x-(\sqrt{17}-\sqrt{239}))(x-(-\sqrt{17}-\sqrt{239}))
				\\ &= x^4-512x^2+49284
			\end{align*}
			Which is a monic degree 4 polynomial with $\sqrt{17}+\sqrt{239}$ as a root, and since the field extension $\Q(\sqrt{17}+\sqrt{239})$ has degree 4 this polynomial must be irreducible.
			
			\item Notice first that $\Q(1+\sqrt[3]{2}+\sqrt[3]{4}) = \Q(\sqrt[3]{2})$, since we have the forward inclusion and $1 < |\Q(1+\sqrt[3]{2}+\sqrt[3]{4}) : \Q| \leq 3$ (It cannot be 2 since its degree must divide 3). Now, we prove the following lemma.
			\begin{lemma}
				Let $F$ be an extension with a fixed algebraic closure $\overline F$, $\alpha, \beta \in \overline F$, $f = \mathrm{Irr}_F(\alpha)$, and $g = \mathrm{Irr}_F(\beta)$. If $F(\alpha) = F(\beta)$, then $F_f = F_g$.
			\end{lemma}
			\begin{proof}
				Let $|F(\alpha) : F| = |F(\beta) : F| = n$, and let $\alpha_2, \ldots, \alpha_n$ be the rest of the roots (not necessarily distinct) of $f$, and $\beta_2, \ldots, \beta_n$ be the rest of the roots of $g$. Since $F_f = F(\alpha, \alpha_2, \ldots, \alpha_n)$ is normal containing $\beta$, it must contain $\beta_2, \ldots, \beta_n$. The reverse inclusion is the same, which completes the proof.
			\end{proof}
			We need only complete the Galois group over the splitting field of $\Q(\sqrt[3]{2})$. But we have already done this in class--the splitting field is $\Q(\sqrt[3]{2}, \zeta)$, where $\zeta$ is a primitive 3rd root of unity, with Galois group $D_3 \cong S_3$. Now, recall that $\sigma: \sqrt[3]{2} \mapsto \sqrt[3]{2}\zeta$ was an automorphism of order 3. The powers of this automorphism will give us the conjugates of $1 + \sqrt[3]{2} + \sqrt[3]{4}$. The other two conjugates are,
			\begin{align*}
				&1 + \sqrt[3]{2}\zeta + \sqrt[3]{4}\zeta^2 \\
				&1 + \sqrt[3]{2}\zeta^2 + \sqrt[3]{4}\zeta
			\end{align*}
			Thus the minimal polynomial is,
			\begin{align*}
				(x-(1 + \sqrt[3]{2} + \sqrt[3]{4}))&(x-(1 + \sqrt[3]{2}\zeta + \sqrt[3]{4}\zeta^2))(x-(1 + \sqrt[3]{2}\zeta^2 + \sqrt[3]{4}\zeta)) 
				\\&= x^3 - 3x^2 - 3x - 1
			\end{align*}
			Notice once again that  this is indeed the minimal polynomial since it has the minimal degree of 3.
		\end{enumerate}
		\subsection*{Problem 4.}
		\begin{enumerate}[label=(\arabic*)]
			\item A reduction of $X^3-X-1$ over $\Q$ would yield a reduction over $\Z$, which would yield an integer root of this polynomial. Then $X(X^2-1) = 1$ would have an integer solution. Thus we must have either $X = 1$ and $X^2-1 = 1$, or $X=-1$ and $X^2 -1 = -1$. Both cases cannot be--for example, if $X=1$ then $X^2-1 = 0 \neq 1$, so $X^3-X-1$ is irreducible over $\Q$. We recall the theorem from class that says the Galois group will be $S_3$ iff the discriminant is not a square. Recall that the formula for the discriminant of a polynomial $f(x) = x^3 + px + q$ is just $-4p^3 - 27q^2$. So, the discriminant of $f(x) = X^3 - X - 1$ is just $-4(-1)^3 - 27(-1)^2 = 4 - 27 = -23$, so the discriminant is not a square since $\Q$ does not contain any complex values. Thus, $\Gal(\Q_f) = S_3$.
			
			\item The roots of this polynomial over $\C = \overline \Q(\sqrt{2})$ are $\sqrt{10}, \sqrt{10}\zeta, \sqrt{10}\zeta^2$ where $\zeta$ is a primitive third root of unity. Clearly the last two are not in $\Q(\sqrt{2})$, and the first isn't, otherwise $\sqrt{5} \in \Q(\sqrt{2})$, and by similar reasoning from problem 1 part (1) this would say that $\sqrt{5/2}$ is a rational number, a contradiction. Thus $X^3-10$ is irreducible. Once again we find the discriminant to be $-4(0^3) -27(10)^2 = -2700$. Again $\Q(\sqrt{2})$ does not contain any complex numbers, thus the discriminant is not a square, so the splitting field's Galois group will be $S_3$.
			
			\item Recall that a cubic polynomial is irreducible iff it splits into a linear and a quadratic factor. In particular, it must have a root. If $X^3-X-t$ was reducible in $\C(t)$, since $\C[t]$ is a UFD, $X^3-X-t$ would be reducible over $\C[t]$. Thus there would be a polynomial $p(t) \in \C[t]$ such that 
			\begin{align*}
				p(t)^3 = p(t) - t
			\end{align*}
			The degree of the LHS is $3\deg p(t)$, and the degree of the RHS is $\leq \max{\deg p(t), 1}$. Thus we have that either $3\deg p(t) \leq \deg p(t)$ thus $\deg p(t) = 0$, or $3\deg p(t) \leq 1$ thus $\deg p(t) = 0$. In any case $p(t) \equiv c \in \C$. This would claim that $c^3 = c - t$, but the degree of the RHS is 1 while the LHS is 0, a contradiction. 
			
			We calculate the discriminant to be $-4(-1)^3 - 27(-t)^2 = 4 - 27t^2$. This is a square iff $f(X) = X^2 - 4 + 27t^2$ has a root in $\C(t)$. Suppose instead that
			\begin{align*}
				a+bt^2 = \qty(\frac{p(t)}{q(t)})^2
			\end{align*}
			With $p(t), q(t) \in \C[t]$. The equation $q(t)^2(a+bt^2) = p^2(t)$ shows that $q^2(t) \mid p^2(t)$, so replace $r(t) \coloneqq \frac{p(t)}{q(t)} \in \C[t]$. Then we have the equation $a+bt^2 = r^2(t)$. We would then have
			\begin{align*}
				a+bt^2 = \qty(\sum_{i=0}^n c_it^i)^2
			\end{align*}
			For some coefficients $c_i$ with $c_n \neq 0$. The largest power of $t$ appearing in the right hand series is just $c_n^2t^{2n}$, and thus $n = 1$ (Since we may pass to an equality in $\C[t] \subset \C(t)$). This would claim that
			\begin{align*}
				a+bt^2 = (z+wt)^2 = z^2 + w^2t^2 + zwt
			\end{align*}
			From here we must have $zw = 0$, i.e. $z = 0$ or $w = 0$. For nonzero $a, b$, a quick check shows that neither of these cases work. In particular, $4-27t^2$ is not a square, thus $\Gal(\C(t)_f/\C(t)) = S_3$. I believe we can generalize the previous procedure to showing the Galois group of the splitting field of $X^3 - aX - bt$ over $\C(t)$ is $S_3$ for any nonzero $a,b$.
		\end{enumerate}
\end{document}