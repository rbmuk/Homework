\documentclass[12pt]{article}
\usepackage[margin=1in]{geometry}
\usepackage{setspace}
\onehalfspacing

% Start of preamble
%==========================================================================================%
% Required to support mathematical unicode
\usepackage[warnunknown, fasterrors, mathletters]{ucs}
\usepackage[utf8x]{inputenc}

\usepackage[dvipsnames,table,xcdraw]{xcolor} % colors
\usepackage{hyperref} % links
\hypersetup{
	colorlinks=true,
	linkcolor=blue,
	filecolor=magenta,      
	urlcolor=cyan,
	pdfpagemode=FullScreen
}

% Standard mathematical typesetting packages
\usepackage{amsmath,amssymb,amscd,amsthm,amsxtra, pxfonts}
\usepackage{mathtools,mathrsfs,xparse}

% Symbol and utility packages
\usepackage{cancel, textcomp}
\usepackage[mathscr]{euscript}
\usepackage[nointegrals]{wasysym}
\usepackage{apacite}

% Extras
\usepackage{physics}  % Lots of useful shortcuts and macros
\usepackage{tikz-cd}  % For drawing commutative diagrams easily
\usepackage{microtype}  % Minature font tweaks
%\usepackage{pgfplots} % plots

\usepackage{enumitem}
\usepackage{titling}

\usepackage{graphicx}

\usepackage{quiver}

% Fancy theorems due to @intuitively on discord
\usepackage{mdframed}
\newmdtheoremenv[
backgroundcolor=NavyBlue!30,
linewidth=2pt,
linecolor=NavyBlue,
topline=false,
bottomline=false,
rightline=false,
innertopmargin=10pt,
innerbottommargin=10pt,
innerrightmargin=10pt,
innerleftmargin=10pt,
skipabove=\baselineskip,
skipbelow=\baselineskip
]{mytheorem}{Theorem}

\newenvironment{theorem}{\begin{mytheorem}}{\end{mytheorem}}

\newtheorem{corollary}{Corollary}
\newtheorem{lemma}{Lemma}

\newtheoremstyle{definitionstyle}
{\topsep}%
{\topsep}%
{}%
{}%
{\bfseries}%
{.}%
{.5em}%
{}%
\theoremstyle{definitionstyle}
\newmdtheoremenv[
backgroundcolor=Violet!30,
linewidth=2pt,
linecolor=Violet,
topline=false,
bottomline=false,
rightline=false,
innertopmargin=10pt,
innerbottommargin=10pt,
innerrightmargin=10pt,
innerleftmargin=10pt,
skipabove=\baselineskip,
skipbelow=\baselineskip,
]{mydef}{Definition}
\newenvironment{definition}{\begin{mydef}}{\end{mydef}}

\newtheorem*{remark}{Remark}

\newtheorem*{example}{Example}
\newtheorem*{claim}{Claim}

% Common shortcuts
\def\mbb#1{\mathbb{#1}}
\def\mfk#1{\mathfrak{#1}}

\def\bN{\mbb{N}}
\def \C{\mbb{C}}
\def \R{\mbb{R}}
\def\bQ{\mbb{Q}}
\def\bZ{\mbb{Z}}
\def \cph{\varphi}
\renewcommand{\th}{\theta}
\def \ve{\varepsilon}
\newcommand{\mg}[1]{\| #1 \|}

% Often helpful macros
\newcommand{\floor}[1]{\left\lfloor#1\right\rfloor}
\newcommand{\ceil}[1]{\left\lceil#1\right\rceil}
\renewcommand{\qed}{\hfill\qedsymbol}
\renewcommand{\ip}[2]{\langle #1, #2 \rangle}
\newcommand{\seq}[2]{\qty(#1_#2)_{#2=1}^{\infty}}

% Sets
\DeclarePairedDelimiterX\set[1]\lbrace\rbrace{\def\given{\;\delimsize\vert\;}#1}

% End of preamble
%==========================================================================================%

% Start of commands specific to this file
%==========================================================================================%

\usepackage{braket}
\newcommand{\Z}{\mbb Z}
\newcommand{\gen}[1]{\left\langle #1 \right\rangle}
\newcommand{\nsg}{\trianglelefteq}
\newcommand{\F}{\mbb F}
\newcommand{\Aut}{\mathrm{Aut}}
\newcommand{\sepdeg}[1]{[#1]_{\mathrm{sep}}}
\newcommand{\Q}{\mbb Q}
\newcommand{\Gal}{\mathrm{Gal}\qty}

%==========================================================================================%
% End of commands specific to this file

\title{Math 505 HW5}
\date{\today}
\author{Anonymous}

\begin{document}
	\maketitle
	\subsection*{Lemma 3.}
	By definition the Galois group is the group of automorphisms of $K$ that fix $k$, so $\Gal(K/k) = \Sigma$ which shows the two above equalities.
	
	\subsection*{Proposition 4.}
	\begin{enumerate}[label=(\arabic*)]
		\item We need only show that $N_{K/k}(\alpha)$ is fixed by every element of the Galois group. Let $\tau \in \Gal(K/k)$, and notice that
		\begin{align*}
			\tau\qty(\prod_{\sigma \in \Gal(K/k)} \sigma(\alpha)) = \prod_{\sigma \in \Gal(K/k)} (\tau \circ \sigma)(\alpha)
		\end{align*}
		Since left multiplication by an element of a group is a permutation, it follows immediately that 
		\begin{align*}
			\Set{\tau \circ \sigma \mid \sigma \in \Gal(K/k)} = \tau\Gal(K/k) = \Gal(K/k)
		\end{align*}
		Which shows that $\tau$ fixes $N_{k/k}(\alpha)$ for every $\tau \in \Gal(K/k)$, completing the proof.
		
		\item Notice that
		\begin{align*}
			\tau\qty(\sum_{\sigma \in \Gal(K/k)} \sigma(\alpha)) = \sum_{\sigma \in \Gal(K/k)} (\tau \circ \sigma)(\alpha)
		\end{align*}
		So for the same reasons as before $\Tr_{K/k}(\alpha) \in k$.
	\end{enumerate}

	\subsection*{Proposition 5.}
	Notice that
	\begin{align*}
		N_{K/k}(\alpha\beta) = \prod_{\sigma \in \Gal(K/k)} \sigma(\alpha\beta) = \prod_{\sigma \in \Gal(K/k)} \sigma(\alpha)\sigma(\beta) &= \prod_{\sigma \in \Gal(K/k)} \sigma(\alpha) \cdot \prod_{\sigma \in \Gal(K/k)} \sigma(\beta)
		\\&=N_{K/k}(\alpha) \cdot N_{K/k}(\beta)
	\end{align*}

	\subsection*{Proposition 5.}
	Once again,
	\begin{align*}
		\Tr_{K/k}(\alpha+\beta) = \sum_{\sigma \in \Gal(K/k)} \sigma(\alpha+\beta) = \sum_{\sigma \in \Gal(K/k)} \sigma(\alpha) + \sigma(\beta) &= \sum_{\sigma \in \Gal(K/k)} \sigma(\alpha) + \sum_{\sigma \in \Gal(K/k)} \sigma(\beta) 
		\\&= \Tr_{K/k}(\alpha) + \Tr_{K/k}(\beta)
	\end{align*}
	
	\subsection*{Example 7.}
	\begin{enumerate}[label=(\arabic*)]
		\item Since the minimal polynomial of $D$ is $x^2-D$ which has degree 2, the Galois group has order 2. We claim that it is generated by the automorphism
		\begin{align*}
			\cph: \sqrt{D} \mapsto -\sqrt{D}
		\end{align*}
		Since $\sqrt{D} \not \in k$, this map fixes $k$, and hence is an automorphism of $k$ sending $\sqrt{D}$ to another root of $x^2-D$. Clearly it has order 2, which shows that the Galois group is just $\gen{\cph}$. Then 
		\begin{align*}
			N_{k(\sqrt{D})/k} = (a+b\sqrt{D})\cph\qty(a+b\sqrt{D}) = (a+b\sqrt{D})(a-b\sqrt{D}) = a^2-Db^2
		\end{align*}
		\item Similarly,
		\begin{align*}
			\Tr_{k(\sqrt{D})/k} = a+b\sqrt{D} + \cph\qty(a+b\sqrt{D}) = a + b\sqrt{D} + a - b\sqrt{D} = 2a
		\end{align*}
	\end{enumerate}
	
	\subsection*{Proposition 8.}
	\begin{enumerate}[label=(\arabic*)]
		\item Let $\alpha_2, \ldots, \alpha_d$ be the other $d-1$ roots of $f$. We claim that for any $\alpha_i$, there are precisely $n/d$ $\sigma \in \Gal(K/k)$ such that $\sigma(\alpha) = \alpha_i$. This was actually shown on homework 2, because each $\sigma \in \Gal(K/k)$ sending $\alpha$ to $\alpha_i$ is just an extension of $\tau: k(\alpha) \to \overline k$ with $\tau(\alpha) = \alpha_i$, and I showed on HW2 that
		\begin{align*}
			\qty|\Set{
				\begin{array}{l}
					\text{extensions of $\tau$ to} \\
					\text{embeddings $E \hookrightarrow L'$}
			\end{array}}| = 
			\qty|\Set{
				\begin{array}{l}
					\text{extensions of $\sigma$ to} \\
					\text{embeddings $E \hookrightarrow L$}
			\end{array}}|
		\end{align*}
		Where $E/F$ is an extension and $\tau: F \to L'$ and also $\sigma: F \to L$ with $L,L'$ (possibly distinct) algebraic closures of $F$. Taking $L=L'=\overline k$, $F = k(\alpha)$, $E = K$, $\sigma = \mathrm{id}_F$, and $\tau$ as above will show that the number of $\sigma \in \Gal(K/k)$ sending $\alpha$ to $\alpha_i$ are in bijection with the number of $\sigma \in \Gal(K/k)$ that send $\alpha$ to $\alpha$, which is precisely the (separable) degree of $K/k(\alpha)$, which, by the following diagram,
		 \[\begin{tikzcd}
		 	K \\
		 	{k(\alpha)} \\
		 	k
		 	\arrow["{n/d}", no head, from=1-1, to=2-1]
		 	\arrow["d", no head, from=2-1, to=3-1]
		 	\arrow["n"', curve={height=18pt}, no head, from=1-1, to=3-1]
		 \end{tikzcd}\]
	 	Is just $n/d$. Thus, in the product
	 	\begin{align*}
	 		N_{K/k}(\alpha) = \prod_{\sigma \in \Gal(K/k)} \sigma(\alpha)
	 	\end{align*}
 		precisely $n/d$ of the terms are $\alpha_i$ for each $i$. So, let $\beta = \prod_{i=1}^d \alpha_i$ (taking $\alpha_1 = \alpha$) and we get that
 		\begin{align*}
 			N_{K/k}(\alpha) = \beta^{n/d}
 		\end{align*}
 		By the formulas for symmetric polynomials, the product of the roots $\beta$ is precisely $(-1)^na_0$. If $n$ is odd, then $n^2/d$ is odd, so $(-1)^n = (-1)^{n^2/d}$. Similarly, if $n$ is even then $(-1)^{n} = (-1)^{n^2/d}$. We have concluded that
 		\begin{align*}
 			N_{K/k}(\alpha) = \beta^{n/d} = (-1)^{n \cdot n/d} a_0^{n/d} = (-1)^n a_0^{n/d}
 		\end{align*}
 	
 		\item Once again, precisely $n/d$ of the terms of
 		\begin{align*}
 			\Tr_{K/k}(\alpha) = \sum_{\sigma \in \Gal(K/k)} \sigma(\alpha)
 		\end{align*}
 		are $\alpha_i$ for each $i$. Thus,
 		\begin{align*}
 			\Tr_{K/k}(\alpha) = \frac nd \sum_{i=1}^d \alpha_i
 		\end{align*}
 		Once again by the explicit formulas for symmetric polynomials, the sum of the roots is just $-a_{d-1}$, whence,
 		\begin{align*}
 			\Tr_{K/k}(\alpha) = -\frac nd a_{d-1}
 		\end{align*}
	\end{enumerate}

	\subsection*{Proposition 9}
	\begin{enumerate}[label=(\arabic*)]
		\item Since $a \in k$, for any $\sigma \in \Gal(K/k)$ $\sigma(a) = a$. Thus,
		\begin{align*}
			N(a\alpha) = \prod_{\sigma \in \Gal(K/k)} \sigma(a\alpha) = \prod_{\sigma \in \Gal(K/k)} a\sigma(\alpha) = a^n\prod_{\sigma \in \Gal(K/k)} \sigma(\alpha) = a^nN_{K/k}(\alpha)
		\end{align*}
	
		\item Similarly,
		\begin{align*}
			\Tr_{K/k}(a\alpha) = \sum_{\sigma \in \Gal(K/k)} \sigma(a\alpha) = \sum_{\sigma \in \Gal(K/k)} a\sigma(\alpha) = a\sum_{\sigma \in \Gal(K/k)} \sigma(\alpha) = a\Tr_{K/k}(\alpha)
		\end{align*}
	\end{enumerate}
	
	\subsection*{Theorem 10.}
	Let $0 \neq \beta \in K$, and write $\alpha = \frac{\beta}{\sigma(\beta)}$. Note that $N_{K/k}(1) = \prod_{\sigma \in \Gal(K/k)} \sigma(1) = 1^n = 1$. This shows that $N_{K/k}(a/b) = N_{K/k}(a)/N_{K/k}(b)$. We showed in Proposition 4 that $\sigma(N_{K/k}(\alpha)) = N_{K/k}(\alpha)$ for each $\sigma$. Similarly, since $\Gal(K/k)\tau = \Gal(K/k)$, we also have that $N_{K/k}(\tau(\alpha)) = N_{K/k}(\alpha)$ for each $\tau \in \Gal(K/k)$. Equivalently, since right multiplication by an element of a group is a permutation of that group, 
	$\Set{\sigma \circ \tau \mid \sigma \in \Gal(K/k)} = \Gal(K/k)$. Thus immediately from the definition,
	\begin{align*}
		N_{K/k}(\tau(\alpha)) = \prod_{\sigma \in \Gal(K/k)} \sigma \circ \tau(\alpha) = N_{K/k}(\tau)
	\end{align*}
	Thus,
	\begin{align*}
		N_{K/k}\qty(\frac{\beta}{\sigma(\beta)}) = \frac{N_{K/k}(\beta)}{N_{K/k}(\beta)} = 1
	\end{align*}
	
	We provide a different proof of the case $n=2$. Let $K = k(\sqrt{D})$ with $\sqrt{D}$ a solution to (by hypothesis, the irreducible) $X^2 - D = 0$. The only nontrivial automorphism in the Galois group is $\sigma: x + y\sqrt{D} \mapsto x - y\sqrt{D}$. Thus, for $\alpha$ of norm 1, we want to find a solution to
	\begin{align*}
		\alpha(x-y\sqrt{D}) &= x+y\sqrt{D} \\
		(\alpha-1)x &= y\sqrt{D}(\alpha+1)
	\end{align*}
	Notice that
	\begin{align*}
		\sigma\qty(\frac{\sqrt{D}(\alpha+1)}{\alpha-1}) = \frac{-\sqrt{D}(\alpha^{-1} + 1)}{\alpha^{-1} - 1} = \frac{-\sqrt{D}(1+\alpha)}{1-\alpha}
	\end{align*}
	Since $\alpha$ has norm 1 we have that $\alpha \sigma(\alpha) = 1$, thus $\sigma(\alpha) = \alpha^{-1}$ (I am a little confused by your question... $\alpha$ is an element of a field so it has an inverse). Thus the coefficient of $y$ is fixed by every element of the Galois group and hence this is an equation in two variables in $k$, thus the kernel of its corresponding linear transformation has dimension at least 1, so there is a (necessarily nonzero) solution.
	
	In the general case we use linear independence of characters. Recall that $\sigma^{(i)}(x) = (\underbrace{\sigma \circ \cdots \circ \sigma}_{\text{$i$ times}})(x)$. Notice that $\sigma \eval_{K^\times}: K^\times \to K^\times$ is a homomorphism of groups, and hence a character of $K^\times$. Notice also that 
	\begin{align*}
		\mathcal{S} = \Set{\sigma^{(0)} = \mathrm{id}, \sigma, \sigma^{(2)}, \ldots, \sigma^{(n-1)}}
	\end{align*}
	is a set of distinct characters since $\sigma$ has order $n$. Thus there is a $\gamma \in K$ so that
	\begin{align*}
		\beta = \alpha \sigma^{(0)}(\gamma) + \alpha \sigma(\alpha) \sigma^{(1)}(\gamma) + \cdots + \underbrace{\qty(\prod_{i=0}^{n-1} \sigma^{(i)}(\alpha))}_{1} \sigma^{(n-1)}(\gamma) \neq 0
	\end{align*}
	since $\mathcal{S}$ is linearly independent by linear independence of characters. We see that,
	\begin{align*}
		\frac{\beta}{\sigma(\beta)} = \frac{\alpha \sigma^{(0)}(\gamma) + \alpha \sigma(\alpha) \sigma^{(1)}(\gamma) + \cdots + \sigma^{(n-1)}(\gamma)}{\sigma(\alpha) \sigma^{(1)}(\gamma) + \sigma(\alpha)\sigma^{(2)}(\alpha) \sigma^{(2)}(\gamma) + \cdots + \prod_{i=1}^{n-1} \sigma^{(i)}(\alpha) \sigma^{(n-1)}(\gamma) + \sigma^{(0)}(\gamma)}
	\end{align*}
	Where the coefficient of the last term is 1 precisely because $N(\alpha) = 1$. After moving the last term on the bottom to the beginning, and multiplying by $\alpha/\alpha$, we can see that this quantity is going to equal $\alpha$, which completes the proof.
\end{document}