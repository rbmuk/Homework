\documentclass[12pt]{article}
\usepackage[margin=1in]{geometry}
\usepackage{setspace}
\onehalfspacing

% Start of preamble
%==========================================================================================%
% Required to support mathematical unicode
\usepackage[warnunknown, fasterrors, mathletters]{ucs}
\usepackage[utf8x]{inputenc}

\usepackage[dvipsnames,table,xcdraw]{xcolor}

% Standard mathematical typesetting packages
\usepackage{amsmath,amssymb,amscd,amsthm,amsxtra, pxfonts}
\usepackage{mathtools,mathrsfs,dsfont,xparse}

% Symbol and utility packages
\usepackage{cancel, textcomp}
\usepackage[mathscr]{euscript}
\usepackage[nointegrals]{wasysym}
\usepackage{apacite}

% Extras
\usepackage{physics}  
\usepackage{tikz-cd} 
\usepackage{microtype}
\usepackage{enumitem}
\usepackage{titling}
\usepackage{graphicx}

% Fancy theorems due to @intuitively on discord
\usepackage{mdframed}
\newmdtheoremenv[
backgroundcolor=NavyBlue!30,
linewidth=2pt,
linecolor=NavyBlue,
topline=false,
bottomline=false,
rightline=false,
innertopmargin=10pt,
innerbottommargin=10pt,
innerrightmargin=10pt,
innerleftmargin=10pt,
skipabove=\baselineskip,
skipbelow=\baselineskip
]{mytheorem}{Theorem}

\newenvironment{theorem}{\begin{mytheorem}}{\end{mytheorem}}

\newtheorem{corollary}{Corollary}
\newtheorem{lemma}{Lemma}

\newtheoremstyle{definitionstyle}
{\topsep}%
{\topsep}%
{}%
{}%
{\bfseries}%
{.}%
{.5em}%
{}%
\theoremstyle{definitionstyle}
\newmdtheoremenv[
backgroundcolor=Violet!30,
linewidth=2pt,
linecolor=Violet,
topline=false,
bottomline=false,
rightline=false,
innertopmargin=10pt,
innerbottommargin=10pt,
innerrightmargin=10pt,
innerleftmargin=10pt,
skipabove=\baselineskip,
skipbelow=\baselineskip,
]{mydef}{Definition}
\newenvironment{definition}{\begin{mydef}}{\end{mydef}}

\newtheorem*{remark}{Remark}

\newtheorem*{example}{Example}

% Common shortcuts
\def\mbb#1{\mathbb{#1}}
\def\mfk#1{\mathfrak{#1}}

\def\bN{\mbb{N}}
\def \C{\mbb{C}}
\def \R{\mbb{R}}
\def\bQ{\mbb{Q}}
\def\bZ{\mbb{Z}}
\def \cph{\varphi}
\renewcommand{\th}{\theta}
\def \ve{\varepsilon}
\newcommand{\mg}[1]{\| #1 \|}

% Often helpful macros
\newcommand{\floor}[1]{\left\lfloor#1\right\rfloor}
\newcommand{\ceil}[1]{\left\lceil#1\right\rceil}
\renewcommand{\qed}{\hfill\qedsymbol}
\renewcommand{\P}{\mathbb P\qty}
\newcommand{\E}{\mathbb{E}\qty}
\newcommand{\Cov}{\mathrm{Cov}\qty}
\newcommand{\Var}{\mathrm{Var}\qty}

% Sets
\usepackage{braket}

\graphicspath{{/}}
\usepackage{float}

\newcommand{\SET}[1]{\Set{\mskip-\medmuskip #1 \mskip-\medmuskip}}

% End of preamble
%==========================================================================================%

% Start of commands specific to this file
%==========================================================================================%

\usepackage{listings}
\usepackage{lstautogobble}
\lstset{
    basicstyle=\ttfamily,
    mathescape=true,
    autogobble=true
}
\newcommand{\doubleplus}{\mathbin{+\mkern-10mu+}}

%==========================================================================================%
% End of commands specific to this file

\title{CSE Template}
\date{\today}
\author{Rohan Mukherjee}

\begin{document}
    \maketitle
    \begin{enumerate}[leftmargin=0pt, label=(\alph*), labelindent=0pt, labelwidth=!, wide]
        \item We have the following.
        
        \begin{lstlisting}
            let R: string[][] = [];
            let i: number = 0;
            {{P_1: R = [], i = 0}}
            {{Inv: R = replace(A[0 .. i-1], M)}}
            while (i !== A.length) {
                if (contains_key(A[i], M)) {
                    const val = get_value(A[i], M);
                    {{Inv, contains-key(A[i], M), val = get-value(A[i], M)}}
                    R.push(val);
                    {{P_2: R[0 .. i-1] = replace(A[0 .. i-1], M), R[i] = val, 
                    contains-key(A[i], M), val = get-value(A[i], M)}}
                } else {
                    {{Inv, !contains-key(A[i], M)}}
                    R.push([A[i]]);
                    {{P_3: R[0 .. i-1] = replace(A[0 .. i-1], M), 
                    not contains-key(A[i], M), R[i] = [A[i]]}}
                }
                {{P_2 or P_3}}
                {{Q: R = replace(A[0 .. i], M)}}
                i = i+1;
                {{Inv}}
            }
            {{P_4: R = replace(A[0 .. i-1], M) and i = n}}
            {{R = replace(A, M)}}
        \end{lstlisting}

        \item We prove that $P_1 \implies $ the loop invariant. Since $i = 0$, we know that $A[0 .. -1] = []$, the empty list. By definition of replace, we have that $replace([], M) = []$. Thus, $R = [] = \rm replace(A[0 .. i-1], M)$ and the loop invariant holds.
        
        Next we prove that $P_4 \implies $ Post. Since $i = n$, we know that $A[0 .. i-1] = A[0 .. n-1] = A$, since $n$ is the length of $A$. Thus $R = \rm replace(A[0 .. i-1], M) = replace(A, M)$ and the postcondition holds.
        
        Finally we prove that $P_2$ or $P_3$ implies $Q$. We do this by cases. In the first case, we assume $P_2$, which is that $R[0 .. i-1] = \rm replace(A[0 .. i-1, M])$, $R[i] = \rm val$, $\rm contains-key(A[i], M)$ is true, and that $\rm val = get-value(A[i], M)$. We have that:
        \begin{align*}
            \begin{aligned}
                \rm replace(A[0 .. i], M) &= \rm replace(A[0 .. i-1] \doubleplus A[i], M) && A[0 .. i] = A[0 .. i-1] \doubleplus A[i] \\
                &= \rm replace(A[0 .. i-1], M) \doubleplus \rm [get-value(A[i], M)] && \text{def replace, contains-key(A[i], M)} \\
                &= R[0 .. i-1] \doubleplus \rm [val] && \text{$\rm val = get-value(A[i], M)$}\\
                &= R[0 .. i-1] \doubleplus [R[i]] && \text{$R[i] = val$} \\
                &= R && \text{$R = R[0 .. i-1] \doubleplus R[i]$}
            \end{aligned}
        \end{align*}
        Thus $P_2 \implies Q$. 

        In the second case, we assume $P_3$, which is that $R[0 .. i-1] = \rm replace(A[0 .. i-1, M])$, $! \rm contains-key(A[i], M)$ is true, and that $R[i] = [A[i]]$. We have that:
        \begin{align*}
            \begin{aligned}
                \rm replace(A[0 .. i], M) &= \rm replace(A[0 .. i-1] \doubleplus A[i], M) && A[0 .. i] = A[0 .. i-1] \doubleplus A[i] \\
                &= \rm replace(A[0 .. i-1], M) \doubleplus \rm [A[i]] && \text{def replace, !$\rm contains-key(A[i], M)$} \\
                &= R[0 .. i-1] \doubleplus \rm [A[i]] && \text{R[0..i-1] = replace(A[0.. i-1])}\\
                &= R[0 .. i-1] \doubleplus [R[i]] && \text{$R[i] = [A[i]]$} \\
                &= R && \text{$R = R[0 .. i-1] \doubleplus [R[i]]$}
            \end{aligned}
        \end{align*}
        Thus $P_3 \implies Q$. Combining these together shows that $P_2$ or $P_3 \implies Q$.
    \end{enumerate}
\end{document}