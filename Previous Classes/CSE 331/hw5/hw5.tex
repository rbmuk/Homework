\documentclass[12pt]{article}
\usepackage[margin=0.5in]{geometry}
\usepackage{setspace}
\onehalfspacing

% Start of preamble
%==========================================================================================%
% Required to support mathematical unicode
\usepackage[warnunknown, fasterrors, mathletters]{ucs}
\usepackage[utf8x]{inputenc}

\usepackage[dvipsnames,table,xcdraw]{xcolor}

% Standard mathematical typesetting packages
\usepackage{amsmath,amssymb,amscd,amsthm,amsxtra, pxfonts}
\usepackage{mathtools,mathrsfs,dsfont,xparse}

% Symbol and utility packages
\usepackage{cancel, textcomp}
\usepackage[mathscr]{euscript}
\usepackage[nointegrals]{wasysym}
\usepackage{apacite}

% Extras
\usepackage{physics}  
\usepackage{tikz-cd} 
\usepackage{microtype}
\usepackage{enumitem}
\usepackage{titling}
\usepackage{graphicx}

% Fancy theorems due to @intuitively on discord
\usepackage{mdframed}
\newmdtheoremenv[
backgroundcolor=NavyBlue!30,
linewidth=2pt,
linecolor=NavyBlue,
topline=false,
bottomline=false,
rightline=false,
innertopmargin=10pt,
innerbottommargin=10pt,
innerrightmargin=10pt,
innerleftmargin=10pt,
skipabove=\baselineskip,
skipbelow=\baselineskip
]{mytheorem}{Theorem}

\newenvironment{theorem}{\begin{mytheorem}}{\end{mytheorem}}

\newtheorem{corollary}{Corollary}
\newtheorem{lemma}{Lemma}

\newtheoremstyle{definitionstyle}
{\topsep}%
{\topsep}%
{}%
{}%
{\bfseries}%
{.}%
{.5em}%
{}%
\theoremstyle{definitionstyle}
\newmdtheoremenv[
backgroundcolor=Violet!30,
linewidth=2pt,
linecolor=Violet,
topline=false,
bottomline=false,
rightline=false,
innertopmargin=10pt,
innerbottommargin=10pt,
innerrightmargin=10pt,
innerleftmargin=10pt,
skipabove=\baselineskip,
skipbelow=\baselineskip,
]{mydef}{Definition}
\newenvironment{definition}{\begin{mydef}}{\end{mydef}}

\newtheorem*{remark}{Remark}

\newtheorem*{example}{Example}

% Common shortcuts
\def\mbb#1{\mathbb{#1}}
\def\mfk#1{\mathfrak{#1}}

\def\bN{\mbb{N}}
\def \C{\mbb{C}}
\def \R{\mbb{R}}
\def\bQ{\mbb{Q}}
\def\bZ{\mbb{Z}}
\def \cph{\varphi}
\renewcommand{\th}{\theta}
\def \ve{\varepsilon}
\newcommand{\mg}[1]{\| #1 \|}

% Often helpful macros
\newcommand{\floor}[1]{\left\lfloor#1\right\rfloor}
\newcommand{\ceil}[1]{\left\lceil#1\right\rceil}
\renewcommand{\qed}{\hfill\qedsymbol}
\renewcommand{\P}{\mathbb P\qty}
\newcommand{\E}{\mathbb{E}\qty}
\newcommand{\Cov}{\mathrm{Cov}\qty}
\newcommand{\Var}{\mathrm{Var}\qty}

% Sets
\usepackage{braket}

\graphicspath{{/}}
\usepackage{float}

\newcommand{\SET}[1]{\Set{\mskip-\medmuskip #1 \mskip-\medmuskip}}

% End of preamble
%==========================================================================================%

% Start of commands specific to this file
%==========================================================================================%

\usepackage{listings}
\usepackage{lstautogobble}
\lstset{
    basicstyle=\ttfamily,
    mathescape=true,
    autogobble=true
}

%==========================================================================================%
% End of commands specific to this file

\title{CSE Template}
\date{\today}
\author{Rohan Mukherjee}

\begin{document}
    \maketitle
    \begin{enumerate}[labelindent=0pt, labelwidth=!, wide]
        \item \begin{enumerate}[leftmargin=0pt, label=(\alph*), labelindent=0pt, labelwidth=!, wide]
            \item 
            We use forward reasoning in the following way:
            \begin{lstlisting}
                {{$c > 0$ and $s > 0$}}
                    e = 5n * c + 1;
                {{$c > 0$ and $s > 0$ and $e=5c+1$}}
                s = s + 3n; 
                {{$c > 0$ and $s - 3> 0$ and $e=5c+1$}}
                    c = c * s; 
                {{$c/s > 0$ and $s>3$ and $e=5c/s+1$}}
                {{e $>$ 1}}
            \end{lstlisting}
            
            We prove that last implication. We have that $c/s > 0$, so we know since $c/s$ is an integer that $c/s \geq 1$. Thus $e = 5c/s+1 \geq 5 \cdot 1 + 1 = 6 > 1$. Thus $e > 1$.
    
            \item 
            \begin{lstlisting}
                {{$s \geq 0$ and $e = 0$}}
                {{$3s + 2 \geq e+2$}}
                s = 3n * s;
                {{$s+2 \geq e+2$}}
                e = e+2n;
                {{$s+2 \geq e$}}
                c = s+2n;
                {{$c \geq e$}}
            \end{lstlisting}
            
            The implication can be proved as follows. Since $s \geq 0$, we know that $3s + 2 \geq 3 \cdot 0 + 2 = 2 \geq 0 + 2 = e = 2$, since $e = 0$. Thus we have that $3s+2 \geq e+2$ from the precondition and we are done.
            \item \begin{lstlisting}
                {{$c \geq 1$ and $s = c^2$}}
                if (s < 20n) {
                    {{$c \geq 1$ and $s = c^2$ and $s<20$}}
                    s = s + 5n;
                    {{$c \geq 1$ and $s - 5 = c^2$ and $s-5<20$}}
                } else if (s < 30n) {
                    {{$c \geq 1$ and $s = c^2$ and $20 \leq s < 30$}}
                    s = (s/c)+1n;
                    {{$c \geq 1$ and $c(s-1)=c^2$ and $20 \leq c(s-1)<30$}}
                } else {
                    {{$c \geq 1$ and $s=c^2$ and $s \geq 30$}}
                    s = s/c;
                    {{$c \geq 1$ and $cs = c^2$ and $cs \geq 30$}}
                }
                {{$c \geq 1$ and $s - 5 = c^2$ and $s-5<20$ 
                or $c \geq 1$ and $c(s-1)=c^2$ and $c(s-1)<30$ 
                or $c \geq 1$ and $cs = c^2$ and $cs \geq 30$}}
                {{$s > 5$}}
            \end{lstlisting}
            We prove the final implication by cases. In the first case, we know that $c \geq 1$, $s-5=c^2$ and $s-5 < 20$. Using the second condition, we see that $s -5 = c^2$ and so $s = 5 + c^2$. Since $c \geq 1$, we know that $c^2 \geq 1$. Using this we see that $s = 5+c^2 \geq 5+1 = 6 > 5$. 
    
            In the second case, we have that $c \geq 1$ and $c(s-1) = c^2$ and $20 \leq c(s-1)<30$. The second condition tells us that $s-1 = c$ after dividing by $c$ ($c$ is nonzero since by the first condition $c \geq 1$). We know from the last condition that $c(s-1) \geq 20$. Using that $s-1=c$ now tells us that $c^2 \geq 20$. Taking square roots on both sides, noting that $c \geq 1$ and hence $c \geq 0$, we have that $c \geq \sqrt{20} > 4$. Since $c$ is an integer, we must have that $c \geq 5$. Since $s = c+1$, we have that $s \geq 5+1 = 6 > 5$, and we are done.
    
            In the last case, we have that $c \geq 1$ and $cs = c^2$ and $cs \geq 30$. Once again using the second condition tells us that $s = c$ because $c \geq 1$ so we can divide on both sides by the nonzero constant $c$. Plugging this into $cs \geq 30$ tells us that $s^2 \geq 30$. Since $s=c$ we have that $s \geq 1$ as well. Taking square roots on both sides of $s^2 \geq 30$ tells us that $s \geq \sqrt{30} > 5$. Thus $s > 5$ as well.
        \end{enumerate}

        \newpage
        \item \begin{enumerate}[label=(\alph*)]
            \item Initially, we know that $x = x_0$, so we have that $4y = 4 \cdot 0 = 0 = x_0 - x$. Similarly, since $x_0 \geq 0$ we know that $x = x_0 \geq 0 \geq -4$, so the invariant holds at the top of the loop.
            \item We use forward reasoning. We have:
            \begin{lstlisting}
                {{Inv: $4y =x_0 - x$ and $x \geq -4$}}
                while (x >= 0) {
                    {{$4y = x_0-x$ and $x \geq -4$ and $x \geq 0$}} $\iff$ {{$4y = x_0-x$ and $x \geq 0$}} 
                    y = y + 1n;
                    {{$4(y-1) = x_0 - x$ and $x \geq 0$}}
                    x = x - 4n;
                    {{$4(y-1) = x_0 - (x+4)$ and $x+4 \geq 0$}}
                }
            \end{lstlisting}
            The fact that the last condition proves the invariant can be shown as follows. $4(y-1) = x_0 - (x+4)$ is the same as $4y - 4 = x_0 - x - 4$, and adding 4 to both sides shows that $4y = x_0 - x$ as required. The second condition is $x + 4 \geq 0$ which is the same as $x \geq -4$.
            \item Once we have exited the loop we will have that $x < 0$. We can use the invariant to see then that $-4 \leq x < 0$. Using the fact that $4y = x_0 - x$, we can use the fact that $x \geq -4$ to then multiply both sides by $-1$ to get that $-x \leq 4$. Thus $4y = x_0 - x \leq x_0 + 4$. Since $x < 0$ here, and since $4y = x_0 - x$, we have again that $-x > 0$ after multiplying both sides by $-1$, so we can apply this to see that $4y = x_0 - x > x_0$. Thus $x_0 < 4y$ and we are done.
        \end{enumerate}
        \item
            (a) Initially, we have that $a = 0$, $b = 0$, and $L = L_0$. So,
            \begin{align*}
                \begin{aligned}
                    &\rm sum-gt(L_0, x) &&= \rm sum-gt(L, x) && \text{$L = L_0$} \\
                    &&&= \rm a + sum-gt(L, x) && \text{$a = 0$}
                \end{aligned}
            \end{align*}
            Similarly, we have that:
            \begin{align*}
                \begin{aligned}
                    &\rm sum-lt(L_0, x) &&= \rm sum-lt(L, x) && \text{$L = L_0$} \\
                    &&&= \rm b + sum-lt(L, x) && \text{$b = 0$}
                \end{aligned}
            \end{align*}
            So the invariant holds at the top of the loop. 

            (b) \begin{lstlisting}
                {{Inv: sum-gt($L_0, x$) = a + sum-gt(L, x) and
                sum-lt($L_0, x$) = b + sum-lt(L, x)}}
                while (L !== nil) {
                    {{sum-gt($L_0, x$) = a + sum-gt(L, x) and
                    sum-lt($L_0, x$) = b + sum-lt(L, x) and
                    L = cons(L.hd, L.tl)}}
                    if (L.hd > x) {
                        {{sum-gt($L_0, x$) = a + sum-gt(L, x) and
                        sum-lt($L_0, x$) = b + sum-lt(L, x) and
                        L = cons(L.hd, L.tl) and L.hd $> x$}}
                        a = a + (L.hd - x);
                        {{sum-gt($L_0, x$) = a - (L.hd - x) + sum-gt(L, x) and
                        sum-lt($L_0, x$) = b + sum-lt(L, x) and
                        L = cons(L.hd, L.tl) and L.hd $> x$}}
                    } else if (L.hd < x) {
                        {{sum-gt($L_0, x$) = a + sum-gt(L, x) and
                        sum-lt($L_0, x$) = b + sum-lt(L, x) and
                        L = cons(L.hd, L.tl) and L.hd $< x$}}
                        b = b + (x-L.hd);
                        {{sum-gt($L_0, x$) = a + sum-gt(L, x) and
                        sum-lt($L_0, x$) = b - (x-L.hd) + sum-lt(L, x) and
                        L = cons(L.hd, L.tl) and L.hd $< x$}}
                    } else {
                        // Do nothing
                        {{sum-gt($L_0, x$) = a + sum-gt(L, x) and
                        sum-lt($L_0, x$) = b + sum-lt(L, x) and
                        L = cons(L.hd, L.tl) and L.hd = x}}
                    }
                    {{sum-gt($L_0, x$) = a - (L.hd - x) + sum-gt(L, x) and
                        sum-lt($L_0, x$) = b + sum-lt(L, x) and
                        L = cons(L.hd, L.tl) and L.hd $> x$ or
                        sum-gt($L_0, x$) = a + sum-gt(L, x) and
                        sum-lt($L_0, x$) = b - (x-L.hd) + sum-lt(L, x) and
                        L = cons(L.hd, L.tl) and L.hd $< x$ or
                        sum-gt($L_0, x$) = a + sum-gt(L, x) and
                        sum-lt($L_0, x$) = b + sum-lt(L, x) and
                        L = cons(L.hd, L.tl) and L.hd = x}}
                    $\implies$
                    {{sum-gt($L_0, x$) = a + sum-gt(L.tl, x) and
                    sum-lt($L_0, x$) = b + sum-lt(L.tl, x)}}
                    L = L.tl;
                    {{sum-gt($L_0, x$) = a + sum-gt(L, x) and
                    sum-lt($L_0, x$) = b + sum-lt(L, x)}}
                }
            \end{lstlisting}
            We used backwards reasoning for the last line L = L.tl at the end there. I put the invariant under that line to make it clear how I am applying the backwards reasoning, and I added an $\implies$ to show the implication that we have to prove. We prove that now by cases. 
        
            In the first case, we have that sum-gt($L_0, x$) = a - (L.hd -x) + sum-gt(L, x) and sum-lt($L_0, x$) = b + sum-lt(L, x) and L = cons(L.hd, L.tl) and L.hd $> x$. We see that:
            \begin{align*}
                \begin{aligned}
                    &\rm sum-gt(L, x) &&= \rm sum-gt(cons(L.hd, L.tl), x) &&\text{\rm L = cons(L.hd, L.tl)} \\
                    &&&= \rm (L.hd - x) + sum-gt(L.tl, x) &&\text{L.hd $>x$ and def of sum-gt}
                \end{aligned}
            \end{align*}
            Plugging this calculation in, we see that sum-gt($L_0, x$) = a - (L.hd - x) + sum-gt(L, x) = a - (L.hd - x) + (L.hd - x) + sum-gt(L.tl, x) = a + sum-gt(L.tl, x) as desired.  Similarly,
            \begin{align*}
                \begin{aligned}
                    &\rm sum-lt(L, x) &&= \rm sum-lt(cons(L.hd, L.tl), x) &&\text{\rm L = cons(L.hd, L.tl)} \\
                    &&&= \rm sum-lt(L.tl, x) &&\text{L.hd $>x$ and def of sum-lt}
                \end{aligned}
            \end{align*}
            So we can plug this into the above to see that sum-lt($L_0, x$) = b + sum-lt(L, x) = b + sum-lt(L.tl, x) as well. This completes the proof of this case.
        
            In the second case, we know that sum-gt($L_0, x$) = a + sum-gt(L, x) and sum-lt($L_0, x$) = b - (x-L.hd) + sum-lt(L, x) and L = cons(L.hd, L.tl) and L.hd $< x$. We see that:
            \begin{align*}
                \begin{aligned}
                    &\rm sum-gt(L, x) &&= \rm sum-gt(cons(L.hd, L.tl), x) &&\text{\rm L = cons(L.hd, L.tl)} \\
                    &&&= \rm sum-gt(L.tl, x) &&\text{L.hd $<x$ and def of sum-gt}
                \end{aligned}
            \end{align*}
            Plugging this calculation in, we see that sum-gt($L_0, x$) = a + sum-gt(L, x) = a + sum-gt(L.tl, x) as desired.  Similarly,
            \begin{align*}
                \begin{aligned}
                    &\rm sum-lt(L, x) &&= \rm sum-lt(cons(L.hd, L.tl), x) &&\text{\rm L = cons(L.hd, L.tl)} \\
                    &&&= \rm (x - L.hd) + sum-lt(L.tl, x) &&\text{L.hd $<x$ and def of sum-lt}
                \end{aligned}
            \end{align*}
            So we can plug this into the above to see that sum-lt($L_0, x$) = b - (x-L.hd) + sum-lt(L, x) = b + sum-lt(L.tl, x) as well. This completes the proof of this case.
        
            In the last case, we know that sum-gt($L_0, x$) = a + sum-gt(L, x) and sum-lt($L_0, x$) = b + sum-lt(L, x) and L = cons(L.hd, L.tl) and L.hd = x. We see that:
            \begin{align*}
                \begin{aligned}
                    &\rm sum-gt(L, x) &&= \rm sum-gt(cons(L.hd, L.tl), x) &&\text{\rm L = cons(L.hd, L.tl)} \\
                    &&&= \rm sum-gt(L.tl, x) &&\text{L.hd $=x$ and def of sum-gt}
                \end{aligned}
            \end{align*}
            Plugging this calculation in, we see that sum-gt($L_0, x$) = a + sum-gt(L, x) = a + sum-gt(L.tl, x) as desired.  Similarly,
            \begin{align*}
                \begin{aligned}
                    &\rm sum-lt(L, x) &&= \rm sum-lt(cons(L.hd, L.tl), x) &&\text{\rm L = cons(L.hd, L.tl)} \\
                    &&&= \rm sum-lt(L.tl, x) &&\text{L.hd $=x$ and def of sum-lt}
                \end{aligned}
            \end{align*}
            So we can plug this into the above to see that sum-lt($L_0, x$) = b + sum-lt(L, x) = b + sum-lt(L.tl, x) as well. This completes the proof of this case and hence the loop invariant.
            \newline 
            \newline
            (c) When we exit the loop, we know that L = nil, that sum-gt($L_0, x$) = a + sum-gt(L, x) and sum-lt($L_0, x$) = b + sum-lt(L,x). Recalling that sum-gt(nil, x) = 0 and sum-lt(nil, x) = 0, we see that sum-gt($L_0, x$) = a + 0 = a and sum-lt($L_0, x$) = b + 0 = b. Thus we have that sum-gt($L_0, x$) = a and sum-lt($L_0, x$) = b as required which completes the proof.

            \newpage
            \item[5.] \begin{enumerate}[leftmargin=0pt, label=(\alph*), labelindent=0pt, labelwidth=!, wide]
                \item Let $P(L)$ for a list $L$ be defined as $\rm contains(a, concat(L, S)) = \rm contains(a, L) \lor contains(a, S)$. We prove this claim by structural induction on $L$. First, if L = nil, then we have that:
                \begin{align*}
                    \begin{aligned}
                        \rm contains(a, concat(L, S)) &= \rm contains(a, concat(nil, S)) && \text{L=nil} \\
                        &= \rm contains(a, S) && \text{def of concat} \\
                    \end{aligned}
                \end{align*}
                Since by definition contains(a, L) = contains(a, nil) = False, we know that contains(a, L) or contains(a, S) = contains(a, S) which shows the base case holds. Now suppose it holds for some list L. Then we have that:
                \begin{align*}
                    \begin{aligned}
                        \rm contains(a, concat(cons(x, L), S)) &= \rm contains(a, cons(x, concat(L, S))) && \text{def of concat} \\
                        &= (a=x) \rm \lor \rm contains(a, concat(L, S)) && \text{def of contains} \\
                        &= ((a=x) \rm \lor \rm contains(a, L)) \rm \lor \rm contains(a, S) && \text{I.H.} \\
                        &= \rm contains(a, cons(x, L)) \rm \lor \rm contains(a, S) && \text{def of contains} 
                    \end{aligned}
                \end{align*}
                This completes the proof by structural induction of this first claim.

                \item Let the claim $P(U)$ for a BST $U$ be defined as contains(a, toList(U)) = (search(a, U) $\neq$ undefined). We prove this claim by structural induction on $U$. First, if $U$ is empty, then we have that:
                \begin{align*}
                    \begin{aligned}
                        \rm contains(a, toList(U)) &= \rm contains(a, toList(empty)) && \text{U=empty} \\
                        &= \rm contains(a, nil) && \text{def of toList} \\
                        &= \rm false && \text{def of contains}
                    \end{aligned}
                \end{align*}
                Similarly,
                \begin{align*}
                    \begin{aligned}
                        \rm search(a, U) &= \rm search(a, empty) && \text{U=empty} \\
                        &= \rm undefined && \text{def of search}
                    \end{aligned}
                \end{align*}
                So $\rm search(a, U) \neq undefined$ is false as well. Thus the base case holds. We now prove that 
                \begin{lstlisting}
                    contains(a, toList(node(b, S, T))) = contains(a, toLisT(S)) or (a == b) 
                                                            or contains(a, toList(T))
                \end{lstlisting}
                Recall that toList(node(b, S, T)) = concat(toList(S), cons(b, toList(T))). Then by part (a), we have that:
            
                \begin{align*}
                    \begin{aligned}
                        \rm contains(a, toList(node(b, S, T))) &= \rm contains(a, concat(toList(S), cons(b, toList(T)))) && \text{def toList} \\
                        &= \rm contains(a, toList(S)) \rm \lor \rm contains(a, cons(b, toList(T))) && \text{part (a)} \\
                        &= \rm contains(a, toList(S)) \rm \lor (a=b) \rm \lor \rm contains(a, toList(T)) && \text{def contains} \\
                    \end{aligned}
                \end{align*}
                Now suppose that $P(S), P(T)$ hold for BSTs $S, T$. 

                First suppose that $a < b$. By the BST invariant, nothing $> b$ is in $S$ and nothing $\leq b$ is in $T$. In particular, $T$ does not have a node with value $a$, since all of its nodes have value $> b$. Thus we see that $\rm contains(a, toList(T)) = $false. Similarly, since $a < b$ we have that $a \neq b$. So the above in this case equals $\rm contains(a, toList(S))$, which is by the IH $\rm search(a, S) \neq undefined$. By the definition of search, we have that $\rm search(a, node(b, S, T)) = \rm search(a, S)$ since $a < b$. So we put these expressions together to get that the above equals $\rm search(a, node(b, S, T)) \neq undefined$ as desired. 

                In the second case, where $a=b$, the above expression would evaluate to true. Similarly, $\rm search(a, node(b, S, T)) = \rm node(a, S, T)$ by definition. This is by definition undefined, so $\newline \rm (search(a, node(b, S, T)) \neq undefined)$ is true as well. This completes the proof in this case.

                In the last case, suppose that $a > b$. Once again, by the BST invariant, nothing $> a$ appears as a node of $S$. So $\rm contains(a, toList(S)) = $false. Similarly $a \neq b$, so in this case this simplifies to $\rm contains(a, toList(T))$. By the IH, this is $\rm search(a, T) \neq undefined$. By the definition of search, $\rm search(a, search(b, S, T)) = \rm search(a, T)$ since $a > b$. So this expression also equals $\rm search(a, node(b, S, T)) \neq undefined$ as required. 
            \end{enumerate}
    \end{enumerate}
\end{document}