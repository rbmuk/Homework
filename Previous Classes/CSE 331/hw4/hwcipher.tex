\documentclass[12pt]{article}
\usepackage[margin=1in]{geometry}
\usepackage{setspace}
\onehalfspacing

% Start of preamble
%==========================================================================================%
% Required to support mathematical unicode
\usepackage[warnunknown, fasterrors, mathletters]{ucs}
\usepackage[utf8x]{inputenc}

\usepackage[dvipsnames,table,xcdraw]{xcolor}

% Standard mathematical typesetting packages
\usepackage{amsmath,amssymb,amscd,amsthm,amsxtra, pxfonts}
\usepackage{mathtools,mathrsfs,dsfont,xparse}

% Symbol and utility packages
\usepackage{cancel, textcomp}
\usepackage[mathscr]{euscript}
\usepackage[nointegrals]{wasysym}
\usepackage{apacite}

% Extras
\usepackage{physics}  
\usepackage{tikz-cd} 
\usepackage{microtype}
\usepackage{enumitem}
\usepackage{titling}
\usepackage{graphicx}

% Fancy theorems due to @intuitively on discord
\usepackage{mdframed}
\newmdtheoremenv[
backgroundcolor=NavyBlue!30,
linewidth=2pt,
linecolor=NavyBlue,
topline=false,
bottomline=false,
rightline=false,
innertopmargin=10pt,
innerbottommargin=10pt,
innerrightmargin=10pt,
innerleftmargin=10pt,
skipabove=\baselineskip,
skipbelow=\baselineskip
]{mytheorem}{Theorem}

\newenvironment{theorem}{\begin{mytheorem}}{\end{mytheorem}}

\newtheorem{corollary}{Corollary}
\newtheorem{lemma}{Lemma}

\newtheoremstyle{definitionstyle}
{\topsep}%
{\topsep}%
{}%
{}%
{\bfseries}%
{.}%
{.5em}%
{}%
\theoremstyle{definitionstyle}
\newmdtheoremenv[
backgroundcolor=Violet!30,
linewidth=2pt,
linecolor=Violet,
topline=false,
bottomline=false,
rightline=false,
innertopmargin=10pt,
innerbottommargin=10pt,
innerrightmargin=10pt,
innerleftmargin=10pt,
skipabove=\baselineskip,
skipbelow=\baselineskip,
]{mydef}{Definition}
\newenvironment{definition}{\begin{mydef}}{\end{mydef}}

\newtheorem*{remark}{Remark}

\newtheorem*{example}{Example}

% Common shortcuts
\def\mbb#1{\mathbb{#1}}
\def\mfk#1{\mathfrak{#1}}

\def\bN{\mbb{N}}
\def \C{\mbb{C}}
\def \R{\mbb{R}}
\def\bQ{\mbb{Q}}
\def\bZ{\mbb{Z}}
\def \cph{\varphi}
\renewcommand{\th}{\theta}
\def \ve{\varepsilon}
\newcommand{\mg}[1]{\| #1 \|}

% Often helpful macros
\newcommand{\floor}[1]{\left\lfloor#1\right\rfloor}
\newcommand{\ceil}[1]{\left\lceil#1\right\rceil}
\renewcommand{\qed}{\hfill\qedsymbol}
\renewcommand{\P}{\mathbb P\qty}
\newcommand{\E}{\mathbb{E}\qty}
\newcommand{\Cov}{\mathrm{Cov}\qty}
\newcommand{\Var}{\mathrm{Var}\qty}

% Sets
\usepackage{braket}

\graphicspath{{/}}
\usepackage{float}

\newcommand{\SET}[1]{\Set{\mskip-\medmuskip #1 \mskip-\medmuskip}}

% End of preamble
%==========================================================================================%

% Start of commands specific to this file
%==========================================================================================%

\DeclareMathOperator{\wormlatinencode}{worm-latin-encode}

%==========================================================================================%
% End of commands specific to this file

\title{CSE HW4}
\date{\today}
\author{Rohan Mukherjee}

\begin{document}
    \maketitle
    \begin{enumerate}[leftmargin=\labelsep]
        \item \begin{enumerate}[label=(\alph*)]
            \item We define the following functions:
            \[\begin{aligned}
                &\rm cipher-encode(nil) &&= \rm nil \\
                &\rm cipher-encode(cons(x, L)) &&= \rm cons(nc(x), cipher-encode(L)) && \text{for any L: List}
            \end{aligned}\]
            and
            \[\begin{aligned}
                &\rm cipher-decode(nil) &&= \rm nil \\
                &\rm cipher-decode(cons(x, L)) &&= \rm cons(pc(x), cipher-decode(L)) && \text{for any L: List}
            \end{aligned}\]
        \end{enumerate}

        \newpage
        \item \begin{enumerate}[label=(\alph*)]
            \item \[\begin{aligned}
                &\rm crazy\_caps\_encode(nil) &&= \rm nil \\
                &\rm crazy\_caps\_encode(cons(x, nil)) &&= \rm cons(x, nil) \\
                &\rm crazy\_caps\_encode(cons(x, cons(y, nil))) &&= \rm cons(x, cons(y, nil)) \\
                &\rm crazy\_caps\_encode(cons(x, cons(y, cons(z, L)))) &&= \rm cons(x, cons(y, cons(uc(z), \\ &&& \qquad \mathrm{crazy\_caps\_encode}(L)))) && \text{for any L: List}
            \end{aligned}\]
            \[\begin{aligned}
                &\rm crazy\_caps\_decode(nil) &&= \rm nil \\
                &\rm crazy\_caps\_decode(cons(x, nil)) &&= \rm cons(x, nil) \\
                &\rm crazy\_caps\_decode(cons(x, cons(y, nil))) &&= \rm cons(x, cons(y, nil)) \\
                &\rm crazy\_caps\_decode(cons(x, cons(y, cons(z, L)))) &&= \rm cons(x, cons(y, cons(lc(z),\\&&& \qquad \rm crazy\_caps\_decode(L)))) && \text{for any L: List}
            \end{aligned}\]
        \end{enumerate}

        \newpage
        \item \begin{enumerate}[label=(\alph*)]
            \item Let $P(S)$ for a list $S$ be the claim that $\rm{keep}(\rm{echo}(S)) = S$. We will prove this by structural induction. First notice that,
            \[\begin{aligned}
                \rm keep(echo(nil)) &= \rm keep(nil) && \text{definition of echo} \\
                &= \rm nil && \text{definition of keep}
            \end{aligned}\]
            Now suppose that $P(L)$ holds for some list $L$. Then,
            \[\begin{aligned}
                \rm keep(echo(cons(a, L))) &= \rm keep(cons(a, cons(a, echo(L)))) && \text{definition of echo} \\
                &= \rm cons(a, drop(cons(a, echo(L)))) && \text{definition of keep} \\
                &= \rm cons(a, keep(echo(L))) && \text{definition of drop} \\
                &= \rm cons(a, L) && \text{inductive hypothesis} \\
            \end{aligned}\]
            Thus, by structural induction, $P(S)$ holds for all lists $S$, which completes the proof.

            \item Notice that:
            \[\begin{aligned}
                \rm keep(cons(1, cons(2, echo(L)))) &= \rm cons(1, drop(cons(2, echo(L)))) && \text{definition of keep} \\
                &= \rm cons(1, keep(echo(L))) && \text{definition of drop} \\
                &= \rm cons(1, L) && \text{by part (a)}
            \end{aligned}\]
        \end{enumerate}

        \newpage
        \item \begin{enumerate}[label=(\alph*)]
            \item We define:
            \[\begin{aligned}
                &\rm prefix(0, L) &&= \rm nil \\
                &\rm prefix(n+1, nil) &&= \rm undefined && \text{for any $n \in \mbb N$}\\
                &\rm prefix(n+1, \rm cons(x, L)) &&= \rm cons(x, \rm prefix(n, L)) && \text{for any $n \in \mbb N$, L: List}
            \end{aligned}\]

            We continue to define:
            \[\begin{aligned}
                &\rm suffix(0, L) &&= L \\
                &\rm suffix(n+1, nil) &&= \rm undefined && \text{for any $n \in \mbb N$}\\
                &\rm suffix(n+1, \rm cons(x, L)) &&= \rm suffix(n, L) && \text{for any $n \in \mbb N$, L: List}
            \end{aligned}\]
        \end{enumerate}

        \newpage
        \item \begin{enumerate}[label=(\alph*)]
            \item We prove this first claim by cases. If $L = \rm nil$, then we have that:
            \begin{align*}
                \begin{aligned}
                    \rm concat(L, cons(b, nil))&= \rm concat(nil, cons(b, nil)) \\
                     &= \rm cons(b, nil) && \text{def of concat} \\
                     &\neq \rm nil
                \end{aligned}
            \end{align*}
            Otherwise, we can write $L = \rm cons(a, R)$ for some list $R$. Then we have,
            \begin{align*}
                \begin{aligned}
                    \rm concat(L, cons(b, nil)) &= \rm concat(\rm cons(a, R), \rm cons(b, nil)) \\
                    &= \rm cons(a, \rm concat(R, \rm cons(b, nil))) && \text{def of concat} \\
                    &\neq \rm nil
                \end{aligned}
            \end{align*}
            and this last list is not nil since it at least contains $a$. Thus, the claim holds.
            \item Let $P(S)$ for a list $S$ be the claim that $\rm last(concat(S, cons(b, nil))) = b$. We prove the claim by structural induction on $S$. First, if $S = \rm nil$, then we have that:
            \begin{align*}
                \begin{aligned}
                    \rm last(concat(S, cons(b, nil))) &= \rm last(concat(nil, cons(b, nil))) && \text{$S = \rm nil$}\\
                    &= \rm last(cons(b, nil)) && \text{def of concat} \\
                    &= b && \text{def of last}
                \end{aligned}
            \end{align*}
            So the base case holds. Now suppose that the claim holds for some list $L$. Then we have that:
            \begin{align*}
                \begin{aligned}
                    \rm last(concat(\rm cons(a, L), \rm cons(b, nil))) &= \rm last(\rm cons(a, \rm concat(L, \rm cons(b, nil)))) && \text{def of concat} \\
                    &= \rm last(\rm concat(L, \rm cons(b, nil))) && \text{def of last} \\
                    &= b && \text{inductive hypothesis}
                \end{aligned}
            \end{align*}
            This completes the proof by structural induction. I don't see where you were meant to use part (a) in the above proof. I believe it might be used to show that the first statement is not undefined, since the inner list is not nil, but I don't see how that is necessary if you order the arguments in this way (you basically repeat part (a) in the proof). 

            \item Notice that,
            \begin{align*}
                \begin{aligned}
                    \rm last(rev(cons(a, R))) &= \rm last(concat(rev(R), cons(a, nil))) && \text{def of rev} \\
                    &= a && \text{by part (b)}
                \end{aligned}
            \end{align*}
        \end{enumerate}
        \item \begin{enumerate}[label=(\alph*)]
            \item We define the function as follows:
            \begin{align*}
                \begin{aligned}
                    & \rm \wormlatinencode(L) &&= \rm nil && \text{$L=$ "bird"} \\
                    & \rm \wormlatinencode(L) &&= \rm L && \text{L: List, $cc(L) = -1$} \\
                    & \rm \wormlatinencode(L) &&= \rm concat(["w", a, "orm"], R) && \text{A} \\
                    & \rm \wormlatinencode(L) &&= \rm concat(suffix(cc(L), L), 
                    \\ &&&\rm \qquad concat(prefix(cc(L), L), ["orm"])) && \text{L: List, $cc(L) \geq 1$}
                \end{aligned}
            \end{align*}
            Where $A$ stands for ``L: List, $\rm cc(L) = 0,\;L = \rm cons(a, R)$ for some $R: \rm List$''. Note that if $cc(L) = 0$, then $L$ has a vowel in the first position, so it is not nil and thus of the above form: $\rm cons(a, R)$ for some other list $R$.
        \end{enumerate}
    \end{enumerate}
\end{document}