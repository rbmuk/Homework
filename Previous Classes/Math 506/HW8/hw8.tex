\documentclass[12pt]{article}
\usepackage[margin=1in]{geometry}
\usepackage{setspace}
\usepackage[dvipsnames,table,xcdraw]{xcolor} % colors
\onehalfspacing{}

% Start of preamble
%==========================================================================================%
% Required to support mathematical unicode
\usepackage[warnunknown, fasterrors, mathletters]{ucs}
\usepackage[utf8x]{inputenc}
\usepackage{R:/sty/quiver}

% Standard mathematical typesetting packages
\usepackage{amsmath,amssymb,amscd,amsthm,amsxtra, pxfonts}
\usepackage{mathtools,mathrsfs,xparse}

% Symbol and utility packages
\usepackage{cancel, textcomp}
\usepackage[mathscr]{euscript}
\usepackage[nointegrals]{wasysym}
\usepackage{apacite}

% Extras
\usepackage{physics}  % Lots of useful shortcuts and macros
\usepackage{tikz-cd}  % For drawing commutative diagrams easily
\usepackage{microtype}  % Minature font tweaks
%\usepackage{pgfplots} % plots

\usepackage{enumitem}
\usepackage{titling}

\usepackage{graphicx}

%\usepackage{quiver}

% Fancy theorems due to @intuitively on discord
\usepackage{mdframed}
\newmdtheoremenv[
backgroundcolor=NavyBlue!30,
linewidth=2pt,
linecolor=NavyBlue,
topline=false,
bottomline=false,
rightline=false,
innertopmargin=10pt,
innerbottommargin=10pt,
innerrightmargin=10pt,
innerleftmargin=10pt,
skipabove=\baselineskip,
skipbelow=\baselineskip]{mytheorem}{Theorem}

\newenvironment{theorem}{\begin{mytheorem}}{\end{mytheorem}}

\newtheorem{corollary}{Corollary}
\newtheorem{lemma}{Lemma}

\newtheoremstyle{definitionstyle}
{\topsep}%
{\topsep}%
{}%
{}%
{\bfseries}%
{.}%
{.5em}%
{}%
\theoremstyle{definitionstyle}
\newmdtheoremenv[
backgroundcolor=Violet!30,
linewidth=2pt,
linecolor=Violet,
topline=false,
bottomline=false,
rightline=false,
innertopmargin=10pt,
innerbottommargin=10pt,
innerrightmargin=10pt,
innerleftmargin=10pt,
skipabove=\baselineskip,
skipbelow=\baselineskip,
]{mydef}{Definition}
\newenvironment{definition}{\begin{mydef}}{\end{mydef}}

\newtheorem*{remark}{Remark}

\newtheorem*{example}{Example}
\newtheorem*{claim}{Claim}

% Common shortcuts
\def\mbb#1{\mathbb{#1}}
\def\mfk#1{\mathfrak{#1}}

\def\bN{\mbb{N}}
\def\C{\mbb{C}}
\def\R{\mbb{R}}
\def\bQ{\mbb{Q}}
\def\bZ{\mbb{Z}}
\def\cph{\varphi}
\renewcommand{\th}{\theta}
\def\ve{\varepsilon}
\newcommand{\mg}[1]{\| #1 \|}

% Often helpful macros
\newcommand{\floor}[1]{\left\lfloor#1\right\rfloor}
\newcommand{\ceil}[1]{\left\lceil#1\right\rceil}
\renewcommand{\qed}{\hfill\qedsymbol}
\renewcommand{\ip}[1]{\langle#1\rangle}
\newcommand{\seq}[2]{\qty(#1_#2)_{#2=1}^{\infty}}

\newcommand{\SET}[1]{\Set{\mskip-\medmuskip #1 \mskip-\medmuskip}}

% End of preamble
%==========================================================================================%

% Start of commands specific to this file
%==========================================================================================%

\usepackage{braket}
\newcommand{\Z}{\mbb Z}
\newcommand{\gen}[1]{\left\langle #1 \right\rangle}
\newcommand{\nsg}{\trianglelefteq}
\newcommand{\F}{\mbb F}
\newcommand{\Aut}{\mathrm{Aut}}
\newcommand{\sepdeg}[1]{[#1]_{\mathrm{sep}}}
\newcommand{\Q}{\mbb Q}
\newcommand{\Gal}{\mathrm{Gal}\qty}
\newcommand{\id}{\mathrm{id}}
\newcommand{\Hom}{\mathrm{Hom}_R}
\newcommand{\A}{\mathbb{A}}

%==========================================================================================%
% End of commands specific to this file

\title{Math 506 HW8}
\date{\today}
\author{Rohan Mukherjee}

\begin{document}
    \maketitle
    \subsection*{Problem 1.}
    We first show that if $\pi: V \to W$ is a morphism of algebraic sets over $\C$ then $\pi$ is continuous w.r.t. the Euclidean topology. Recall that every morphism $\pi$ arises by being the restriction of a polynomial map $f = (f_1, \ldots, f_m)$ from $\A^n \to \A^m$. Since polynomial maps are certainly continous in the Euclidean topology, and restrictions of continous maps are continuous, the result follows. 

    Let $F = Z(f_1, \ldots, f_n)$ be a Zariski-closed set, and let $z_n \to z$ w.r.t. the Euclidean distance metric with each $z_i \in F$. Then,
    \begin{align*}
        0 = f(z_i) \to f(z)
    \end{align*}
    Since polynomials are continous, which shows that $z \in F$ and $F$ is closed in the Euclidean topology. 

    Finally, we claim that the open interval $[-1,1] \subset \C$ is Euclidean closed but not Zariski closed. Suppose it was of the form $Z(f_1, \ldots, f_n)$ for some polynomials $f_i: \C \to \C$. Since complex-valued polynomials in one variable have only finitely many roots if they are nonzero, we see that each polynomial has to be zero and the above set is just all of $\C$, a contradiction. So the Euclidean topology is finer than the Zariski.

    By what we proved in class (equivalently, exercise 15), we have a surjective morphism $\pi: V \to \C^n$ for some $n \geq 0$. We easily see that $n = 0$ otherwise $\pi(V) = \C^n$ would be compact, which is nonsense since $\C^n$ is unbounded. Thus we have a surjection $\pi: V \to \SET{0}$ with finite fibers, and since clearly $V = \pi^{-1}(0)$, $V$ must be finite, as desired. Similarly, every finite subset of Euclidean space is compact [for every point $v \in V$, and any open cover $\mathscr{G} = \SET{G_\alpha}_{\alpha}$, we can just find one open set $G_v$ containing $v$ and then clearly $V \subset \bigcup_{v\in V} G_v$], so we are done.

    \subsection*{Problem 2.}
    We see that
    \begin{align*}
        \frac{\Z\qty[\sqrt{-3}]}{(2)} \cong \frac{\Z[x]}{(x^2+3)} \bigg/ \frac{(2,x^2+3)}{(x^2+3)} \cong \frac{\Z[x]}{(2,x^2+3)} = \frac{\Z[x]}{(2)+(x^2+3)} \cong \frac{\F_2[x]}{(x^2+3)} = \frac{\F_2[x]}{(x+1)^2}
    \end{align*}
    By the third isomorphism theorem there is a 1-1 correspondence between the prime ideals containing $(2)$ and prime ideals of $\Z\qty[\sqrt{-3}]\big/(2) \cong \frac{\F_2[x]}{(x+1)^2} = R$. Since the last ring has 4 elements being $0,1,x,1+x$, and is a PID being a quotient of a PID, we see that the only ideals are $(0), (1) = R, (x), (1+x)$. Notice that $x^2 = 1 \in (x)$, so the only potential nontrivial ideal is $(1+x)$. Finally notice that,
    \begin{align*}
        \frac{\F_2[x]}{(x+1)^2} \bigg / \frac{(x+1)}{(x+1)^2} \cong \frac{\F_2[x]}{(x+1)} \cong \F_2
    \end{align*}
    Thus the ideal $(1+x)$ is maximal and hence prime. Since the ring $R$ has only one prime ideal, we see that there is precisely one prime ideal containing $(2)$, and walking back the isomorphism $a+b\sqrt{-3} \mapsto a+bx$, we see that this ideal is just $(2, 1+\sqrt{-3}) = P$. Recall that $P^2$ is the ideal generated by finite sums of the form $p_1 \cdot p_2$ for $p_1, p_2 \in P$. When $\SET{p_1, \ldots, p_n}$ is a list of generators for $P$, every element of the form $p_1 \cdot p_2$ can be written as $\sum r_i p_i \cdot \sum r_i' p_i' = \sum r_ir_j' p_i p_j'$. Thus we see that, if $P = (p_1, \ldots, p_n)$ then \begin{align*}
        P^2 = \begin{pmatrix}
            p_1^2 & p_1p_2 & \cdots & p_1p_n \\
            p_2p_1 & p_2^2 & \cdots & p_2p_n \\
            \vdots & \vdots & \ddots & \vdots \\
            p_np_1 & p_np_2 & \cdots & p_n^2
        \end{pmatrix}
    \end{align*}
    For our case this means that $P^2 = (2^2, (1+\sqrt{-3})^2 , 2(1+\sqrt{-3}))$. Notice that $(1+\sqrt{-3})^2 = -2 + 2\sqrt{-3} = 4 - 2(1+\sqrt{-3})$, so $P^2 = (4, 2(1+\sqrt{-3}))$. If we somehow had
    \begin{align*}
        2 = 4\alpha + 2(1+\sqrt{-3})\beta
    \end{align*}
    Then we would have
    \begin{align*}
        1 = 2\alpha + (1+\sqrt{-3})\beta
    \end{align*}
    But this would say the prime, proper ideal $P$ is the whole ring, a contradiction. Thus $(2)$ sits strictly between $P$ and $P^2$, and hence cannot be a product of prime ideals in $R$, since if $(2)$ was of the form $\prod_i I_i$ we would see that $(2) \subset \bigcap_i I_i$, so each prime ideal in the product has to contain $(2)$ and we showed this is not possible.

    \subsection*{Problem 3.}
    \begin{enumerate}[label=(\alph*)]
        \item Let $S = \Q$ and $R = \Z$, and take the ideal $I = 2\Z$. Then we see that $SI \cap R = \Q \cap \Z = \Z$, which is strictly bigger than $I$.
        \item We use the above example again. We know that $I = 2\Z$ is a prime ideal of $R$, and that the only ideals of $\Q$ are just $\Q$, $0$. Since clearly $\Q \cap \Z = \Z$ is not $2\Z$, there is no ideal $J$ of $\Q$ such that $J \cap R = I$.
        \item Let $k$ be a field and take $S = k[x]$ and $R = k$. Taking the maximal ideal $\mathfrak{m} = (x)$ of $S$, we see that $\mathfrak m \cap R = R$ is not a maximal ideal in $R$.
        \item We use the above example again. Writing now $J = (x)$, we have that $(J \cap R)S = RS = S$ is not equal to $J$, being much bigger.
    \end{enumerate}

    \subsection*{Problem 4.}
    Define the map $\cph: \A^1 \to V$ by $t \mapsto (t^3,t^4,t^5)$. Notice that $xz-y^2 = t^8 - t^8 = 0$, $yz-x^3 = t^9 - t^9 = 0$, and finally $z^2 - x^y = t^{10} - t^{10} = 0$, so $\cph$ does indeed have the correct range. Now define,
    \begin{align*}
        \psi(x, y, z) = \begin{cases}
            \frac{y}{x} & x \neq 0 \\
            0 & x = 0
        \end{cases}
    \end{align*}
    We can immediately see that $\psi \circ \cph = \id$. We need to show that $\cph \circ \psi(x,y,z) = ((\frac{y}{x})^3, (\frac{y}{x})^4, (\frac{y}{x})^5) = (x,y,z)$. We first see that if $x = 0$ then $y = 0$ from the equation $y^2 = xz$ and $z = 0$ from the equation $z^2=x^2y$. Similarly if $y = 0$ or $z = 0$ then they are all 0, so the only point in $V$ with any coordinate being 0 is just $(0,0,0)$. Thus we can assume that $x,y,z \neq 0$. From here we see that $yz = x^3$ and $xz = y^2$, so $y^2 = x\frac{x^3}{y}$ which shows that $y^3 = x^4$. Thus $\qty(\frac{y}{x})^4 = y \cdot \frac{y^3}{x^4} = y$, and thus $z = \frac{y^2}{x} = \frac{y^2 \cdot y^3}{x \cdot x^4} = \frac{y^5}{x^5}$. So $\psi$ is a 2-sided inverse of $\cph$ and hence $\cph$ is bijective.

    We calculate the Jacobian of the algebraic set $V$ to be
    \begin{align*}
        \begin{pmatrix}
            z & -2y & x \\
            -3x^2 & z & y \\
            -2xy & -x^2 & 2z
        \end{pmatrix}
    \end{align*}
    We see that the tangent space at $(0,0,0)$ is the kernel of the above Jacobian evaluated at $(0,0,0)$, which is easily seen to be all of $\A^3$ (in particular, 3-dimensional). However, the tangent at the point $(1,1,1)$ is the kernel of the Jacobian at $(1, 1, 1)$, which equals:
    \begin{align*}
        \begin{pmatrix}
            1 & -2 & 1 \\
            -3 & 1 & 1 \\
            -2 & -1 & 2
        \end{pmatrix}
    \end{align*}
    This matrix has a nullity of 1, hence its kernel is 1-dimensional and thus $(0,0,0)$ is a singular point of $V$, which shows that $V$ is not isomorphic to $\A^1$.
\end{document}