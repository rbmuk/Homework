\documentclass[12pt]{article}
\usepackage[margin=1in]{geometry}
\usepackage{setspace}
\onehalfspacing{}

% Start of preamble
%==========================================================================================%
% Required to support mathematical unicode
\usepackage[warnunknown, fasterrors, mathletters]{ucs}
\usepackage[utf8x]{inputenc}
\usepackage{R:/sty/quiver}

%\usepackage[dvipsnames,table,xcdraw]{xcolor} % colors
\usepackage{hyperref} % links
\hypersetup{
	colorlinks=true,
	linkcolor=blue,
	filecolor=magenta,      
	urlcolor=cyan,
	pdfpagemode=FullScreen
}

% Standard mathematical typesetting packages
\usepackage{amsmath,amssymb,amscd,amsthm,amsxtra, pxfonts}
\usepackage{mathtools,mathrsfs,xparse}

% Symbol and utility packages
\usepackage{cancel, textcomp}
\usepackage[mathscr]{euscript}
\usepackage[nointegrals]{wasysym}
\usepackage{apacite}

% Extras
\usepackage{physics}  % Lots of useful shortcuts and macros
\usepackage{tikz-cd}  % For drawing commutative diagrams easily
\usepackage{microtype}  % Minature font tweaks
%\usepackage{pgfplots} % plots

\usepackage{enumitem}
\usepackage{titling}

\usepackage{graphicx}

%\usepackage{quiver}

% Fancy theorems due to @intuitively on discord
\usepackage{mdframed}
\newmdtheoremenv[
backgroundcolor=NavyBlue!30,
linewidth=2pt,
linecolor=NavyBlue,
topline=false,
bottomline=false,
rightline=false,
innertopmargin=10pt,
innerbottommargin=10pt,
innerrightmargin=10pt,
innerleftmargin=10pt,
skipabove=\baselineskip,
skipbelow=\baselineskip]{mytheorem}{Theorem}

\newenvironment{theorem}{\begin{mytheorem}}{\end{mytheorem}}

\newtheorem{corollary}{Corollary}
\newtheorem{lemma}{Lemma}

\newtheoremstyle{definitionstyle}
{\topsep}%
{\topsep}%
{}%
{}%
{\bfseries}%
{.}%
{.5em}%
{}%
\theoremstyle{definitionstyle}
\newmdtheoremenv[
backgroundcolor=Violet!30,
linewidth=2pt,
linecolor=Violet,
topline=false,
bottomline=false,
rightline=false,
innertopmargin=10pt,
innerbottommargin=10pt,
innerrightmargin=10pt,
innerleftmargin=10pt,
skipabove=\baselineskip,
skipbelow=\baselineskip,
]{mydef}{Definition}
\newenvironment{definition}{\begin{mydef}}{\end{mydef}}

\newtheorem*{remark}{Remark}

\newtheorem*{example}{Example}
\newtheorem*{claim}{Claim}

% Common shortcuts
\def\mbb#1{\mathbb{#1}}
\def\mfk#1{\mathfrak{#1}}

\def\bN{\mbb{N}}
\def\C{\mbb{C}}
\def\R{\mbb{R}}
\def\bQ{\mbb{Q}}
\def\bZ{\mbb{Z}}
\def\cph{\varphi}
\renewcommand{\th}{\theta}
\def\ve{\varepsilon}
\newcommand{\mg}[1]{\| #1 \|}

% Often helpful macros
\newcommand{\floor}[1]{\left\lfloor#1\right\rfloor}
\newcommand{\ceil}[1]{\left\lceil#1\right\rceil}
\renewcommand{\qed}{\hfill\qedsymbol}
\renewcommand{\ip}[2]{\langle#1, #2\rangle}
\newcommand{\seq}[2]{\qty(#1_#2)_{#2=1}^{\infty}}

\newcommand{\SET}[1]{\Set{\mskip-\medmuskip #1 \mskip-\medmuskip}}

% End of preamble
%==========================================================================================%

% Start of commands specific to this file
%==========================================================================================%

\usepackage{braket}
\newcommand{\Z}{\mbb Z}
\newcommand{\gen}[1]{\left\langle #1 \right\rangle}
\newcommand{\nsg}{\trianglelefteq}
\newcommand{\F}{\mbb F}
\newcommand{\Aut}{\mathrm{Aut}}
\newcommand{\sepdeg}[1]{[#1]_{\mathrm{sep}}}
\newcommand{\Q}{\mbb Q}
\newcommand{\Gal}{\mathrm{Gal}\qty}
\newcommand{\id}{\mathrm{id}}
\newcommand{\Hom}{\mathrm{Hom}_R}

%==========================================================================================%
% End of commands specific to this file

\title{Math 506 HW2}
\date{\today}
\author{Rohan Mukherjee}

\begin{document}
	\maketitle
	\begin{enumerate}
		\item We need only verify that $\Hom(M, M)$ has inverses. Let $T \neq 0 \in \Hom(M, M)$ be a linear map. Then $\ker T$ is a submodule of $M$. Since $M$ is irreducible, and $T$ is nonzero, we must have $\ker T = 0$, and similarly $\Im T = M$. Thus $T$ has a two-sided inverse $T^{-1}$ that is also a linear transformation, which shows that $\Hom(M, M)$ is a division ring.
		
		\item We shall use the vertices $\SET{-1, 1}^2$ of the square. Recall that $r \in D_8$ will result in a 90 degree rotation of the square, and $s$ will result in a reflection about the $x$-axis. Thus, we calculate the image of $r$ to be:
		\[T = \begin{pmatrix}
			0 & -1 \\
			1 & 0 \\
			\end{pmatrix}\]
		And similarly, flipping across the $x$-axis would be represnted by:
		\[S = \begin{pmatrix}
			1 & 0 \\
			0 & -1 \\
			\end{pmatrix}\]
		Thus we can define a (faithful) degree 2 representation of $D_8$ by sending $r \mapsto T$ and $s \mapsto S$.

		\item Let $T = \pi(x^2)$. Since $x$ has order 4 in $Q_8$, and $\pi$ is an embedding, we have that $T^2 - I = 0$, meaning that the minimal polynomial of $T$ divides $x^2-1$. It cannot be $x-1$ for obivous reasons, and we wish to show that it is $x+1$, so suppose by contradiction that it were $x^2-1$. Then $T$ is similar to the following matrix:
		\[D = \begin{pmatrix}
			1 & 0 \\
			0 & -1 \\
			\end{pmatrix}\]
		Recall that $Z(Q_8) = \gen{x^2}$. Let $T = PDP^{-1}$. Then $\cph = P\pi P^{-1}$ is another embedding of $Q_8$ into $\mathrm{GL}_2(\R)$, and now we have $\cph(x^2) = D$. Let $\cph(y) = \begin{pmatrix}
			a & b \\
			c & d \\
		\end{pmatrix}$. Then we have:
		\begin{align*}
			\begin{pmatrix}
				a & -b \\
				-c & d \\
			\end{pmatrix} = \cph(x^2)\cph(y)\cph(x^2) = \cph(y) = \begin{pmatrix}
				a & b \\
				c & d \\
			\end{pmatrix}
		\end{align*}
		This in particular shows that $b = c = 0$, hence $\cph(y)$ is diagonal. However, the condition that $\cph(y) \neq I$ and $\cph(y^4) = I$ would show that $a \neq 1$ is a fourth root of $1$, which is a contradiction. Thus the minimal polynomial of $T$ is $x+1$, and $\cph(x) = -I$. In particular, $\pi(x) = P^{-1}(-I)P = -I$ as well. Thus the minimal polynomial of $\pi(x)$ is $x^2+1=0$, and it's rational canonical form is:
		\[\begin{pmatrix}
			0 & -1 \\
			1 & 0 \\
		\end{pmatrix}\]
		Replacing $\pi$ with a conjugate shows that $\pi(x)$ is the above matrix. A quick calculation shows that if $T^2 = -I$, then $T$ is of the form:
		\begin{align*}
			\begin{pmatrix}
				a & b \\
				c & -a
			\end{pmatrix}
		\end{align*}
		So let $\pi(y)$ equal the above matrix for a suitable choice of $a,b,c,d$. Rewriting the last relation for the quarternion group, we get $y = x^{-1}yx^{-1}$. Applying $\pi$ from above gives us the following equation:
		\begin{align*}
			\begin{pmatrix}
				a & b \\
				c & -a
			\end{pmatrix} = \begin{pmatrix}
				0 & 1 \\
				-1 & 0
			\end{pmatrix}\begin{pmatrix}
				a & b \\
				c & -a
			\end{pmatrix}\begin{pmatrix}
				0 & 1 \\
				-1 & 0
			\end{pmatrix} = \begin{pmatrix}
				a & c \\
				b & -a
			\end{pmatrix}
		\end{align*}
		This of course tells us that $b = c$. Thus $\pi(y^2)$ is just
		\begin{align*}
				(a^2+b^2)I
		\end{align*}
		We cannot possibly have $a^2+b^2 = -1$ when $a, b \in \R$, which completes the proof. I note that the above contradictions could easily be used to find a degree 2 representation $\pi: Q_8 \to \mathrm{GL}_2(\C)$.
		
		\item Assume by contradiction that $V$ was decomposable, i.e. $V = V_1 \oplus V_2$ for some nonzero submodules $V_1, V_2$. Let $S = \mathrm{span}\Set{v_i \mid i \in \Z}$. Define the following projection:
		\begin{align*}
			\pi: V \to S \quad \text{by} \quad v_i \mapsto v_i, \; w_i \mapsto 0
		\end{align*}
		And let $\sum b_i v_i \in \pi(V_1) \cap \pi(V_2)$. Finding $\sum a_iv_i + \sum c_iw_i \in V_1$ and $\sum a_i'v_i' + \sum c_i'w_i' \in V_2$ with the above image, by applying our projection we see that 
		\begin{align*}
			\sum b_iv_i = \sum a_iv_i = \sum a_i'v_i'
		\end{align*}
		Which shows that terms of each sum is equal by linear independence of the $v_i$. Since $V_1 + V_2 = V$, it must be that $\sum b_iv_i \in V_1$ or $V_2$, so suppose without loss of generality it is the first case. Then we have the updated set of equations:
		\begin{align*}
			\sum b_iv_i = \pi\qty(\sum b_iv_i) = \pi \qty(\sum b_iv_i + \sum c_i'w_i')
		\end{align*}
		Where $\sum b_iv_i \in V_1$ and $\sum b_iv_i + \sum c_i'w_i' \in V_2$. Once again, if $\sum c_i'w_i' \in V_1$, we get a contradiction since then $\sum b_iv_i + \sum c_i'w_i' \in V_1 \cap V_2$, where the sum would be nonzero since $\sum b_iv_i \neq 0$ and the $v_i, w_i$ are linearly independent. Similarly if $\sum c_i'w_i' \in V_2$, we get a contradiction. Thus we have verified that $S = P_1 \oplus P_2$. 

		Now note that $x-1, y-1 \neq 0$, so it acts as an invertible linear transformation. In particular, $(x-1)P_1 \cap (x-1)P_2 = (x-1)(P_1 \cap P_2) = (y-1)P_1 \cap (y-1)P_2 = 0$ by injectivity. Also note that for each $w_i \in T$, $w_i$ is the image of $(x-1)v_{i-1}$ and $(y-1)v_i$ per a simple calculation. These facts together show that $T = (x-1)P_1 \oplus (x-1)P_2 = (y-1)P_1 \oplus (y-1)P_2$.

		Now take $\sum a_iw_i \in (x-1)P_1 \cap (y-1)P_2$, and suppose that (WLOG) $\sum a_iw_i \in V_1$. Then $\sum a_iw_i = (x-1)\sum a_iv_{i-1}$ where $\sum a_iv_{i-1} \in P_1$, and $\sum a_iw_i = (y-1)\sum a_iv_i$ where $\sum a_iv_i \in P_2$. Since the projections trivially intersect, it must be that $\sum a_iv_{i-1} \in V_1$ and similarly $\sum a_iv_i \in V_2$. Since $V_1, V_2$ are $K$-stable, $(y-1)\sum a_iv_i \in V_2$ and not in $V_1$, showing that $\sum a_iw_i = 0$. Then $(x-1)P_1 = T \cap (x-1)P_1 = (y-1)P_1 \cap (x-1)P_1$, showing that $(x-1)P_1 \subset (y-1)P_1$. Applying this logic in reverse shows that $(x-1)P_1 = (y-1)P_1$ and that $(x-1)P_2 = (y-1)P_2$.

		Since $P_1 \oplus P_2 = S$, suppose WLOG that $v_0 \in P_1$. By the above equality, we have that $(x-1)v_0 = w_1 \in (y-1)P_1$. Thus $(y-1)^{-1} w_1 = v_1 \in P_1$. Similarly, $(y-1)v_0 = w_0 \in (x-1)P_1$, showing that $(x-1)^{-1} w_0 = v_{-1} \in P_1$. By induction $v_i \in P_1$ for all $i \in \bZ$, which shows that $P_1 = S$. At last, if any of the $v_i \in V_2$, then $v_i \in P_1 \cap P_2$ a contradiction, so $V$ contains all the $v_i$. Since $V$ is $K$-stable, $(y-1)v_i = w_i \in V_1$ for all $i$, which shows that $V_1 = V$, a contradiction. Thus $V$ is indecomposable.
	\end{enumerate}
\end{document}
