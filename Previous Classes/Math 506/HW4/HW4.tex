\documentclass[12pt]{article}
\usepackage[margin=1in]{geometry}
\usepackage{setspace}
\onehalfspacing{}

% Start of preamble
%==========================================================================================%
% Required to support mathematical unicode
\usepackage[warnunknown, fasterrors, mathletters]{ucs}
\usepackage[utf8x]{inputenc}
%\usepackage{R:/sty/quiver}

\usepackage[dvipsnames,table,xcdraw]{xcolor} % colors
\usepackage{hyperref} % links
\hypersetup{
colorlinks=true,
linkcolor=blue,
filecolor=magenta,
urlcolor=cyan,
pdfpagemode=FullScreen
}

% Standard mathematical typesetting packages
\usepackage{amsmath,amssymb,amscd,amsthm,amsxtra, pxfonts}
\usepackage{mathtools,mathrsfs,xparse}

% Symbol and utility packages
\usepackage{cancel, textcomp}
\usepackage[mathscr]{euscript}
\usepackage[nointegrals]{wasysym}
\usepackage{apacite}

% Extras
\usepackage{physics}  % Lots of useful shortcuts and macros
\usepackage{tikz-cd}  % For drawing commutative diagrams easily
\usepackage{microtype}  % Minature font tweaks
%\usepackage{pgfplots} % plots

\usepackage{enumitem}
\usepackage{titling}

\usepackage{graphicx}

%\usepackage{quiver}

% Fancy theorems due to @intuitively on discord
\usepackage{mdframed}
\newmdtheoremenv[
backgroundcolor=NavyBlue!30,
linewidth=2pt,
linecolor=NavyBlue,
topline=false,
bottomline=false,
rightline=false,
innertopmargin=10pt,
innerbottommargin=10pt,
innerrightmargin=10pt,
innerleftmargin=10pt,
skipabove=\baselineskip,
skipbelow=\baselineskip]{mytheorem}{Theorem}

\newenvironment{theorem}{\begin{mytheorem}}{\end{mytheorem}}

\newtheorem{corollary}{Corollary}
\newtheorem{lemma}{Lemma}

\newtheoremstyle{definitionstyle}
{\topsep}%
{\topsep}%
{}%
{}%
{\bfseries}%
{.}%
{.5em}%
{}%
\theoremstyle{definitionstyle}
\newmdtheoremenv[
backgroundcolor=Violet!30,
linewidth=2pt,
linecolor=Violet,
topline=false,
bottomline=false,
rightline=false,
innertopmargin=10pt,
innerbottommargin=10pt,
innerrightmargin=10pt,
innerleftmargin=10pt,
skipabove=\baselineskip,
skipbelow=\baselineskip,
]{mydef}{Definition}
\newenvironment{definition}{\begin{mydef}}{\end{mydef}}

\newtheorem*{remark}{Remark}

\newtheorem*{example}{Example}
\newtheorem*{claim}{Claim}

% Common shortcuts
\def\mbb#1{\mathbb{#1}}
\def\mfk#1{\mathfrak{#1}}

\def\bN{\mbb{N}}
\def\C{\mbb{C}}
\def\R{\mbb{R}}
\def\bQ{\mbb{Q}}
\def\bZ{\mbb{Z}}
\def\cph{\varphi}
\renewcommand{\th}{\theta}
\def\ve{\varepsilon}
\newcommand{\mg}[1]{\| #1 \|}

% Often helpful macros
\newcommand{\floor}[1]{\left\lfloor#1\right\rfloor}
\newcommand{\ceil}[1]{\left\lceil#1\right\rceil}
\renewcommand{\qed}{\hfill\qedsymbol}
\renewcommand{\ip}[1]{\langle#1\rangle}
\newcommand{\seq}[2]{\qty(#1_#2)_{#2=1}^{\infty}}

\newcommand{\SET}[1]{\Set{\mskip-\medmuskip #1 \mskip-\medmuskip}}

% End of preamble
%==========================================================================================%

% Start of commands specific to this file
%==========================================================================================%

\usepackage{braket}
\newcommand{\Z}{\mbb Z}
\newcommand{\gen}[1]{\left\langle #1 \right\rangle}
\newcommand{\nsg}{\trianglelefteq}
\newcommand{\F}{\mbb F}
\newcommand{\Aut}{\mathrm{Aut}}
\newcommand{\sepdeg}[1]{[#1]_{\mathrm{sep}}}
\newcommand{\Q}{\mbb Q}
\newcommand{\Gal}{\mathrm{Gal}\qty}
\newcommand{\id}{\mathrm{id}}
\newcommand{\Hom}{\mathrm{Hom}_R}
\newcommand{\GL}{\mathrm{GL}}
\newcommand{\Ind}{\mathrm{Ind}}

%==========================================================================================%
% End of commands specific to this file

\title{Math 506 HW4}
\date{\today}
\author{Rohan Mukherjee}

\begin{document}
    \maketitle
    \begin{enumerate}
        \item If we consider the group $G = D_3$, we see that $srsr^{-1} = srrs = sr^2s = r^{-2} = r$, so $\gen{r} \subset [G,G]$. Since $D_3/\gen{r} \cong \Z_2$, we have that $[G,G] = \gen{r}$ since $[G,G]$ is the smallest subgroup with abelian quotient. The one-dimensional irreducible representations of $D_3$ are thus the irreducible representations of $\Z/2$, which are just going to be the trivial representation and the representation sending $r \to 1$, $s \to -1$. Finally, we know the last irreducible representation of $D_3$ to be the representation sending $r$ to the matrix that rotates by $\frac{2\pi}{3}$ radians about the origin, and $s$ to the matrix that flips across the $x$-axis. This is the representation that sends 
        \begin{align*}
            r \to \begin{pmatrix} -\frac{1}{2} & -\frac{\sqrt{3}}{2} \\ \frac{\sqrt{3}}{2} & -\frac{1}{2} \end{pmatrix}, \quad s \to \begin{pmatrix} 1 & 0 \\ 0 & -1 \end{pmatrix}.
        \end{align*} Since the first matrix is orthogonal, it has determinant $\pm 1$, and since it is orientation preserving it has determinant $1$. The second matrix can easily be seen to have matrix $-1$. The three conjugacy classes of $D_3$ are $\SET{e}, \SET{r, r^2}, \SET{s, sr, sr^2}$. Clearly, for a 1-dimensinoal representation, the determinant equals the trace. By our above calculations, if we call the only 1-dimensional non-trivial character $\pi_2=\chi_2$, we know that $\chi_2(e) = 1$, $\chi_2(r) = 1$ and $\chi_2(s) = -1$. If we call the 2-dimensional representation $\pi_3$, we also see that $\det(\pi_3(e)) = 1$, $\det(\pi_3(r)) = 1$ and $\det(\pi_3(s)) = -1$. Since $\pi_2, \pi_3$ are inequivalent irreducible representations, we have disproven the claim.
        
        \item Recall that if $\pi: G \to \GL_n(\C)$, then $\pi(g)$ is diagonalizable for any $g$, and that every eigenvalue of $\pi(g)$ is a root of unity. Suppose that the elements on the diagonal are $\lambda_1, \ldots, \lambda_n$ (where repretition is possible). If ever $\lambda \neq 1$, then $\lambda$ has a non-zero imaginary part. If we were to have $\chi(g) = \chi(1) = n$, we would need the sum of the eigenvalues to be $n$. The only way to have a sum of $n$ norm 1 numbers to be $n$ is if all the numbers are themselves 1, otherwise the weight of the imaginary part would take from the real part and we couldn't possibly have $n$ as the sum. This shows that if $\chi(g) = \chi(1)$, then $\pi(g)$ is similar to the identity matrix and hence just the identity matrix, so $g \in \ker \pi$. Now, if $g \neq 1$, then $g$ is clearly not conjugate to the identity, so by the orthogonality relations we have that
        \begin{align*}
            \sum_{\chi_i} \chi_i(g) \overline{\chi_i(1)} = 0.
        \end{align*}
        If somehow $\chi_i(g) = \chi_i(1)$ for all irreducible characters $\chi_i$, then the above sum would evaluate to
        \begin{align*}
            \sum_{\chi_i} \chi_i(1) \overline{\chi_i(1)} = |C_G(e)| = |G|
        \end{align*}
        which is certainly nonzero. This completes the proof.

        \item Recall that if $g_1, \ldots, g_m$ are distinct representatives from each of the $m$ left cosets of $H$, 
        \begin{align*}
            c_G(g) = \Ind_H^G(c_H)(g) \coloneqq \sum_{g_i^{-1}gg_i \in H} c_H(g_i^{-1}gg_i).
        \end{align*} Notice first that if $xh_1x^{-1} = yh_2y^{-1}$, then we have $y^{-1}xh_1x^{-1}y = h_2$, so either $y^{-1}x \in H$ or $h_1 = h_2 = 1$. This means that $xHx^{-1} \cap yHy^{-1} = 1$ if $x$ and $y$ are in different left cosets of $H$, and if $x \in yH$, then $x^{-1}Hx = y^{-1}Hy$ so there are precisely $|G : H|$ distinct conjugacy classes of $H$. We proceed with the following calculation:
        \begin{align*}
            \ip{c_G, c_G} &= \frac{1}{|G|} \sum_{g \in G} c_G(g) \overline{c_G(g)} = \frac{1}{|G|} \sum_{g \in G} \qty(\sum_{g_i^{-1}gg_i \in H} c_H(g_i^{-1}gg_i) \overline{\sum_{g_j^{-1}gg_j \in H} c_H(g_j^{-1}gg_j)}) \\
            &= \frac{1}{|G|} \sum_{g \in G} \qty(\sum_{g_i^{-1}gg_i \in H} c_H(g_i^{-1}gg_i) \sum_{g_j^{-1}gg_j \in H} \overline{c_H(g_j^{-1}gg_j)}) \\
            &= \frac{1}{|G|} \sum_{g \in G} \sum_{\substack{g_igg_i^{-1} \in H \\ g_jgg_j^{-1} \in H}} c_H(g_i^{-1}gg_i) \overline{c_H(g_j^{-1}gg_j)} \\
        \end{align*}
        As before, if $g_i \neq g_j$ and $g_igg_i^{-1}, g_jgg_j^{-1} \in H$ we have $g \in g_i^{-1}Hg_i \cap g_j^{-1}Hg_j = 1$ so $g = 1$ and since $c_H(1) = 0$, the only terms that contribute to the sum are those where $g_i = g_j$. This means that the above sum is just
        \begin{align*}
            \frac{1}{|G|} \sum_{g \in G} \sum_{g_igg_i^{-1} \in H} |c_H(g_i^{-1}gg_i)|^2
        \end{align*}
        We want this to equal 
        \begin{align*}
            \frac{1}{|H|} \sum_{h \in H} |c_H(h)|^2 = \frac{1}{|H|} \sum_{h \in H} |c_H(h)|^2 = \ip{c_H, c_H}.
        \end{align*}
        Thus, we count the number of times $|c_H(h)|^2$ for $h \neq 1$ shows up in the first sum. We showed above that there are precisely $|G : H|$ distinct conjugacy classes of $H$. The above also shows that $h$ has precisely $|G : H|$ distinct conjugates in $G$, namely, $g_i^{-1}hg_i$ for the $g_i$ as above. Thus the term $|c_H(h)|^2$ shows up precisely $|G : H|$ times for each $h$. This shows that
        \begin{align*}
            \frac{1}{|G|} \sum_{g \in G} \sum_{g_igg_i^{-1} \in H} |c_H(g_i^{-1}gg_i)|^2 = \frac{|G : H|}{|G|} \sum_{h \in H} |c_H(h)|^2 = \frac{1}{|H|} \sum_{h \in H} |c_H(h)|^2 = \ip{c_H, c_H}.
        \end{align*}

        We first seek to show that if $c_H$ is any class function on $H$, then $\Ind_H^G(c_H)$ restricts to itself on $H$. This is true because the condition $ghg^{-1} \in H$ is equivalent to $h \in g^{-1}Hg$, and by hypothesis this is only true if $g \in H$. Thus,
        \begin{align*}
            \Ind_H^G(c_H)(h) = \frac{1}{|H|} \sum_{h' \in H} c_H((h')^{-1}hh') = \frac{1}{|H|} \sum_{h' \in H} c_H(h) = c_H(h).
        \end{align*}
        Notice that if $\chi_1, \chi_2$ are two characters of $H$, then,
        \begin{align*}
            \Ind_H^G(a\chi_1 + b\chi_2) &= \frac{1}{|H|} \sum_{x \;:\; xgx^{-1} \in H} (a\chi_1(xgx^{-1}) + b\chi_2(xgx^{-1})) \\&= a\frac{1}{|H|} \sum_{x \;:\; xgx^{-1} \in H} \chi_1(xgx^{-1}) + b\frac{1}{|H|} \sum_{x \;:\; xgx^{-1} \in H} \chi_2(xgx^{-1}) = a\Ind_H^G(\chi_1) + b\Ind_H^G(\chi_2).
        \end{align*}
        Holding for any complex numbers $a, b \in \C$. In particular, $c_\chi = \Ind_H^G(\chi - d\chi_1)$ is an integer linear combination of two characters. This tells us that $c_\chi = c_1 \chi_1 + \sum_{i=2}^m c_i\chi_i$ for some $c_i \in \Z$ with the $\chi_i$ being the distinct irreducible characters of $G$. In the setting that $\chi$ is not the trivial character on $H$, and noting that restricting the trivial character clearly gives the trivial character, by Frobenius reciprocity we see that
        \begin{align*}
            \ip{c_\chi, \chi_1}_G = \ip{\chi - d, \chi_1}_H = \ip{\chi, \chi_1}_H - d\ip{\chi_1, \chi_1}_H = -d.
        \end{align*}
        Where we recall that $\ip{\chi_1, \chi_2} = 0$ if $\chi_1, \chi_2$ are distinct irreducible characters. Thus $c_\chi = -d\chi_1 + \sum_{i=2}^m c_i\chi_i$. By the previous result, we know that $\mg{c_\chi}_G^2 = \mg{\chi - d}_H^2$. Now, since $\chi$ and the trivial character are distinct irreducibles, we clearly have that $\mg{\chi - d\chi_1}_H^2 = d^2+1$. It follows then that $\mg{c_\chi}_G^2 = d^2+1$. Since distinct characters are orthogonal, this tells us that $\mg{\sum_{i=2}^m c_i\chi_i}_G^2 = 1$, so precisely one of the $c_i = \pm 1$ for $i \geq 2$, and the rest are 0. Recalling that the restrictiion of $\Ind_H^G(c_H) = c_H$ for any class function $c_H$ on $H$,
        \begin{align*}
            c_\chi(1) = \chi_H(1) - d = 0
        \end{align*}
        This shows that $c_i = 1$, since for any character $\chi_G$ of $G$, $\chi_G(1)$ equals the dimension of the representation, and is importantly positive. Thus we can write $c_\chi = \chi_G - \chi_G(1)$ for some irreducible representation $\chi_G$ of $G$. In the case where $\chi$ is the trivial character on $H$, we are inducing the 0 class function, so taking the trivial character on $G$ will suffice for $\chi_G$. Notably, we have induced a degree $d$ irreducible character and ended up with another degree $d$ irreducible character of $G$. Now define
        \begin{align*}
            N = \SET{g \in G \mid xgx^{-1} \not \in H \text{ for all } x \in G} \cup \SET{1}.
        \end{align*}
        Let $n \in N$. Then $n$ is not conjugate to anything in $H$. In particular, if $c_H$ is any class function, $\Ind_H^G(c_H)(n) = \frac{1}{|G|} \sum_{xgx^{-1} \in H} c_H(xgx^{-1}) = 0$ because the sum is empty. Conversly, if $g \neq 1 \in G$ is so that $xgx^{-1} = h \in H$ for some $x \in G$, we see first that $h \neq 1$. Thus by question 2 there is an irreducible character $\chi_H$ of $H$ so that $\chi_H(h) \neq \chi_H(1)$. In particular, if we define $c_H = \chi_H - \chi_H(1)$, $\chi_H(h) \neq 0$. Write $c_G = \Ind_H^G \chi_H = \chi_G - \chi_G(1)$ for some irreducible character $\chi_G$ of $G$. Then,
        \begin{align*}
            c_G(h) = \sum_{g_ihg_i^{-1}} c_H(g_ihg_i^{-1})
        \end{align*}
        For the final time, $g_ihg_i^{-1} \in H$ means $h \in g_i^{-1}Hg_i$ which is only true if $g_i = 1$. Thus the above sum evaluates to $c_H(h) \neq 0$. Since a class function on $G$ is constant on conjugacy classes, we see that $c_G(xgx^{-1}) = c_G(g) \neq 0$. If we write $\pi_G$ as the representation of $\chi_G$, this means that $g \not \in \pi_G$. Thus $N = \bigcap_{\pi_G} \ker \pi_G$, which shows that $N$ is the intersection of normal subgroups of $G$ and hence itself normal.

        \item Let $\lambda = (2, 1)$. If we let $T_1 = \begin{array}{c|c}
            1 & 2 \\ \hline
            3 &
        \end{array}$, $T_2 = \begin{array}{c|c}
            1 & 3 \\ \hline
            2 &
        \end{array}$, we can see that the Tabloids of $\lambda$ are 
        \begin{align*}
            a = \SET{T_1} = \SET{\begin{array}{c|c}
                1 & 2 \\ \hline
                3 & 
            \end{array}, 
            \begin{array}{c|c}
                2 & 1 \\ \hline
                3 &
            \end{array}}, 
            c = \SET{T_2} = \SET{\begin{array}{c|c}
                1 & 3 \\ \hline
                2 & 
            \end{array},
            \begin{array}{c|c}
                3 & 1 \\ \hline
                2 &
            \end{array}},
            b = \SET{\begin{array}{c|c}
                2 & 3 \\ \hline
                1 &
            \end{array},
            \begin{array}{c|c}
                3 & 2 \\ \hline
                1 &
            \end{array}}.
        \end{align*}
        We see, noting that the only nontrivial element of the column group is $(13)$, that $V_{T_1} = a-b$ and by similar reasoning $V_{T_2} = c-b$. We claim that $\mathrm{span}\SET{V_{T_1}+V_{T_2}}$ is a submodule of $S^\lambda$. Note that $(12)V_{T_1} = a - c$, $(12)V_{T_2} = b - c$. Thus $(12)(V_{T_1}+V_{T_2}) = a - c + b - c = a + b - 2c = a + c - 2b = V_{T_1}+V_{T_2}$, since $1 = -2$ in characteristic 3. Similarly, one calculates $(123)V_{T_1} = b-c$ and $(123)V_{T_2} = a-c$, so $(123)(V_{T_1}+V_{T_2}) = b - c + a - c = a + b - 2c = a + c - 2b = V_{T_1}+V_{T_2}$. Thus $\mathrm{span}\SET{V_{T_1}+V_{T_2}}$ is a submodule of $S^\lambda$, which shows that for this choice of $\lambda$ $S^\lambda$ is not irreducible. Notably, this representation collapses into the identity representation of $S_3$, since every element of $S_3$ acts as the identity on the basis elements of $S^\lambda$.
    \end{enumerate}
\end{document}