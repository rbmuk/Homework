\documentclass[12pt]{article}
\usepackage[margin=1in]{geometry}

% Start of preamble
%==========================================================================================%
% Required to support mathematical unicode
\usepackage[warnunknown, fasterrors, mathletters]{ucs}
\usepackage[utf8x]{inputenc}

\usepackage[dvipsnames,table,xcdraw]{xcolor} % colors
\usepackage{hyperref} % links
\hypersetup{
	colorlinks=true,
	linkcolor=blue,
	filecolor=magenta,      
	urlcolor=cyan,
	pdfpagemode=FullScreen
}

% Standard mathematical typesetting packages
\usepackage{amsmath,amssymb,amscd,amsthm,amsxtra, pxfonts}
\usepackage{mathtools,mathrsfs,dsfont,xparse}

% Symbol and utility packages
\usepackage{cancel, textcomp}
\usepackage[mathscr]{euscript}
\usepackage[nointegrals]{wasysym}
\usepackage{apacite}

% Extras
\usepackage{physics}  % Lots of useful shortcuts and macros
\usepackage{tikz-cd}  % For drawing commutative diagrams easily
\usepackage{microtype}  % Minature font tweaks
%\usepackage{pgfplots} % plots

\usepackage{enumitem}
\usepackage{titling}

\usepackage{graphicx}

% Fancy theorems due to @intuitively on discord
\usepackage{mdframed}
\newmdtheoremenv[
backgroundcolor=NavyBlue!30,
linewidth=2pt,
linecolor=NavyBlue,
topline=false,
bottomline=false,
rightline=false,
innertopmargin=10pt,
innerbottommargin=10pt,
innerrightmargin=10pt,
innerleftmargin=10pt,
skipabove=\baselineskip,
skipbelow=\baselineskip
]{mytheorem}{Theorem}

\newenvironment{theorem}{\begin{mytheorem}}{\end{mytheorem}}

\newtheorem{corollary}{Corollary}
\newtheorem{lemma}{Lemma}

\newtheoremstyle{definitionstyle}
{\topsep}%
{\topsep}%
{}%
{}%
{\bfseries}%
{.}%
{.5em}%
{}%
\theoremstyle{definitionstyle}
\newmdtheoremenv[
backgroundcolor=Violet!30,
linewidth=2pt,
linecolor=Violet,
topline=false,
bottomline=false,
rightline=false,
innertopmargin=10pt,
innerbottommargin=10pt,
innerrightmargin=10pt,
innerleftmargin=10pt,
skipabove=\baselineskip,
skipbelow=\baselineskip,
]{mydef}{Definition}
\newenvironment{definition}{\begin{mydef}}{\end{mydef}}

\newtheorem*{remark}{Remark}

\newtheorem*{example}{Example}

% Common shortcuts
\def\mbb#1{\mathbb{#1}}
\def\mfk#1{\mathfrak{#1}}

\def\bN{\mbb{N}}
\def \C{\mbb{C}}
\def \R{\mbb{R}}
\def\bQ{\mbb{Q}}
\def\bZ{\mbb{Z}}
\def \cph{\varphi}
\renewcommand{\th}{\theta}
\def \ve{\varepsilon}
\newcommand{\mg}[1]{\| #1 \|}

% Often helpful macros
\newcommand{\floor}[1]{\left\lfloor#1\right\rfloor}
\newcommand{\ceil}[1]{\left\lceil#1\right\rceil}
\renewcommand{\qed}{\hfill\qedsymbol}
\renewcommand{\ip}[2]{\langle #1, #2 \rangle}
\newcommand{\seq}[2]{\qty(#1_#2)_{#2=1}^{\infty}}

% Sets
\DeclarePairedDelimiterX\set[1]\lbrace\rbrace{\def\given{\;\delimsize\vert\;}#1}

% End of preamble
%==========================================================================================%

% Start of commands specific to this file
%==========================================================================================%

%==========================================================================================%
% End of commands specific to this file

\title{Answer to P2}
\date{\today}
\author{Rohan Mukherjee}

\begin{document}
	\maketitle
	\begin{enumerate}[leftmargin=\labelsep]
		\item[9.] Define the function $f$ by $f(x) = e^{-1/x^2}$ if $x \neq 0$, and $f(0) = 0$. 
		\begin{enumerate}
			\item Show that $\lim_{x \to 0} f(x)/x^n = 0$ for all $n > 0$. 
			
			\begin{align*}
				\lim_{x \to 0} \frac{e^{-1/x^2}}{x^n} = \lim_{y \to \infty} \frac{y^{n/2}}{e^{y}} = 0
			\end{align*}
			
			\item Show that $f$ is differentiable at 0 and that $f'(0) = 0$.
			
			\begin{align*}
				f'(0) = \lim_{h \to 0} \frac{f(h)-f(0)}{h} = \frac{e^{-1/h^2}}{h} = 0 \quad \text{ by part (a)}
			\end{align*}
			
			\item Show by induction on $k$ that for $x \neq 0$, $f^{(k)}(x) = P(1/x)e^{-1/x^2}$, where $P$ is a polynomial of degree $3k$. 
			
			We see that, for $x \neq 0$, $f'(x) = -\frac{2}{x^3} e^{-1/x^2}$, and that $-2/x^3$ is a polynomial in $1/x$ of degree 3.
			
			Next, suppose that $f^{(k)}(x) = P(1/x)e^{-1/x^2}$ for some polynomial $P(x)$ of degree $3k$ and all $x \neq 0$. We see that:
			\begin{align*}
				f^{(k+1)}(x) = e^{-1/x^2} \qty(P(1/x) \cdot \frac{-2}{x^3} + P'(1/x) \cdot \frac{-1}{x^2})
			\end{align*}
			$P'(x)$ is a polynomial of degree $3k-1$ and thus $P'(1/x) \cdot -1/x^2$ is a polynomial in $1/x$ of degree $3k+1$. $P(1/x) \cdot -2/x^3$ is a polynomial of $3k+3$, thus $P(1/x) \cdot -2/x^3 + P'(1/x) \cdot -1/x^2$ is a polynomial of degree $3k+3$, completing the inductive step.
			
			\item We recall that $f'(0) = 0$. Suppose that $f^{(k)}(0) = 0$ for some $k \geq 2$. Then
			\begin{align*}
				f^{(k+1)}(0) = \lim_{h \to 0} \frac{f^{(k)}(h) - f^{(k)}(0)}{h} = \lim_{h \to 0} \frac{P(1/h) \cdot e^{-1/h^2}}{h}
			\end{align*}
			for some polynomial $P(x)$. Then $1/hP(1/h)$ is a polynomial in $1/h$ with no constant term, and thus may be written in the form $\sum_{k=1}^{N} a_kh^{-k}$. Now,
			\begin{align*}
				\lim_{h \to 0} \frac{P(1/h) \cdot e^{-1/h^2}}{h} = \lim_{h \to 0} e^{-1/h^2} \cdot \sum_{k=1}^{N} a_kh^{-k} = \sum_{k=1}^{N} a_k \lim_{h \to 0} e^{-1/h^2} \cdot h^{-k}
			\end{align*}
			Part (a) shows that the limit inside of the sum is always 0, thus this quantity is 0, which completes the inductive step.
		\end{enumerate}
	\end{enumerate}
\end{document}
