\documentclass[12pt]{article}
\usepackage[margin=1in]{geometry}

% Start of preamble
%==========================================================================================%
% Required to support mathematical unicode
\usepackage[warnunknown, fasterrors, mathletters]{ucs}
\usepackage[utf8x]{inputenc}

\usepackage[dvipsnames,table,xcdraw]{xcolor} % colors
\usepackage{hyperref} % links
\hypersetup{
	colorlinks=true,
	linkcolor=blue,
	filecolor=magenta,      
	urlcolor=cyan,
	pdfpagemode=FullScreen
}

% Standard mathematical typesetting packages
\usepackage{amsmath,amssymb,amscd,amsthm,amsxtra, pxfonts}
\usepackage{mathtools,mathrsfs,dsfont,xparse}

% Symbol and utility packages
\usepackage{cancel, textcomp}
\usepackage[mathscr]{euscript}
\usepackage[nointegrals]{wasysym}
\usepackage{apacite}

% Extras
\usepackage{physics}  % Lots of useful shortcuts and macros
\usepackage{tikz-cd}  % For drawing commutative diagrams easily
\usepackage{microtype}  % Minature font tweaks
%\usepackage{pgfplots} % plots

\usepackage{enumitem}
\usepackage{titling}

\usepackage{graphicx}

% Fancy theorems due to @intuitively on discord
\usepackage{mdframed}
\newmdtheoremenv[
backgroundcolor=NavyBlue!30,
linewidth=2pt,
linecolor=NavyBlue,
topline=false,
bottomline=false,
rightline=false,
innertopmargin=10pt,
innerbottommargin=10pt,
innerrightmargin=10pt,
innerleftmargin=10pt,
skipabove=\baselineskip,
skipbelow=\baselineskip
]{mytheorem}{Theorem}

\newenvironment{theorem}{\begin{mytheorem}}{\end{mytheorem}}

\newtheorem{corollary}{Corollary}
\newtheorem{lemma}{Lemma}

\newtheoremstyle{definitionstyle}
{\topsep}%
{\topsep}%
{}%
{}%
{\bfseries}%
{.}%
{.5em}%
{}%
\theoremstyle{definitionstyle}
\newmdtheoremenv[
backgroundcolor=Violet!30,
linewidth=2pt,
linecolor=Violet,
topline=false,
bottomline=false,
rightline=false,
innertopmargin=10pt,
innerbottommargin=10pt,
innerrightmargin=10pt,
innerleftmargin=10pt,
skipabove=\baselineskip,
skipbelow=\baselineskip,
]{mydef}{Definition}
\newenvironment{definition}{\begin{mydef}}{\end{mydef}}

\newtheorem*{remark}{Remark}

\newtheorem*{example}{Example}

% Common shortcuts
\def\mbb#1{\mathbb{#1}}
\def\mfk#1{\mathfrak{#1}}

\def\bN{\mbb{N}}
\def \C{\mbb{C}}
\def \R{\mbb{R}}
\def\bQ{\mbb{Q}}
\def\bZ{\mbb{Z}}
\def \cph{\varphi}
\renewcommand{\th}{\theta}
\def \ve{\varepsilon}
\newcommand{\mg}[1]{\| #1 \|}

% Often helpful macros
\newcommand{\floor}[1]{\left\lfloor#1\right\rfloor}
\newcommand{\ceil}[1]{\left\lceil#1\right\rceil}
\renewcommand{\qed}{\hfill\qedsymbol}
\renewcommand{\ip}[2]{\langle #1, #2 \rangle}
\newcommand{\seq}[2]{\qty(#1_#2)_{#2=1}^{\infty}}

% Sets
\DeclarePairedDelimiterX\set[1]\lbrace\rbrace{\def\given{\;\delimsize\vert\;}#1}

% End of preamble
%==========================================================================================%

% Start of commands specific to this file
%==========================================================================================%

%==========================================================================================%
% End of commands specific to this file

\title{Math 441 HW3}
\date{\today}
\author{Rohan Mukherjee}

\begin{document}
	\maketitle
	\begin{enumerate}[leftmargin=\labelsep]
		\item Notice that $(A \times B)^c = \set{(x,y) \in X \times Y \given (x,y) \not \in A \times B} = \set{(x, y) \in X \times Y \given x \not \in A \lor y \not \in B} = \set{(x, y) \in X \times Y \given x \in A^c} \cup \set{(x, y) \in X \times Y \given y \in B^c} = (A^c \times Y) \cup (X \times B^c)$. Since $A$ is closed $A^c$ is open, and since $B$ is closed $B^c$ is open. Thus the above is the union of two open sets (by definition of the product topology) and hence open, and hence $A \times B$ is closed.
		
		\item \begin{enumerate}
			\item We recall that an arbitrary intersection of closed sets is closed. Since $A \subset B \subset \overline{B}$, $\overline{B}$ is a closed set containing $A$, and hence $\overline{A} \subset \overline{B}$ by definition of intersection.
			\item Notice that $A \cup B \subset \overline{A} \cup \overline{B}$, thus $\overline{A \cup B} \subset \overline{\overline{A} \cup \overline{B}}$. Notice also that if $C$ is closed, then $C$ will be one of the sets in the intersection of all closed sets containing $C$, and thus $C = \overline{C}$, hence $\overline{\overline{A} \cup \overline{B}} = \overline{A} \cup \overline{B}$ as the finite union of closed sets is closed. For the reverse direction, notice that $A \subset A \cup B$, thus by part (a) $\overline{A} \subset \overline{A \cup B}$. Similarly, $\overline{B} \subset \overline{A \cup B}$, and $\overline{A \cup B} = \overline{A} \cup \overline{B}$.
			\item By the above, since $A_\alpha \in \bigcup_{\alpha \in \Delta} A_\alpha$, we have that $\overline{A_\alpha} \subset \overline{\bigcup_{\alpha \in \Delta} A_\alpha}$, and since this holds for all $\alpha \in \Delta$, we have that $\bigcup_{\alpha \in \Delta} \overline{A_\alpha} \subset \overline{\bigcup_{\alpha \in \Delta} A_\alpha}$. A clean counterexample is as follows: $[1/n, 1]$ is closed in the standard topology of $\R$ for all $n \in \bN$, thus $\overline{[1/n, 1]} = [1/n, 1]$ by the above, yet
			\begin{align*}
				\bigcup_{n \in \bN} [1/n, 1] = (0, 1]
			\end{align*}
			who's closure is $[0, 1]$, which is strictly larger than the LHS.
		\end{enumerate}
	
		\item Given two distinct points $(x_1, y_1)$ and $(x_2, y_2)$ in $X \times Y$, since $X$ is Hausdorff there exist an open $X_1 \subset X$ with $x_1 \in X_1$ and another open $X_2 \subset X$ with $x_2 \in X_2$ and $X_1 \cap X_2 = \emptyset$. Thus $(x_1, y_1) \in X_1 \times Y$ and also $(x_2, y_2) \in X_2 \times Y$. Finally, $(X_1 \times Y) \cap (X_2 \times Y) = (X_1 \cap X_2) \times Y = \emptyset \times Y = \emptyset$, and hence we are done, as both $X_1 \times Y$ and $X_2 \times Y$ are open in the product topology. We remark that we never used the condition that $Y$ is Hausdorff.
		
		\item Suppose that $X$ is Hausdorff. Note that $\Delta^c = \set{(x, y) \in X \times X \given x \neq y}$, and take any point $(x, y) \in \Delta^c$. By the Hausdorff condition there exists a $U_1 \subset X$ so that $x \in U_1$ and $U_2 \subset X$ so that $y \in U_2$ with $U_1 \cap U_2 = \emptyset$. I claim that $U_1 \times U_2 \subset \Delta^c$. Indeed, the condition $U_1 \cap U_2 = \emptyset$ shows that the $x$ and $y$ values of this product are never equal. By the union lemma $\Delta^c$ is open.
		
		For the reverse direction, suppose that $\Delta^c$ is open, and let $x \neq y$ both in $X$. Since the product topology has basis
		\begin{align*}
			\mathscr{B} = \set{U \times V \given U \subset X \text{ open}, \; V \subset X \text{ open}}
		\end{align*}
		we may write $\Delta^c = \bigcup U_\alpha \times V_\alpha$. Since $(x, y) \in \Delta^c$, $(x, y) \in U \times V$ for some $U, \; V$ both open in $X$. Notice that $U \cap V = \emptyset$, since if there was some point $x \in U \cap V$, then $(x, x) \in U \times V$, which would say $(x, x) \in \Delta^c$, which can't be. Thus $X$ is Hausdorff.
		
		\item \begin{enumerate}
			\item Suppose there was an $a \in A^{\circ} \cap \partial A$. Since $A^{\circ}$ is open, we have found an open neighborhood of $a$ fully contained in $A$, which contradicts the fact that $a$ is a boundary point of $A$. Thus $A^{\circ} \cap \partial A = \emptyset$. Second, we recall that $\overline{A} = A \cup \partial A$. I claim that $A^{\circ} \cup \partial A = A \cup \partial A$. The forward direction is clear. Given any $a \in A \cup \partial A$, either $a \in A$, or $a \in \partial A$. The second case is trivial. Now, either there is an open neighborhood $U$ of $a$ so that $U \subset A$, or every open neighborhood $U$ of $a$ intersects $A^c$. In the first case, $U$ is an open neighborhood contained in $A$, thus $U \subset A^{\circ}$ by definition. The second case states precisely that $a$ is a boundary point of $A$ (Note: any open neighborhood $U$ of $a$ intersects $A$ since $a \in A$), thus $a \in \partial A$, and we are done.
			
			\item I showed above $A = \overline{A}$ iff $A$ is closed, and similarly, if $A$ is open, then $A$ is an open set contained in $A$, thus $A^{\circ} \supset A$. The reverse direction is by definition, hence $A = A^{\circ}$. Thus $A = A^{\circ}$ iff $A$ is open (again, the interior is clearly open). If $\partial A = \emptyset$, then $\overline{A} = A^{\circ}$ by part (a). Since $A^{\circ} \subset A \subset \overline{A} = A^{\circ}$, we have $A = A^{\circ} = \overline{A}$, which shows that $A$ is clopen. For the reverse direction, note that if $A$ is clopen then $A = \overline{A} = A^{\circ}$, and hence $A^{\circ} = \overline{A} = A^{\circ} \cup \partial A$. Thus, $\emptyset = A^\circ \cap \partial A = (A^{\circ} \cup \partial A) \cap \partial A = (A^\circ \cap \partial A) \cup (\partial A \cap \partial A) = \emptyset \cup \partial A = \partial A$.
			
			\item Since $A$ is open, by the above $\overline{A} = A^\circ \cup \partial A = A \cup \partial A$. Since $A^\circ \cap \partial A = A \cap \partial A = \emptyset$, we have that $\partial A \subset A^c$, thus $(A \cup \partial A) \setminus A = (A \cup \partial A) \cap A^c = (A \cap A^c) \cup (\partial A \cap A^c) = \partial A \cap A^c = \partial A$, thus $\overline{A} \setminus A = \partial A$. For the reverse direction, if $\partial A = \overline{A} \setminus A$, let $x \in A$. By construction $x \not \in \partial A$. This says that there is a neighborhood $U$ of $x$ so that $V \cap A = \emptyset$ or $U \cap A^c = \emptyset$. The first condition is obviously false, since $x \in A$, thus $U \cap A^c = \emptyset$, or $U \subset A$. By the union lemma $A$ is open, and we are done.
			
			\item This is not true. Consider $X = \set{1, 2}$ with the topology $\mathcal{T} = \set{\emptyset, \set{1}, \set{1, 2}}$. It suffices to prove finite intersection on two sets. So let $U, V \in \mathcal{T}$. If either is empty, their intersection is empty. If neither are empty, they are either both $\set{1}$, both $\set{1, 2}$, or one is $\set{1}$ and the other is $\set{1, 2}$. In the first and last case the intersection is $\set{1}$, and in the middle case the intersection is $\set{1, 2}$. For arbitrary union, either one of the sets is $\set{1, 2}$, or all are not $\set{1, 2}$. In the first case the union is $\set{1, 2}$, in the second case, if all sets are the empty set, the union is the empty set. Else, at least one is $\set{1}$, thus the union is $\set{1}$, since it can't be any bigger by the last sentence. Now, the only closed sets are $\set{\emptyset, \set{2}, \set{1, 2}}$. The only one of those containing $\set{1}$ is $\set{1, 2}$, thus $\overline{\set{1}} = \set{1, 2}$. Since $\set{1, 2}$ is open, $\set{1, 2}^\circ = \set{1, 2}$. This is strictly bigger than $\set{1}$, so we have disproven the claim.
		\end{enumerate}
	
		\item The constant function 1 is never 0, thus $V(1) = \emptyset$. The constant function 0 is always 0, so $V(0) = \R^n$. If $x \in V(f) \cap V(g)$, then $f(x) = 0$ and $f(g) = 0$. Thus $x \in V(f, g)$. If $x \in V(f, g)$, then $f(x) = 0$ and $g(x) = 0$, thus $x \in V(f)$ and $x \in V(g)$, hence $x \in V(f) \cap V(g)$. Finally, if $x \in V(f) \cup V(g)$, then $f(x) = 0$ or $g(x) = 0$. WLOG the first case is true, thus $(f \cdot g)(x) = f(x) g(x) = 0 \cdot g(x) = 0$. If $(f \cdot g)(x) = 0$, then since $\R$ is an integral domain $f(x) = 0$ or $g(x) = 0$. Thus $x \in V(f) \cup V(g)$.
		
		\item 
		One recalls that nonconstant polynomials of finite degree in one variable have only finitely many roots. Thus, for any polynomial $f \in \R[x]$, we have three cases: $V(f) = \R$, $V(f) = \emptyset$, or $V(f)$ is a finite set. I claim that $V(x^2+y^2-1)$, the circle, is not closed in $\R \times \R$. First we shall show that $V(x^2+y^2-1) \not \subset V(f) \times V(g)$, unless $V(f) = V(g) = \R$. WLOG $V(f) \neq \R$. If $V(f) = \emptyset$, we are done, since the product of the empty set with anything is empty. Else, $V(f)$ is finite. The circle has uncountably many points at different $x$-values, thus will most certainly have a point with $x$ value not in $V(f)$. Now, suppose that $V(x^2+y^2-1) = \bigcap_{f_\alpha, g_\alpha \in \Delta} V(f_\alpha) \times V(g_\alpha)$. If all $V(f_\alpha) \times V(g_\alpha) = \R^2$, we certainly don't have equality, so there exists one $V(f) \times V(g) \neq \R^2$. This would say that $V(x^2+y^2-1) \subset V(f) \times V(g)$, with not both $\R$, which we proved above impossible. In the general case where we have $V(T) \times V(G)$, we could indeed run the same argument and say there must be one that is not all of $\R^2$, which then $V(x^2+y^2-1) \subset V(T) \times V(G) \subset V(f) \times V(g)$ for some $f \in T$ and $g \in G$ not both equivalently 0 but we proved that false. Notice that $V(T) \cup V(G) = \bigcap_{f \in T} V(f) \cup \bigcap_{g \in G} V(g) = \bigcap_{f \in T, g \in G} V(f) \cup V(g) = \bigcap_{f \in T, g \in G} V(f) \cup V(g) = \bigcap_{f \in T, g \in G} V(fg)$, and if we let $U = \set{fg \given f \in T, g \in G}$, we would have this union equal to $V(fg)$. Thus finite union may be expressed as another $V(T)$. So if we had $V(x^2+y^2-1)$ = finite union, we would have it equal to $V(T)$, which we showed above impossible (from the arbitrary intersection).
	\end{enumerate}
\end{document}
