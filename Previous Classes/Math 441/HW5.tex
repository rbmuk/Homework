\documentclass[12pt]{article}
\usepackage[margin=1in]{geometry}

% Start of preamble
%==========================================================================================%
% Required to support mathematical unicode
\usepackage[warnunknown, fasterrors, mathletters]{ucs}
\usepackage[utf8x]{inputenc}

\usepackage[dvipsnames,table,xcdraw]{xcolor} % colors
\usepackage{hyperref} % links
\hypersetup{
	colorlinks=true,
	linkcolor=blue,
	filecolor=magenta,      
	urlcolor=cyan,
	pdfpagemode=FullScreen
}

% Standard mathematical typesetting packages
\usepackage{amsmath,amssymb,amscd,amsthm,amsxtra, pxfonts}
\usepackage{mathtools,mathrsfs,dsfont,xparse}

% Symbol and utility packages
\usepackage{cancel, textcomp}
\usepackage[mathscr]{euscript}
\usepackage[nointegrals]{wasysym}
\usepackage{apacite}

% Extras
\usepackage{physics}  % Lots of useful shortcuts and macros
\usepackage{tikz-cd}  % For drawing commutative diagrams easily
\usepackage{microtype}  % Minature font tweaks
\usepackage{braket}

\usepackage{enumitem}
\usepackage{titling}

\usepackage{graphicx}

% Fancy theorems due to @intuitively on discord
\usepackage{mdframed}
\newmdtheoremenv[
backgroundcolor=NavyBlue!30,
linewidth=2pt,
linecolor=NavyBlue,
topline=false,
bottomline=false,
rightline=false,
innertopmargin=10pt,
innerbottommargin=10pt,
innerrightmargin=10pt,
innerleftmargin=10pt,
skipabove=\baselineskip,
skipbelow=\baselineskip
]{mytheorem}{Theorem}

\newenvironment{theorem}{\begin{mytheorem}}{\end{mytheorem}}

\newtheorem{corollary}{Corollary}
\newtheorem{lemma}{Lemma}

\newtheoremstyle{definitionstyle}
{\topsep}%
{\topsep}%
{}%
{}%
{\bfseries}%
{.}%
{.5em}%
{}%
\theoremstyle{definitionstyle}
\newmdtheoremenv[
backgroundcolor=Violet!30,
linewidth=2pt,
linecolor=Violet,
topline=false,
bottomline=false,
rightline=false,
innertopmargin=10pt,
innerbottommargin=10pt,
innerrightmargin=10pt,
innerleftmargin=10pt,
skipabove=\baselineskip,
skipbelow=\baselineskip,
]{mydef}{Definition}
\newenvironment{definition}{\begin{mydef}}{\end{mydef}}

\newtheorem*{remark}{Remark}

\newtheorem*{example}{Example}

% Common shortcuts
\def\mbb#1{\mathbb{#1}}
\def\mfk#1{\mathfrak{#1}}

\def\bN{\mbb{N}}
\def \C{\mbb{C}}
\def \R{\mbb{R}}
\def\bQ{\mbb{Q}}
\def\bZ{\mbb{Z}}
\def \cph{\varphi}
\renewcommand{\th}{\theta}
\def \ve{\varepsilon}
\newcommand{\mg}[1]{\| #1 \|}

% Often helpful macros
\newcommand{\floor}[1]{\left\lfloor#1\right\rfloor}
\newcommand{\ceil}[1]{\left\lceil#1\right\rceil}
\renewcommand{\qed}{\hfill\qedsymbol}
\renewcommand{\ip}[2]{\langle #1, #2 \rangle}
\newcommand{\seq}[2]{\qty(#1_#2)_{#2=1}^{\infty}}

% End of preamble
%==========================================================================================%

% Start of commands specific to this file
%==========================================================================================%

\renewcommand{\S}{\mbb S}

%==========================================================================================%
% End of commands specific to this file

\title{Math 441 Final HW}
\date{\today}
\author{Rohan Mukherjee}

\begin{document}
	\maketitle
	\begin{enumerate}[leftmargin=\labelsep]
		\item We recall that $\S^n \coloneqq \set{x \in \R^n | |x| = 1}$. I claim then that $\S^n \subset [-1,1]^n$. Indeed, if $x \not \in [-1,1]^n$, then $x$ has a coordinate who's value is strictly greater in absolute value than 1. Thus its square is strictly greater than 1, contradicting the fact that $|x| = 1$. $[-1,1] \cong [0, 1]$ via the homeomorphism $x \to x/2 + 1/2$, which is a homeomorphism since it is a linear bijection (and linear maps are continuous and have continuous inverses). Since the product of compact sets are compact, $[-1,1]^n$ is compact. I claim that $\S^n$ is closed. Indeed, if $\S^n \ni x_n \to x \in \R^n$, we have that 
		\begin{align*}
			\mg{x_n} = 1 \text{ for all $n \in \bN$}
		\end{align*}
		Thus, since $x \mapsto \mg{x}$ is continuous, we have that
		\begin{align*}
			1 = \lim_{n \to \infty} \mg{x_n} = \mg{\lim_{n \to \infty} x_n} = \mg{x}
		\end{align*}
		Which shows that $\S^n$ is closed, thus $\S^n$ is compact as it is a closed subset of a compact space. Similarly, we recall that $\R \mbb P^n \coloneqq \S^n / \sim$, where $x \sim -x$. The quotient topology is defined so that the projection is a quotient map--in particular, it is continuous (and onto). Thus $\R \mbb P^n$ is the continuous image of a compact set, and hence compact.
		
		\item We shall construct the topology in the most obvious way: consider the topology generated by the base containing elements of the form
		\begin{align*}
			B = \Set{(a_{ij})_{1 \leq i \leq m, \; 1 \leq j \leq n} | a_{ij} \in U_{ij} \subset \R, \; U_{ij} \text{ open }}
		\end{align*}
		First we need to show that this is indeed a base. It is clear that if we take each $U_{ij} = \R$, then we would get the entire space, thus the basis elements cover $\mathrm{Mat}(m \times n;\; \R)$. Next notice that given $B_1 = \Set{(a_{ij}) | a_{ij} \in U_{ij} \subset \R, \; U_{ij} \text{ open }}$ and $B_2 = \Set{(a_{ij}) | a_{ij} \in B_{ij} \subset \R, \; B_{ij} \text{ open }}$, we have that
		\begin{align*}
			B_1 \cap B_2 = \Set{(a_{ij}) | a_{ij} \in U_{ij} \cap V_{ij}}
		\end{align*}
		And since $U_{ij} \cap V_{ij}$ is always the finite intersection of open sets, it too is open. Thus $B_1 \cap B_2$ is another element in the base, and we are done with that part. Now consider the ``identity'' map $(a_{ij}) \mapsto (a_{11}, a_{21}, \ldots, a_{mn})$. This map is obviously a bijection. If $\prod_{i=1}^{mn} U_i$ is a subbasis element for $\R^{mn}$, pulling back would give us the set 
		\begin{align*}
			\set{(a_{ij}) | a_{ij} \in U_{ij}}
		\end{align*}
		(where the ordering and column/row jumping is clear) which is of course open since it is a basis element. Now, if $B = \Set{(a_{ij}) | a_{ij} \in U_{ij} \subset \R, \; U_{ij} \text{ open }}$ is a subbasis element for the topology we just constructed, it would get mapped to $\prod_{1 \leq i \leq m, 1 \leq j \leq n} U_{ij}$, which is indeed open in $\R^{mn}$. A bjiective open map is a homeomorphism, thus we are done.
		
		\item \begin{enumerate}
			\item We use the fact from linear algebra that the determinant is a continuous map from $\mathrm{Mat}(n \times n;\; \R)$ to $\R$. Indeed, the determinant is a polynomial in the entries of the matrix, and since we equipped the matrix with the topology of (effectively) $\R^{n^2}$, this map is going to be continuous because the same polynomial from $\R^{n^2}$ to $\R$ is continuous. Each matrix in $\mathrm{GL_n}(\R)$ has nonzero determinant, and matrices with 0 determinant are not invertible. Thus $\mathrm{GL_n}(\R) = \det^{-1}(\R \setminus 0)$, and since $\R \setminus 0 = (-\infty, 0) \cup (0, \infty)$ is the union of two open sets, it is open thus  $\mathrm{GL_n}(\R)$ is open. 
			
			\item Furthermore, restricting domain preserves continuity, thus $\mathrm{det}: \mathrm{GL_n}(\R) \to \R$ is continuous. If $\mathrm{GL_n}(\R)$ were connected, then it's image under a continuous map would be connected. However, its image is $\R \setminus 0$, as for any $x \neq 0$, the matrix with 1's along the main diagonal with the first 1 replaced by $x$ has determinant equal to $x$, and 0 is not the determinant of any invertible matrix, by definition, and since $\R \setminus 0$ is clearly not connected, we have reached a contradiction.
		\end{enumerate}
		\item \begin{enumerate}
			\item Given the Euclidean topology on $\C$, we recall that $\R^2 \setminus K$ where $K$ is countable is path-connected from in class. Call $\iota$ the prescribed map. Since $\set{\iota(\alpha_1), \ldots, \iota(\alpha_k)}$ is countable, we have that $\R^2 \setminus \set{\iota(\alpha_1), \ldots, \iota(\alpha_k)}$ is path-connected. Since we defined the topology on $\C$ to be so that $\iota$ is a homeomorphism, and since the restriction of a homeomorphism is a homeomorphism, $\iota(\R^2 \setminus \set{\iota(\alpha_1), \ldots, \iota(\alpha_k)}) = \C \setminus \set{\alpha_1, \ldots, \alpha_k}$ is path-connected. We remark this can also be used to show that $\C \setminus K$ where $K \subset \C$ is countable is path-connected.
			
			\item $\det(A + z(I-A))$ is a polynomial in $z$, that is not equivalently 0 (since we showed it is nonzero somewhere). By the fundamental theorem of algebra it has finitely many roots, say $\set{\alpha_1, \ldots, \alpha_k}$. Now, since $\C \setminus \set{\alpha_1, \ldots, \alpha_k}$ is path-connected, we can find a path $\gamma: [0, 1] \to \C \setminus \set{\alpha_1, \ldots, \alpha_k}$ with $\gamma(0) = 0$, $\gamma(1) = 1$, and $P(\gamma(t)) \neq 0$ for all $t \in [0,1]$, since $0, 1 \in \C \setminus \set{\alpha_1, \ldots, \alpha_k}$ (we showed they were not roots), and since $\gamma$ maps to the subset of $\C$ precisely defined as such so that it excludes the roots of $P$.
			
			\item Since $A + z(I-A)$ is clearly a continuous map, and since $\gamma(t)$ is continuous, $A + \gamma(t)(I-A)$ is continuous. We see immediately that $A + \gamma(0)(I-A) = A$, and $A + \gamma(1)(I-A) = I$. By the gluing lemma, given any matrix $A, B$, we could glue together two of these paths to get a path from $A$ to $I$ then to $B$, thus $\mathrm{GL_n}(\C)$ is path-connected, and we are done.
		\end{enumerate}
	\end{enumerate}
\end{document}
