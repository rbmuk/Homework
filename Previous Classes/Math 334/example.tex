\documentclass[12pt]{article}
\usepackage[margin=1in]{geometry}

% Start of preamble
%==========================================================================================%
% Required to support mathematical unicode
\usepackage[warnunknown, fasterrors, mathletters]{ucs}
\usepackage[utf8x]{inputenc}

% Always typeset math in display style
\everymath{\displaystyle}

% Standard mathematical typesetting packages
\usepackage{amsfonts, amsthm, amsmath, amssymb}
\usepackage{mathtools}  % Extension to amsmath

% Symbol and utility packages
\usepackage{cancel, textcomp}
\usepackage[mathscr]{euscript}
\usepackage[nointegrals]{wasysym}

% Extras
\usepackage{physics}  % Lots of useful shortcuts and macros
\usepackage{tikz-cd}  % For drawing commutative diagrams easily
\usepackage{color}  % Add some color to life
\usepackage{microtype}  % Minature font tweaks
%\usepackage{pgfplots} % plots

\usepackage{enumitem}
\usepackage{titling}

% Common shortcuts
\def\mbb#1{\mathbb{#1}}
\def\mfk#1{\mathfrak{#1}}

\def\bN{\mbb{N}}
\def\bC{\mbb{C}}
\def\bR{\mbb{R}}
\def\bQ{\mbb{Q}}
\def\bZ{\mbb{Z}}

% Sometimes helpful macros
\newcommand{\func}[3]{#1\colon#2\to#3}
\newcommand{\vfunc}[5]{\func{#1}{#2}{#3},\quad#4\longmapsto#5}
\newcommand{\floor}[1]{\left\lfloor#1\right\rfloor}
\newcommand{\ceil}[1]{\left\lceil#1\right\rceil}

% Some standard theorem definitions
\newtheorem{theorem}{Theorem}[section]
\newtheorem{corollary}{Corollary}[theorem]
\newtheorem{lemma}[theorem]{Lemma}

\theoremstyle{definition}
\newtheorem{definition}{Definition}[section]

\theoremstyle{remark}
\newtheorem*{remark}{Remark}

% End of preamble
%==========================================================================================%

% Start of commands specific to this file
%==========================================================================================%

\newcommand{\R}{\mathbb{R}}
\renewcommand{\ip}[2]{\langle #1, #2 \rangle}
\newcommand{\mg}[1]{\| #1 \|}
\newcommand{\linf}[1]{\max_{1\leq i \leq #1}}
\newcommand{\ve}{\varepsilon}

%==========================================================================================%
% End of commands specific to this file

\title{Math 334 HW 2}
\date{\today}
\author{Rohan Mukherjee}

\begin{document}
	\maketitle
	\begin{enumerate}[leftmargin=\labelsep]
		\item 
		\begin{lemma}
			$\R$ is open.
		\end{lemma}
		\begin{proof}
			Let $x \in \R$. Choose $r = 1$. Note that
			\begin{align*}
				B(x, r) \subset \R
			\end{align*}
		because everything in $B(x, r)$ is a real number. So $x$ is an interior point, and $R$ is open.
		\end{proof}
		Let $S_i = [-i, i], i \in \mathbb{N}$. Note that
		\begin{align*}
			\bigcap_{i=2}^{\infty} \left[\frac1i, 1-\frac1i\right] = (0, 1)
		\end{align*}
		and that each $\left[\frac1i, 1-\frac1i\right]$ is closed. 
		
		Let $\bigcup_{i=1}^{\infty} G_i$ be an infinite union of open sets. Note that if $x \in \bigcup_{i=1}^{\infty} G_i$, then $x \in G_j$, for some $j \in \mathbb{N}$. Then, because $G_j$ is open, $x$ is an interior point, and we see that everything in this infinite union is an interior point.
		
		Let $\bigcap_{i=1}^{n} H_i$ be a finite intersection of closed sets. Consider the complement of this set, i.e. $\bigcup_{i=1}^{n} H_i^C$. Note that each $H_i^C$ is open (complement of a closed set is open). The proof above showing that an infinite union of open sets is open also works in the finite case, so we see that this finite intersection of open sets is open. Then its complement is closed, i.e. $\bigcap_{i=1}^{n} H_i$ is closed. Consider $\bigcap_{i=2}^{\infty} \left[\frac1i, 1-\frac1i\right] = (0, 1)$, and note that each $\left[\frac1i, 1-\frac1i\right]$ is closed. 
		
		\item 
		First, we define another type of continuity, something we will call $\infty$ continuity:
		\begin{definition}
			A function $f: \R^n \to \R^m$ is $\infty$-continuous if $\forall \varepsilon > 0, \exists \delta > 0$ such that $\max_{1\leq i \leq n} |x_i-y_i| < \delta \implies \linf{m} |f(x)_i-f(y)_i| < \varepsilon$.
			\end{definition}
		\begin{lemma}
			For all $x \in \R^n$, $\linf{n} |x_i| \leq \|x\|_p \leq \sqrt[p]{n} \linf{n} |x_i|$.
		\end{lemma}
		\begin{proof}
			Let $x \in \R^n$. 
			\begin{align*}
				\linf{n} |x_i| = \sqrt[p]{\left(\linf{n} |x_i|\right)^p} \leq \sqrt[p]{|x_1|^p+|x_2|^p + \ldots + |x_n|^p} = \mg{x}_p
			\end{align*}
			The inequality is true because we are just adding a bunch of positive things under the p-th root, which will definitely make the number bigger. Next, because every $x_i \leq \linf{n} |x_i|$, we have that
			\begin{align*}
				\mg{x}_p \leq \sqrt[p]{n \cdot \left(\linf{n} |x_i|\right)^p} = \sqrt[p]{n} \linf{n} |x_i|
			\end{align*}
		\end{proof}
		We now claim that being $p$-continuous is equivalent to being $\infty$-continuous.
		\begin{proof}
		    \fbox{$\implies$}
			
			Suppose that $f: \R^n \to \R^m$ is $\infty$-continuous, and let $\ve > 0$. Choose $\delta$ so that $\forall y \in \R^m$ with $\linf{n} |x_i-y_i| < \delta$, we have that $\linf{n} |f(x)_i - f(y)_i| < \ve / \sqrt[p]{n}$. Then for all $y \in \R^m$ with $\mg{x-y}_p < \delta$, we also know that $\linf{n}|x_i-y_i| < \delta$ by Lemma 0.2. Now note that $\mg{f(x)-f(y)}_p \leq \sqrt[p]{n}\linf{m} |f(x)_i -f(y)_i| < \sqrt[p]{n} \cdot \frac{\ve}{\sqrt[p]{n}}=\ve$. 	
			
			\fbox{$\impliedby$}
			
			Suppose now that $f:\R^n \to \R^m$ is $p$-continuous. Choose $\delta$ so that $\forall y \in \R^n$ with $\mg{x-y}_p < \sqrt[p]{n}\delta \implies \mg{f(x)-f(y)}_p < \ve$. Now note that $\forall y \in \R^n$ with $ \linf{n}|x_i-y_i| < \delta$, we have that $\mg{x-y}_p < \sqrt[p]{n}\delta$ and so $\linf{n} |f(x)_i-f(y)_i| \leq \mg{x-y}_p < \ve$.
		\end{proof}
		Let $1\leq p < \infty$. By the above, we have that $2$-continuous $\iff$ $\infty$-continuous $\iff$ $p$-continuous. Now let $1 \leq p,q < \infty$. By the above again, we see that $p$-continuous $\iff$ $\infty$-continuous $\iff$ $q$-continuous.
		\end{enumerate}
		
\end{document}