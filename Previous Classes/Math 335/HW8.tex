\documentclass[12pt]{article}
\usepackage[margin=1in]{geometry}

% Start of preamble
%==========================================================================================%
% Required to support mathematical unicode
\usepackage[warnunknown, fasterrors, mathletters]{ucs}
\usepackage[utf8x]{inputenc}

% Always typeset math in display style
%\everymath{\displaystyle}

% GROUPOIDS FONT!
%\usepackage{eulervm}
%\usepackage{charter}

% Standard mathematical typesetting packages
\usepackage{amsthm, amsmath, amssymb}
\usepackage{mathtools}  % Extension to amsmath

% Symbol and utility packages
\usepackage{cancel, textcomp}
\usepackage[mathscr]{euscript}
\usepackage[nointegrals]{wasysym}

% Extras
\usepackage{physics}  % Lots of useful shortcuts and macros
\usepackage{tikz-cd}  % For drawing commutative diagrams easily
\usepackage{color}  % Add some color to life
\usepackage{microtype}  % Minature font tweaks
%\usepackage{pgfplots} % plots

\usepackage{enumitem}
\usepackage{titling}

\usepackage{graphicx}

% Common shortcuts
\def\mbb#1{\mathbb{#1}}
\def\mfk#1{\mathfrak{#1}}

\def\bN{\mbb{N}}
\def\bC{\mbb{C}}
\def\bR{\mbb{R}}
\def\bQ{\mbb{Q}}
\def\bZ{\mbb{Z}}

% Sometimes helpful macros
\newcommand{\floor}[1]{\left\lfloor#1\right\rfloor}
\newcommand{\ceil}[1]{\left\lceil#1\right\rceil}
\DeclarePairedDelimiterX\set[1]\lbrace\rbrace{\def\given{\;\delimsize\vert\;}#1}

% Some standard theorem definitions
\newtheorem{theorem}{Theorem}[section]
\newtheorem{corollary}{Corollary}[theorem]
\newtheorem{lemma}[theorem]{Lemma}

\theoremstyle{definition}
\newtheorem{definition}{Definition}[section]

\theoremstyle{remark}
\newtheorem*{remark}{Remark}

% End of preamble
%==========================================================================================%

% Start of commands specific to this file
%==========================================================================================%

\newcommand{\R}{\mathbb{R}}
\renewcommand{\ip}[2]{\langle #1, #2 \rangle}
\newcommand{\mg}[1]{\| #1 \|}
\newcommand{\linf}[1]{\max_{1\leq i \leq #1}}
\newcommand{\ve}{\varepsilon}
\renewcommand{\qed}{\hfill\qedsymbol}
\newcommand{\seq}[2]{\qty(#1_#2)_{#2=1}^{\infty}}
\newcommand{\vp}{\varphi}

%==========================================================================================%
% End of commands specific to this file

\title{Math 335 HW8}
\date{\today}
\author{Rohan Mukherjee}

\begin{document}
	\maketitle
	\begin{enumerate}[leftmargin=\labelsep]
		\item We first note that 
		\begin{align*}
			\int_{\partial S} -ydx = \int_{\partial S} \begin{pmatrix}
				-y \\ 0
			\end{pmatrix} \cdot d\textbf{x}
		\end{align*}
		We can parameterize the boundary by 4 curves, but one must note that this parameterization traces clockwise. So we must multiply by a $-1$ in the integral, which actually ends up making things nicer.
		\begin{align*}
			&\gamma_1(t) = (a, t) &\quad 0 \leq t \leq f(a) \\
			&\gamma_2(t) = (t, f(t)) &\quad  a \leq t \leq b \\
			&\gamma_3(t) = (b, f(b)-t) &\quad  0 \leq t \leq f(b) \\
			&\gamma_4(t) = (b-t, 0) &\quad  0 \leq t \leq b-a
		\end{align*}
		We wish then to evalute the integral
		\begin{align*}
			\int_{- \partial S} \begin{pmatrix}
				y \\ 0
			\end{pmatrix} \cdot d\textbf{x}
		\end{align*}
		where I put a $-$ in front of $\partial S$ to signify that we are going clockwise (this is like swapping the bounds of the integral). By our parameterization above, we get that this integral equals
		\begin{align*}
			\int_0^{f(a)} 
			\begin{pmatrix}
				t \\ 0
			\end{pmatrix} \cdot \begin{pmatrix}
			0 \\ 1
			\end{pmatrix}dt + 
			\int_a^b \begin{pmatrix}
			f(t)\\ 0
			\end{pmatrix} \cdot \begin{pmatrix}
			1 \\ f'(t)
			\end{pmatrix}dt +
			\int_0^{f(b)} 
			\begin{pmatrix}
				f(b)-t \\ 0
			\end{pmatrix} \cdot 
			\begin{pmatrix}
				0 \\ -1
			\end{pmatrix}dt \\
			+ \int_0^{b-a} \begin{pmatrix}
				0 \\ 0
			\end{pmatrix} \cdot 
			\begin{pmatrix}
			-1 \\ 0
			\end{pmatrix}dt
		\end{align*}
		A keen eye notices that the first, third, and fourth integral equal 0 (the dot product ends up being 0 in all of those cases). The resultant integral is precisely 
		\begin{align*}
			\int_a^b f(t)dt
		\end{align*}
		which is exactly what we wanted to show.
		
		\item By noting that $\pdv{g}{n} = \grad g \cdot n$, where $n$ is the outward normal vector to the surface, we see that 
		\begin{align*}
			\int_{\partial S} f \pdv{g}{n}ds &= \int_{\partial S} f \cdot \grad g \cdot n ds \\
			&= \int_{\partial S} 
			\begin{pmatrix}
				f \cdot \pdv{g}{x} \\
				f \cdot \pdv{g}{y}
			\end{pmatrix} \cdot n ds
		\end{align*}
		By Corollary 5.17 in the book, we see that this integral equals
		\begin{align*}
			\int_{S} \pdv{x} f \cdot \pdv{g}{x} + \pdv{y} f \cdot \pdv{g}{y} dA \\ &= \int_S \pdv{f}{x} \pdv{g}{x} + \pdv[2]{g}{x} f + \pdv{f}{y}\pdv{g}{y} + \pdv[2]{g}{y} fdA
			\\ &= \int_S f\qty(\pdv[2]{g}{x} + \pdv[2]{g}{y}) + \begin{pmatrix}
				\pdv{f}{x} \\
				\pdv{f}{y}
			\end{pmatrix} \cdot \begin{pmatrix}
			\pdv{g}{x} \\
			\pdv{g}{y}
		\end{pmatrix}dA \\
	&= \int_S f\qty(\pdv[2]{g}{x}+\pdv[2]{g}{y}) + \grad{f} \cdot \grad{g}dA
		\end{align*}	
		which completes the proof.
		
		
		\item We parameterize the ellipsoid by 
		\begin{align*}
			G(\theta, \vp) = (a\sin(\phi)\cos(\theta), a\sin(\phi)\sin(\theta), b\cos(\phi))
		\end{align*}
		This is just spherical coordinates but scaled up a little in two directions. Plugging this into the equation $x^2/a^2+y^2/a^2+z^2/b^2=1$ verifies that it works (it is also spherical coordinates). We then see clearly that $\theta \in [0, 2\pi]$, and that $\phi \in [0, \pi]$, by simply comparing this to spherical--we want to be integrating over the entire shape, but this time it is an ellipsoid. This does not change the angles. A very long calculation, which I have done but will not type up, shows that
		\begin{align*}
			\| \pdv{G}{\theta}  \cross \pdv{G}{\vp} \| = a\sin(\vp)\sqrt{a^2\cos^2(\phi)+b^2\sin^2(\phi)}
		\end{align*}
		Our surface area is therefore
		\begin{align*}
			\int_0^{2\pi} \int_0^{\pi} a\sin(\vp)\sqrt{a^2\cos^2(\phi)+b^2\sin^2(\phi)}d\phi &= 2\pi a \int_0^{\pi} \sin(\vp)\sqrt{a^2\cos^2(\phi)+b^2\sin^2(\phi)}d\phi 
		\end{align*}
		Letting $u = \cos(\vp)$, and noting that this transformation is bijective on $\vp \in [0, \pi]$, seeing that $\sin^2(\phi) = 1 - u^2$, and finally that $du = -\sin(\vp)d\vp$, we may conclude that
		\begin{align*}
			2\pi a \int_0^{\pi} \sin(\vp)\sqrt{a^2\cos^2(\phi)+b^2\sin^2(\phi)}d\vp  &= 2\pi \int_{-1}^1 \sqrt{b^2 + (a^2-b^2)u^2}du \\
			&= 2\pi ab \int_{-1}^1 \sqrt{1+\frac{a^2-b^2}{b^2}u^2}du
		\end{align*}
		Finally, noting  that (The proof of this theorem is left as a trivial exercise to the reader)
		\begin{align*}
			\int_{-1}^{1} \sqrt{1+cu^2}du = \sqrt{c+1}+\frac{\mathrm{sinh}^{-1}(\sqrt{c})}{\sqrt{c}}
		\end{align*}
		We conclude that this integral equals
		\begin{align*}
			2\pi ab \cdot \qty(\frac ab + \frac{b\cdot \mathrm{sinh}^{-1}(\sqrt{(a^2-b^2)/b^2})}{\sqrt{a^2-b^2}})
			&= 2\pi a^2 + \frac{2\pi a b^2}{\sqrt{a^2-b^2}} \mathrm{sinh}^{-1}\qty(\frac{\sqrt{a^2-b^2}}{b})
		\end{align*}
		
		
		\item Because the upper half of the unit sphere has radial symmetry, it must necessarily have a center of mass on the $z$-axis (If it was elsewhere, this would mean its heavier i.e. more volume around a point that's not the z-axis, which doesn't make any sense). So what's left is to calculate the center of mass z-coordinate. We do this by parameterizing the unit sphere by 
		\begin{align*}
			G(s, t) = (s, t, \sqrt{1-s^2-t^2})
		\end{align*}
		The projection of the upper half of the unit sphere is going to be $B(0, 1) \subset \R^2$, which is exactly what values $(s, t)$ can take on, as the point $(s, t)$ is the upper half of the unit sphere projected down. So $(s, t) \in B(0, 1)$.
		It is now clear that 
		\begin{align*}
			\pdv{G}{s} = \qty(1, 0, \frac{-s}{\sqrt{1-s^2-t^2}}) \\
			\pdv{G}{t} = \qty(0, 1, \frac{-t}{\sqrt{1-s^2-t^2}})
		\end{align*}
		Taking the cross product of these two gives us 
		\begin{align*}
			\pdv{G}{s} \cross \pdv{G}{t} = \begin{pmatrix}
				\frac{s}{\sqrt{1-s^2-t^2}} \\
				\frac{t}{\sqrt{1-s^2-t^2}} \\
				1
			\end{pmatrix}
		\end{align*}
		Who's magnitude is now
		\begin{align*}
			\sqrt{\frac{s^2+t^2+1-s^2-t^2}{1-s^2-t^2}} &= \frac{1}{\sqrt{1-s^2-t^2}}
		\end{align*}
		Where we did $1^2 = 1 = \frac{1-s^2-t^2}{1-s^2-t^2}$ and simplified.
		Our integral is now
		\begin{align*}
			\qty(\int_{\partial S} zdS) / \int_{\partial S} 1dS &= \int_{B(0, 1)} \frac{\sqrt{1-s^2-t^2}}{\sqrt{1-s^2-t^2}}dA / \int_{\partial S} 1dS \\
			&= \int_{B(0, 1)} 1dA / \int_{\partial S} 1dS \\
			&= \pi / \int_{\partial S} 1dS \\
			&= \pi / 2\pi \\
			&= \frac12
		\end{align*}
		Where I converted to polar (the bounds are obvious are we are integrating over the unit ball in $\R^2$). Note that we also showed in class that the surface area of the upper hemisphere of the unit sphere is $2\pi$. So the center of mass is going to be $(0, 0, 1/2)$, which is really nice, and even makes a lot of intuitive sense.
		
		\item Writing $f = f(x, y, z)$, and $g = g(x, y, z)$, we see that $\grad f \cross \grad g = \begin{pmatrix}
			\pdv{f}{y}\pdv{g}{z} - \pdv{g}{y}\pdv{f}{z} \\
			\pdv{g}{x}\pdv{f}{z} - \pdv{f}{x}\pdv{g}{z} \\
			\pdv{f}{x}\pdv{g}{y} - \pdv{g}{x}\pdv{f}{y}
		\end{pmatrix}$ just from the definition of the cross product. For a function $h(x, y, z)$, $\mathrm{div}(h) = \pdv{h}{x} + \pdv{h}{y} + \pdv{h}{z}$. We see that the derivative of the first coordinate of $\grad f \cross \grad g$ w.r.t. $x$ would give us (by the product rule)
		\begin{align*}
			\pdv{f}{xy}\pdv{g}{z} + \pdv{g}{xz}\pdv{f}{y} - \pdv{g}{yx}\pdv{f}{z} - \pdv{f}{zx}\pdv{g}{y}
		\end{align*}
		Differentiating the second coordinate w.r.t. $y$, also through the product rule, gives us
		\begin{align*}
			\pdv{g}{xy}\pdv{f}{z}+\pdv{f}{zy}\pdv{g}{x}-\pdv{f}{xy}\pdv{g}{z}-\pdv{g}{zy}\pdv{f}{x}
		\end{align*}
		Differentiating the third coordinate w.r.t. $z$ gives us
		\begin{align*}
			\pdv{f}{xz}\pdv{g}{y}+\pdv{g}{yz}\pdv{f}{x}-\pdv{g}{xz}\pdv{f}{y}-\pdv{f}{yz}\pdv{g}{x}
		\end{align*}
		Noting that the order of differentiation doesn't matter, if you look carefully, all terms cancel (after adding them) (for example, the 4th term in row 1 cancels with the 1st term in row 3), so $\mathrm{div}(\grad f \cross \grad g) = 0$.
		\end{enumerate}
\end{document}