\documentclass[12pt]{article}
\usepackage[margin=1in]{geometry}
\usepackage{setspace}
\onehalfspacing

% Start of preamble
%==========================================================================================%
% Required to support mathematical unicode
\usepackage[warnunknown, fasterrors, mathletters]{ucs}
\usepackage[utf8x]{inputenc}

\usepackage[dvipsnames,table,xcdraw]{xcolor} % colors
\usepackage{hyperref} % links
\hypersetup{
	colorlinks=true,
	linkcolor=blue,
	filecolor=magenta,      
	urlcolor=cyan,
	pdfpagemode=FullScreen
}

% Standard mathematical typesetting packages
\usepackage{amsmath,amssymb,amscd,amsthm,amsxtra, pxfonts}
\usepackage{mathtools,mathrsfs,xparse}

% Symbol and utility packages
\usepackage{cancel, textcomp}
\usepackage[mathscr]{euscript}
\usepackage[nointegrals]{wasysym}
\usepackage{apacite}

% Extras
\usepackage{physics}  % Lots of useful shortcuts and macros
\usepackage{tikz-cd}  % For drawing commutative diagrams easily
\usepackage{microtype}  % Minature font tweaks
%\usepackage{pgfplots} % plots

\usepackage{enumitem}
\usepackage{titling}

\usepackage{graphicx}

% Fancy theorems due to @intuitively on discord
\usepackage{mdframed}
\newmdtheoremenv[
backgroundcolor=NavyBlue!30,
linewidth=2pt,
linecolor=NavyBlue,
topline=false,
bottomline=false,
rightline=false,
innertopmargin=10pt,
innerbottommargin=10pt,
innerrightmargin=10pt,
innerleftmargin=10pt,
skipabove=\baselineskip,
skipbelow=\baselineskip
]{mytheorem}{Theorem}

\newenvironment{theorem}{\begin{mytheorem}}{\end{mytheorem}}

\newtheorem{corollary}{Corollary}
\newtheorem{lemma}{Lemma}

\newtheoremstyle{definitionstyle}
{\topsep}%
{\topsep}%
{}%
{}%
{\bfseries}%
{.}%
{.5em}%
{}%
\theoremstyle{definitionstyle}
\newmdtheoremenv[
backgroundcolor=Violet!30,
linewidth=2pt,
linecolor=Violet,
topline=false,
bottomline=false,
rightline=false,
innertopmargin=10pt,
innerbottommargin=10pt,
innerrightmargin=10pt,
innerleftmargin=10pt,
skipabove=\baselineskip,
skipbelow=\baselineskip,
]{mydef}{Definition}
\newenvironment{definition}{\begin{mydef}}{\end{mydef}}

\newtheorem*{remark}{Remark}

\newtheorem*{example}{Example}

% Common shortcuts
\def\mbb#1{\mathbb{#1}}
\def\mfk#1{\mathfrak{#1}}

\def\bN{\mbb{N}}
\def \C{\mbb{C}}
\def \R{\mbb{R}}
\def\bQ{\mbb{Q}}
\def\bZ{\mbb{Z}}
\def \cph{\varphi}
\renewcommand{\th}{\theta}
\def \ve{\varepsilon}
\newcommand{\mg}[1]{\| #1 \|}

% Often helpful macros
\newcommand{\floor}[1]{\left\lfloor#1\right\rfloor}
\newcommand{\ceil}[1]{\left\lceil#1\right\rceil}
\renewcommand{\qed}{\hfill\qedsymbol}
\renewcommand{\ip}[2]{\langle #1, #2 \rangle}
\newcommand{\seq}[2]{\qty(#1_#2)_{#2=1}^{\infty}}

% Sets
\DeclarePairedDelimiterX\set[1]\lbrace\rbrace{\def\given{\;\delimsize\vert\;}#1}

% End of preamble
%==========================================================================================%

% Start of commands specific to this file
%==========================================================================================%

\usepackage{braket}
\newcommand{\Z}{\mbb Z}
\newcommand{\gen}[1]{\left\langle #1 \right\rangle}
\newcommand{\nsg}{\trianglelefteq}
\newcommand{\F}{\mbb F}
\newcommand{\Aut}{\mathrm{Aut}}

%==========================================================================================%
% End of commands specific to this file

\title{Math 504 HW7}
\date{\today}
\author{Rohan Mukherjee}

\begin{document}
	\maketitle
	\begin{enumerate}[leftmargin=\labelsep]
		\item Let $I \subset R$ be an ideal. If $I = (0)$ we are done, so suppose $I$ contains a nonzero element. Define $\mathcal S = \Set{\sum r_i x_i \mid r_i \in R, x_i \in x} \setminus 0$. Associate to $\mathcal S$ the set $N = \Set{N(y) \mid y \in S}$. $N$ is a nonempty set of $\Z_{\geq 0}$ and hence it has a (not necessarily unique) minimal element $d$. By definition, there exists an element $z \in \mathcal S$ such that $N(z) = d$. We claim of course that $(d) = I$. Suppose otherwise, that there was an element $a \in I$ so that $a \not \in (d)$. We have that $a = dq + r$ for some $r = 0$ or $N(r) < N(d)$, but since $a \not \in (d)$, we can't have $r = 0$. Notice now that $a - dq$ is an $R$-linear combination of elements of $I$, and hence $a-dq \in I$. But then $a-dq = r$ is an element of $I$ with smaller norm than $d$, a contradiction. 
		
		\item 
		\begin{enumerate}
			\item We shall show that $\Z[i]$ is a Euclidean Domain. Let $a+bi, c+di \in \Z[i]$ with $c+di \neq 0$. Notice that,
			\begin{align*}
				\frac{a+bi}{c+di} = \frac{ac+bd}{c^2+d^2} + i\frac{bc-ad}{c^2+d^2}
			\end{align*}
			Now define $r = \frac{ac+bd}{c^2+d^2} \in \mbb Q$ and $s = \frac{bc-ad}{c^2+d^2} \in \mbb Q$. If both $r,s$ are integers we are done, else the sets $[k, k+1)$ for $k \in \Z$ partition $\R$, so we must have $r \in [k, k+1)$ for some integer $k$. From here, either $r \in [k, k+1/2)$, or $r \in [k+1/2, k)$. In the first case, we have $|k-r| \leq 1/2$, and in the second we have $|k+1-r| \leq 1/2$, so, possibly replacing $k$ with $k+1$, we have found an integer within $1/2$ of $r$. Similarly we can find an integer $l$ such that $|l-s| \leq 1/2$. Write $k = r + \ve$ and $l = s + \delta$, where $|\ve| \leq 1/2$ and $|\delta| \leq 1/2$, thus
			\begin{align*}
				N(a+bi-(k+li)(c+di)) &= N(a+bi - (r+si + \ve + \delta i)(c+di)) 
				\\&= N(a+bi - a - bi + (\ve+\delta i)(c+di)) = N((\ve+\delta i)(c+di)) \\
				&= N(\ve+\delta i)N(c+di) = (\ve^2+\delta^2)N(c+di) \leq \frac12N(c+di)
			\end{align*}
			In particular, $N(a+bi-(k+li)(c+di)) < N(c+di)$, completing the proof.
			
			\item Let $x$ be a unit. Then $N(x x^{-1}) = N(x)N(x^{-1}) = N(x)N(x^{-1}) = 1$ (We are using elementary facts from complex analysis about $N(a+bi)=a^2+b^2$). The only units in $\Z$ are $\pm1$, and the only positive one of those is just 1. So, units in $\Z[i]$ are precisely those elements $a+bi \in \Z[i]$ with $N(a+bi) = a^2+b^2=1$. From here we can only have $a = \pm 1$ with $b =0$ or $a = 0$ with $b = \pm 1$. These yield $\pm 1, \pm i$ as the only units.
			
			\item We first classify which primes are irreducible. $2 = (1+i)(1-i)$, so we reduce to odd primes. If $p$ is an odd prime and is reducible, then $p = ab$ for $a,b$ not units. Then $N(ab) = N(a)N(b) = p^2$, and since we are now working in the integers, we must have $N(a) = p$ (it cannot be 1, else it would be a unit). This would say that $p = x^2+y^2$ for some integers $x,y$, which is true iff $p \equiv 1 \mod 4$. Now if $p \equiv 3 \mod 4$, then $p$ is not the sum of two squares, so it is irreducible. Now, suppose that $x$ were irreducible, and notice that $x \overline x = N(x) = p_1^{\alpha_1} \cdots p_n^{\alpha_n}$. Since $x$ is irreducible, we must have $p_1 \in (x)$, thus $p_1 = yx$ for some $y$. If $y$ is a unit we are done (we are back to the prime case), so suppose otherwise. Then $N(x) = p_1$ through a similar line of reasoning as above. Now we claim that elements with prime order are irreducible. If $x = ab$, with $N(x) = p$, then $N(a)N(b) = N(x) = p$, so one of $N(a), N(b)$ must be 1, i.e. $x$ is irreducible. Thus the irreducible elements in $\Z[i]$ are those with prime order, and primes (in $\Z$) that are congruent to $3 \mod 4$.
		\end{enumerate}
	
		\item Define the following norm on $\Z[w]$ as $N(a+bw+cw^2) = \frac12((a-b)^2 + (b-c)^2 + (c-a)^2)$. Notice that,
		\begin{align*}
			(a+bw+cw^2)(a+cw+bw^2) &= a^2 + b^2 + c^2 + w(ab + ac + bc) + w^2(ab + ac + bc) 
			\\ &= a^2 + b^2 + c^2 - ab - bc - ca = \frac12\qty((a-b)^2 + (b-c)^2 + (c-a)^2)
		\end{align*}
		Notice that $\overline{a+bw+cw^2} = a+cw+bw^2$, where $\overline{x+iy}$ is just the normal complex conjugate. Thus the above norm is precisely the same one as for complex numbers, and hence it is multiplicative. Notice also that the norm of a complex number is 0 iff the complex number is 0, and in particular $a+bw+cw^2 = 0$ iff $a = b = c$. We now explicitly calculate the quotient: 
		\begin{align*}
			\frac{x+yw+zw^2}{a+bw+cw^2} &= \frac{(x+yw+zw^2)(a+cw+bw^2)}{(a-b)^2+(a-c)^2+(b-c)^2} 
			\\& = \frac{ax+cx+bz+w(ay+bx+cz) + w^2(az+by+cx)}{(a-b)^2+(a-c)^2+(b-c)^2}
		\end{align*}
		Thus define $p = (ax+cx+bz)/((a-b)^2+(a-c)^2+(b-c)^2), q = (ay+bx+cz)/((a-b)^2+(a-c)^2+(b-c)^2)$, and $r = (az+by+cx)/((a-b)^2+(a-c)^2+(b-c)^2)$. Find $i$ within $1/2$ of $p$, $j$ within $1/2$ of $q$, and $k$ within $1/2$ of $r$. Write $i = p + \ve_1$, $j = q + \ve_2$, and $k = r + \ve_3$. Now,
		\begin{align*}
			&N(x+yw+zw^2 - (i + jw + kw^2 + (\ve_1 + \ve_2w + \ve_3w^2))(a+bw+cw^2)) \\
			&= N(\ve_1 + \ve_2w + \ve_3w^2)N((a+bw+cw^2)) \leq \frac{3}{4}N(a+bw+cw^2)
		\end{align*}
		Which completes the proof.
		
		\item Write $f(X) = a_0 + a_1X + \cdots a_nX^n$ and $g(X) = b_0 + b_1X + \cdots + b_mX^m$, WLOG $m \leq n$ (otherwise we are already done). Consider $h_1(X) = a_nb_m^{-1}x^{n-m}g(X)$. Then the leading term of $h_1(X)$ is just $a_nb_m^{-1}b_mX^{n-m}X^m = a_nX^m$. Thus, we must have that $f(X) - h_1(X)$ has degree $\leq n-1$. If it has degree less than $m$ we are done, else we can do the same thing as above but with $f(X) - h_1(X)$ taking the place of $X$ to find a function $h_2(X)$ which is a multiple of $g(X)$ canceling the highest order term of $f(X)-h_1(X)$. Thus, $f(X)-h_1(X)-h_2(X)$ has degree at most $\deg(f(X)-h(X)) - 1 \leq n-2$. We can now repeat this $k$ times until $r(X) = f(X) - \sum_{i=1}^k h_i(X)$ has degree less than $m$ or is equivalently 0 ($k$ is finite because in the worst case this process takes $n-m$ steps). Thus, $f(X) = \sum_{i=1}^k h_i(X) + r(X)$, and since the $h_i$ are divisible by $g$, we are done.
		
		\item \begin{enumerate}
			\item We prove the claim by induction. The polynomial $a_0+a_1X$ has at only one root, because we can solve $a_0+a_1x = 0$ where $x \in F$ to get that $x = a_1^{-1}a_0$. Now suppose that a polynomial of degree $n-1 \geq 0$ has at most $n-1$ roots, and let $f(X)$ be a polynomial of degree $n$. If $f$ has no roots we are done, so suppose it had a root, $a$. Then $f(X) = h(X)(X-a) + r$, where $\deg r < \deg(x-a) = 1$ or $r = 0$. Degree 0 elements are just elements of $F$, so plugging in $x = a$ will yield $0 = f(a) = h(a)(a-a) + r$, which tells us that $r = 0$. Now, $h(X)$ has at most $n-1$ roots, thus $f(X) = h(X)(X-a)$ has at most $n-1+1 = n$ roots, which completes the proof. 
			
			\item Notice first that $f(X)$ has either nonnegative degree or is equivalently 0. In the second case we are done, so suppose the first case. Label its degree $n \geq 0$. If $f(X)$ has degree 0, and is not 0, then it has no roots, so we are done. If $n \geq 1$, then by the last part we proved a polynomial of degree $n$ has at most $n$ roots, but $f(X)$ has infinitely many roots--since $f(x) = 0$ (as a function from $F \to F$) for all $x \in F$. This is a contradiction.
			
			\item The counterexample is as follows: $f(X) = X^2 + X \in \Z/2[X]$. Notice that $f(X) = 0$ for all $X \in \Z/2[X]$ but $f \neq 0$ (in $\Z/2[X]$).
		\end{enumerate}
	
		\newpage
		\item Let $P$ be a group of order $|p|^2$. Then $P$ is abelian, and in particular, $P \cong \Z/p^2$ or $\Z/p \times \Z/p$. We claim that $Z(P) \neq \gen{1}$. By the class equation, if $g_1, \ldots, g_r$ are representatives of the non-central conjugacy classes,
		\begin{align*}
			|P| = |Z(P)| + \sum_{i=1}^r |P : \mathcal{C}_{g_i}|
		\end{align*}
		Since each $C_{g_i} \neq P$ by hypothesis, we must have $p \mid |C_{g_i}|$, and thus $p \mid |P| - \sum_{i=1}^r |P : \mathcal{C}_{g_i}| = |Z(P)|$. So, $Z(P) \neq \gen{1}$, and in particular, $p \mid Z(P)$. We have only two cases for $Z(P)$: $p$ or $p^2$. In the latter case we are done, so suppose the former. Then $|P/Z(P)| = p$, so $P/Z(P) \cong \Z/p$, and in particular it is cyclic, so $P$ is abelian. Now, either $P$ has an element of order $p^2$, or all elements have order dividing $p$. In the first case $P$ is cyclic and isomorphic to $\Z/p^2$. Since the only element with order 1 is $e$, we can find an element $x$ of order $p$. Now taking $y \in P - \gen{x}$, we can see that $\gen{x} < \gen{x, y} \leq P$, so in particular $\gen{x, y}$ divides $p^2$ and is not 1 or $p$, hence it equals $p^2$ and $\gen{x,y} = P$. Since $P$ is abelian, we have $\gen{x} \gen{y} = \Set{x^\alpha y^\beta \mid \alpha,\beta \in \Z} = \gen{x, y}$, and also since $P$ is abelian we have $\gen{x} \nsg P$ and $\gen{y} \nsg P$. Thus $P = \gen{x} \times \gen{y} = \Z/p \times \Z/p$, and we are done.
	\end{enumerate}
\end{document}