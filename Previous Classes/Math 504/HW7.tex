\documentclass[12pt]{article}
\usepackage[margin=1in]{geometry}
\usepackage{setspace}
\onehalfspacing

% Start of preamble
%==========================================================================================%
% Required to support mathematical unicode
\usepackage[warnunknown, fasterrors, mathletters]{ucs}
\usepackage[utf8x]{inputenc}

\usepackage[dvipsnames,table,xcdraw]{xcolor} % colors
\usepackage{hyperref} % links
\hypersetup{
	colorlinks=true,
	linkcolor=blue,
	filecolor=magenta,      
	urlcolor=cyan,
	pdfpagemode=FullScreen
}

% Standard mathematical typesetting packages
\usepackage{amsmath,amssymb,amscd,amsthm,amsxtra}
\usepackage{mathtools,mathrsfs,xparse}

% Symbol and utility packages
\usepackage{cancel, textcomp}
\usepackage[mathscr]{euscript}
\usepackage[nointegrals]{wasysym}
\usepackage{apacite}

% Extras
\usepackage{physics}  % Lots of useful shortcuts and macros
\usepackage{tikz-cd}  % For drawing commutative diagrams easily
\usepackage{microtype}  % Minature font tweaks
%\usepackage{pgfplots} % plots

\usepackage{enumitem}
\usepackage{titling}

\usepackage{graphicx}

% Fancy theorems due to @intuitively on discord
\usepackage{mdframed}
\newmdtheoremenv[
backgroundcolor=NavyBlue!30,
linewidth=2pt,
linecolor=NavyBlue,
topline=false,
bottomline=false,
rightline=false,
innertopmargin=10pt,
innerbottommargin=10pt,
innerrightmargin=10pt,
innerleftmargin=10pt,
skipabove=\baselineskip,
skipbelow=\baselineskip
]{mytheorem}{Theorem}

\newenvironment{theorem}{\begin{mytheorem}}{\end{mytheorem}}

\newtheorem{corollary}{Corollary}
\newtheorem{lemma}{Lemma}

\newtheoremstyle{definitionstyle}
{\topsep}%
{\topsep}%
{}%
{}%
{\bfseries}%
{.}%
{.5em}%
{}%
\theoremstyle{definitionstyle}
\newmdtheoremenv[
backgroundcolor=Violet!30,
linewidth=2pt,
linecolor=Violet,
topline=false,
bottomline=false,
rightline=false,
innertopmargin=10pt,
innerbottommargin=10pt,
innerrightmargin=10pt,
innerleftmargin=10pt,
skipabove=\baselineskip,
skipbelow=\baselineskip,
]{mydef}{Definition}
\newenvironment{definition}{\begin{mydef}}{\end{mydef}}

\newtheorem*{remark}{Remark}

\newtheorem*{example}{Example}

% Common shortcuts
\def\mbb#1{\mathbb{#1}}
\def\mfk#1{\mathfrak{#1}}

\def\bN{\mbb{N}}
\def \C{\mbb{C}}
\def \R{\mbb{R}}
\def\bQ{\mbb{Q}}
\def\bZ{\mbb{Z}}
\def \cph{\varphi}
\renewcommand{\th}{\theta}
\def \ve{\varepsilon}
\newcommand{\mg}[1]{\| #1 \|}

% Often helpful macros
\newcommand{\floor}[1]{\left\lfloor#1\right\rfloor}
\newcommand{\ceil}[1]{\left\lceil#1\right\rceil}
\renewcommand{\qed}{\hfill\qedsymbol}
\renewcommand{\ip}[2]{\langle #1, #2 \rangle}
\newcommand{\seq}[2]{\qty(#1_#2)_{#2=1}^{\infty}}

% Sets
\DeclarePairedDelimiterX\set[1]\lbrace\rbrace{\def\given{\;\delimsize\vert\;}#1}

% End of preamble
%==========================================================================================%

% Start of commands specific to this file
%==========================================================================================%

\usepackage{braket}
\newcommand{\Z}{\mbb Z}
\newcommand{\gen}[1]{\left\langle #1 \right\rangle}
\newcommand{\nsg}{\trianglelefteq}
\newcommand{\F}{\mbb F}
\newcommand{\Aut}{\mathrm{Aut}}

%==========================================================================================%
% End of commands specific to this file

\title{Math 504 HW7}
\date{\today}
\author{Rohan Mukherjee}

\begin{document}
	\maketitle
	\begin{enumerate}[leftmargin=\labelsep]
		\item We claim that if $\mathcal C$ is a conjugacy class of $G$, and $H \nsg G$, then either $\mathcal C \subset H$ or $\mathcal C \cap H = \emptyset$. If there is an $x \in \mathcal C \cap H$, then $gxg^{-1} \in H$ as well, and since every $y \in \mathcal C$ is of the form $gxg^{-1}$, this shows that $\mathcal C \subset H$. Since $G = Z(G) \cup \mathcal C_{g_1} \cup \cdots \cup \mathcal C_{g_r}$ where $g_1, \ldots, g_r$ are the representatives of the distinct noncentral conjugacy classes of $G$, we have that $H = (Z(G) \cap H) \cup \mathcal C_{g_1} \cup \cdots \cup \mathcal C_{g_s}$, where we have potentially re-ordered the $g_i$ to be so that $\mathcal C_{g_i} \subset H$ for all $1 \leq i \leq s$ and $\mathcal C_{g_i} \cap H = \emptyset$ for all $s+1 \leq i \leq r$. The above is a disjoint union, since the $\mathcal C_{g_i}$ are disjoint and each were disjoint from $Z(G)$, thus they must be disjoint from $Z(G) \cap H$. We have deduced that
		\begin{align*}
			|H| = |Z(G) \cap H| + \sum_{i=1}^s |\mathcal C_{g_i}|
		\end{align*}
		We recall that the orbit of the conjugacy class containing $g_i$ is just $|G : C_G(g_i)|$, and since $g_i \not \in Z(G)$ by hypothesis we must have $|G : C_G(g_i)| > 1$. Since $G$ is a $p$-group, $p \mid |G : C_G(g_i)|$, and since $p \mid |H|$ as well, $p$ divides $|H| - \sum_{i=1}^s |\mathcal C_{g_i}| = |Z(G) \cap H|$, which shows that $Z(G) \cap H \neq \gen{1}$.
		
		\item We claim that $\gen{a, b \mid a^4=b^4=1, bab^{-1}=a^{-1}} \cong Z_4 \rtimes Z_4$. Clearly $\gen{a^4} = \Set{e, a, a^2, a^3}$, since free groups are defined on words, and similarly $\gen{b^4} = \Set{e, b, b^2, b^3}$. Since $a \neq b$, it follows immediately that $\gen{a} \cap \gen{b} = \gen{1}$. We also claim that $\gen{a} \nsg \gen{a, b \mid a^4=b^4=1, bab^{-1}=a^{-1}}$. We need only show that every element is in the normalizer of $a$, since then we could show it by induction. Elements of the form $a^\alpha$ commute with $a$, and $b^\beta a b^{-\beta} = a^{-\beta}$, so every word in our free group will normalize $\gen{a}$. Since $\gen{a} \cong Z_4$ and $\gen{b} \cong Z_4$ we may use Theorem 12 in chapter 4 to conclude the above claim.
	\end{enumerate}
\end{document}