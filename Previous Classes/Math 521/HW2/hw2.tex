\documentclass[12pt]{article}
\usepackage[margin=1in]{geometry}
\usepackage{setspace}
\onehalfspacing{}
\usepackage[dvipsnames,table,xcdraw]{xcolor} % colors

% Start of preamble
%==========================================================================================%
% Required to support mathematical unicode
\usepackage[warnunknown, fasterrors, mathletters]{ucs}
\usepackage[utf8x]{inputenc}

\usepackage{dsfont}

% Standard mathematical typesetting packages
\usepackage{amsmath,amssymb,amscd,amsthm,amsxtra,amsfonts}
\usepackage{mathtools,mathrsfs,xparse}

% Symbol and utility packages
\usepackage{cancel, textcomp}
\usepackage[mathscr]{euscript}
\usepackage[nointegrals]{wasysym}
\usepackage{apacite}

% Extras
\usepackage{physics}  % Lots of useful shortcuts and macros
\usepackage{tikz-cd}  % For drawing commutative diagrams easily
\usepackage{microtype}  % Minature font tweaks
%\usepackage{pgfplots} % plots

\usepackage{enumitem}
\usepackage{titling}

\usepackage{graphicx}

%\usepackage{quiver}

% Fancy theorems due to @intuitively on discord
\usepackage{mdframed}
\newmdtheoremenv[
backgroundcolor=NavyBlue!30,
linewidth=2pt,
linecolor=NavyBlue,
topline=false,
bottomline=false,
rightline=false,
innertopmargin=10pt,
innerbottommargin=10pt,
innerrightmargin=10pt,
innerleftmargin=10pt,
skipabove=\baselineskip,
skipbelow=\baselineskip]{mytheorem}{Theorem}

\newenvironment{theorem}{\begin{mytheorem}}{\end{mytheorem}}

\newtheorem{corollary}{Corollary}
\newtheorem{lemma}{Lemma}

\newtheoremstyle{definitionstyle}
{\topsep}%
{\topsep}%
{}%
{}%
{\bfseries}%
{.}%
{.5em}%
{}%
\theoremstyle{definitionstyle}
\newmdtheoremenv[
backgroundcolor=Violet!30,
linewidth=2pt,
linecolor=Violet,
topline=false,
bottomline=false,
rightline=false,
innertopmargin=10pt,
innerbottommargin=10pt,
innerrightmargin=10pt,
innerleftmargin=10pt,
skipabove=\baselineskip,
skipbelow=\baselineskip,
]{mydef}{Definition}
\newenvironment{definition}{\begin{mydef}}{\end{mydef}}

\newtheorem*{remark}{Remark}

\newtheorem*{example}{Example}
\newtheorem*{claim}{Claim}

% Common shortcuts
\def\mbb#1{\mathbb{#1}}
\def\mfk#1{\mathfrak{#1}}

\def\bN{\mbb{N}}
\def\C{\mbb{C}}
\def\R{\mbb{R}}
\def\bQ{\mbb{Q}}
\def\bZ{\mbb{Z}}
\def\cph{\varphi}
\renewcommand{\th}{\theta}
\def\ve{\varepsilon}
\newcommand{\mg}[1]{\| #1 \|}

% Often helpful macros
\newcommand{\floor}[1]{\left\lfloor#1\right\rfloor}
\newcommand{\ceil}[1]{\left\lceil#1\right\rceil}
\renewcommand{\qed}{\hfill\qedsymbol}
\renewcommand{\ip}[1]{\langle#1\rangle}
\newcommand{\seq}[2]{\qty(#1_#2)_{#2=1}^{\infty}}

\newcommand{\SET}[1]{\Set{\mskip-\medmuskip #1 \mskip-\medmuskip}}

% End of preamble
%==========================================================================================%

% Start of commands specific to this file
%==========================================================================================%

\usepackage{braket}
\newcommand{\Z}{\mbb Z}
\newcommand{\gen}[1]{\left\langle #1 \right\rangle}
\newcommand{\nsg}{\trianglelefteq}
\newcommand{\F}{\mbb F}
\newcommand{\Aut}{\mathrm{Aut}}
\newcommand{\sepdeg}[1]{[#1]_{\mathrm{sep}}}
\newcommand{\Q}{\mbb Q}
\newcommand{\Gal}{\mathrm{Gal}\qty}
\newcommand{\id}{\mathrm{id}}
\newcommand{\Hom}{\mathrm{Hom}_R}
\newcommand{\1}{\mathds 1}
\newcommand{\N}{\mathbb N}
\renewcommand{\P}{\mathbb P \qty}
\newcommand{\E}{\mathbb E \qty}
\newcommand{\Var}{\mathrm{Var}}

%==========================================================================================%
% End of commands specific to this file

\title{Math 521 HW1}
\date{\today}
\author{Rohan Mukherjee}

\begin{document}
    \maketitle
    \begin{enumerate}
        \item Let $(X,Y)$ be the random vector that takes on $(0,0)$, $(1, 1)$, $(2, 0)$ each with probability $1/3$. Then,
        \begin{align*}
            \E[XY] &= \frac 13 \\
            \E[X] &= 1 \\
            \E[Y] &= \frac 13 \\
        \end{align*}
        Hence $\mathrm{Cov}(X,Y) = 0$. However, noticing that $Y \cdot \1_{|Y| \leq 1} = Y$,
        \begin{align*}
            \E[X \cdot \1_{|X| \leq 1} Y] &= 0 + \frac 13 \cdot 1 \cdot 1 + 0 = \frac 13 \\
            \E[X \cdot \1_{|X| \leq 1}] &= \frac 13 \\
            \E[Y] &= \frac 13
        \end{align*}
        thus, $\mathrm{Cov}(X \cdot \1_{|X| \leq 1}, Y) = \frac 13 - \frac 19$.

        \item We see that, by Fubini's theorem,
        \begin{align*}
            \E[f(X)] &= \int_\R f(x) \mu_X(dx) = \int_\R \int_0^x f'(y)dy \mu_X(dx) \\ &= \int_\R \int_y^\infty f'(y)\mu_X(dx)dy = \int_\R f'(y) \P(X \geq y)dy
        \end{align*}

        \item For $m \leq n$, we know that
        \begin{align*}
            \E[\frac{S_m}{S_n}] = \sum_{i=1}^m \E[\frac{X_i}{S_n}]
        \end{align*}
        Now, notice that:
        \begin{align*}
            \E[\frac{X_1}{X_1+X_2}] = \int_{\R^2} \frac{x}{x+y} d(\mu_{X_1} \times \mu_{X_2}) = \int_{\R^2} \frac{x}{x+y} d(\mu_{X_2} \times \mu_{X_1}) = \E[\frac{X_2}{X_1+X_2}]
        \end{align*}
        Since $\mu_{X_1} = \mu_{X_2}$. Extending this result to the case of $n$ variables, we have, for $1 \leq i \leq n$,
        \begin{align*}
            \E[\frac{X_i}{S_n}] = \E[\frac{X_1}{S_n}]
        \end{align*}
        Thus,
        \begin{align*}
            1 = \E[\frac{S_n}{S_n}] = \sum_{i=1}^n \E[\frac{X_i}{S_n}] = n \E[\frac{X_1}{S_n}]
        \end{align*}
        And thus $\E[\frac{X_1}{S_n}] = \frac 1n$. In conclusion we have that $\E[S_m/S_n] = m/n$.

        On the other hand, if $m > n$, we have that:
        \begin{align*}
            \E[\frac{S_m}{S_n}] = \E[\frac{S_n}{S_n}] + \E[\frac{S_m - S_n}{S_n}] = 1 + \E[\frac{\sum_{i=n+1}^m X_i}{S_n}]
        \end{align*}
        Now, each $X_i$ for $i > n$ is independent of $S_n$, and thus we have that:
        \begin{align*}
            \E[\frac{\sum_{i=n+1}^m X_i}{S_n}] = \sum_{i=n+1}^m \E[\frac{X_i}{S_n}] = \sum_{i=n+1}^m \E[X_i]\E[\frac{1}{S_n}] = (m-n)\E[X_1]\E[\frac{1}{S_n}]
        \end{align*}
        Which finally shows that:
        \begin{align*}
            \E[S_m/S_n] = 1 + (m-n)\E[X_1]\E[1/S_n]
        \end{align*}

        \item Recall that $X_n \to X$ in probability iff given any subsequence $X_{n_m}$ of $X_n$, there is a further subsequence $X_{n_{m_k}} \to X$ a.s. So, let $X_{n_m} + Y_{n_m}$ be a subsequence of $X_n+Y_n$. Find a subsequence of $X_{n_m}$, say $X_{n_{m_k}} \to X$ a.s.. Further, find a subsequence of $Y_{n_{m_k}}$ say $Y_{n_{m_{k_l}}} \to Y$ a.s.. Then, we have that $X_{n_{m_{k_l}}} + Y_{n_{m_{k_l}}} \to X+Y$ a.s., which shows that $X_n+Y_n \to X +Y$ in probability.

        Similarly, let $X_{n_m}Y_{n_m}$ be a subsequence of $X_nY_n$. Find a subsequence of $X_{n_m}$, say $X_{n_{m_k}} \to X$ a.s.. Further, find a subsequence of $Y_{n_{m_k}}$ say $Y_{n_{m_{k_l}}} \to Y$ a.s.. Then, we have that $X_{n_{m_{k_l}}}Y_{n_{m_{k_l}}} \to XY$ a.s., which shows that $X_nY_n \to XY$ in probability.

        \item For a fixed $k$, we have that,
        \begin{align*}
            \E[(X_k-\overline X)^2] &= \E[X_k^2] - 2\E[X_k \overline X] + \E[\overline X^2]
        \end{align*}
        First,
        \begin{align*}
            \E[X_k \overline X] = \frac 1n \sum_i \E[X_kX_i] = \frac 1n\E[X_k^2] + \frac{n-1}n\E[X_k]^2
        \end{align*}
        Second,
        \begin{align*}
            \E[\overline X^2] = \frac{1}{n^2}\sum_{i,j} \E[X_iX_j] = \frac{1}{n^2} \qty(n\E[X_k^2] + n(n-1)\E[X_k]^2) = \frac{1}{n}\E[X_k^2] + \frac{n-1}{n}\E[X_k]^2
        \end{align*}
        Adding these together shows that,
        \begin{align*}
            \E[(X_k-\overline X)^2] &= \E[X_k^2] - \frac 2n \E[X_k^2] - \frac{2(n-1)}n \E[X_k]^2 + \frac 1n\E[X_k^2] + \frac{n-1}{n}\E[X_k]^2 
            \\&= \frac{n-1}n \E[X_k^2] - \frac{n-1}{n} \E[X_k]^2 = \frac{n-1}{n}\Var(X_1)
        \end{align*}
        Thus,
        \begin{align*}
            \E[\overline V_n] = \frac{1}{n-1} \sum_{i=1}^n \E[(X_i-\overline X)^2] = \sigma^2
        \end{align*}

        Notice that
        \begin{align*}
            \overline V_n &= \frac{1}{n-1} \sum_{i=1}^n (X_k-\overline X_n)^2 = \frac{1}{n-1} \sum_{i=1}^n \qty(X_k^2 - 2X_k\overline X_n + \overline X_n^2) \\
            &= \frac{1}{n-1} \sum_{i=1}^n X_k^2 - \frac{n}{n-1}\overline X_n^2 
        \end{align*}
        Assume WLOG that $\E[X_1]=0$ (possibly after shifting). Then,
        \begin{align*}
            \frac{1}{n} \sum_{i=1}^n X_k^2 \to \sigma^2
        \end{align*}
        in probability by the weak law of large numbers. Similarly, $\frac{n}{n-1} \to 1$ in probability so that $\frac{1}{n-1} \sum_{i=1}^n X_k^2 \to \sigma^2$ in probability. Also by the weak large of large numbers, $\overline X_n \to 0$ in probability. Thus, $\overline X_n^2 \to 0$ in probablity (since the product of two sequences converging to 0 in probability also converges to 0 in probability, by the last question). Thus $\frac{n}{n-1} \overline X_n^2 \to 0$ in probability. Thus the sum of these two terms converges to $\sigma^2$ in probability, which completes the proof.

        \item We shall show that 
        \begin{align*}
            (f(X)-f(Y))(g(X)-g(Y)) \geq 0
        \end{align*}
        For each $\omega \in \Omega$, either $X(\omega) \leq Y(\omega)$ or vice versa. In the first case, both terms in the above product are $\leq 0$ and the product is $\geq 0$. In the second, both are $\geq 0$, and so the product is $\geq 0$.
        Thus,
        \begin{align*}
            \E[f(X)g(X)] - \E[f(X)g(Y)] - \E[f(Y)g(X)] + \E[f(Y)g(Y)] \\
            = \E[(f(X)-f(Y))(g(X)-g(Y))] \geq 0
        \end{align*}
        Now notice that $\E[f(Y)g(Y)] = \E[f(X)g(X)]$, and because $X,Y$ are independent, $\E[f(X)g(Y)] = \E[f(Y)g(X)] = \E[f(X)]\E[g(X)]$. This completes the proof.

        \item By Cauchy-Schwarz, we have that:
        \begin{align*}
            \E[X \cdot \1_{X > 0}]^2 &\leq \E[X^2]\E[1_{X>0}] = \E[X^2]\P(X>0)
        \end{align*}
        because $X \geq 0$, we know that $X \cdot \1_{X > 0} = X$ (For if $X = 0$, then both sides are just 0). So we have that:
        \begin{align*}
            \frac{\E[X]^2}{\E[X^2]} \leq \P(X > 0)
        \end{align*}

        \item Let $_mS_n = \sum_{k=m}^n \1_{A_k}$. Then we have that:
        \begin{align*}
            \E[_mS_n] = \sum_{k=m}^n \P(A_k) = \sum_{k=1}^n \P(A_k) - \sum_{k=1}^{m-1} \P(A_k)
        \end{align*}
        And,
        \begin{align*}
            \E[_mS_n^2] &= \sum_{m \leq k,j \leq n} \E[\1_{A_k}\1_{A_j}] = \sum_{m \leq k,j \leq n} \P(A_k \cap A_j) \\
            &= \sum_{1 \leq k,j \leq n} \P(A_k \cap A_j) - 2\sum_{k=1}^n \sum_{j=1}^{m-1} \P(A_k \cap A_j) + \sum_{1 \leq k,j \leq m-1} \P(A_k \cap A_j) 
        \end{align*}
        We use the following lemma.
        \begin{lemma}
            For $a_n \uparrow \infty$ and $b_n \uparrow \infty$, if $\limsup_{n \to \infty} \frac{a_n^2}{b_n} = \alpha > 0$, then $\limsup_{n \to \infty} \frac{a_n}{b_n} = 0$ (and hence $\lim_{n \to \infty} \frac{a_n}{b_n} = 0$).
        \end{lemma}
        \begin{proof}
            Suppose instead that $\limsup_{n \to \infty} \frac{a_n}{b_n} = \beta > 0$. Then, eventually $a_n/b_n \geq \beta/2$. Then eventually,
            \begin{align*}
                \frac{a_n^2}{b_n} \geq \frac \beta 2 a_n \uparrow \infty
            \end{align*}
            a contradiction.
        \end{proof}
        We use this to show that a number of terms are negligible in the limsup. First,
        \begin{align*}
            \limsup_{n \to \infty} \frac{\E[_1S_n]^2}{\E[_1S_n^2]} = \alpha > 0
        \end{align*}
        So by the above lemma $\E[_1S_n]/\E[_1S_n^2] \to 0$. Also,
        \begin{align*}
            \sum_{k=1}^n \sum_{j=1}^{m-1} \P(A_k \cap A_j) \leq \sum_{k=1}^n (m-1)\P(A_k) = (m-1)\E[_1S_n]
        \end{align*}
        Combining the above, we have that:
        \begin{align*}
            \limsup_{n \to \infty} \frac{\E[_mS_n^2]}{\E[_1S_n^2]} = \limsup_{n \to \infty} \frac{\E[_1S_n^2] - 2\sum_{k=1}^n \sum_{j=1}^{m-1} \P(A_k \cap A_j) + \E[_1S_{m-1}^2]}{\E[_1S_n^2]} = 1
        \end{align*}
        And also, because the bottom destroys all the lower order terms to the right of the first term on the top,
        \begin{align*}
            \limsup_{n \to \infty} \frac{\E[_mS_n]^2}{\E[_1S_n^2]} &= \limsup_{n \to \infty} \frac{\E[_1S_n]^2 - 2\E[_1S_n]\E[_1S_{m-1}] + \E[_1S_{m-1}]^2}{\E[_1S_n^2]} 
            \\&= \limsup_{n \to \infty} \frac{\qty(\sum_{1 \leq k \leq n} \P(A_k))^2}{\sum_{1 \leq k,j \leq n} \P(A_k \cap A_j)} = \alpha
        \end{align*}
        Thus,
        \begin{align*}
            \limsup_{n \to \infty} \frac{\E[_mS_n]^2}{\E[_mS_n^2]} = \alpha
        \end{align*}
        after dividing top and bottom by $\E[_1S_n^2]$. 
        By using the previous exercise, and noting that $_mS_n(\omega) > 0$ iff $\omega \in \bigcup_{k = m}^n A_k$, we have that:
        \begin{align*}
            \P(\bigcup_{k \geq m} A_k) = \P(\limsup_{n \to \infty} \bigcup_{k=m}^n A_k) \geq \limsup_{n \to \infty} \P(\bigcup_{k=m}^n A_k) \geq \limsup_{n \to \infty} \frac{\E[_mS_n]^2}{\E[_mS_n^2]} = \alpha
        \end{align*}
        Since this holds for any $m$, and $\bigcup_{k \geq m} A_k \downarrow \SET{A_k \text{ i.o}}$, we have that:
        \begin{align*}
            \P(A_k \text{ i.o.}) \geq \alpha.
        \end{align*}
    \end{enumerate}
\end{document}