\documentclass[12pt]{article}
\usepackage[margin=1in]{geometry}

% Start of preamble
%==========================================================================================%
% Required to support mathematical unicode
\usepackage[warnunknown, fasterrors, mathletters]{ucs}
\usepackage[utf8x]{inputenc}

\usepackage[dvipsnames,table,xcdraw]{xcolor} % colors
\usepackage{hyperref} % links
\hypersetup{
	colorlinks=true,
	linkcolor=blue,
	filecolor=magenta,      
	urlcolor=cyan,
	pdfpagemode=FullScreen
}

% Standard mathematical typesetting packages
\usepackage{amsmath,amssymb,amscd,amsthm,amsxtra, pxfonts}
\usepackage{mathtools,mathrsfs,dsfont,xparse}

% Symbol and utility packages
\usepackage{cancel, textcomp}
\usepackage[mathscr]{euscript}
\usepackage[nointegrals]{wasysym}
\usepackage{apacite}

% Extras
\usepackage{physics}  % Lots of useful shortcuts and macros
\usepackage{tikz-cd}  % For drawing commutative diagrams easily
\usepackage{microtype}  % Minature font tweaks
%\usepackage{pgfplots} % plots

\usepackage{enumitem}
\usepackage{titling}

\usepackage{graphicx}

% Fancy theorems due to @intuitively on discord
\usepackage{mdframed}
\newmdtheoremenv[
backgroundcolor=NavyBlue!30,
linewidth=2pt,
linecolor=NavyBlue,
topline=false,
bottomline=false,
rightline=false,
innertopmargin=10pt,
innerbottommargin=10pt,
innerrightmargin=10pt,
innerleftmargin=10pt,
skipabove=\baselineskip,
skipbelow=\baselineskip
]{mytheorem}{Theorem}

\newenvironment{theorem}{\begin{mytheorem}}{\end{mytheorem}}

\newtheorem{corollary}{Corollary}

\newmdtheoremenv[
backgroundcolor=Cerulean!30,
linewidth=2pt,
linecolor=Cerulean,
topline=false,
bottomline=false,
rightline=false,
innertopmargin=10pt,
innerbottommargin=10pt,
innerrightmargin=10pt,
innerleftmargin=10pt,
skipabove=\baselineskip,
skipbelow=\baselineskip,
]{mylemma}{Lemma}
\newenvironment{lemma}{\begin{mylemma}}{\end{mylemma}}

\newtheoremstyle{definitionstyle}
{\topsep}%
{\topsep}%
{}%
{}%
{\bfseries}%
{.}%
{.5em}%
{}%
\theoremstyle{definitionstyle}
\newmdtheoremenv[
backgroundcolor=Violet!30,
linewidth=2pt,
linecolor=Violet,
topline=false,
bottomline=false,
rightline=false,
innertopmargin=10pt,
innerbottommargin=10pt,
innerrightmargin=10pt,
innerleftmargin=10pt,
skipabove=\baselineskip,
skipbelow=\baselineskip,
]{mydef}{Definition}
\newenvironment{definition}{\begin{mydef}}{\end{mydef}}

\newtheorem*{remark}{Remark}

\newtheorem*{example}{Example}

% Common shortcuts
\def\mbb#1{\mathbb{#1}}
\def\mfk#1{\mathfrak{#1}}

\def\bN{\mbb{N}}
\def \C{\mbb{C}}
\def \R{\mbb{R}}
\def\bQ{\mbb{Q}}
\def\bZ{\mbb{Z}}
\def \cph{\varphi}
\renewcommand{\th}{\theta}
\def \ve{\varepsilon}
\newcommand{\mg}[1]{\| #1 \|}

% Often helpful macros
\newcommand{\floor}[1]{\left\lfloor#1\right\rfloor}
\newcommand{\ceil}[1]{\left\lceil#1\right\rceil}
\renewcommand{\qed}{\hfill\qedsymbol}
\renewcommand{\ip}[2]{\langle #1, #2 \rangle}
\newcommand{\seq}[2]{\qty(#1_#2)_{#2=1}^{\infty}}

% Sets
\DeclarePairedDelimiterX\set[1]\lbrace\rbrace{\def\given{\;\delimsize\vert\;}#1}

% End of preamble
%==========================================================================================%

% Start of commands specific to this file
%==========================================================================================%

%==========================================================================================%
% End of commands specific to this file

\title{Math 425 HW7}
\date{\today}
\author{Rohan Mukherjee}

\begin{document}
	\maketitle
	\begin{lemma}
		Let $A: \R^n \to \R^m$. If for every $x, y \in \R^n$, and $c \in \R$ we have that 
		\begin{align*}
			A(cx+y) = cAx + Ay
		\end{align*}
		then $A$ is linear.
	\end{lemma}
	
	\begin{proof}
		Choosing $y = 0$, we see that for every $x \in \R^n$, and $c \in \R$, we have that $A(cx) = A(cx+0) = cAx + A0 = cAx$. Choosing $c = 1$, we see that for every $x, y \in \R^n$, we have that $A(x+y) = A(1 \cdot x + y) = 1 \cdot Ax + Ay = Ax + Ay$.
	\end{proof}
	Since I used this lemma countless times without mentioning it, I thought I should add (and prove) it here.

	\begin{enumerate}[leftmargin=\labelsep]
		\item Denote $S = \set{x_1, \ldots, x_n}$. We recall that 
		\begin{align*}
			\mathrm{span}(S) = \set[\Big]{\sum_{i=1}^n {a_ix_i} \given a_i \in \R, x_i \in X}
		\end{align*}
		Given $\sum_{i=1}^n {a_ix_i}, \sum_{i=1}^n {b_ix_i} \in S$, we know that
		\begin{align*}
			\sum_{i=1}^n {a_ix_i} + \sum_{i=1}^n {b_ix_i} = \sum_{i=1}^n a_ix_i + b_ix_i = \sum_{i=1}^n (a_i+b_i)x_i \in \mathrm{span}(S)
		\end{align*}
		Also, given $c \in \R$,
		\begin{align*}
			c\sum_{i=1}^n {a_ix_i} = \sum_{i=1}^n {(ca_i)x_i} \in \mathrm{span}(S)
		\end{align*}
		which shows that $\mathrm{span}(S)$ is a vector space.
		
		\item Let $x, y \in \R^n$ and $c \in \R$. Then,
		\begin{align*}
			BA(cx+y) = B(cAx + Ay) = cBAx + BAy
		\end{align*}
		So indeed, $BA$ is linear. Next, by definition $A^{-1}$ is defined as the left inverse of $A$. Since $A$ is onto, it has a right inverse, say $B$ (We are thinking of $A, A^{-1}, B$ as functions). Now,
		\begin{align*}
			Bx = A^{-1}ABx = A^{-1}x
		\end{align*}
		Since this holds for all $x \in \R^n$, it follows that $B = A^{-1}$, and hence $A^{-1}$ is also a right inverse of $A$. Note now that, for any $x, y \in \R^n$ and $c \in \R$, $x = Az_1$ for some $z_1 \in \R^n$ and $y = Az_2$ for some $z_2 \in \R^n$ (since $A$ is onto), and so
		\begin{align*}
			A^{-1}(cx+y) = A^{-1}(cAz_1 + Az_2) = A^{-1}(A(cz_1) + Az_2) = A^{-1}(A(cz_1 + z_2)) = cz_1 + z_2 = cA^{-1}x + A^{-1}y
		\end{align*}
		Since $A^{-1}x = A^{-1}Az_1 = z_1$, and similarly $A^{-1}y = z_2$.
		
		\item Let $x, y \in \R^n$, and suppose that $Ax = Ay$. By lineararity, $A(x-y) = 0$. Then $x-y = 0$, which says that $x = y$, so $A$ is indeed 1--1.
		
		\item Let $A: \R^n \to \R^m$ be any linear transformation, $\mathrm{null}(A)$ be its nullspace, and $\mathrm{range}(A)$ be its range. First, let $x, y \in \mathrm{null}(A)$ and $c \in \R$. Then,
		\begin{align*}
			A(cx + y) = cAx + Ay = c \cdot 0 + 0 = 0 + 0 = 0
		\end{align*}
		So $cx + y \in \mathrm{null}(A)$, which shows that $\mathrm{null}(A)$ is a vector space. Similarly, let $x, y \in \mathrm{range}(A)$, and $c \in \R$. Then $x = Az_1$ and $y = Az_2$ for some $z_1, z_2 \in \R^n$. Now,
		\begin{align*}
			cx + y = cAz_1 + Az_2 = A(cz_1 + z_2)
		\end{align*}
		So, $cx + y$ is the image of $cz_1 + z_2$ which shows that it is in the range.
		
		\item Let $a_i = Ae_i$ for $i \in \set{1, \ldots, n}$ where $e_i \in \R^n$ is the vector with a 1 in position $i$ and a 0 everywhere else. We notice that for any vector $x = \begin{pmatrix}
			x_1 \\
			\vdots \\
			x_n
		\end{pmatrix} \in \R^n$, we have that $x = \sum_{i=1}^n x_ie_i$, and since $A$ is linear,
		\begin{align*}
			Ax = A\qty(\sum_{i=1}^n x_ie_i) = \sum_{i=1}^n x_i A(e_i) = \sum_{i=1}^n x_i a_i
		\end{align*}
		If we let $y = \begin{pmatrix}
			a_1 \\
			\vdots \\
			a_n
		\end{pmatrix}$, we know that 
		\begin{align*}
			x \cdot y = \sum_{i=1}^n x_ia_i
		\end{align*}
		which shows that $Ax = x \cdot y$ for every $x$. Note also that $y$ is unique, since $A$ is a function (i.e., $Ae_i$ can only have 1 value). If $\mg{y} = 0$, then $y = 0$ and so $A = 0$ since it maps everything to 0, and then given any vector $x \in \R^n$ with $\mg{x} = 1$, we have that $\mg{Ax} = \mg{0} = 0$, and since $\mg{A} \geq 0$, it follows that $\mg{A} = 0$. Indeed, $\mg{A} = 0 = \mg{y}$. So now suppose that $\mg{y} \neq 0$.
		
		 We also notice that (by Cauchy-Schwartz),
		\begin{align*}
			\mg{Ax} = \mg{x \cdot y} \leq \mg{x} \cdot \mg{y}
		\end{align*}
		Taking the sup over all $x \in \R^n$ with $\mg{x} = 1$ yields $\mg{A} \leq \mg{y}$. We also note that since $\mg{y} \neq 0$, $y/\mg{y}$ is a vector of norm 1, and
		\begin{align*}
			\frac{1}{\mg{y}} \mg{Ay} = \mg{A \frac{y}{\mg{y}}} = \frac{y}{\mg{y}} \cdot y = \frac{\mg{y}^2}{\mg{y}} = \mg{y}
		\end{align*}
		So $\mg{A} \geq \mg{y}$. Therefore, $\mg{A} = \mg{y}$.
		
		\item Suppose that $A \not \equiv 0$. Then there is some $x \in \R^n$ so that $Ax \neq 0$. It is clear that $x \neq 0$. Letting $z = x / \mg{x}$, we see that $\mg{z} = 1$ and that $Az = A x / \mg{x} = 1/\mg{x} Ax \neq 0$, since $\frac{1}{\mg{x}} \neq 0$. Since $\mg{\cdot}$ is a norm, we see that $\mg{Az} > 0$. Since
		\begin{align*}
			\mg{Ax - Ay} < C\mg{x-y}^r
		\end{align*}
		holds for all $x, y \in \R^n$. We may choose $y = 0$ to see that
		\begin{align*}
			\mg{Ax} = \mg{Ax - A0} < C\mg{x-0}^r = C\mg{x}^r
		\end{align*}
		holds for all $x \in \R^n$ and $c \in \R$. In particular, for all $\ve > 0$ we have that
		\begin{align*}
			\ve \mg{Az} = \mg{A(\ve z)} < C \mg{\ve z}^r = C\ve^r \cdot \mg{z}^r = C\ve^r
		\end{align*}
		Which tells us that 
		\begin{align*}
			\mg{Az} < C \ve^{r-1}
		\end{align*}
		Choosing $\ve < \qty(\frac{\mg{Az}}{C})^{\frac{1}{r-1}}$ yields a contradiction (Notice: since the power $(r-1)^{-1} > 0$, it follows that $\ve^{r-1} < \frac{\mg{Az}}{C}$ since $x^{\frac{1}{r-1}}$ is increasing on $x > 0$). 
	\end{enumerate}
\end{document}
