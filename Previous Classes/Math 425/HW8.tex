\documentclass[12pt]{article}
\usepackage[margin=1in]{geometry}
\usepackage{setspace}
\onehalfspacing

% Start of preamble
%==========================================================================================%
% Required to support mathematical unicode
\usepackage[warnunknown, fasterrors, mathletters]{ucs}
\usepackage[utf8x]{inputenc}

\usepackage[dvipsnames,table,xcdraw]{xcolor} % colors
\usepackage{hyperref} % links
\hypersetup{
	colorlinks=true,
	linkcolor=blue,
	filecolor=magenta,      
	urlcolor=cyan,
	pdfpagemode=FullScreen
}

% Standard mathematical typesetting packages
\usepackage{amsmath,amssymb,amscd,amsthm,amsxtra, pxfonts}
\usepackage{mathtools,mathrsfs,dsfont,xparse}

% Symbol and utility packages
\usepackage{cancel, textcomp}
\usepackage[mathscr]{euscript}
\usepackage[nointegrals]{wasysym}
\usepackage{apacite}

% Extras
\usepackage{physics}  % Lots of useful shortcuts and macros
\usepackage{tikz-cd}  % For drawing commutative diagrams easily
\usepackage{microtype}  % Minature font tweaks
%\usepackage{pgfplots} % plots

\usepackage{enumitem}
\usepackage{titling}

\usepackage{graphicx}

% Fancy theorems due to @intuitively on discord
\usepackage{mdframed}
\newmdtheoremenv[
backgroundcolor=NavyBlue!30,
linewidth=2pt,
linecolor=NavyBlue,
topline=false,
bottomline=false,
rightline=false,
innertopmargin=10pt,
innerbottommargin=10pt,
innerrightmargin=10pt,
innerleftmargin=10pt,
skipabove=\baselineskip,
skipbelow=\baselineskip
]{mytheorem}{Theorem}

\newenvironment{theorem}{\begin{mytheorem}}{\end{mytheorem}}

\newmdtheoremenv[
backgroundcolor=BurntOrange!30,
linewidth=2pt,
linecolor=BurntOrange,
topline=false,
bottomline=false,
rightline=false,
innertopmargin=10pt,
innerbottommargin=10pt,
innerrightmargin=10pt,
innerleftmargin=10pt,
skipabove=\baselineskip,
skipbelow=\baselineskip
]{mycorollary}{Corollary}

\newenvironment{corollary}{\begin{mycorollary}}{\end{mycorollary}}


\newmdtheoremenv[
backgroundcolor=OrangeRed!30,
linewidth=2pt,
linecolor=OrangeRed,
topline=false,
bottomline=false,
rightline=false,
innertopmargin=10pt,
innerbottommargin=10pt,
innerrightmargin=10pt,
innerleftmargin=10pt,
skipabove=\baselineskip,
skipbelow=\baselineskip
]{mylemma}{Lemma}

\newenvironment{lemma}{\begin{mylemma}}{\end{mylemma}}

\newtheoremstyle{definitionstyle}
{\topsep}%
{\topsep}%
{}%
{}%
{\bfseries}%
{.}%
{.5em}%
{}%
\theoremstyle{definitionstyle}
\newmdtheoremenv[
backgroundcolor=Violet!30,
linewidth=2pt,
linecolor=Violet,
topline=false,
bottomline=false,
rightline=false,
innertopmargin=10pt,
innerbottommargin=10pt,
innerrightmargin=10pt,
innerleftmargin=10pt,
skipabove=\baselineskip,
skipbelow=\baselineskip,
]{mydef}{Definition}
\newenvironment{definition}{\begin{mydef}}{\end{mydef}}

\newtheorem*{remark}{Remark}

\newtheorem*{example}{Example}

% Common shortcuts
\def\mbb#1{\mathbb{#1}}
\def\mfk#1{\mathfrak{#1}}

\def\bN{\mbb{N}}
\def \C{\mbb{C}}
\def \R{\mbb{R}}
\def\bQ{\mbb{Q}}
\def\bZ{\mbb{Z}}
\def \cph{\varphi}
\renewcommand{\th}{\theta}
\def \ve{\varepsilon}
\newcommand{\mg}[1]{\| #1 \|}

% Sometimes helpful macros
\newcommand{\floor}[1]{\left\lfloor#1\right\rfloor}
\newcommand{\ceil}[1]{\left\lceil#1\right\rceil}
\renewcommand{\qed}{\hfill\qedsymbol}

% Sets
\DeclarePairedDelimiterX\set[1]\lbrace\rbrace{\def\given{\;\delimsize\vert\;}#1}

% End of preamble
%==========================================================================================%

% Start of commands specific to this file
%==========================================================================================%

\renewcommand{\ip}[2]{\langle #1, #2 \rangle}
\newcommand{\linf}[1]{\max_{1\leq i \leq #1}}
\newcommand{\seq}[2]{\qty(#1_#2)_{#2=1}^{\infty}}

%==========================================================================================%
% End of commands specific to this file

\title{Math 425 HW8}
\date{\today}
\author{Rohan Mukherjee}

\begin{document}
	\maketitle
	\begin{enumerate}[leftmargin=\labelsep]
		\item Suppose that $f$ has a local maximum at $\textbf{x} \in E$ and that $f'(\textbf{x}) \neq 0$. Then, $\qty|f'(\textbf{x})| > 0$. Now let $\ve > 0$. We expand $f$ as:
		\begin{align*}
			f(\textbf{x}+\textbf{h}) = f(\textbf{x}) + \grad{f}(\mathbf{x}) \cdot \textbf{h} + E(h)
		\end{align*}
		So that $\qty|{E(\textbf{h})/\textbf{h}}| \to 0$. Now let $\ve > 0$. We shall consider $\textbf{h} = \ve f'(\mathbf{x}) \neq 0$ since $f'(\mathbf{x}) \neq 0$. Indeed, we see that 
		\begin{align*}
			f(x+\ve f'(\mathbf{x})) = f(\textbf{x}) + f'(\mathbf{x}) \cdot f'(\mathbf{x}) \ve + E(f'(\mathbf{x}) \cdot \ve) = f(x) + \ve\qty|f(x)|^2 + E(f'(\mathbf{x}) \cdot \ve)
		\end{align*}
		Also, since $\qty|E(\textbf{h})/\textbf{h}| \to 0$, we can find a $\delta > 0$ so that $\qty|E(\textbf{h})/\textbf{h}| < \qty|f'(\textbf{x})|/2$. It follows that $\qty|E(\textbf{h})| < \frac12\qty|f'(\textbf{x}) \textbf{h}|$ for all $\qty|\textbf{h}| < \delta$. Now, for all $\ve < \delta / \qty|\grad{f}(\textbf{x})|$, we have that $\qty|E(\ve f'(\textbf{x}))| < \frac{\ve}{2} \qty|f(\textbf{x})|^2$, and in particular, $E(\ve \grad{f}(\textbf{x})) > -\ve/2 |f(\textbf{x})|^2$. It follows then that
		\begin{align*}
			f(x+\ve f'(\textbf{x})) > f(\textbf{x}) + \frac{\ve}2 |f(\textbf{x})|^2
		\end{align*}
		So it cannot be the case that $\textbf{x}$ was a local minimum of $f$, a contradiction. Thus, $f'(\textbf{x}) = 0$.
		
		\item Fix $x \in E$, and let $A = f^{-1}(\set{f(x)})$, and $B = A^c$, i.e. $B = \set{b \in E \given f(b) \neq f(x)}$. Suppose that $f$ is non-constant, i.e. that $B$ is nonempty. We shall show that $A$ is a clopen subset of $E$, which will immediately show that $E$ cannot be connected. First, $A$ is obviously closed since it is the preimage of a closed set (singletons are closed sets). $B$ is therefore open. Suppose that $B$ were not closed. Then there would be some sequence $y_n \to y$ so that $y_n \in B$ and $y \not \in B$, i.e. $y \in A$. Now find $N$ sufficiently large so that $|y_n - y| < 1$ for all $n \geq N$. Now we shall consider $N_1(y)$. First, we prove the following lemma:
		\begin{lemma}
			Let $r > 0$ and $x \in \R^n$. Then $N_r(x)$ is convex.
		\end{lemma}
		\begin{proof}
			Letting $x, y \in N_r(x)$, we see for any $0 \leq t \leq 1$,
			\begin{align*}
				\mg{tx+(1-t)y} \leq t\mg{x} + (1-t)\mg{y} \leq tr + (1-t)r = r
			\end{align*}
		Which completes the proof.
		\end{proof}
		So now let $z \in N_1(y)$. Since $N_1(y)$ is convex, we may apply the mean value theorem to see that, for all $i \in \set{1, \ldots, n}$,
		\begin{align*}
			e_i \cdot (f(z) - f(y)) = e_i \cdot \qty[f'(\xi)(b-a)]
		\end{align*}
		for some $\xi \in \R^n$. Now, 
		\begin{align*}
			 \mg{e_i \cdot \qty[f'(\xi)(b-a)]} \leq 1 \cdot \mg{f'(\xi)(b-a)} \leq \mg{f'(\xi)} \cdot \mg{b-a} = 0
		\end{align*}
		This shows that the $i$th coordinate of $f(z) - f(y)$ is 0. Since this holds for all $i \in \set{1, \ldots, n}$, $f(z) = f(y)$. But now, since $|y_N - y| < 1$, we have that $f(y_N) = f(y) = f(x)$, a contradiction, since $y_N \not \in A$. So, looking back at what we were contradicting, we see that $f$ is constant.
		
		
		\item Let $f, g: \R^n \to \R$. We notice, by the regular product rule, that
		\begin{align*}
			\pdv{x_i} fg = g\pdv{x_i} f + f \pdv{x_i}g
		\end{align*}
		We therefore see that
		\begin{align*}
			\grad(fg) = \begin{pmatrix}
				\pdv{x_1} fg \\
				\ldots \\
				\pdv{x_n} fg
			\end{pmatrix} = \begin{pmatrix}
				g\pdv{x_1} f + f \pdv{x_1}g \\
				\ldots \\
				g \pdv{x_n}f + f \pdv{x_n} g
			\end{pmatrix} = g\begin{pmatrix}
				\pdv{x_1} f \\
				\ldots \\
				\pdv{x_n} f
			\end{pmatrix} + f \begin{pmatrix}
				\pdv{x_1} g \\
				\ldots \\
				\pdv{x_n} g
			\end{pmatrix}
			= g \grad f + f \grad g
		\end{align*}
		Similarly, if $f \neq 0$, $\dv{x_i} \frac1f = \frac{-1}{f^2} \cdot \pdv{f}{x_i}$ by the regular quotient rule. It follows then that
		\begin{align*}
			\grad(1/f) = \begin{pmatrix}
				\frac{-1}{f^2} \pdv{f}{x_1} \\
				\ldots \\
				\frac{-1}{f^2} \pdv{f}{x_n}
			\end{pmatrix}
			= \frac{-1}{f^2} \grad f
		\end{align*}
	
		\item 
		\begin{align*}
			f'(0) = \lim_{h \to 0} \frac{h + 2h^2\sin(\frac1h) - 0}{h} = 1 + \lim_{h \to 0} 2h\sin(\frac1h)
		\end{align*}
		Now,
		\begin{align*}
			\qty|2h\sin(\frac1h)| \leq 2h
		\end{align*}
		And so $\qty|2h\sin(\frac1h)| \to 0$, which shows that $f'(0) = 1$. For $t \neq 0$, we can calculate $f'$ using regular derivative rules as
		\begin{align*}
			f'(t) = 1 - 2\cdot \cos(\frac1t) + 4t \sin(\frac1t) 
		\end{align*}
		By the triangle inequality, this is bounded above by $1 + 2 + 4 = 7$ on $(-1, 1) \setminus 0$, and we already showed it equals 1 at 0. First we notice that $f'\qty(\frac{1}{2\pi n}) = -1$ for all $n \in \bN$. Fix a neighborhood of 0, say $(-t, t)$. First find $n > 0$ so that $\frac{1}{2\pi n} < t$. We also notice that $\color{red}{f\qty(\frac{1}{2\pi n}) = \frac{1}{2\pi n}}$ at all $n \in \bN$. We know that $f$ is continuously differentiable outside of 0, in particular at $x = \frac{1}{2\pi (n+1)} < \frac{1}{2\pi n}$. So find a $\delta > 0$ so that $y \in (x-\delta, x)$ means $f'(y) - f'(x) < 1/2$ and $f(y) - f(x) < f\qty(\frac{1}{2\pi n}) - f(x)$ (which is \textcolor{red}{obviously} positive), in particular, $f(y) < f(\frac{1}{2\pi n})$, and also $f'(y) < -1/2$. Now choosing $z = x - \delta/2$, since $f'(y) < -1/2$ on all of $(x-\delta, x)$, it follows by the mean value theorem that $f(x) - f(z) = f'(\xi)(z-x)$ for some $\xi \in (x-\delta, x)$, and this tells us that $f(x) - f(z) \leq -\frac12(z-x) < 0$, so in particular $f(z) > f(x)$. One also notes that $f(z) < f(\frac{1}{2\pi n})$ by how we defined $\delta$. Then $f(z) \in (f(x), f(\frac{1}{2\pi n}))$, and by the intermediate value theorem there is some $\eta \in (x, \frac{1}{2\pi n})$ so that $f(z) = f(x)$. This tells us that $f$ is not injective, which completes the proof.
	\end{enumerate}
\end{document}
