\documentclass[12pt]{article}
\usepackage[margin=1in]{geometry}
\usepackage{setspace}
\onehalfspacing

% Start of preamble
%==========================================================================================%
% Required to support mathematical unicode
\usepackage[warnunknown, fasterrors, mathletters]{ucs}
\usepackage[utf8x]{inputenc}

% Always typeset math in display style
%\everymath{\displaystyle}

% GROUPOIDS FONT!
\usepackage{eulervm}
\usepackage{charter}

% Standard mathematical typesetting packages
\usepackage{amsthm, amsmath, amssymb}
\usepackage{mathtools}  % Extension to amsmath

% Symbol and utility packages
\usepackage{cancel, textcomp}
\usepackage[mathscr]{euscript}
\usepackage[nointegrals]{wasysym}

% Extras
\usepackage{physics}  % Lots of useful shortcuts and macros
% `tikz-cd` is necessary to draw commutative diagrams.
\RequirePackage{tikz-cd}
% `amssymb` is necessary for `\lrcorner` and `\ulcorner`.
\RequirePackage{amssymb}
% `calc` is necessary to draw curved arrows.
\usetikzlibrary{calc}
% `pathmorphing` is necessary to draw squiggly arrows.
\usetikzlibrary{decorations.pathmorphing}

% A TikZ style for curved arrows of a fixed height, due to AndréC.
\tikzset{curve/.style={settings={#1},to path={(\tikztostart)
			.. controls ($(\tikztostart)!\pv{pos}!(\tikztotarget)!\pv{height}!270:(\tikztotarget)$)
			and ($(\tikztostart)!1-\pv{pos}!(\tikztotarget)!\pv{height}!270:(\tikztotarget)$)
			.. (\tikztotarget)\tikztonodes}},
	settings/.code={\tikzset{quiver/.cd,#1}
		\def\pv##1{\pgfkeysvalueof{/tikz/quiver/##1}}},
	quiver/.cd,pos/.initial=0.35,height/.initial=0}

% TikZ arrowhead/tail styles.
\tikzset{tail reversed/.code={\pgfsetarrowsstart{tikzcd to}}}
\tikzset{2tail/.code={\pgfsetarrowsstart{Implies[reversed]}}}
\tikzset{2tail reversed/.code={\pgfsetarrowsstart{Implies}}}
% TikZ arrow styles.
\tikzset{no body/.style={/tikz/dash pattern=on 0 off 1mm}}

\usepackage{color}  % Add some color to life
\usepackage{microtype}  % Minature font tweaks
%\usepackage{pgfplots} % plots

\usepackage{enumitem}
\usepackage{titling}

\usepackage{graphicx}
\usepackage{xcolor}

% Common shortcuts
\def\mbb#1{\mathbb{#1}}
\def\mfk#1{\mathfrak{#1}}

\def\bN{\mbb{N}}
\def\bC{\mbb{C}}
\def\bR{\mbb{R}}
\def\bQ{\mbb{Q}}
\def\bZ{\mbb{Z}}

% Sometimes helpful macros
\newcommand{\floor}[1]{\left\lfloor#1\right\rfloor}
\newcommand{\ceil}[1]{\left\lceil#1\right\rceil}
\DeclarePairedDelimiterX\set[1]\lbrace\rbrace{\def\given{\;\delimsize\vert\;}#1}

% Some standard theorem definitions
\newtheorem{theorem}{Theorem}[section]
\newtheorem{corollary}{Corollary}[theorem]
\newtheorem{lemma}[theorem]{Lemma}

\theoremstyle{definition}
\newtheorem{definition}{Definition}[section]

\theoremstyle{remark}
\newtheorem*{remark}{Remark}

% End of preamble
%==========================================================================================%

\newcommand{\A}{\textcolor{magenta}{A}}
\newcommand{\B}{\textcolor{blue}{B}}

% Start of commands specific to this file
%==========================================================================================%

\newcommand{\R}{\mathbb{R}}
\renewcommand{\ip}[2]{\langle #1, #2 \rangle}
\newcommand{\mg}[1]{\| #1 \|}
\newcommand{\linf}[1]{\max_{1\leq i \leq #1}}
\newcommand{\ve}{\varepsilon}
\renewcommand{\qed}{\hfill\qedsymbol}
\newcommand{\seq}[2]{\qty(#1_#2)_{#2=1}^{\infty}}
\newcommand\setItemnumber[1]{\setcounter{enumi}{\numexpr#1-1\relax}}
\newcommand{\justif}[1]{&\quad &\text{(#1)}}
\newcommand{\ra}{\rightarrow}
\newcommand{\E}{\mbb E}
\renewcommand{\P}{\mbb P \qty}


%==========================================================================================%
% End of commands specific to this file

\title{CSE 521 HW1}
\date{\today}
\author{Rohan Mukherjee}

\begin{document}
	\maketitle
	\textbf{\huge{Collaborators: Lukshya Ganjoo, Alex Albors Juez}}
	
	\begin{enumerate}[leftmargin=\labelsep]
		\item Let $G = (V, E)$ be an undirected graph with $n = |V|$ vertices, let $k = \min_{S \subset V} |E(S, V \setminus S)|$ be the size of the min cut, and let $\alpha$ be fixed and sufficiently small (i.e., $k\alpha \leq |E|$, note that this forces $\alpha \leq n/2$ by the hand-shake lemma). Let $(S, V \setminus S)$ be an arbitrary $\alpha$-approximate min cut. We show that
		\begin{align*}
			\P[\text{Karger's algorithm finds } (S, V \setminus S)] \geq \frac{\Theta(\alpha!)}{n^{2\alpha - 1}}
		\end{align*}
		
		First, notice that
		\begin{align*}
			\P[e \in E(S, V \setminus S)] = \frac{|E(S, V \setminus S)|}{|E|} \leq \frac{\alpha k}{|E|} \leq \frac{\alpha k}{nk/2} = \frac{2\alpha}{n}
		\end{align*}
		Letting $A_i$ denote the event that the uniformly random edge chosen by Krager's algorithm is not in $E(S, V \setminus S)$, we see that
		\begin{align*}
			\P[\text{alg finds } (S, V \setminus S)] &= \P[A_1 \mid A_2] \cdot \P[A_2 \mid A_1] \cdots \P[A_{n-3} \mid A_1, A_2, \ldots, A_{n-3}] \\
			&= \qty(1 - \frac{2\alpha}{n}) \cdot \qty(1-\frac{2\alpha - 1}{n-1}) \cdots \qty(1 - \frac{2\alpha}{4})\qty(1 - \frac{2\alpha}{3}) \\
			&= \frac{n-2\alpha}{n} \cdot \frac{n-2\alpha - 1}{n-1} \cdots \frac{4-2\alpha}{4} \cdot \frac{3 - 2\alpha}{3} 
		\end{align*}
		Extracting the first $2\alpha-1$ factors from the bottom, we can cancel $n - 2 - (2\alpha-1)$ factors that become $\geq 1$, and we have the remaining $2\alpha-1$ factors on top that equate to something $\Theta(\alpha!)$. We get that this probability is $\geq \Theta(\alpha!)/n^{2\alpha-1}$. We see there must be at most $\leq n^{2\alpha-1}/\Theta(\alpha!) \leq n^{2\alpha}$ $\alpha$-approximate min cuts.
		
		More detail:
		\begin{align*}
			&= \prod_{i=0}^{n-3} \frac{n-2\alpha-i}{n-i} \\
			&= \prod_{0 \leq i < 2\alpha} \frac{1}{n-i} \cdot \prod_{0 \leq i \leq n-3-2\alpha} \frac{n-2\alpha-i}{n-\floor{2\alpha}-i} \prod_{0 \leq i < 2\alpha} (i+3-2\alpha) \\
			&\geq \frac{\Theta(\alpha!)}{n^{2\alpha-1}}
		\end{align*}
		Basically, we have around $2\alpha-1$ factors on the bottom at the start that don't cancel, we have $n-2-2\alpha$ factors in the top/bottom that cancel to something $\geq 1$, and we have the remaining $2\alpha-1$ factors on top that turn into $\Theta(\alpha!)$. Since probabilities are $\leq 1$ we must have $\leq \frac{n^{2\alpha}}{\Theta(\alpha!)} \leq n^{2\alpha}$ $\alpha$-approximate min cuts for large enough $\alpha$. For the reader/grader, does the first level of detail work? This was sort of a pain to work through the algebra.
		
		\item We shall construct a graph where, if the algorithm cuts one particular edge, the algorithm fails to find the min s-t cut. The first type of graph that came to my mind is the following:
		\[\begin{tikzcd}
			&& {v_1} \\
			\\
			t && {v_2} && s \\
			\\
			&& {v_3}
			\arrow[curve={height=6pt}, no head, from=3-5, to=3-3]
			\arrow[curve={height=-6pt}, no head, from=3-5, to=3-3]
			\arrow[no head, from=3-1, to=3-3]
			\arrow[curve={height=6pt}, no head, from=1-3, to=3-5]
			\arrow[curve={height=6pt}, no head, from=3-5, to=1-3]
			\arrow[curve={height=6pt}, no head, from=3-5, to=5-3]
			\arrow[curve={height=-6pt}, no head, from=3-5, to=5-3]
			\arrow[no head, from=3-1, to=1-3]
			\arrow[no head, from=3-1, to=5-3]
		\end{tikzcd}\]
		We could have any number of the $v_n$'s. The crucial part to the proof is that if we contract an edge of the form $tv_i$, 2 new edges go from $s$ to the new supernode $tv_i$. This is a contradiction however as you could've broken the path by simply cutting the $tv_i$ edge, not contracting it (which would result in 1 less edge). We see that the algorithm fails in this case iff it contracts an edge of the form $tv_i$, of which there are $n/3$ of those (if the number of edges in the graph is $n$). So, defining $A_i$ to be the event that the algorithm doesn't contract any of the $tv_\alpha$ edges in step $i$, for $i = 1$ to $n/3$, we would get that
		$\P[A_1] = 1 - \frac{n/3}{n}$, $\P[A_2 \mid A_1] = 1 - \frac{n/3}{n-1}$, and in general $\P[A_i \mid A_1, \ldots, A_{i-1}] = 1 - \frac{n/3}{n-i+1} = \frac{2n/3-i+1}{n-i+1}$. Putting this together gives us
		\begin{align*}
			\P[\text{alg doesn't fail}] &= \P[A_1] \cdot \P[A_2 \mid A_1] \cdots \P[A_{n/3} \mid A_1, \ldots, A_{n/3-1}] \\
			&= \frac{2n/3}{n} \cdot \frac{2n/3-1}{n-1} \cdots \frac{n/3+1}{2n/3+1} \\
			&= \frac{(2n/3)!}{n!} \cdot \frac{(2n/3)!}{(n/3)!} \approx \frac{(2n/3e)^{4n/3}}{(n/e)^n(n/3e)^{n/3}} \\
			&= \frac{(2n)^{4n/3} \cdot (1/3e)^{4n/3}}{n^{4n/3} \cdot e^n \cdot (1/3e)^{n/3}} \\
			&= \frac{2^{4n/3}}{3^n} \approx 0.839947665^n
		\end{align*}
		Indeed, a quite small exponential decay. Also, by utilizing the following graph:
		\[\begin{tikzcd}
			& \bullet && \bullet && \bullet \\
			t & \bullet && \bullet && \bullet & s \\
			& \bullet && \bullet && \bullet
			\arrow[no head, from=2-1, to=3-2]
			\arrow[no head, from=3-2, to=3-4]
			\arrow[no head, from=3-4, to=3-6]
			\arrow[no head, from=3-6, to=2-7]
			\arrow[no head, from=2-6, to=2-7]
			\arrow[no head, from=2-4, to=2-6]
			\arrow[no head, from=2-2, to=2-4]
			\arrow[no head, from=2-1, to=2-2]
			\arrow[no head, from=2-1, to=1-2]
			\arrow[no head, from=1-2, to=1-4]
			\arrow[no head, from=1-4, to=1-6]
			\arrow[no head, from=1-6, to=2-7]
		\end{tikzcd}\]
		We can show that the probability this algorithm finds a specific $s-t$ cut is very small. For this graph, an $s-t$ cut can be found by cutting 1 edge on the 1st path, 1 on the second, and one on the third. If we let $n$ be the number of edges, we can do this in $(n/3)^3$ ways. In general, for any $n$ and any $k$, we could get at least $(n/k)^k = \Theta(n^k)$ (the only important variable being $n$) $s-t$ cuts for the generalized version of this graph. Thus the probability our algorithm finds one specific $s-t$ cut is no more than $1/\Theta(n^k)$ for every $k$. Although I am not sure, I believe this tells us that the probability it finds a specific $s-t$ cut is necessarily exponential since it is below all polynomials.
	\end{enumerate}
\end{document}
