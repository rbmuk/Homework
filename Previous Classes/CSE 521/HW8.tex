\documentclass[12pt]{article}
\usepackage[margin=1in]{geometry}

% Start of preamble
%==========================================================================================%
% Required to support mathematical unicode
\usepackage[warnunknown, fasterrors, mathletters]{ucs}
\usepackage[utf8x]{inputenc}

\usepackage[dvipsnames,table,xcdraw]{xcolor}
\usepackage{hyperref} 
\hypersetup{
	colorlinks=true,
	linkcolor=blue,
	filecolor=magenta,      
	urlcolor=cyan,
	pdfpagemode=FullScreen
}

% Standard mathematical typesetting packages
\usepackage{amsmath,amscd,amsthm,amsxtra, pxfonts}
\usepackage{mathtools,mathrsfs,dsfont,xparse}

% Symbol and utility packages
\usepackage{cancel, textcomp}
\usepackage[nointegrals]{wasysym}
\usepackage{apacite}

% Extras
\usepackage{physics}  
\usepackage{tikz-cd} 
\usepackage{microtype}
\usepackage{enumitem}
\usepackage{titling}
\usepackage{graphicx}

% Fancy theorems due to @intuitively on discord
\usepackage{mdframed}
\newmdtheoremenv[
backgroundcolor=NavyBlue!30,
linewidth=2pt,
linecolor=NavyBlue,
topline=false,
bottomline=false,
rightline=false,
innertopmargin=10pt,
innerbottommargin=10pt,
innerrightmargin=10pt,
innerleftmargin=10pt,
skipabove=\baselineskip,
skipbelow=\baselineskip
]{mytheorem}{Theorem}

\newenvironment{theorem}{\begin{mytheorem}}{\end{mytheorem}}

\newtheorem{corollary}{Corollary}
\newtheorem{lemma}{Lemma}

\newtheoremstyle{definitionstyle}
{\topsep}%
{\topsep}%
{}%
{}%
{\bfseries}%
{.}%
{.5em}%
{}%
\theoremstyle{definitionstyle}
\newmdtheoremenv[
backgroundcolor=Violet!30,
linewidth=2pt,
linecolor=Violet,
topline=false,
bottomline=false,
rightline=false,
innertopmargin=10pt,
innerbottommargin=10pt,
innerrightmargin=10pt,
innerleftmargin=10pt,
skipabove=\baselineskip,
skipbelow=\baselineskip,
]{mydef}{Definition}
\newenvironment{definition}{\begin{mydef}}{\end{mydef}}

\newtheorem*{remark}{Remark}

\newtheorem*{example}{Example}

% Common shortcuts
\def\mbb#1{\mathbb{#1}}
\def\mfk#1{\mathfrak{#1}}

\def\bN{\mbb{N}}
\def \C{\mbb{C}}
\def \R{\mbb{R}}
\def\bQ{\mbb{Q}}
\def\bZ{\mbb{Z}}
\def \cph{\varphi}
\renewcommand{\th}{\theta}
\def \ve{\varepsilon}
\newcommand{\mg}[1]{\| #1 \|}

% Often helpful macros
\newcommand{\floor}[1]{\left\lfloor#1\right\rfloor}
\newcommand{\ceil}[1]{\left\lceil#1\right\rceil}
\renewcommand{\qed}{\hfill\qedsymbol}
\renewcommand{\P}{\mathrm{Pr}\qty}
\newcommand{\E}{\mathbb{E}\qty}

% Sets
\usepackage{braket}

% Code
\usepackage{listings}
\usepackage{color}

\definecolor{dkgreen}{rgb}{0,0.6,0}
\definecolor{gray}{rgb}{0.5,0.5,0.5}
\definecolor{mauve}{rgb}{0.58,0,0.82}

\lstset{frame=tb,
	language=Python,
	aboveskip=1mm,
	belowskip=1mm,
	showstringspaces=false,
	columns=flexible,
	basicstyle={\small\ttfamily},
	numbers=none,
	numberstyle=\tiny\color{gray},
	keywordstyle=\color{blue},
	commentstyle=\color{dkgreen},
	stringstyle=\color{mauve},
	breaklines=true,
	breakatwhitespace=true,
	tabsize=3
}

% End of preamble
%==========================================================================================%

% Start of commands specific to this file
%==========================================================================================%

\usepackage{adjustbox}
\usepackage{algorithm}
\usepackage{algpseudocode}
\usepackage{tikz}
\renewcommand{\ip}[1]{\langle #1 \rangle}

%==========================================================================================%
% End of commands specific to this file

\title{CSE 521 Last Homework}
\date{\today}
\author{Rohan Mukherjee, Lukshya Ganjoo}

\begin{document}
	\maketitle
	\begin{enumerate}[leftmargin=\labelsep]
		\item Recall that $x \geq |y|$ iff $x \geq y$ and $x \geq -y$. This motivates the following: $A \in \R^{5 \times 7}$ shall be the matrix with $|\ve_{ij}|$ in entry $ij$, $a \in \R^5$ shall be the aptitude of the people, and $e \in \R^7$ shall be the easiness of the classes. Next, we shall set the entries of $G$ to be 0 if they are undefined. So, we do not want to include the error in those spots. Therefore, define $F_{ij} = \mbb I_{G_{ij} > 0}$. Notice that $\tr(AF^T) = \sum_{i,j} |\ve_{ij}| \cdot \mbb I_{G_{ij} > 0}$ which is precisely the quantity we want. Finally, our other constraints shall be $G - a\mathbf 1^T - \mathbf 1e^T \leq A$ and $-G + a\mathbf 1^T + \mathbf 1e^T \leq A$, since the $ij$ coordinate the first is just $G_{ij} - a_i - e_j \leq A_{ij}$ and the second is $-(G_{ij} + a_i + e_j) \leq A_{ij}$, which describes exactly what we want. Note also that since the only entries of $A$ that affect the sum are the ones where $G_{ij}$ is defined, the extra constraints on the other coordinates do not matter.
		
		So, we have the following linear program:
		\begin{align*}
			\text{minimize}& \quad \tr(AF^T) \\
			\text{s.t. }& \quad  G - a\mathbf 1^T - \mathbf 1e^T \leq A \\
			&-(G - a\mathbf 1^T - \mathbf 1e^T) \leq A
		\end{align*}
	
		Here is my code:
		\begin{lstlisting}[language=matlab]
			G = [0 2.66 3 3.33 3.66 2 0; 2.66 3.66 0 0 4.33 1.66 3
			2.66 0 3.33 0 3.66 3 3.33; 4.33 0 2.66 4 0 3.66 0;
			0 2.66 1.33 3.33 0 3 2.33];
			Filter = transpose(G > 0);
			cvx_begin
				variables T(5,7) a(5) e(1,7)
				minimize trace(T*Filter)
				subject to
					G - a*ones(1, 7) - ones(5, 1)*e <= T;
					-G  + a*ones(1, 7) + ones(5, 1)*e <= T;
					T >= 0;
			cvx_end
		\end{lstlisting}
	
		I got the calculated easiness to be:
		0.960037046829499
		
		1.63003704798536
		
		1.03457764835811
		
		2.30003704582580
		
		2.63003704559624
		
		1.17342758396096
		
		1.30003704812862
		
		and the aptitudes of the students to be:
		1.02996295373911
		
		1.69996295314997
		
		1.82657241603184
		
		2.17422512162083
		
		1.02996295245312
		
		\item We construct a convex program. let $|E| = m$, and define to edge $k$, where $k$ connects vertex $i$ with vertex $j$,
		\begin{align*}
			a^k_l = \begin{cases}
				1, \; l = i \\
				-1, \; l = j \\
				0, \; \text{o.w}
			\end{cases}
		\end{align*}
		Finally define $L(w) = \sum_{k=1}^m w_k a^k (a^k)^T = A\mathrm{diag}(w)A^T$, where $A = [a_1 \ldots a_m]$. We claim the function $f(w) = \lambda_2(L(w)) = \min_{\ip{x, \textbf{1}} = 0} \frac{x^TLx}{x^Tx}$ is concave. Notice that $\mathrm{diag}((w+v)/2) = \frac12 (\mathrm{diag}(w) + \mathrm{diag}(v))$, so,
		\begin{align*}
			f\qty(\frac{w+v}{2}) &= \min_{\ip{x, \mathbf 1} = 0} \frac12 \frac{x^T(\mathrm{diag}(w) + \mathrm{diag}(v))x}{x^Tx} \\
			&= \frac12 \min_{\ip{x, \mathbf 1} = 0} \qty( \frac{x^T\mathrm{diag}(w)x}{x^Tx} + \frac{x^T\mathrm{diag}(v)x}{x^Tx}) \geq \frac12 \qty(\min_{\ip{x, \mathbf 1} = 0}  \frac{x^T\mathrm{diag}(w)x}{x^Tx} + \min_{\ip{x, \mathbf 1} = 0}  \frac{x^T\mathrm{diag}(v)x}{x^Tx}) \\
			&= \frac12(f(v)+f(w))
		\end{align*}
		We can now formulate the following linear program.
		\begin{align*}
			\text{maximize} \quad f(w) \\
			\text{s.t.} \quad \mathrm{diag}(L) = \mathbf 1
		\end{align*}
		Since $\mathrm{diag}(L)$ is a linear function in $w$ as we showed above, the constraint is simply a linear system of equalities, where we are trying to maximize a convex function. The above is a convex program, and hence can be solved using the ellipsoid method in polynomial time. We now prove the correctness of this algorithm. We claim that $I - L$ is the adjacency matrix of the weighted graph. Notice first that the diagonal entry is going to be the sum of the weights of the edges connecting to vertex $i$, and by our constraint this diagonal entry is always 1. So, $I-L$ has diagonal entries all 0s. The $ij$ entry of this matrix is 0 if there is no edge connecting $i$ to $j$ by construction, and $(-w_l \cdot a^k(a^k)^T)_{ij}$ if there is an edge connecting $i$ to $j$, and this quantity is just $w_l$ since $a^k$ will be 1 in the $i$th coordinate and $-1$ in the $j$th coordinate. Now, $L$ is symmetric, which proves the previous claim. Finally, the second largest eigenvalue of $I-L$ is the second smallest eigenvalue of $L$, and it will be minimized when the second smallest eigenvalue of $L$ is maximized, which is what the above program does.
	\end{enumerate}
\end{document}