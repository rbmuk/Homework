\documentclass[12pt]{article}
\usepackage[margin=1in]{geometry}

% Start of preamble
%==========================================================================================%
% Required to support mathematical unicode
\usepackage[warnunknown, fasterrors, mathletters]{ucs}
\usepackage[utf8x]{inputenc}

\usepackage[dvipsnames,table,xcdraw]{xcolor}
\usepackage{hyperref} 
\hypersetup{
	colorlinks=true,
	linkcolor=blue,
	filecolor=magenta,      
	urlcolor=cyan,
	pdfpagemode=FullScreen
}

% Standard mathematical typesetting packages
\usepackage{amsmath,amssymb,amscd,amsthm,amsxtra}
\usepackage{mathtools,mathrsfs,dsfont,xparse}

% Symbol and utility packages
\usepackage{cancel, textcomp}
\usepackage[mathscr]{euscript}
\usepackage[nointegrals]{wasysym}
\usepackage{apacite}

% Extras
\usepackage{physics}  
\usepackage{tikz-cd} 
\usepackage{microtype}
\usepackage{enumitem}
\usepackage{titling}
\usepackage{graphicx}

% Fancy theorems due to @intuitively on discord
\usepackage{mdframed}
\newmdtheoremenv[
backgroundcolor=NavyBlue!30,
linewidth=2pt,
linecolor=NavyBlue,
topline=false,
bottomline=false,
rightline=false,
innertopmargin=10pt,
innerbottommargin=10pt,
innerrightmargin=10pt,
innerleftmargin=10pt,
skipabove=\baselineskip,
skipbelow=\baselineskip
]{mytheorem}{Theorem}

\newenvironment{theorem}{\begin{mytheorem}}{\end{mytheorem}}

\newtheorem{corollary}{Corollary}
\newtheorem{lemma}{Lemma}

\newtheoremstyle{definitionstyle}
{\topsep}%
{\topsep}%
{}%
{}%
{\bfseries}%
{.}%
{.5em}%
{}%
\theoremstyle{definitionstyle}
\newmdtheoremenv[
backgroundcolor=Violet!30,
linewidth=2pt,
linecolor=Violet,
topline=false,
bottomline=false,
rightline=false,
innertopmargin=10pt,
innerbottommargin=10pt,
innerrightmargin=10pt,
innerleftmargin=10pt,
skipabove=\baselineskip,
skipbelow=\baselineskip,
]{mydef}{Definition}
\newenvironment{definition}{\begin{mydef}}{\end{mydef}}

\newtheorem*{remark}{Remark}

\newtheorem*{example}{Example}

% Common shortcuts
\def\mbb#1{\mathbb{#1}}
\def\mfk#1{\mathfrak{#1}}

\def\bN{\mbb{N}}
\def \C{\mbb{C}}
\def \R{\mbb{R}}
\def\bQ{\mbb{Q}}
\def\bZ{\mbb{Z}}
\def \cph{\varphi}
\renewcommand{\th}{\theta}
\def \ve{\varepsilon}
\newcommand{\mg}[1]{\| #1 \|}

% Often helpful macros
\newcommand{\floor}[1]{\left\lfloor#1\right\rfloor}
\newcommand{\ceil}[1]{\left\lceil#1\right\rceil}
\renewcommand{\qed}{\hfill\qedsymbol}
\renewcommand{\P}{\mathrm{Pr}\qty}
\newcommand{\E}{\mathbb{E}\qty}

% Sets
\usepackage{braket}

% algorithms
\usepackage{algorithm}
\usepackage{algpseudocode}

% End of preamble
%==========================================================================================%

% Start of commands specific to this file
%==========================================================================================%

\renewcommand{\S}{\mbb S}

%==========================================================================================%
% End of commands specific to this file

\title{CSE 521 HW6 (Midterm)}
\date{\today}
\author{Rohan Mukherjee}

\begin{document}
	\maketitle
	\begin{enumerate}[leftmargin=\labelsep]
		\item 
		\begin{enumerate}
			\item Consider the following greedy algorithm:
			\begin{algorithm}
				\caption{$1/4$-net}
				\begin{algorithmic}
					\State $N \gets \Set{(1, 0, \ldots, 0)}$
					\While{there is some $x \in \overline B(0, 1)$ with $d(x, N) > 1/4$}
						\State $N \gets N \cup \Set{x}$
					\EndWhile \\
					\Return $N$
				\end{algorithmic}
			\end{algorithm}
		
			We can see that at the end, every point in the ball is within $1/4$ of $N$. We also see that every point in $N$ is at least $1/4$ away from all other points. This means that the balls centered at each $x \in N$ with radius $1/8$ are disjoint, and the union of all these balls is contained in $\overline B(1+1/8, 0)$, which has volume $c_n \cdot (1+1/8)^n = c_n \cdot (9/8)^n = 9^n \cdot c_n \cdot 1/8^n$. Each ball has volume $c_n \cdot (1/8)^n$, so we must have no more than $9^n = 2^{\log_2(9)n} = 2^{O(n)}$ balls.
			
			\item Let $x \in \overline B(0, 1)$ be the maximizer of $|x^TMx|$ and decompose $x = y + z$ where $y \in N$ and $|z| < 1/4$. Then
			\begin{align*}
				\sigma_1 = |x^TMx| &= |(y+z)^TM(y+z)| = |(y+z)^T(My+Mz)| \\
				&= |y^TMy + z^TMz + z^TMy + y^TMz| \\
				&\leq |y^TMy| + |z^TMz| + 2|z^TMy| \\
				&\leq |y^TMy| + \frac14 \sigma_1 + 2 \cdot \frac14 \sigma_1 = |y^TMy| + \frac34 \sigma_1
			\end{align*}
			Hence $\sigma_1 \leq 4|y^TMy|$, thus $\sigma_1 \leq 4\max_{y \in N} |y^TMy|$.
			
			\item Notice that
			\begin{align*}
				\E[\sum r_ia_i] = \sum a_i \E[r_i] = 0
			\end{align*}
			Since $\E[r_i] = 0$. Therefore, by Hoeffding's inequality,
			\begin{align*}
				\P[|\sum r_ia_i - 0| \geq t] \leq 2\exp(\frac{-2t^2}{\sum (2a_i)^2}) = 2\exp(-\frac{t^2}{2\sum a_i^2})
			\end{align*}
			Since $-1 \leq r_i \leq 1$ we have $-|a_i| \leq a_ir_i \leq |a_i|$.
			
			\item Fix $y \in N$. Then,
			\begin{align*}
				y^T(A-\E[A])y = \sum_{i, j} (A-\E[A])_{ij} y_i y_j
			\end{align*}
			Now notice that $A_{ij}$ is a Bernoulli random variable with $p = \frac12$ for $i \neq j$. Then $\E[A_{ij}] = \frac12$, and $A_{ij} = \E[A_{ij}] = 0$ for $i = j$. Then for $i \neq j$, \begin{align*}
				(A-\E[A])_{ij} = A_{ij} - \E[A_{ij}] = \begin{cases}
					\frac12, \; \text{w.p } \frac12 \\
					-\frac12, \; \text{o.w.}
				\end{cases}
			\end{align*}
			That is, $(A-\E[A])$ is 1/2 times a Radamacher random variable. Thus we write $A_{ij} = \sigma_{ij}$ for $i \neq j$ and $A_{ij} = 0$ otherwise. Now,
			\begin{align*}
				y^T(A-\E[A])y = \sum_{i \neq j} \frac{y_iy_j}{2} \sigma_{ij}
			\end{align*}
			By part c) we can conclude that
			\begin{align*}
				\P[\qty|\sum_{i \neq j} \frac{y_iy_j}{2} \sigma_{ij}| \geq C\sqrt{n}] \leq 2\exp(-\frac{C^2n}{\frac{2}{4}\sum_{i \neq j} y_i^2 y_j^2})
			\end{align*}
			Notice now that,
			\begin{align*}
				\sum_{i \neq j} y_i^2y_j^2 \leq \sum_{i, j} y_i^2 y_j^2 = \sum_{i} y_i^2 \cdot \sum_{j} y_j^2 = \mg{y}^4 \leq 1
			\end{align*}
			So,
			\begin{align*}
				\P[\qty|\sum_{i \neq j} \frac{y_iy_j}{2} \sigma_{ij}| \geq C\sqrt{n}] \leq 2e^{-2C^2n}
			\end{align*}
			By the union bound we have that
			\begin{align*}
				\Pr[\max_{y \in N}|y^T(A - \E[A])y| \geq C\sqrt{n}] \leq 2 \cdot 9^n e^{-C^2n}
			\end{align*}
			Choosing $C = 4\sqrt{\ln(18)}$ yields
			\begin{align*}
				\Pr[4\max_{y \in N} |y^T(A - \E[A])y| \geq C\sqrt{n}] \leq 2^{-n}
			\end{align*}
			Now, by part b), since $\sigma_1(A - \E[A]) = \max_{x \in \S^n} |x^T(A - \E[A])x| \leq 4\max_{y \in N}|y^T(A - \E[A])y|$, we have that
			\begin{align*}
				\Pr[\mg{A - \E[A]} \leq C\sqrt{n}] &\geq \Pr[4\max_{y \in N} |y^T(A - \E[A])y| \geq C\sqrt{n}] \\
				&\geq 1 - 2^{-n}
			\end{align*}
			\end{enumerate}
	\end{enumerate}
\end{document}
