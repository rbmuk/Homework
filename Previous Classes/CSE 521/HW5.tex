\documentclass[12pt]{article}
\usepackage[margin=1in]{geometry}

% Start of preamble
%==========================================================================================%
% Required to support mathematical unicode
\usepackage[warnunknown, fasterrors, mathletters]{ucs}
\usepackage[utf8x]{inputenc}

\usepackage[dvipsnames,table,xcdraw]{xcolor}
\usepackage{hyperref} 
\hypersetup{
	colorlinks=true,
	linkcolor=blue,
	filecolor=magenta,      
	urlcolor=cyan,
	pdfpagemode=FullScreen
}

% Standard mathematical typesetting packages
\usepackage{amsmath,amssymb,amscd,amsthm,amsxtra, pxfonts}
\usepackage{mathtools,mathrsfs,dsfont,xparse}

% Symbol and utility packages
\usepackage{cancel, textcomp}
\usepackage[mathscr]{euscript}
\usepackage[nointegrals]{wasysym}
\usepackage{apacite}

% Extras
\usepackage{physics}  
\usepackage{tikz-cd} 
\usepackage{microtype}
\usepackage{enumitem}
\usepackage{titling}
\usepackage{graphicx}

% Fancy theorems due to @intuitively on discord
\usepackage{mdframed}
\newmdtheoremenv[
backgroundcolor=NavyBlue!30,
linewidth=2pt,
linecolor=NavyBlue,
topline=false,
bottomline=false,
rightline=false,
innertopmargin=10pt,
innerbottommargin=10pt,
innerrightmargin=10pt,
innerleftmargin=10pt,
skipabove=\baselineskip,
skipbelow=\baselineskip
]{mytheorem}{Theorem}

\newenvironment{theorem}{\begin{mytheorem}}{\end{mytheorem}}

\newtheorem{corollary}{Corollary}
\newtheorem{lemma}{Lemma}

\newtheoremstyle{definitionstyle}
{\topsep}%
{\topsep}%
{}%
{}%
{\bfseries}%
{.}%
{.5em}%
{}%
\theoremstyle{definitionstyle}
\newmdtheoremenv[
backgroundcolor=Violet!30,
linewidth=2pt,
linecolor=Violet,
topline=false,
bottomline=false,
rightline=false,
innertopmargin=10pt,
innerbottommargin=10pt,
innerrightmargin=10pt,
innerleftmargin=10pt,
skipabove=\baselineskip,
skipbelow=\baselineskip,
]{mydef}{Definition}
\newenvironment{definition}{\begin{mydef}}{\end{mydef}}

\newtheorem*{remark}{Remark}

\newtheorem*{example}{Example}

% Common shortcuts
\def\mbb#1{\mathbb{#1}}
\def\mfk#1{\mathfrak{#1}}

\def\bN{\mbb{N}}
\def \C{\mbb{C}}
\def \R{\mbb{R}}
\def\bQ{\mbb{Q}}
\def\bZ{\mbb{Z}}
\def \cph{\varphi}
\renewcommand{\th}{\theta}
\def \ve{\varepsilon}
\newcommand{\mg}[1]{\| #1 \|}

% Often helpful macros
\newcommand{\floor}[1]{\left\lfloor#1\right\rfloor}
\newcommand{\ceil}[1]{\left\lceil#1\right\rceil}
\renewcommand{\qed}{\hfill\qedsymbol}
\renewcommand{\P}{\mathrm{Pr}\qty}
\newcommand{\E}{\mathbb{E}\qty}

% Sets
\usepackage{braket}

% End of preamble
%==========================================================================================%

% Start of commands specific to this file
%==========================================================================================%

%==========================================================================================%
% End of commands specific to this file

\title{CSE 521 HW5}
\date{\today}
\author{Rohan Mukherjee}

\begin{document}
	\maketitle
	\begin{enumerate}[leftmargin=\labelsep]
		\item \begin{enumerate}
			\item We first claim that $P M P^T \in \R^{m \times m}$ is symmetric. Indeed,
			\begin{align*}
				(PMP^T)^T = (P^T)^T M^T P^T = P M P^T
			\end{align*}
			We recall that symmetric $M \in \R^{n \times n}$ is PSD iff $x^T M x \geq 0$ for all $x \in \R^n$. Thus, given an arbitrary $x \in \R^m$,
			\begin{align*}
				x^T PMP^T x = (P^Tx)^T M (P^Tx) \geq 0
			\end{align*}
			Since $P^Tx \in \R^n$ is just another vector.
			
			\item I claim that if $M$ has $k$ positive eigenvalues, then $M + vv^T$ has at most $k + 1$ positive eigenvalues for any $v$. Indeed, enumerate the eigenvalues of $M$ as $\lambda_1, \ldots, \lambda_n$, and the eigenvalues of $M + vv^T$ as $\beta_1, \ldots, \beta_n$. We have the following inequality:
			\begin{align*}
				\lambda_n \leq \beta_n \leq \cdots \leq \lambda_1 \leq \beta_1
			\end{align*}
			In particular, since $\lambda_{k+1} \leq 0$, we have $\beta_{k+2} \leq \lambda_{k+1} \leq 0$, and in general $\beta_{i} \leq 0$ for $i > k + 2$. Thus there can be at most $k + 1$ positive eigenvalues. My second claim is that if $M$ has $k$ positive eigenvalues, then $M - vv^T$ has at most $k$ positive eigenvalues. This follows similarly as last time, since $\lambda_{k+1} \leq \beta_{k+1} \leq 0$. With these two simple facts we can finish the problem. Write $M = \sum_{i=1}^n \lambda_i u_iu_i^T$. Clearly \begin{align*}
				PMP^T = \sum_{i=1}^n \lambda_i (Pu_i)(Pu_i)^T = \sum_{i=1}^n \mathrm{sgn}(\lambda_i)(\sqrt{|\lambda_i|}Pu_i)(\sqrt{|\lambda_i|}Pu_i)^T.
			\end{align*}
			Starting from the 0 matrix, which has 0 positive eigenvalues, we now add each of the matrices in the above sum one at a time. For each $\lambda_i > 0$, we can pick up at most $1$ positive eigenvalue, and for each $\lambda_i < 0$, can pick up at most 0 eigenvalues. Since there are exactly $k$ positive eigenvalues, we conclude that $PMP^T$ has at most $k$ positive eigenvalues. Forsooth, the claim is upon us.
			\end{enumerate}
		
		\item
		\begin{enumerate}
			\item If $\sigma_i$ are the singular values of $A$, we know that $\sigma_i^2$ are the eigenvalues of $A^TA$. What can we say about $(A^TA)^T (A^TA) = A^TAA^TA$? Clearly, if $v_i$ is the eigenvector associated with $\sigma_i^2$, we have that
			\begin{align*}
				A^TAA^TA v_i = A^TA \sigma_i^2 v_i = \sigma_i^4 v_i
			\end{align*}
			So the eigenvalues of $A^TAA^TA$ are precisely the $\sigma_i^4$. Now,
			\begin{align*}
				\mg{A^TA}_F^2 = \sum_{i=1}^n \lambda_i((A^TA)^T(A^TA)) = \sum_{i=1}^n \sigma_i^4 \leq \sigma_1^2 \sum_{i=1}^n \sigma_i^2 = \mg{A} \cdot \mg{A}_F
			\end{align*}
			We remark that this also gives equality conditions. Equality holds iff every singular value is equal.
		
			\item First, $\mg{A\sigma}^2 = \sigma^T A^T A \sigma$. Second, notice that
			\begin{align*}
				\E[\mg{A\sigma}^2] &= \E[\sum_{i=1}^n \langle a_i, \sigma \rangle^2] = \E[\sum_{i=1}^n \qty(\sum_{j=1}^n a_j^i \sigma_j)^2] \\
				&= \sum_{i=1}^n \sum_{1 \leq j, k \leq n} \E[a_j^i a_k^i \sigma_j \sigma_k]
			\end{align*}
			Next notice that $\E[\sigma_i] = 0$ and that $\E[\sigma_i^2] = \frac12 \cdot 1^2 + \frac12 \cdot (-1)^2 = 1$. Thus, this equals,
			\begin{align*}
				\sum_{i=1}^n \sum_{j=1}^n (a_j^i)^2 = \mg{A}_F^2
			\end{align*}
			Now,
			\begin{align*}
				\P[\qty|\sigma^T A^T A \sigma - \E[\sigma^T A^TA\sigma]| > \ve \E[\sigma^T A^TA\sigma]] &= \P[\qty|\mg{A\sigma}^2 - \mg{A}_F^2| > \ve \mg{A}_F^2] \\ &\leq 2\exp(-c\frac{\ve^2 \mg{A}_F^4}{\mg{A^TA}_F^2}) \\
				&\leq 2\exp(-c\frac{\ve^2 \mg{A}_F^4}{\mg{A}^2 \mg{A}_F^2}) \\
				&= 2\exp(-c \frac{\ve^2 \mg{A}_F^2}{\mg{A}^2})
			\end{align*}
			Very nice.
			
			\item We see clearly that $d(\sigma, E) = \mg{\sigma - \Pi_E \sigma}$, as was shown in class and can be easily seen geometrically. Thus, $d(\sigma, E) = \mg{(I - \Pi_E)\sigma} = \mg{\Pi_E^\perp \sigma}$. Finding $w_1, \ldots, w_d$ an orthonormal basis for $E$, we have that $\Pi_E = \sum_{i=1}^d w_iw_i^T$. Extending this to a basis of $\R^n$ as $w_1, \ldots, w_d, v_1, \ldots, v_{n-d}$, we have that $\Pi_E^\perp = \sum_{i=1}^{n-d} v_iv_i^T$. Notice now that
			\begin{align*}
				\big\| \sum_{i=1}^{n-d} v_iv_i^T \big\|_F^2 &= \Tr(\sum_{1 \leq i,j \leq n-d} v_iv_i^T v_j v_J^T) = \Tr(\sum_{i=1}^{n-d} v_iv_i^T) = \sum_{i=1}^n \Tr(v_i v_i^T) \\
				&= \sum_{i=1}^{n-d} \Tr(v_i^T v_i) = \sum_{i=1}^{n-d} \mg{v_i}^2 = n-d
			\end{align*}
			We also claim that $\mg{\Pi_E^\perp} \leq 1$. This follows since $\mg{\Pi_E^\perp}$ equals the max eigenvalue of $\Pi_E^\perp (\Pi_E^\perp)^T = \Pi_E^\perp$, and the max eigenvalue of a projection matrix is at most 1. We conclude that
			\begin{align*}
				\P[\qty|\frac{d(\sigma, E)}{n-d} - 1| > \ve] &= \P[\qty|\frac{\mg{\Pi_E^\perp \sigma}}{\mg{\Pi_E^\perp}_F} - 1| > \ve] \leq 2\exp(-c \frac{\ve^2 \mg{\Pi_E^\perp}_F^2}{\mg{\Pi_E^\perp}}) \\ &= 2\exp(-c \frac{\ve^2(n-d)}{\mg{\Pi_E^\perp}}) \\
				&\leq 2\exp(-c \ve^2 (n-d))
			\end{align*}
		\end{enumerate}
	\end{enumerate}
\end{document}
