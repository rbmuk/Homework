\documentclass[12pt]{article}
\usepackage[margin=1in]{geometry}

% Start of preamble
%==========================================================================================%
% Required to support mathematical unicode
\usepackage[warnunknown, fasterrors, mathletters]{ucs}
\usepackage[utf8x]{inputenc}

% Always typeset math in display style
%\everymath{\displaystyle}

% Standard mathematical typesetting packages
\usepackage{amsmath,amssymb,amscd,amsthm,amsxtra, pxfonts}
\usepackage{mathtools,mathrsfs,dsfont,xparse}

% Symbol and utility packages
\usepackage{cancel, textcomp}
\usepackage[mathscr]{euscript}
\usepackage[nointegrals]{wasysym}

% Extras
\usepackage{physics}  % Lots of useful shortcuts and macros
\usepackage{tikz-cd}  % For drawing commutative diagrams easily
\usepackage{color}  % Add some color to life
\usepackage{microtype}  % Minature font tweaks
%\usepackage{pgfplots} % plots

\usepackage{enumitem}
\usepackage{titling}

\usepackage{graphicx}

% Common shortcuts
\def\mbb#1{\mathbb{#1}}
\def\mfk#1{\mathfrak{#1}}

\def\bN{\mbb{N}}
\def \C{\mbb{C}}
\def \R{\mbb{R}}
\def\bQ{\mbb{Q}}
\def\bZ{\mbb{Z}}
\def \cph{\varphi}
\renewcommand{\th}{\theta}
\def \ve{\varepsilon}
\newcommand{\mg}[1]{\| #1 \|}

% Sometimes helpful macros
\newcommand{\floor}[1]{\left\lfloor#1\right\rfloor}
\newcommand{\ceil}[1]{\left\lceil#1\right\rceil}
\renewcommand{\qed}{\hfill\qedsymbol}

% Sets
\DeclarePairedDelimiterX\set[1]\lbrace\rbrace{\def\given{\;\delimsize\vert\;}#1}

% Some standard theorem definitions
\newtheorem{theorem}{Theorem}[section]
\newtheorem{corollary}{Corollary}[theorem]
\newtheorem{lemma}[theorem]{Lemma}

\theoremstyle{definition}
\newtheorem{definition}{Definition}[section]

\theoremstyle{remark}
\newtheorem*{remark}{Remark}

% End of preamble
%==========================================================================================%

% Start of commands specific to this file
%==========================================================================================%

\renewcommand{\ip}[2]{\langle #1, #2 \rangle}
\newcommand{\linf}[1]{\max_{1\leq i \leq #1}}
\newcommand{\seq}[2]{\qty(#1_#2)_{#2=1}^{\infty}}
\renewcommand{\Re}{\mathfrak{R}\qty}

%==========================================================================================%
% End of commands specific to this file

\title{Template}
\date{\today}
\author{Rohan Mukherjee}

\begin{document}
	\maketitle
	\begin{enumerate}[leftmargin=\labelsep]
		\item \begin{enumerate}
			\item We recognize that the only places where $\cos(z) = 0$ in $|z| \leq 2$ is $z = \pm \pi/2$ (This would mean that $e^{iz} + e^{-iz} = 0$, i.e. that $e^{2iz} = e^{i\pi}$, i.e. that $z = \pi/2$ or $-\pi/2$). We know that 
			\begin{align*}
				\int_{|z| = 2} \frac{z}{\cos(z)}dz = 2\pi i \qty(\mathrm{Res}\qty(\frac{z}{\cos(z)}, \pi/2) + \Res(\frac{z}{\cos(z)}, -\pi/2))
			\end{align*}
			We see that 
			\begin{align*}
				\lim_{z \to \pi/2} (z-\pi/2) \frac{z}{\cos(z)} = \pi/2 \lim_{z \to \pi/2} \frac{z-\pi/2}{\cos(z)} = \pi/2 \lim_{z \to \pi/2} \frac1{-\sin(z)} = -\pi/2
			\end{align*}
			And also that $\Res(z/\cos(z), -\pi/2) = -\pi/2$ as well (through a similar calculation). We conclude that $\int_{|z| = 2} z/\cos(z)dz = 2\pi i \cdot - \pi = -2\pi^2 i$.
			
			\item We note that
			\begin{align*}
				\int_{|z-1/2| = 3/2} \frac{\tan(z)}{z}dz = \int_{|z-1/2| = 3/2} \frac{\sin(z)}{\cos(z)z}dz
			\end{align*}
			In $|z-1/2| = 3/2$, the denominator is zero at $\pi/2$, and 0, with both poles being order 0. We notice that 0 is a removable singularity, since the function does not blow up there, so the residue at 0 is just 0. Now,
			\begin{align*}
				\lim_{z \to \pi/2} (z-\pi/2) \frac{\sin(z)}{z\cos(z)} = 2/\pi \cdot \lim_{z \to \pi/2} \frac{z-\pi/2}{\cos(z)} \overset{L'H}{=} -2/\pi
			\end{align*}
			We conclude by the residue theorem that the integral is just $2\pi i \cdot -2/\pi = -4i$.
			
			\item We look to evaluate
			\begin{align*}
				\int_{\Gamma} \frac{1}{z^4+1}dz
			\end{align*}
			Where $\Gamma$ is the semicircle of radius $R$ in the upper half plane. 
			
			\begin{tikzpicture}
				\def\R{2} % set the radius R
				\draw[->] (-4,0) -- (4,0) node[right] {$\mathrm{Re}(z)$}; % x-axis
				\draw[->] (0,-1) -- (0,3) node[above] {$\mathrm{Im}(z)$}; % y-axis
				\draw[thick,blue] (\R,0) arc (0:180:\R); % semicircle
				\draw[->,thick] (0,0) -- ({\R*cos(45)},{\R*sin(45)}) node[midway,above left] {$R$}; % line labeled R
				\draw[thick,blue] (-\R,0) -- (\R,0); % blue line on the real axis
				\node[blue] at ({\R*cos(135)},{\R*sin(135)+0.3}) {$\Gamma$}; % label for the whole contour
			\end{tikzpicture}
		
			Yes, I had ChatGPT draw this picture. AI is nuts! Anyways, we can parameterize the circle part of it by $\gamma(t) = Re^{it}$ where $0 \leq t \leq \pi$ (as usual). We see that
			\begin{align*}
				\int_\gamma \frac{1}{z^4+1}dz = \int_0^\pi \frac{iRe^{it}}{R^4e^{i4t}+1}dt
			\end{align*}
			Followed by the usual computation: $\qty|\int_0^\pi \frac{iRe^{it}}{R^4e^{i4t}+1}dt| \leq \int_0^\pi \qty|\frac{iRe^{it}}{R^4e^{i4t}+1}|dt \leq \int_0^\pi \frac{R}{R^4-1}dt \overset{R \to \infty}{\to} 0$.
			
			The only poles of $1/(z^4+1)$ in this contour are at $e^{i\pi/4}$ and $e^{3i\pi/4}$. We see that the residues are therefore equal to
			$\Res(f, e^{i\pi/4}) = \lim_{z \to e^{i\pi/4}} \frac{z-e^{i\pi/4}}{z^4+1} \overset{L'H}{=} \lim_{z \to e^{i\pi/4}} \frac{1}{4z^3} = \frac14 e^{-3i\pi/4}$. Similarly, $\Res(f, e^{3i\pi/4}) = \frac14 e^{-\pi i/4}$. By the residue theorem we see that \begin{align*}
				\int_\R \frac{1}{x^4+1}dx = \int_{\Gamma} \frac{1}{z^4+1}dz = \frac{\pi i}{2} \qty(e^{-\pi i/4} + e^{-3\pi i/4}) = \frac{\pi}{\sqrt{2}}
			\end{align*}
			Very nice!
			\item I will take less care with the calculations this time since they are almost the same as last time. Using the same contour, and the same $\gamma$ we get
			\begin{align*}
				\int_\gamma \frac{z^2}{z^4+1}dz = \int_0^\pi \frac{R^2e^{2it}}{R^4e^{4it}+1} iRe^{it}dt \approx \int_0^\pi \frac1Rdt \to 0
			\end{align*}
			One could do the reverse triangle inequality trick again to make this rigorous, but I don't want to. Now we are left to calculate the residues, which again are the same as last time. We see that
			\begin{align*}
				\Res(\frac{z^2}{z^4+1}, e^{i\pi/4}) = \lim_{z \to e^{i\pi/4}} (z-e^{i\pi/4}) \frac{z^2}{z^4+1} = e^{i\pi/2} \cdot \frac14 e^{-3i\pi/4} = \frac i4 e^{-3i\pi/4}
			\end{align*}
			Similarly, $\Res(\frac{z^2}{z^4+1}, e^{3i\pi/4}) = \frac{-i}4 e^{-i\pi/4}$. We conclude that
			\begin{align*}
				\int_{\R} \frac{x^2}{x^4+1}dx = 2\pi i \qty(\frac i4 e^{-3i\pi/4} - \frac i4 e^{-i\pi/4}) = -\frac{\pi i}{2} \frac{i}{2\sqrt{2}} = \frac{\pi}{4\sqrt{2}}
			\end{align*}
		\end{enumerate}
		\item First, we see that
		\begin{align*}
			\int_\R \frac{\cos(x)}{(1+x^2)^2}dx = \Re(\int_{\R} \frac{e^{ix}}{(1+x^2)^2}dx)
		\end{align*}
		Factor $(1+x^2)^2 = (x+i)^2(x-i)^2$. We shall use the same contour as I used in the last two parts of the last problem. First,
		\begin{align*}
			\int_\gamma \frac{e^{iz}}{(1+z^2)^2}dz = \int_0^\pi \frac{e^{Ri\cos(t)} \cdot e^{-R\sin(t)}}{(1+R^2e^{2it})^2} \cdot iRe^{it}dt
		\end{align*}
		Like usual, $|1+R^2e^{2it}| \geq R^2 - 1$, and choosing $R$ large enough so that the RHS is positive gives $|1+R^2e^{2it}|^2 \geq (R^2-1)^2$. Also, on $[0, \pi]$, $\sin(t) \geq 0$, so $|e^{-R\sin(t)}| \leq 1$. We conclude that
		\begin{align*}
			\qty|\int_0^\pi \frac{e^{Ri\cos(t)} \cdot e^{-R\sin(t)}}{(1+R^2e^{2it})^2} \cdot iRe^{it}dt| \leq \int_0^\pi \frac{R}{(R^2-1)^2}dt \overset{R \to \infty}{\to} 0
		\end{align*}
		We are left to calculate $\Res(f, i)$. This time the pole is of order 2, so we must use the derivative. We see that
		\begin{align*}
			\Res(f, i) &= \lim_{z \to i} \frac{1}{1!} \dv{z} (z-i)^2 e^{iz}{(1+z^2)^2} = \lim_{z \to i} \dv{z} e^{iz}{(z+i)^2} \\
			&= \lim_{z \to i} \frac{ie^{iz}(z+3i)}{(z+i)^3} = \frac{ie^{-1} \cdot 4i}{8i^{-1}} = \frac{1}{2ie}
		\end{align*}
		Therefore, by the residue theorem
		\begin{align*}
			\int_{\R} \frac{\cos(x)}{(x^2+1)^2}dx = \Re(\int_\R \frac{e^{ix}}{(x^2+1)^2}dx) = \Re(2\pi i \frac{1}{2ie}) = \frac{\pi}{e}
		\end{align*}
		A magical formula!
		
		\item Since $\cos(x) = \cos(-x)$, we can take WLOG $a > 0$ (i.e. we can integrate $\cos(|a|x)$ instead). We see that 
		\begin{align*}
			\int_\R \frac{\cos(ax)}{x^2+b^2}dx = \Re(\int_\R \frac{e^{iax}}{x^2+b^2}dx)
		\end{align*}
		And once again we will reuse the contour in the last 3 problems. Using $\gamma(t) = Re^{it}$, we see that 
		\begin{align*}
			\int_\gamma f(z)dz = \int_0^\pi \frac{e^{iaRe^{it}}}{R^2e^{2it} + b^2} iRe^{it}dt
		\end{align*}
		Once again, $e^{iaRe^{it}} = e^{iaR\cos(t)} e^{-aR\sin(t)}$. Since $aR\sin(t)$ is positive on $[0, \pi]$ (remember, we took $a$ to be positive!), $|e^{iaRe^{it}}| \leq 1$. Also, $|R^2e^{2it} + b^2| \geq R^2 - b^2$, so we see that
		\begin{align*}
			\qty|\int_0^\pi \frac{e^{iaRe^{it}}}{R^2e^{2it} + b^2} iRe^{it}dt| \leq \int_0^\pi \frac{R}{R^2-b^2}dt \overset{R \to \infty}{\to} 0
		\end{align*}
		Like usual. Next, we see that $\frac{e^{aiz}}{z^2+b^2}$ has one pole at $z = ib$ with order 1 inside our contour. We are left with calculating the residue of our function there, which can be done as follows:
		\begin{align*}
			\Res(\frac{e^{aiz}}{z^2+b^2}, ib) = \lim_{z \to ib} (z-ib)\frac{e^{aiz}}{(z+ib)(z-ib)} = e^{-ab} \lim_{z \to ib} \frac{1}{z+ib} = \frac{e^{-ab}}{2ib}
		\end{align*}
		By the residue theorem, $\int_\R \frac{\cos(ax)}{x^2+b^2}dx = \Re(\int_\R \frac{e^{iax}}{x^2+b^2}dx)  = \Re(2\pi i \cdot \frac {e^{-ab}}{2ib}) = \frac{\pi e^{-ab}}{b}$, as claimed. Remember, we could've just put $|a|$ everywhere if $a$ was negative, so the identity in the question holds.
		
		\item The idea for this one is to revert the transformation we did to get it into the complex plane. First, since $\sin(x-\pi) = -\sin(x)$, we see that $\sin(x-\pi) = \sin^2(x)$, and so 
		\begin{align*}
			\int_{-\pi}^{\pi} \frac{1}{1+\sin^2(\theta)}d\theta = \int_0^{2\pi} \frac{1}{1+\sin^2(\theta)}d\theta
		\end{align*}
		Now note that $\sin^2(\theta) = \frac12 (1 - \cos(2\theta)) = \frac12 - \frac14 \qty(e^{2i\theta}+e^{-2i\theta})$. Plugging this in gives us
		\begin{align*}
			\int_0^{2\pi} \frac{1}{1+\sin^2(\theta)}d\theta &= \int_0^{2\pi} \frac{4}{6-e^{2i\theta}-e^{-2i\theta}}d\theta
		\end{align*}
		Now let $z = e^{2i\theta}$. Note that $dz = 2ie^{2i\theta}d\theta$, i.e. that $\frac{-idz}{2z} = d\theta$. Our contour will be the unit circle traced counterclockwise, twice. So if we let $\gamma$ = the unit circle, we get that this integral equals (the two integrals and the divide by two cancels)
		\begin{align*}
			\oint_\gamma \frac{-4i}{6z-z^2-1}dz = \oint_\gamma \frac{4i}{z^2-6z+1}dz
		\end{align*}
		The only root of $z^2-6z+1$ inside the unit circle is $-2\sqrt{2}+3$. So, we wish to calculate the residue of our function there. We see that
		\begin{align*}
			\Res(\frac{4i}{z^2-6z+1}, -2\sqrt{2}+3) = 4i \lim_{z \to -2\sqrt{2}+3} \frac{1}{z-2\sqrt{2}-3} = \frac{-i}{\sqrt{2}}
		\end{align*}
		Now, by the residue theorem,
		\begin{align*}
			\int_0^{2\pi} \frac1{1+\sin^2(\theta)}d\theta = \oint_\gamma \frac{4i}{z^2-6z+1}dz = 2\pi i \cdot \frac{-i}{\sqrt{2}} = \pi\sqrt{2}
		\end{align*}
		Which is very nice. CA is powerful!!
		
		\item We are going to do the same first step as last time. Since $\cos(\theta-\pi) = -\cos(\theta)$, we see that
		\begin{align*}
			\int_{-\pi}^\pi \frac{1-r^2}{1-2r\cos(\theta)+r^2}d\theta = (1-r^2)\int_0^{2\pi} \frac{1}{1+2r\cos(\theta)+r^2}d\theta \\
			= (1-r^2)\int_0^{2\pi} \frac{1}{1+re^{i\theta}+re^{-i\theta}+r^2}d\theta
		\end{align*}
		Now let $z = e^{i\theta}$ (We may assume that $r \neq 0$, if $r = 0$, the integral is obviously $2\pi$ as the integrand is just  $1$). Once again $\gamma$ is the standard parametrization of the unit circle. Clearly $dz = ie^{i\theta}d\theta$, so $\frac{-idz}{z} = d\theta$. Forgetting about the $(1-r^2)$, our integral becomes 
		\begin{align*}
			\oint_\gamma \frac{-i}{z+rz^2+r+r^2z}dz = \oint_\gamma \frac{-i}{(z+r)(rz+1)}dz 
		\end{align*}
		Since $0 < |r| < 1$, $\qty|\frac1r| > 1$, so the only pole inside our unit circle is $z = -r$. We then wish to calculate
		\begin{align*}
			\Res(\frac{-i}{(z+r)(rz+1)}, -r) = -i \lim_{z \to -r} \frac{1}{rz+1} = \frac{-i}{1-r^2}
		\end{align*}
		By the residue theorem,
		\begin{align*}
			\oint_\gamma \frac{-i}{(z+r)(rz+1)}dz = 2\pi i \cdot \frac{-i}{1-r^2} = \frac{2\pi}{1-r^2}
		\end{align*}
		So, our original integral equals $2\pi$. This is very nice!!! I believe you could also do all these integrals without complex analysis by the Weirstrass substitution, but it would be a LOT more work. This stuff is magical in comparison!
	\end{enumerate}
\end{document}
