\documentclass[12pt]{article}
\usepackage[margin=1in]{geometry}

% Start of preamble
%==========================================================================================%
% Required to support mathematical unicode
\usepackage[warnunknown, fasterrors, mathletters]{ucs}
\usepackage[utf8x]{inputenc}

% Always typeset math in display style
\everymath{\displaystyle}

% Standard mathematical typesetting packages
\usepackage{amsfonts, amsthm, amsmath, amssymb}
\usepackage{mathtools}  % Extension to amsmath

% Symbol and utility packages
\usepackage{cancel, textcomp}
\usepackage[mathscr]{euscript}
\usepackage[nointegrals]{wasysym}

% Extras
\usepackage{physics}  % Lots of useful shortcuts and macros
\usepackage{tikz-cd}  % For drawing commutative diagrams easily
\usepackage{color}  % Add some color to life
\usepackage{microtype}  % Minature font tweaks
%\usepackage{pgfplots} % plots

% Common shortcuts
\def\mbb#1{\mathbb{#1}}
\def\mfk#1{\mathfrak{#1}}

\def\bN{\mbb{N}}
\def\bC{\mbb{C}}
\def\bR{\mbb{R}}
\def\bQ{\mbb{Q}}
\def\bZ{\mbb{Z}}

% Sometimes helpful macros
\newcommand{\func}[3]{#1\colon#2\to#3}
\newcommand{\vfunc}[5]{\func{#1}{#2}{#3},\quad#4\longmapsto#5}
\newcommand{\floor}[1]{\left\lfloor#1\right\rfloor}
\newcommand{\ceil}[1]{\left\lceil#1\right\rceil}

% Some standard theorem definitions
\newtheorem{theorem}{Theorem}[section]
\newtheorem{corollary}{Corollary}[theorem]
\newtheorem{lemma}[theorem]{Lemma}

\theoremstyle{definition}
\newtheorem{definition}{Definition}[section]

\theoremstyle{remark}
\newtheorem*{remark}{Remark}

% End of preamble
%==========================================================================================%

% Start of commands specific to this file
%==========================================================================================%

\usetikzlibrary{decorations.markings}
\newcommand{\indef}{\int_{0}^{\infty}}
\newcommand{\ve}{\varepsilon}

%==========================================================================================%
% End of commands specific to this file

\title{Elegant Integrals}
\date{\today}
\author{Rohan Mukherjee}

\begin{document}
    \maketitle

    \section{Gamma and Zeta Functions}
    We begin with a discussion of some important special functions, and prove some facts about them.
    \begin{definition}[Gamma Function]\label{Gamma}
        The \textit{Gamma Function} is defined as follows:
        \begin{align*}
            \Gamma(x+1) = \indef t^xe^{-t}dt
        \end{align*}    
    \end{definition}

    \begin{theorem}[Gamma-Factorial Identity]
        For all natural numbers n, denoted $n \in \bN$, $\Gamma(n) = (n-1)!$
    \end{theorem}

    \begin{proof}
        First I show that $\Gamma(n+1)=n\Gamma(n)$. 
        Note that this is the same thing as saying $n! = n(n-1)!$
        Suppose that $n \in \bN$, and that $n \geq 2$. 
        Let $u = t^n$ and $dv = e^{-t}dt$. Then $du = nt^{n-1}$ and $v = -e^{-t}$. So 
        \begin{align*}
            \Gamma(n+1) = \indef t^xe^{-t}dt &= -t^ne^{-t}\eval_{0}^{\infty} + n\indef t^{n-1}e^{-t}dt \\ 
            &= \lim_{t \to \infty} -\frac{t^n}{e^{t}} + 0 + n\indef t^{n-1}e^{-t}dt \\ 
            &= n\indef t^{n-1}e^{-t}dt \\ 
            &= n\Gamma(n)
        \end{align*}
        Now I will show that $\Gamma(1) = (1-1)! = 1$. Observe that
        \begin{align*}
           \Gamma(1) = \indef t^{0}e^{-t}dt &= \indef e^{-t}dt \\ 
           &= -e^{-t} \eval_{0}^{\infty} \\ 
           &= -\lim_{t \to \infty} e^{-t} + 1 \\ 
           &= 1
        \end{align*}
    \end{proof}

    The next tool I will be introducing is the \textit{Riemann Zeta Function}, denoted $\zeta(s)$. 
    For our purposes, the only definition we will have to know is the next one.

    \begin{definition}[Riemann Zeta Function]\label{Zeta}
        The \textit{Riemann Zeta Function}, denoted $\zeta(s)$, is defined as follows:
        \begin{align*}
           \zeta(s) = \sum_{n=1}^{\infty} \frac{1}{n^s}
        \end{align*}
        Put simply, it is the value of the p-series with $p = s$.
    \end{definition}

    There are some important implications of this function in the field of complex analysis, where you can make sense of things like
    \begin{align*}
        \zeta(-1) = \sum_{n=1}^{\infty} \frac{1}{n^{-1}} = 1 + 2 + 3 + \cdots = -\frac{1}{12}
    \end{align*}
    But that is beyond the scope of this article.

    The most famous result--one that shows up all the time, is that
    \begin{align*}
        \zeta(2) = \sum_{n=1}^{\infty} \frac{1}{n^2} = \frac{\pi^2}{6}
    \end{align*}

    Sadly showing that this is true is really complicated, so I will ask you to take it as a fact.
    If you want, you can look up ``3blue1brown basel problem'' to see a really elegant solution.
    The wikipedia article also does a good job showing it, but uses some theorems that are a lot less intuitive.

    So now we will move onto our first integral.

    \section{Problems Part 1}
    This first one is a fairly modified version of one from the MIT 2020 integration bee qualifying exam.
    This was the last (and hardest) problem on the exam.
    \begin{align*}
        \indef \frac{x}{e^x-1}dx
    \end{align*}

    You are probably wondering where in the world the Gamma function or the Zeta function that I mentioned above are in this.
    The solution, as the title of this article states, is very elegant, and those functions show up in amazing places.

    We begin by multiplying the top and bottom by $e^{-x}$, 
    \begin{align*}
        \indef \frac{x}{e^x-1}\cdot \frac{e^{-x}}{e^{-x}}dx &= \indef x\frac{e^{-x}}{1-e^{-x}}dx
    \end{align*}

    At this point thinking of this would be ludicrous.
    A (very) keen eye will notice that $\frac{e^{-x}}{1-e^{-x}}$ looks suspiciously like a geometric series.
    Unsurprisingly, that is exactly what I will be doing here. So,
    \begin{align*}
        \indef x\frac{e^{-x}}{1-e^{-x}} &= \indef x\sum_{n=1}^{\infty} e^{-nx}dx \\ 
        &= \indef \sum_{n=1}^{\infty} xe^{-nx}dx
    \end{align*}
    Now with no justification I will be interchanging the sum and the integral. 
    If you are interested in finding out why this is justified, I recommend you look up ``Monotone Convergence Theorem'' or ``Dominated Convergence Theorem''.
    I have not learned those yet so I will not be showing them here.
    
    \begin{align*}
        \indef \sum_{n=1}^{\infty} xe^{-nx}dx = \sum_{n=1}^{\infty} \indef xe^{-nx}dx
    \end{align*}
    Now I will make the substitution $u = nx$ which means that $du = ndx$ and $x = u/n$. 
    Note that when $x = 0$, $u = 0$, and as $x \to \infty$, $u \to \infty$. So 
    \begin{align*}
        \sum_{n=1}^{\infty} \indef xe^{-nx}dx &= \sum_{n=1}^{\infty} \frac{u}{n}e^{-u}\frac{du}{n} \\ 
        &= \sum_{n=1}^{\infty} \frac{1}{n^2} \indef ue^{-u}du
    \end{align*}
    At this point I would recommend you go look back at Definition~\ref{Gamma}, because without doing anything weelse we already now the value of this integral.
    After doing that, I hope you can see why I included this integral in this article.
    \begin{align*}
        \sum_{n=1}^{\infty} \indef ue^{-u}du &= \sum_{n=1}^{\infty} \frac{1}{n^2} \Gamma(2) \\ 
        &= \sum_{n=1}^{\infty} \frac{1}{n^2} \\ 
        &= \zeta(2) \text{ (See Definition~\ref{Zeta})} \\ 
        &= \frac{\pi^2}{6}
    \end{align*}
    This is an amazing answer. I also amazed that MIT put a similar question on a qualifying exam--I don't believe anyone could think of a solution like this wihtout having seen it before.

    \section{Symmetries}
    A lot of hard integrals can be solved using simple symmetries. 
    First I will talk about a \textbf{very} important identity.

    \begin{theorem}[King's Property]
        \begin{align*}
            \int_{a}^{b} f(x)dx = \int_{a}^{b} f(a+b-x)dx
        \end{align*}
    \end{theorem}

    \begin{proof}
        The proof of this theorem is very simple.
        Let $u = a + b - x$. Then $du = -dx$.
        When $x = a, u = a + b - a = b$.
        When $x = b, u = a + b - b = a$. So,
        \begin{align*}
            \int_{a}^{b} f(x)dx = \int_{a}^{b} f(a+b-u)(-du)
        \end{align*}
        Recall that 
        \begin{align*}
        -\int_{b}^{a} f(x)dx = \int_{a}^{b} f(x)dx     
        \end{align*}
        Also note that u here is a dummy variable, so we can replace it with $x$.
        So,
        \begin{align*}
            \int_{b}^{a}f(a+b-u)(-du) = \int_{a}^{b} f(a+b-x)dx
        \end{align*}
    \end{proof}
    So with this under our belt, we can easy handle some very scary looking integrals.

    \section{Problems Part 2}
    The first problem we will be tackling is 
    \begin{align*}
        \int_{0}^{\pi/2} \sin^2(\sin(x)) + \cos^2(\cos(x))dx
    \end{align*}
    Utilizing Kings Property, we get that 
    \begin{align*}
        I = \int_{0}^{\pi/2} \sin^2(\sin(x))+\cos^2(\cos(x))dx &= \int_{0}^{\pi/2} \sin^2(\sin(\pi/2 - x)) + \cos^2(\cos(\pi/2 - x))dx \\ 
        &= \int_{0}^{\pi/2} \sin^2(\cos(x)) + \cos^2(\sin(x))dx
    \end{align*}
    This next step demonstrates precisely why this technique is so powerful.
    By adding $I$ to both sides, we get 
    \begin{align*}
        2I &= \int_{0}^{\pi/2} \sin^2(\sin(x)) + \cos^2(\cos(x)) + \int_{0}^{\pi/2} \sin^2(\cos(x)) + \cos^2(\sin(x))dx \\ 
        &= \int_{0}^{\pi/2} \sin^2(\sin(x)) + \cos^2(\sin(x)) + \sin^2(\cos(x)) + \cos^2(\cos(x))dx \\ 
        &= \int_{0}^{\pi/2} 1 + 1dx \\ 
        &= \pi
    \end{align*}
    Therefore, through an extremely elegant solution, we get that $I = \pi/2$.

    The next problem is similar, but equally (if not more) scary looking.
    \begin{align*}
        \int_{0}^{\pi/2} \frac{\sin^n(x)}{\sin^n(x) + \cos^n(x)}dx \text{, for any integer n}
    \end{align*}
    We begin by using King's Property once again:
    \begin{align*}
        I = \int_{0}^{\pi/2} \frac{\sin^n(x)}{\sin^n(x) + \cos^n(x)}dx &= \int_{0}^{\pi/2} \frac{\sin^n(\pi/2 - x)}{\sin^n(\pi/2 - x) + \cos^n(\pi/2 - x)}dx \\ 
        &= \int_{0}^{\pi/2} \frac{\cos^n(x)}{\cos^n(x) + \sin^n(x)}dx
    \end{align*}
    Adding $I$ to both sides, we get:
    \begin{align*}
        2I &= \int_{0}^{\pi/2} \frac{\sin^n(x)}{\sin^n(x) + \cos^n(x)} + \frac{\cos^n(x)}{\cos^n(x) + \sin^n(x)}dx \\ 
        &= \int_{0}^{\pi/2} \frac{\sin^n(x) + \cos^n(x)}{\sin^n(x) + \cos^n(x)}dx \\ 
        &= \int_{0}^{\pi/2} 1dx \\ 
        &= \frac{\pi}{2}
    \end{align*}
    So $I$ = $\pi/4$.
    
    Once again, this is an amazing result that is actually quite short because of how powerful this integration techinque is.

    The next problem we will be examining is this
    \begin{align*}
        \int_{0}^{1} \frac{\ln(x+1)}{x^2+1}dx
    \end{align*}
    \section{Important Trig Identity}
    Now we will be looking at a very important trig identity.
    \begin{theorem}[Sum of Trigonometric Functions]\label{sum of trig functions}
        \begin{align*}
            a\sin(x)+b\cos(x) = \sqrt{a^2+b^2}\sin(x+\varphi) \text{, where } \tan(\varphi) = \frac{b}{a}
        \end{align*}
    \end{theorem}
    This might look pretty complicated, but if you examine some base cases, it ends up being pretty nice. For example,
    \begin{align*}
        \sin(x)+\cos(x) = \sqrt{2}\sin(x+\pi/4)
    \end{align*}
    \begin{proof}
        The first thing we will do to prove this theorem is assume that
        \begin{align*}
            a\sin(x)+b\cos(x) = c\sin(x+\varphi)
        \end{align*}
        Now all we have to do is show that $c = \sqrt{a^2+b^2}$ and that $\tan(\varphi) = b/a$.
        If we expand the right side, we get 
        \begin{align*}
            a\sin(x)+b\cos(x) &= c(\sin(x)\cos(\varphi) + \sin(\varphi)\cos(x)) \\ 
            &= c\sin(x)\cos(\varphi) + c\sin(\varphi)\cos(x) \\ 
            &= c\cos(\varphi)\sin(x) + c\sin(\varphi)\cos(x)
        \end{align*}
        Now all we have to do is match the coefficients.
        So,
        \begin{align}
            a = c\cos(\varphi) \\ 
            b = c\sin(\varphi)
        \end{align}
        The next thing I will do is square both equations and add them to each other. We get: 
        \begin{align*}
            a^2 + b^2 &= c^2\cos^2(\varphi) + c^2\sin^2(\varphi) \\ 
            &= c^2
        \end{align*}
        So we have concluded that $c = \sqrt{a^2+b^2}$. 
        By backsubstituting $c$, and dividing equation (2) by (1), we get:
        \begin{align*}
            \frac{b}{a} &= \frac{\sqrt{a^2+b^2}\sin(\varphi)}{\sqrt{a^2+b^2}\cos(\varphi)} \\ 
            &= \tan(\varphi)
        \end{align*}

        Which completes the proof.
    \end{proof}

    This next integral is quite advanced, but can be ripped apart by the trig identity I just talked about.
    You might immediately try to use King's Property, but in this integral patience is key.
    To start, I will let $x = \tan(\theta)$.
    Then $dx = \sec^2(\theta)d\theta$.
    When $x = 0$, $\theta = 0$.
    When $x = 1$, $\theta = \pi/4$.
    \begin{align*}
        \int_{0}^{1} \frac{\ln(x+1)}{x^2+1}dx &= \int_{0}^{\pi/4} \frac{\ln(\tan(\theta)+1)}{\sec^2(\theta)}\cdot \sec^2(\theta)d\theta \\ 
        &= \int_{0}^{\pi/4} \ln(\frac{\sin(\theta)+\cos(\theta)}{\cos(\theta)})d\theta \\ 
        &= \int_{0}^{\pi/4} \ln(\sin(\theta)+\cos(\theta)) - \ln(\cos(\theta))d\theta
    \end{align*}
    Now we will use the identity that I just talked about.
    By Theorem~\ref{sum of trig functions}, $\sin(x) + \cos(x) = \sqrt{1^2+1^2}\sin(x + \tan^{-1}(1/1)) = \sqrt{2}\sin(x+\pi/4)$.
    Plugging this in, we get 
    \begin{align*}
        \int_{0}^{\pi/4} \ln(\sin(\theta)+\cos(\theta)) - \ln(\cos(\theta)) &= \int_{0}^{\pi/4} \ln(\sqrt{2}\sin(\theta+\pi/4)) - \ln(\cos(\theta))d\theta \\ 
        &= \int_{0}^{\pi/4} \ln(\sqrt{2}) + \ln(\sin(\theta + \pi/4)) - \ln(\cos(\theta))d\theta \\ 
        &= \frac{\pi \ln(\sqrt{2})}{4} + \underbrace{\int_{0}^{\pi/4} \ln(\sin(\theta+\pi/4)) - \ln(\cos(\theta))d\theta}_{I} \\
        &= \frac{\pi \ln(2)}{8} + I
    \end{align*}
    
    Now we can use King's Property.
    \begin{align*}
        I &= \int_{0}^{\pi/4} \ln(\sin(\theta+\pi/4)) - \ln(\cos(\theta))d\theta \\ 
        &= \int_{0}^{\pi/4} \ln(\sin(\pi/4 - \theta + \pi/4)) - \ln(\cos(\pi/4 - \theta))d\theta \\ 
        &= \int_{0}^{\pi/4} \ln(\cos(\theta)) - \ln(\cos(\pi/2 - (\pi/4 + \theta)))d\theta \\ 
        &= \int_{0}^{\pi/4} \ln(\cos(\theta)) - \ln(\sin(\theta + \pi/4))d\theta
    \end{align*}
    Finally, adding $I$ to both sides, we get 
    \begin{align*}
        2I &= \int_{0}^{\pi/4} \ln(\sin(\theta + \pi/4)) - \ln(\cos(\theta)) + \ln(\cos(\theta)) - \ln(\sin(\theta+\pi/4))d\theta \\ 
        &= \int_{0}^{\pi/4} 0d\theta \\ 
        &= 0
    \end{align*}
    Which means that $I = 0$. 
    Therefore, our original integral 
    \begin{align*}
        \int_{0}^{1} \frac{\ln(x+1)}{x^2+1}dx = \frac{\pi \ln(2)}{8}
    \end{align*}
    This is another elegant result.
    This was on the Putnam Exam, an international math exam for undergraduate students in college.
    That's all for now, I hope you enjoyed reading this magazine!
%    \section{Contour Integration}
%    The first thing we need to know in this section is Cauchy's Residue Theorem.
%    \begin{theorem}[Cauchy's Residue Theorem]
%        If $C$ is a simple closed, positively oriented contour in the complex plane, and $f$ is analytic except for some points $z_1, z_2, \dots, z_n$ inside the contour $C$, then 
%        \begin{align*}
%            \oint_C f(z)dz = 2\pi i\sum_{k=1}^{n} \mathrm{Res}_f(z_k)
%        \end{align*}
%    \end{theorem}
%    The proof of this theorem is very complicated, so I will not be proving it here (that is, I don't know how). 
%    Another thing that we have to know is what $\mathrm{Res}_f(z_k)$ means.
%    \begin{theorem}[Calculation of Residue]
%        Suppose that $z_0$ is a singularity of $f(z)$ of multiplicity 1. Then 
%        \begin{align*}
%            \mathrm{Res}_f(z_0) = \lim_{z \to z_0} (z-z_0)f(z)
%        \end{align*}
%    \end{theorem}
%    You might be wondering what some of the words mean in Theorem 6.1. 
%    Simple closed means that there are no gaps in the curve, and that it does not cross over itself.
%    Positively oriented means the curve travels counterclockwise.
%    If the curve is negatively oriented, we can just multiply the integral by -1. 
%    And finally, you can interpret contour as just a simple loop. 
%    Analytic pretty much means differentiable as a complex function, but that ends up being really complicated so I won't be talking about that.
%    A singularity pretty much means that you are dividing by zero.  
%    Finally, the circle in the integral symbol means that you are integrating over a closed curve. 
%    You technically do need the tools of vector calculus to understand this, but even without it it's not super confusing.
%
%    So with these tools, the next integral that we will be exploring is 
%    \begin{align*}
%        \int_{0}^{\infty} \frac{\cos(x)}{1+x^2}dx = \frac{\pi}{e}
%    \end{align*}
%    This is immediately a beautiful result. A $\pi$? A $e$?
%    We begin by changing this to a contour integral. 
%    Note that 
%    \begin{align*}
%        \int_{0}^{\infty} \frac{\cos(x)}{1+x^2}dx= \Re\qty(\indef \frac{\cos(x)}{x^2+1}dx + i\indef \frac{\sin(x)}{x^2+1}dx) = \Re\qty(\indef \frac{e^{ix}}{x^2+1}dx) 
%    \end{align*}
%    This is where I will bring in the contour integration. 
%    This is the contour that we will be integrating over:
%
%    \begin{tikzpicture}[decoration={markings,
%		mark=at position 1cm with {\arrow[line width=1pt]{>}},
%		mark=at position 3cm with {\arrow[line width=1pt]{>}},
%		mark=at position 6.3cm with {\arrow[line width=1pt]{>}},
%		mark=at position 9cm with {\arrow[line width=1pt]{>}}
%		}
%		]
%		% The axes
%		\draw[help lines,->] (-3,0) -- (3,0) coordinate (xaxis);
%		\draw[help lines,->] (0,-1) -- (0,3) coordinate (yaxis);
%		
%		% The path
%		\path[draw,line width=0.8pt,postaction=decorate] (-2, 0) -- (2, 0) arc (0:180:2);
%		
%		% The labels
%		\node[below] at (xaxis) {$x$};
%		\node[left] at (yaxis) {$y$};
%		\node[below left] {$O$};
%		\node at (0.3, 2.3) {$\Gamma_R$};
%        \node at (0.3, -0.5) {$\gamma_R$};
%        \node at (2.1, -0.2) {$R$};
%        \node at (-2.1, -0.2) {$-R$};
%	\end{tikzpicture}
%
%    Using this picture, note that 
%    \begin{align*}
%        \lim_{R \to \infty} \int_{\gamma_{R}} \frac{e^{iz}}{z^2+1}dz = \indef \frac{e^{ix}}{x^2+1}dx
%    \end{align*}

	Suppose that $(n_k)$ is a sequence of increasing positive integers satisfying $\lim_{k \to \infty} \frac{n_k}{n_{k-1}} = 1$, $(a_n)$ is a bounded sequence, and 
	\begin{align*}
		\lim_{k \to \infty} \frac{1}{n_k}\sum_{i=1}^{n_k} a_i = a
	\end{align*}
	then 
	\begin{align*}
		\lim_{n\to \infty} \frac 1n \sum_{i=1}^n a_i = a
	\end{align*}
	
	\begin{proof}
		Fix $\varepsilon > 0$ and find $K > 0$ so that $(1-\ve)n_{k} < n_{k-1} < (1+\ve)n_{k}$ holds for all $k > K$. Then if $D$ is a global constant bounding $a_n-a$, fixing $n > n_K$ and finding $k$ so that $n_{k-1} \leq n \leq n_k$, we have that $(1-\ve)n_{k} \leq n$, so 
		\begin{align*}
			\frac 1n \qty|\sum_{i=1}^n (a_i-a)| &\leq \frac 1n \qty(\qty|\sum_{i=1}^{n_k} (a_i-a)| + \qty|\sum_{i=n}^{n_k} (a_i-a)|) \\
			&\leq \frac{1}{(1-\ve)n_k} \qty|\sum_{i=1}^{n_k} (a_i-a)| + \frac{(n_k-n)}{n} D \\
			&\leq \frac{\ve}{(1-\ve)} + D\ve\frac{n_k}{n_{k-1}} \frac{n_{k-1}}{n} \leq C_1\ve + D\ve (1+\ve) \leq C_2\ve
		\end{align*}
		for some constant $C_2$. This completes the proof.
	\end{proof}

\end{document}
