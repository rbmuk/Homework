\documentclass[12pt]{article}
\usepackage[margin=1in]{geometry}
\usepackage{setspace}
\onehalfspacing{}

% Start of preamble
%==========================================================================================%
% Required to support mathematical unicode
\usepackage[warnunknown, fasterrors, mathletters]{ucs}
\usepackage[utf8x]{inputenc}

% Standard mathematical typesetting packages
\usepackage{amsmath,amssymb,amscd,amsthm,amsxtra, pxfonts}
\usepackage{mathtools,mathrsfs,xparse}

% Symbol and utility packages
\usepackage{cancel, textcomp}
\usepackage[mathscr]{euscript}
\usepackage[nointegrals]{wasysym}
\usepackage{apacite}

% Extras
\usepackage{physics}  % Lots of useful shortcuts and macros
\usepackage{tikz-cd}  % For drawing commutative diagrams easily
\usepackage{microtype}  % Minature font tweaks
%\usepackage{pgfplots} % plots

\usepackage{enumitem}
\usepackage{titling}

\usepackage{graphicx}

% Common shortcuts
\def\mbb#1{\mathbb{#1}}
\def\mfk#1{\mathfrak{#1}}

\def\bN{\mbb{N}}
\def\C{\mbb{C}}
\def\R{\mbb{R}}
\def\bQ{\mbb{Q}}
\def\bZ{\mbb{Z}}
\def\cph{\varphi}
\renewcommand{\th}{\theta}
\def\ve{\varepsilon}
\newcommand{\mg}[1]{\| #1 \|}

% Often helpful macros
\newcommand{\floor}[1]{\left\lfloor#1\right\rfloor}
\newcommand{\ceil}[1]{\left\lceil#1\right\rceil}
\renewcommand{\qed}{\hfill\qedsymbol}
\renewcommand{\ip}[1]{\langle#1\rangle}
\newcommand{\seq}[2]{\qty(#1_#2)_{#2=1}^{\infty}}

\newcommand{\SET}[1]{\Set{\mskip-\medmuskip #1 \mskip-\medmuskip}}

% End of preamble
%==========================================================================================%

% Start of commands specific to this file
%==========================================================================================%

\usepackage{braket}
\newcommand{\Z}{\mbb Z}
\newcommand{\gen}[1]{\left\langle #1 \right\rangle}
\newcommand{\nsg}{\trianglelefteq}
\newcommand{\F}{\mbb F}
\newcommand{\Aut}{\mathrm{Aut}}
\newcommand{\sepdeg}[1]{[#1]_{\mathrm{sep}}}
\newcommand{\Q}{\mbb Q}
\newcommand{\Gal}{\mathrm{Gal}\qty}
\newcommand{\id}{\mathrm{id}}
\newcommand{\Hom}{\mathrm{Hom}_R}
\newcommand{\1}{\mathds 1}
\newcommand{\N}{\mathbb N}
\renewcommand{\P}{\mathbb P \qty}
\newcommand{\E}{\mathbb E \qty}
\newcommand{\Var}{\mathrm{Var}}
\newcommand{\argmax}{\mathrm{argmax}}
\newcommand{\Vol}{\mathrm{Vol}}

%==========================================================================================%
% End of commands specific to this file

\title{Math 582 HW1}
\date{\today}
\author{Rohan Mukherjee}

\begin{document}
    \maketitle
    \begin{enumerate}
        \item Let $A = \begin{pmatrix}
            a_1^T \\
            \vdots \\
            a_n^T
        \end{pmatrix}$. For $n \geq 3$, consider the quantity $(a_1 +a_2)^T (a_2+a_3) = a_2^T a_2 = n$ by orthogonality. However, 
        \begin{align*}
            (a_1+a_2)^T(a_2+a_3) = \sum_i (a_{1i}+a_{2i})(a_{2i}+a_{3i})
        \end{align*}
        $a_{1i}+a_{2i}$ and $a_{2i}+a_{3i}$ are both divisible by 2 (being either 0, 2 or -2 each), so this sum is divisible by 4. But this means that $4 \mid n$. 
        \item The matrix is as follows: 
        \begin{align*}
            \frac{1}{\sqrt{3}} \begin{pmatrix}
                1 & 1 & 1 & 0 \\
                1 & 1 & -1 & 0 \\
                1 & -1 & 1 & 0 \\
                1 & -1 & -1 & 0
            \end{pmatrix}
        \end{align*}
        Since every vector $x \in \SET{\pm 1}^4$ will have a prefix of one of those rows (up to a possible sign change), we can always get a value of $\frac{3}{\sqrt{3}} = \sqrt{3}$. An important paper by Steinerberger discussing the bad science problem shows that when we normalize the rows, the Kolmos constant cannot be finite. There is even an explicit construction giving $\geq \sqrt{\log_2(n+1)}$, yet, it alludes me on how to achieve $17\%$ better $\sqrt{2\ln(n+1)}$.

        For when the columns are normalized, consider the following $5 \times 5$ matrix:
        \begin{align*}
            \begin{pmatrix}
                1/2 & 1/2 & 1/2 & 0 & 0 \\
                1/2 & 1/2 & -1/2 & 0 & 0\\
                1/2 & -1/2 & 1/2 & 0 & 0\\
                1/2 & -1/2 & -1/2 & 0 & 0 \\
                0 & 0 & 0 & 1 & 1 \\
            \end{pmatrix}
        \end{align*}
        Every vector $x \in \SET{\pm 1}^5$ will have a prefix of one of the top 4 rows (up to a possible sign change). This will then give an inner product of $\frac{3}{2}$. The last row is just a throwaway. This shows the Kolmos constant with rows is at least $\frac{3}{2} > \sqrt{2}$. After readinng Kunisky's paper, I have been enlightened even further. Let $A_0 \in \R^{n \times n}$ be any starting matrix with normalized columns. Define 
        \begin{align*}
            A_{n+1} = \frac{1}{\sqrt{2}} \begin{pmatrix}
                A_n & I \\
                -A_n & I
            \end{pmatrix}
        \end{align*}
        Further define $K(A) = \inf_{x \in \SET{\pm 1}^n} \mg{Ax}_\infty$. Then I claim that $K(A_n) = C \cdot 2^{-n/2} + 2/(\sqrt{2}-1)$. 

        This follows by letting $x_n = K(a_n)$, then noticing that $x_n = \frac{1}{\sqrt{2}} x_{n-1} + \frac{1}{\sqrt{2}}$. This is inhomogeneous linear recurrence, which can be solved easily. The constant part satisfies $c = c/\sqrt{2} + \frac{1}{\sqrt{2}}$, which gives $c = 2/(\sqrt{2}-1)$. The exponential part satisfies $b_n = b_{n-1}/\sqrt{2}$, which gives $b_n = C \cdot 2^{-n/2}$. Their sum $K(A_n)$ is then $C \cdot 2^{-n/2} + 2/(\sqrt{2}-1)$, which is the desired result. Taking $n \to \infty$ gives a sequence of matrices whose $K$ value goes upto $\frac{2}{\sqrt{2} - 1}$, which is around 4.8, much bigger than 2.
        
        \item First I prove the following lemma. Let $W,V$ be subspaces, and consider the map $f: V \to (V+W)/W$ sending $v$ to $\overline v$. As every vector in $V+W$ can be written as $v+w$, passing to a quotient gives just $\overline v$, so this map is surjective. By the first isomorphism theorem, $\dim V = \dim \ker f + \dim \Im f$. So, $\dim V \geq \dim \Im f = \dim (V+W)/W = \dim (V+W) - \dim W$. Thus, $\dim (V+W) \leq \dim V + \dim W$. In fact we prove the stronger result that equality holds iff $V \cap W = \SET{0}$. This is because equality holds when $\ker f = \SET{0}$, and the kernel can be described by the set of $v$ so that $v \in W$, which is just $V \cap W$.

        Now, let $A = (a_1, \ldots, a_n)$ and $B = (b_1, \ldots, b_n)$. Clearly,
        \begin{align*}
            \mathrm{span} \SET{a_1, \ldots, a_n} \subset \mathrm{span} \SET{a_1+b_1, \ldots, a_n+b_n, b_1, \ldots, b_n}
        \end{align*} The first subset has dimension $\rank A$ and the last has dimension $\leq \rank (A+B) + \rank B$. This shows that $\rank A - \rank B \leq \rank (A+B)$, and similarly the reverse holds, so $|\rank A - \rank B| \leq \rank (A+B)$.

        Now, also $\mathrm{span} \SET{a_1+b_1, \ldots, a_n+b_n} \subset \mathrm{span} \SET{a_1, \ldots, a_n, b_1, \ldots, b_n}$. The first subset has dimension $\rank (A+B)$ and the last has dimension $\leq \rank A + \rank B$. This shows that $\rank (A+B) \leq \rank A + \rank B$, which completes the proof.

        \item Let $B_r(x) \subset \R^n$ be the $\ell^1$ ball of radius $r$ around the point $x$. I claim that $\Vol(B_r(x)) = \frac{2^n}{n!} r^n$. We prove this by induction. Suppose that it is true for $n-1$. Then the volume of the 1-ball in $n$ dimensions is just:
        \begin{align*}
            \int_{-1}^1 \frac{2^n}{n!} r^n dr = \frac{2^n}{n!} \frac{1}{n+1} r^{n+1} \Big|_{-1}^1 = \frac{2^{n+1}}{(n+1)!}
        \end{align*}
        By simple scaling map and translational invariance of volume, we see that the volume of the $r$-ball around any point is just $\frac{2^{n+1}}{(n+1)!} r^{n+1}$.

        I give second proof for fun. We know that there are $2^n$ quadrants in in $n$ dimensions, so we only look at the first quadrant $x_i \geq 0$ for all $i$. Then we just have to show that the volume of $\sum x_i \leq 1$ is $\frac{1}{n!}$ when $x_i \geq 0$ for all $i$. 

        Since this is linear algebra class, we use linear algebra. Consider linear map sending $(x_1, \ldots, x_n)$ to $(x_1, x_1+x_2, \ldots, x_1+\cdots+x_n)$. The matrix $A$ is just:
        \begin{align*}
            \begin{pmatrix}
                1 & 0 & 0 & \cdots & 0 \\
                1 & 1 & 0 & \cdots & 0 \\
                1 & 1 & 1 & \cdots & 0 \\
                \vdots & \vdots & \vdots & \ddots & \vdots \\
                1 & 1 & 1 & \cdots & 1
            \end{pmatrix}
        \end{align*}
        Clearly, the image of $S = \SET{x_1+\cdots + x_n \leq 1}$ under $A$ is just $T = \SET{0 \leq x_1 \leq x_2 \leq \cdots \leq x_n \leq 1}$. Intuitively, for almost every point $x \in [0,1]^n$, there is a unique $\pi \in S_n$ so that $x_{\pi(1)} \leq x_{\pi(2)} \leq \cdots \leq x_{\pi(n)}$ (just look at the numbers and order them in increasing order; points where there are multiple correspond to where two of the coordinates are equal, and these have measure 0). Since the $n!$ permutations of the indices of $T$ all have the same volume, whose volume is a (almost) disjoint union of $n!$ copies summing to the volume of $[0,1]^n$ which is 1, we see that the volume of $T$ is just $\frac{1}{n!}$. Since $\det A = 1$, this means the volume of $S$ is $(\det A)^{-1} \frac{1}{n!} = \frac{1}{n!}$ too. Incorporating the other quadrants, the volume of $|x_1| + \cdots + |x_n| \leq 1$ is just $\frac{2^n}{n!}$. 

        Assume that $S$ is countably infinite (otherwise throw away some points).
        
        We now use a very standard volume argument. Consider the set $U = B_{3/2}(x_1)$. I claim that $\bigcup_i B_{1/2}(x_i) \subset U$. This is because if $y \in B_{1/2}(x_i)$, then $|y-x_1| \leq |y-x_i| + |x_i-x_1| < 3/2$. The open balls $B_{1/2}(x_1), B_{1/2}(x_2), \ldots$ are disjoint (if $y \in B_{1/2}(x_1) \cap B_{1/2}(x_2)$, then $|x_1-x_2| \leq |y-x_1| + |y-x_2| < 1$, a contradiction), and so the volume of their union is the sum of the volumes. This sum is:
        \begin{align*}
            \sum_{i=1}^\infty \Vol(B_{1/2}(x_i)) = \sum_{i=1}^\infty \frac{2^n}{n!} 2^{-n} = \sum_{i=1}^\infty \frac{1}{n!} = \infty
        \end{align*}
        But $S$ is contained in $B_3/2(x_1)$ which has finite volume $\leq 2^n/n! \cdot (3/2)^n = 3^n/n!$, a contradiction. Further, this shows that $|S| \leq 3^n$.
    \end{enumerate}
\end{document}