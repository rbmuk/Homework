\documentclass[12pt]{article}
\usepackage[margin=1in]{geometry}
\usepackage{setspace}
\onehalfspacing{}
\usepackage[dvipsnames,table,xcdraw]{xcolor} % colors

% Start of preamble
%==========================================================================================%
% Required to support mathematical unicode
\usepackage[warnunknown, fasterrors, mathletters]{ucs}
\usepackage[utf8x]{inputenc}

% Standard mathematical typesetting packages
\usepackage{amsmath,amssymb,amscd,amsthm,amsxtra}
\usepackage{mathtools,mathrsfs,xparse,newtxtext,newtxmath}

% Symbol and utility packages
\usepackage{cancel, textcomp}
\usepackage[mathscr]{euscript}
\usepackage[nointegrals]{wasysym}
\usepackage{apacite}

% Extras
\usepackage{physics}  % Lots of useful shortcuts and macros
\usepackage{tikz-cd}  % For drawing commutative diagrams easily
\usepackage{microtype}  % Minature font tweaks
%\usepackage{pgfplots} % plots

\usepackage{enumitem}
\usepackage{titling}

\usepackage{graphicx}

%\usepackage{quiver}

% Fancy theorems due to @intuitively on discord
\usepackage{mdframed}
\newmdtheoremenv[
backgroundcolor=NavyBlue!30,
linewidth=2pt,
linecolor=NavyBlue,
topline=false,
bottomline=false,
rightline=false,
innertopmargin=10pt,
innerbottommargin=10pt,
innerrightmargin=10pt,
innerleftmargin=10pt,
skipabove=\baselineskip,
skipbelow=\baselineskip]{mytheorem}{Theorem}

\newenvironment{theorem}{\begin{mytheorem}}{\end{mytheorem}}

\newtheorem{corollary}{Corollary}
\newtheorem{lemma}{Lemma}

\newtheoremstyle{definitionstyle}
{\topsep}%
{\topsep}%
{}%
{}%
{\bfseries}%
{.}%
{.5em}%
{}%
\theoremstyle{definitionstyle}
\newmdtheoremenv[
backgroundcolor=Violet!30,
linewidth=2pt,
linecolor=Violet,
topline=false,
bottomline=false,
rightline=false,
innertopmargin=10pt,
innerbottommargin=10pt,
innerrightmargin=10pt,
innerleftmargin=10pt,
skipabove=\baselineskip,
skipbelow=\baselineskip,
]{mydef}{Definition}
\newenvironment{definition}{\begin{mydef}}{\end{mydef}}

\newtheorem*{remark}{Remark}

\newtheorem*{example}{Example}
\newtheorem*{claim}{Claim}

% Common shortcuts
\def\mbb#1{\mathbb{#1}}
\def\mfk#1{\mathfrak{#1}}

\def\bN{\mbb{N}}
\def\C{\mbb{C}}
\def\R{\mbb{R}}
\def\bQ{\mbb{Q}}
\def\bZ{\mbb{Z}}
\def\cph{\varphi}
\renewcommand{\th}{\theta}
\def\ve{\varepsilon}
\newcommand{\mg}[1]{\| #1 \|}

% Often helpful macros
\newcommand{\floor}[1]{\left\lfloor#1\right\rfloor}
\newcommand{\ceil}[1]{\left\lceil#1\right\rceil}
\renewcommand{\qed}{\hfill\qedsymbol}
\renewcommand{\ip}[1]{\langle#1\rangle}
\newcommand{\seq}[2]{\qty(#1_#2)_{#2=1}^{\infty}}

\newcommand{\SET}[1]{\Set{\mskip-\medmuskip #1 \mskip-\medmuskip}}

% End of preamble
%==========================================================================================%

% Start of commands specific to this file
%==========================================================================================%

\usepackage{braket}
\newcommand{\Z}{\mbb Z}
\newcommand{\gen}[1]{\left\langle #1 \right\rangle}
\newcommand{\nsg}{\trianglelefteq}
\newcommand{\F}{\mbb F}
\newcommand{\Aut}{\mathrm{Aut}}
\newcommand{\sepdeg}[1]{[#1]_{\mathrm{sep}}}
\newcommand{\Q}{\mbb Q}
\newcommand{\Gal}{\mathrm{Gal}\qty}
\newcommand{\id}{\mathrm{id}}
\newcommand{\Hom}{\mathrm{Hom}_R}
\newcommand{\1}{\mathds 1}
\newcommand{\N}{\mathbb N}
\renewcommand{\P}{\mathbb P \qty}
\newcommand{\E}{\mathbb E \qty}
\newcommand{\Var}{\mathrm{Var}}
\everymath{\displaystyle}
\newcommand{\argmax}{\mathrm{argmax}}

%==========================================================================================%
% End of commands specific to this file

\title{Math HW5}
\date{\today}
\author{Rohan Mukherjee}

\begin{document}
    \maketitle
    \begin{enumerate}
        \item We need to show that:
        \begin{align*}
            \max_{\pi \in S_n} \sum_{i=1}^n \lambda_{\pi(i)}^2 + A_{ii}^2 - 2\lambda_{\pi(i)}A_{ii} \geq \frac{1}{n-1}\sum_{i=1}^n R_i^2
        \end{align*}
        Then notice that
        \begin{align*}
            \frac{1}{n-1} \sum_{i=1}^n R_i^2 = \frac{1}{n-1} \sum_{i=1}^n \qty(\sum_{j \neq i} |A_{ij}|)^2 \leq \sum_{i=1}^n \sum_{j \neq i} |A_{ij}|^2 = \mg{A}_F^2 - \sum_{i=1}^n A_{ii}^2 = \sum_{i=1}^n \lambda_i^2 - \sum_{i=1}^n A_{ii}^2
        \end{align*}
        So a strictly stronger inequality is to show that:
        \begin{align*}
            \sum_{i=1}^n \lambda_i^2 - \sum_{i=1}^n A_{ii}^2 \leq \max_{\pi \in S_n} \sum_{i=1}^n \lambda_{\pi(i)}^2 + A_{ii}^2 - 2\lambda_{\pi(i)}A_{ii}
        \end{align*}
        Or equivalently,
        \begin{align*}
            0 \leq \max_{\pi \in S_n} \sum_{i=1}^n 2A_{ii}^2 - 2 \lambda_{\pi(i)}A_{ii}
        \end{align*}
        Put uniform measure on $S_n$. Then,
        \begin{align*}
            E_{\pi}\qty(\sum_{i=1}^n \lambda_{\pi(i)}A_{ii}) = \sum_{i=1}^n A_{ii} E_{\pi}(\lambda_{\pi(i)})
        \end{align*}
        For a fixed $i$, $\pi(i)$ is uniformly distributed in $[n]$. Thus this equals:
        \begin{align*}
            \sum_{i=1}^n A_{ii} \frac{1}{n} \qty(\sum_{j=1}^n \lambda_j) = \frac{1}{n} \Tr(A)^2 = \frac{1}{n} \qty(\sum_{i=1}^n A_{ii})^2 \leq \sum_{i=1}^n A_{ii}^2
        \end{align*}
        Since it happens on average, there exists permutation $\pi \in S_n$ that makes it happen. This completes the proof.

        \item This is false. For any starting matrix $A$, consider $B = -A$. Then $1/2A + 1/2B = 0$, whose largest eigenvalue 0 is not simple as long as $n \geq 2$.

        \item Since $\lambda_{n-1}$ is closer to $\lambda_n$ than $\lambda_i$ for $i \leq n-1$, we have that:
        \begin{align*}
            \lambda_n' \leq \frac{n-1}{\lambda_n - \lambda_{n-1}}
        \end{align*}
        By dropping all the positive terms, we also have that:
        \begin{align*}
            \lambda_{n-1}' \geq \frac{1}{\lambda_{n-1} - \lambda_n}
        \end{align*}
        Thus,
        \begin{align*}
            \lambda_n' - \lambda_{n-1}' \leq \frac{n}{\lambda_n - \lambda_{n-1}}
        \end{align*}
        Let $v = \lambda_n - \lambda_{n-1}$. The problem tells us that $v > 0$ always. So,
        \begin{align*}
            vv' \leq n
        \end{align*}
        Integrating both sides,
        \begin{align*}
            v^2(t)/2 - v^2(0)/2 = \int_{0}^t vv' dx \leq \int_0^t n dx = nt
        \end{align*}
        For $t \geq T$ for some $T$ depending only on $v(0)$, we have $\lambda_n - \lambda_{n-1} = v(t) \leq 2\sqrt{nt}$ (More specifically, for $t \geq v^2(0)/2n = T$, $v^2(0)/2 \leq nt$). We seek to repeat this process for all the other gaps. By dropping the negative term, and using that $\lambda_{n-2}$ is closer to $\lambda_{n-1}$ than $\lambda_i$ for $i \leq n-2$, we have that:
        \begin{align*}
            \lambda_{n-1}' \leq \frac{n-2}{\lambda_{n-1} - \lambda_{n-2}}
        \end{align*}
        By a similar logic, we have that (recall that by making the denominator smaller, we are pushing the fraction far in the negative direction, giving a better lower bound):
        \begin{align*}
            \lambda_{n-2}' \geq \frac{2}{\lambda_{n-2}-\lambda_{n-1}}
        \end{align*}
        As before this means that $\lambda_{n-1} - \lambda_{n-2} \leq 2\sqrt{nt}$ for $T$ sufficiently large depending only on the initial conditions.

        Then, (up to some constant depending only on $\lambda_i(0)$ for each $i$):
        \begin{align*}
            \lambda_n - \lambda_1 = \sum_{i=1}^{n} \lambda_i - \lambda_{i-1} \leq 2n^{3/2} \sqrt{t}
        \end{align*}
        Now, 
        \begin{align*}
            \lambda_n' \geq \frac{1}{\lambda_n - \lambda_1} \geq \frac{1}{2n^{3/2} \sqrt{t}}
        \end{align*}
        Simutlaenously,
        \begin{align*}
            \lambda_1' \leq \frac{1}{\lambda_1 - \lambda_n} \leq \frac{-1}{2n^{3/2} \sqrt{t}}
        \end{align*}
        Integrating these inequalities shows that (up to another constant depending on $\lambda_i(0)$ and $\lambda_n(0)$):
        \begin{align*}
            \lambda_1 \leq \frac{-C}{n^{3/2}} \sqrt{t} \leq \frac{C}{n^{3/2}} \sqrt{t} \leq \lambda_n \quad \text{for $t \geq T$}
        \end{align*} where $T$ depends only on the $\lambda_i(0)$. Indeed, in particular, the solutions are unbounded (also this is better than $\log t$).

        \item There are two answers here, one that is more fun than the other. The first is to just consider $X = 0$. Since $A$ has an eigenvalue 1 with multiplicity 2 it must not be 0. Then $A+\ve X = A$ also has eigenvalue 1 with multiplicity 2 always. The other answer is to consider the Jordan normal form of the matrix $A$:
        \begin{align*}
            A = P\begin{pmatrix}
                1 & 1 & \mathbf{0} \\
                0 & 1 & \mathbf{0} \\
                \mathbf{0} & \mathbf{0} & B
            \end{pmatrix}P^{-1}
        \end{align*}
        From here, taking
        \begin{align*}
            X = P\begin{pmatrix}
                0 & 1 & \mathbf{0} \\
                0 & 0 & \mathbf{0} \\
                \mathbf{0} & \mathbf{0} & B
            \end{pmatrix}P^{-1}
        \end{align*}
        Works for all $\ve > 0$. You can just read the eigenvalues off the main diagonal for the $2 \times 2$ upper triangular block in the top left.
    \end{enumerate}
\end{document}