\documentclass[12pt]{article}
\usepackage[margin=1in]{geometry}
\usepackage{setspace}
\onehalfspacing{}
\usepackage[dvipsnames,table,xcdraw]{xcolor} % colors

% Start of preamble
%==========================================================================================%
% Required to support mathematical unicode
\usepackage[warnunknown, fasterrors, mathletters]{ucs}
\usepackage[utf8x]{inputenc}

% Standard mathematical typesetting packages
\usepackage{amsmath,amssymb,amscd,amsthm,amsxtra, pxfonts}
\usepackage{mathtools,mathrsfs,xparse}

% Symbol and utility packages
\usepackage{cancel, textcomp}
\usepackage[mathscr]{euscript}
\usepackage[nointegrals]{wasysym}
\usepackage{apacite}

% Extras
\usepackage{physics}  % Lots of useful shortcuts and macros
\usepackage{tikz-cd}  % For drawing commutative diagrams easily
\usepackage{microtype}  % Minature font tweaks
%\usepackage{pgfplots} % plots

\usepackage{enumitem}
\usepackage{titling}

\usepackage{graphicx}

%\usepackage{quiver}

% Fancy theorems due to @intuitively on discord
\usepackage{mdframed}
\newmdtheoremenv[
backgroundcolor=NavyBlue!30,
linewidth=2pt,
linecolor=NavyBlue,
topline=false,
bottomline=false,
rightline=false,
innertopmargin=10pt,
innerbottommargin=10pt,
innerrightmargin=10pt,
innerleftmargin=10pt,
skipabove=\baselineskip,
skipbelow=\baselineskip]{mytheorem}{Theorem}

\newenvironment{theorem}{\begin{mytheorem}}{\end{mytheorem}}

\newtheorem{corollary}{Corollary}
\newtheorem{lemma}{Lemma}

\newtheoremstyle{definitionstyle}
{\topsep}%
{\topsep}%
{}%
{}%
{\bfseries}%
{.}%
{.5em}%
{}%
\theoremstyle{definitionstyle}
\newmdtheoremenv[
backgroundcolor=Violet!30,
linewidth=2pt,
linecolor=Violet,
topline=false,
bottomline=false,
rightline=false,
innertopmargin=10pt,
innerbottommargin=10pt,
innerrightmargin=10pt,
innerleftmargin=10pt,
skipabove=\baselineskip,
skipbelow=\baselineskip,
]{mydef}{Definition}
\newenvironment{definition}{\begin{mydef}}{\end{mydef}}

\newtheorem*{remark}{Remark}

\newtheorem*{example}{Example}
\newtheorem*{claim}{Claim}

% Common shortcuts
\def\mbb#1{\mathbb{#1}}
\def\mfk#1{\mathfrak{#1}}

\def\bN{\mbb{N}}
\def\C{\mbb{C}}
\def\R{\mbb{R}}
\def\bQ{\mbb{Q}}
\def\bZ{\mbb{Z}}
\def\cph{\varphi}
\renewcommand{\th}{\theta}
\def\ve{\varepsilon}
\newcommand{\mg}[1]{\| #1 \|}

% Often helpful macros
\newcommand{\floor}[1]{\left\lfloor#1\right\rfloor}
\newcommand{\ceil}[1]{\left\lceil#1\right\rceil}
\renewcommand{\qed}{\hfill\qedsymbol}
\renewcommand{\ip}[1]{\langle#1\rangle}
\newcommand{\seq}[2]{\qty(#1_#2)_{#2=1}^{\infty}}

\newcommand{\SET}[1]{\Set{\mskip-\medmuskip #1 \mskip-\medmuskip}}

% End of preamble
%==========================================================================================%

% Start of commands specific to this file
%==========================================================================================%

\usepackage{braket}
\newcommand{\Z}{\mbb Z}
\newcommand{\gen}[1]{\left\langle #1 \right\rangle}
\newcommand{\nsg}{\trianglelefteq}
\newcommand{\F}{\mbb F}
\newcommand{\Aut}{\mathrm{Aut}}
\newcommand{\sepdeg}[1]{[#1]_{\mathrm{sep}}}
\newcommand{\Q}{\mbb Q}
\newcommand{\Gal}{\mathrm{Gal}\qty}
\newcommand{\id}{\mathrm{id}}
\newcommand{\Hom}{\mathrm{Hom}_R}
\newcommand{\1}{\mathds 1}
\newcommand{\N}{\mathbb N}
\renewcommand{\P}{\mathbb P \qty}
\newcommand{\E}{\mathbb E \qty}
\newcommand{\Var}{\mathrm{Var}}
\everymath{\displaystyle}
\newcommand{\argmax}{\mathrm{argmax}}
\usepackage{float}

%==========================================================================================%
% End of commands specific to this file

\title{Math Template}
\date{\today}
\author{Rohan Mukherjee}

\begin{document}
    \maketitle
    \begin{enumerate}
        \item Writing 
        \begin{align*}
            A = \begin{pmatrix}
                a_1^T \\
                \vdots \\
                a_n^T
            \end{pmatrix}
        \end{align*}
        we can see that,
        \begin{align*}
            \left\|A\frac{a_i}{|a_i|}\right\| = \frac{1}{\mg{a_i}}\sum_{j=1}^n (a_j^Ta_i)^2 \geq |a_i|
        \end{align*}
        Since this holds for every row, $\mg{A}_{op} \geq \sup_{1 \leq i \leq m} \qty(\sum_{j=1}^n |A_{ij}|^2)$.

        For $\mg{x} = 1$,
        \begin{align*}
            \mg{ABx} = \mg{Bx} \cdot \mg{A(Bx/\mg{Bx})} \leq \mg{Bx} \cdot \mg{A}_{op} \leq \mg{A}_{op}\mg{B}_{op}
        \end{align*}

        Notice from that last part,
        \begin{align*}
            \mg{ABx} \leq \mg{Bx} \cdot \mg{A}_{op}
        \end{align*}
        Now, $\sup_{\mg{x} = 1} \mg{Bx}^2 = \sup_{\mg{x} = 1} x^T B^T B x$ is the largest eigenvalue of $B^TB$, say $\lambda$. Recall that $\mg{B}_F^2$ can be written as $\Tr(B^TB)$. By properties of trace, $\Tr(B^TB) = \sum_i \lambda_i$ where $\lambda_i$ is the $i$th largest eigenvalue of $B^TB$ (breaking ties arbitrarily). Then clearly, since $B^TB$ is PSD, $\lambda_1 \leq \sum \lambda_i$. This shows that $\mg{B}_{op} \leq \mg{B}_F$. Using that above completes the proof.

        \item Consider any centrally symmetric convex set $\Omega$, which is compact, and let $B$ be the largest ellipse contained in $\Omega$. Via an affine transformation, we can ensure that $B$ is the unit ball. Let $C = \inf \SET{t > 0 \mid \Omega \subset tB}$. Assume by contradiction that $C$ were large. Then there is some point of $\Omega$ of magnitude at least bigger than $C$. By connecting this point to the origin, we can find a point in $\Omega$ with norm precisely equal to $C$. By symmetry, the opposite of this point is also in $\Omega$, and after a rotation, we may assume that these two points are $(\pm C, 0)$. By connecting the top and bottom of the unit circle with these points, we get the following shape fully contained in $\Omega$:

        \begin{figure}[H]
            \centering
            \begin{tikzpicture}
                % Axes
                \draw[->] (-9,0) -- (9,0) node[right] {$x$};
                \draw[->] (0,-2) -- (0,2) node[above] {$y$};
                
                % Parallelogram
                \draw[thick] (0,1) -- (7,0) -- (0,-1) -- (-7,0) -- cycle;
                
                % Dots at vertices
                \filldraw[red] (0,1) circle (2pt) node[above] {$(0,1)$};
                \filldraw[red] (7,0) circle (2pt) node[above right] {$(C,0)$};
                \filldraw[red] (0,-1) circle (2pt) node[below] {$(0,-1)$};
                \filldraw[red] (-7,0) circle (2pt) node[above left] {$(-C,0)$};
            \end{tikzpicture}
        \end{figure}

        Now the idea is simply to show that there is a ellipse other than the unit ball that has larger volume contained in this smaller shape. Consider the ellipse:
        \begin{align*}
            \frac{x^2}{C^2} + y^2 = \frac{1}{2}
        \end{align*}
        It just might be that these coefficients were specifically chosen to make this algebra work better. To show that this ellipse is fully contained in our shape, we need only consider $0 \leq x \leq C$, and then show that all the points the boundary of that ellipse are below the line $y = -1/C x + 1$ in the first quadrant. Plugging in, we need to show that, if $(x,y)$ is a point on the ellipse, then:
        \begin{align*}
            y \leq (-1/C)x + 1
        \end{align*}
        This is equivalent to (since all the values are positive):
        \begin{align*}
            y^2 \leq \frac{x^2}{C^2} - \frac{2}{C}x + 1
        \end{align*}
        plugging in the value for $y$, 
        \begin{align*}
            \frac 12 - \frac{x^2}{C^2} &\leq \frac{x^2}{C^2} - \frac{2}{C}x + 1 \\
             \iff 0 &\leq \frac{2x^2}{C^2} - \frac{2}{C}x + \frac 12 = \frac{(C-2x)^2}{2C^2}
        \end{align*}
        This shows that the ellipse is fully contained in the set. Rewriting the ellipse in a familiar form:
        \begin{align*}
            \frac{x^2}{(C/\sqrt{2})^2} + \frac{y^2}{(1/\sqrt{2})^2} = 1
        \end{align*}
        The area of this ellipse becomes $\pi C/2$. So we have a larger ellipse contained in $\Omega$ whenever $C > 2$, a contradiction. Thus $C = 2$ suffices.
        

        \item Consider $g(x) = \frac 12 x^T Ax + b^Tx$. By compactness $g$ has a maxmium on the sphere $S^{n-1}$, say $y$. The only way that $g(y)$ could be maximal is if the gradient of $g$ were perpendicular to the sphere, because otherwise we could walk in a projected direction and increase our value. The perpendicular to the sphere is precisely $y$, so we need $\nabla g(y) = \lambda y$ for some $y$. Notice that $\nabla g(x) = \frac 12 (A+A^T)x + b$, and since $A$ is symmetric, we get $\nabla g(x) = Ax + b$. Thus we have found a vector $y$ with $Ay + b = \lambda y$ as desired.
        
        \item We prove by Rolle's theorem. Since $p(x)$ is $d$ dimensional and $d$ distinct roots, we know that $p'(x)$ has $\leq d-1$ distinct roots, since it has dimension $d-1$. Labeling the roots of $p(x)$ as $x_1, \ldots, x_d$, we know by Rolle's theorem that since $p(x_i) = p(x_{i+1})$, there is some $y_i$ so that $p'(y_i) = 0$. Then $x_1 < y_1 < \cdots < y_{d-1} < x_d$. This gives $d-1$ distinct roots for $p'(x)$, so these must be all the roots, and we are done.
    \end{enumerate}
\end{document}