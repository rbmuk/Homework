\documentclass[12pt]{article}
\usepackage[margin=1in]{geometry}
\usepackage{setspace}
\onehalfspacing{}
\usepackage[dvipsnames,table,xcdraw]{xcolor} % colors

% Start of preamble
%==========================================================================================%
% Required to support mathematical unicode
\usepackage[warnunknown, fasterrors, mathletters]{ucs}
\usepackage[utf8x]{inputenc}

% Standard mathematical typesetting packages
\usepackage{amsmath,amssymb,amscd,amsthm,amsxtra}
\usepackage{mathtools,mathrsfs,xparse,newtxtext,newtxmath}

% Symbol and utility packages
\usepackage{cancel, textcomp}
\usepackage[mathscr]{euscript}
\usepackage[nointegrals]{wasysym}
\usepackage{apacite}

% Extras
\usepackage{physics}  % Lots of useful shortcuts and macros
\usepackage{tikz-cd}  % For drawing commutative diagrams easily
\usepackage{microtype}  % Minature font tweaks
%\usepackage{pgfplots} % plots

\usepackage{enumitem}
\usepackage{titling}

\usepackage{graphicx}

%\usepackage{quiver}

% Fancy theorems due to @intuitively on discord
\usepackage{mdframed}
\newmdtheoremenv[
backgroundcolor=NavyBlue!30,
linewidth=2pt,
linecolor=NavyBlue,
topline=false,
bottomline=false,
rightline=false,
innertopmargin=10pt,
innerbottommargin=10pt,
innerrightmargin=10pt,
innerleftmargin=10pt,
skipabove=\baselineskip,
skipbelow=\baselineskip]{mytheorem}{Theorem}

\newenvironment{theorem}{\begin{mytheorem}}{\end{mytheorem}}

\newtheorem{corollary}{Corollary}
\newtheorem{lemma}{Lemma}

\newtheoremstyle{definitionstyle}
{\topsep}%
{\topsep}%
{}%
{}%
{\bfseries}%
{.}%
{.5em}%
{}%
\theoremstyle{definitionstyle}
\newmdtheoremenv[
backgroundcolor=Violet!30,
linewidth=2pt,
linecolor=Violet,
topline=false,
bottomline=false,
rightline=false,
innertopmargin=10pt,
innerbottommargin=10pt,
innerrightmargin=10pt,
innerleftmargin=10pt,
skipabove=\baselineskip,
skipbelow=\baselineskip,
]{mydef}{Definition}
\newenvironment{definition}{\begin{mydef}}{\end{mydef}}

\newtheorem*{remark}{Remark}

\newtheorem*{example}{Example}
\newtheorem*{claim}{Claim}

% Common shortcuts
\def\mbb#1{\mathbb{#1}}
\def\mfk#1{\mathfrak{#1}}

\def\bN{\mbb{N}}
\def\C{\mbb{C}}
\def\R{\mbb{R}}
\def\bQ{\mbb{Q}}
\def\bZ{\mbb{Z}}
\def\cph{\varphi}
\renewcommand{\th}{\theta}
\def\ve{\varepsilon}
\newcommand{\mg}[1]{\| #1 \|}

% Often helpful macros
\newcommand{\floor}[1]{\left\lfloor#1\right\rfloor}
\newcommand{\ceil}[1]{\left\lceil#1\right\rceil}
\renewcommand{\qed}{\hfill\qedsymbol}
\renewcommand{\ip}[1]{\langle#1\rangle}
\newcommand{\seq}[2]{\qty(#1_#2)_{#2=1}^{\infty}}

\newcommand{\SET}[1]{\Set{\mskip-\medmuskip #1 \mskip-\medmuskip}}

% End of preamble
%==========================================================================================%

% Start of commands specific to this file
%==========================================================================================%

\usepackage{braket}
\newcommand{\Z}{\mbb Z}
\newcommand{\gen}[1]{\left\langle #1 \right\rangle}
\newcommand{\nsg}{\trianglelefteq}
\newcommand{\F}{\mbb F}
\newcommand{\Aut}{\mathrm{Aut}}
\newcommand{\sepdeg}[1]{[#1]_{\mathrm{sep}}}
\newcommand{\Q}{\mbb Q}
\newcommand{\Gal}{\mathrm{Gal}\qty}
\newcommand{\id}{\mathrm{id}}
\newcommand{\Hom}{\mathrm{Hom}_R}
\newcommand{\1}{\mathds 1}
\newcommand{\N}{\mathbb N}
\renewcommand{\P}{\mathbb P \qty}
\newcommand{\E}{\mathbb E \qty}
\newcommand{\Var}{\mathrm{Var}}
\everymath{\displaystyle}
\newcommand{\argmax}{\mathrm{argmax}}

%==========================================================================================%
% End of commands specific to this file

\title{Math HW5}
\date{\today}
\author{Rohan Mukherjee}

\begin{document}
    \maketitle
    \begin{enumerate}
        \item Consider $b_n = \frac{a_n}{n} - a_1$. Now,
        \begin{align*}
            b_{n+1} \leq \frac{a_n + a_1 - (n+1)a_1}{n+1} = \frac{a_n - na_1}{n+1} \leq b_n
        \end{align*} 
        Also, since $a_n$ was assumed to be non-negative, $b_n \geq -a_1$, so it is bounded below. Now the monotone convergence theorem implies that $b_n$ converges so $a_n$ does too.

        Further, if $a_n$ is not bounded below, we know that it converges to $-\infty$ by the previous argument. 

        Now, if $\mg{\cdot}$ is a matrix norm, then by submultiplicativity $\mg{AB} \leq \mg{A} \mg{B}$. Thus,
        \begin{align*}
            \mg{A^{m+n}} \leq \mg{A^m} \mg{A^n} 
        \end{align*}
        So by the previous theorem applied to $a_n = \log(\mg{A^n})$, we know that $a_n/n$ converges to some $a$ (possibly $-\infty$ as it is monotonically decreasing), and hence $e^a = \lim_{n \to \infty} \mg{A^n}^{1/n}$ exists as well.

        \item For convienience I will work with $Ax > 0$ instead (consider $x' = -x)$. First, if there is some $x \in \R^n$ with $Ax > 0$ and also some $y \geq 0$ with $y \neq 0$ with $A^Ty = 0$, then $x^TA^Ty = (Ax)^Ty = 0$, however this must be strictly positive. Now, by considering $\eta = \min (Ax)_i$, $Ax > 0$ is equivalent to there being some $\eta$ with $Ax - \eta 1 \geq 0$. Writing $x' = (x, \eta)$, this is the same as $[Ax \; -1]x' \geq 0$ with $\ip{-e_{n+1}, x'} < 0$ (the last coordiante must be positive). 

        By Farkas lemma, if this doesn't happen, then there is some $y \geq 0$ so that $\begin{pmatrix}
            A^T \\
            -1^T
        \end{pmatrix}y = -e_{n+1}$. This implies that $A^Ty = 0$ with $\sum y_i = 1$, in particular it is not 0, completing the proof.
        
        \item Consider the probability simplex $\Delta = \SET{(x_1, x_2, x_3) \in \R^3_{\geq 0} : x_1+x_2+x_3 = 1}$. Give it the natural triangulation, and consider a function $f: \Delta \to \Delta$ that is continuous. Since $\Delta$ is compact, $f$ is uniformly continuous. Now suppose that $f$ does not have a fixed point--by compactness, there is some $c > 0$ so that $|f(x)-x| \geq c$ for all $x \in \Delta$. Now, by uniform continuity, there is some $\delta > 0$ so that $|f(x)-f(y)| < c/100$ whenever $|x-y| < \delta$. Now label each vertex of the triangulation via $c(v) = \inf \SET{i : (f(v)-v)_i < 0}$. This always exists, as $f(x)$ goes to $\Delta$, we know that its entries sum to 1, and so does $x \in \Delta$, so $f(x) - x$ has entries summing to 0, and by hypothesis this is not the zero vector.

        By Sperner's triangle theorem, there is a 1-2-3 triangle. We assume the grid has been chosen so small so that the diameter of this triangle is $<\delta$. Call the vertices of this triangle $x_1, x_2, x_3$ with corresponding $v_i = f(x_i)$. After possibly permuting the $v_i$, we know these are of the form:
        \begin{align*}
            v_1 = \begin{pmatrix}
                - \\ \\ \\
            \end{pmatrix}, \;
            v_2 = \begin{pmatrix}
                + \\ - \\ \\
            \end{pmatrix}. \;
            v_3 = \begin{pmatrix}
                + \\ + \\ -
            \end{pmatrix}
        \end{align*}
        We shall now argue that since these are all within $c/100$ of each other, they cannot possibly exhibit this behavior, given that $|v_i| \geq c$. Since $|v_i| \geq c$, there must be a coordinate $i$ so that $|v_i| \geq c/3$ (otherwise the norm is $< c/3 < c)$. We now do a grand case distinction. I will say a coordinate is big if it is $>c/3$ in absolute value. If $v_{11}$ is big, since $v_{21}$ has opposite sign, $|v_1 - v_2| > c/3$, a contradiction as it should be $<c/100$. Similar logic shows that if $v_{22}$ is big, then $|v_2 - v_3| > c/3$. The same logic works if any of the plusses are big ($v_{21} \to v_{21}$, $v_{31} \to v_{11}$, and $v_{32} \to v_{22}$). Now I will show that $v_2$ cannot have both top coordinates small. If $v_{21}, v_{22}$ are both smaller than $c/3$ in absolute value, we know that $v_{23}$ is bigger than $c/3$ in absolute value. But, $|v_{23} + v_{22} + v_{21}| \geq |v_{23}| - |v_{22} + v_{21}| > 0$, since $v_{22} - v_{21}$ is smaller than $c/3$ in absolute value since they point in different directions, and $v_{23}$ is bigger than $c/3$. So one of the previous cases applies, and we are done. 

        \item Let $z_1, \ldots, z_n$ be arbitrary points on the the circle. Consider the vandermonde matrix:
        \[
            V^T = \begin{pmatrix}
            1 & z_1 & z_1^2 & \cdots & z_1^{n-1} \\
            1 & z_2 & z_2^2 & \cdots & z_2^{n-1} \\
            \vdots & \vdots & \vdots & \ddots & \vdots \\
            1 & z_n & z_n^2 & \cdots & z_n^{n-1}
            \end{pmatrix}
        \]
        We discussed at great length that $\det(V) = \prod_{i < j} (z_j-z_i)$. By Hadamard's determinant inequality, as this matrix is $n \times n$ with all entries bounded by 1 in norm,
        \[
            |\det(V)| = \prod_{i < j} |z_j-z_i| \leq n^{n/2}
        \]
        Now,
        \begin{align*}
            \lim_{n \to \infty} n^{n/2 \cdot \frac{1}{{n \choose 2}}} = \lim_{n \to \infty} n^{1/(n-1)} = 1
        \end{align*}
        This shows that it goes to 1. 

        I present a second proof that gets $\leq \sqrt{3}$. Recall the circumradius formula for triangles, once again, that $4RA = abc$. In our case, since all the points are on a circle, $R = 1$, so $4A = abc$. Now, by doing the same angle trick, and using Jensens inequality we get that $A \leq \frac{3\sqrt{3}}{4}$ (equilateral is the maximizer). So $abc \leq 3^{3/2}$. 

        We now instead consider the product
        \begin{align*}
            \prod_{i < j < k} |z_i-z_j| \cdot |z_i - z_k| \cdot |z_j - z_k| \leq 3^{\frac 32 \cdot {n \choose 3}}
        \end{align*}
        By what we just proved, since $z_i,z_j,z_k$ make a triangle. Now we wonder how many times each term $|z_i - z_j|$ appears in this product. Once we have fixed vertices $i,j$, we just have to pick the last vertex $k$, which can be done in $n-2$ ways, so we have the equality:
        \begin{align*}
            \prod_{i < j < k} |z_i-z_j| \cdot |z_i - z_k| \cdot |z_j - z_k| = \prod_{i < j} |z_i - z_j|^{n-2}
        \end{align*}
        Taking the $n-2$th root on both sides, we get:
        \begin{align*}
            \prod_{i < j} |z_i - z_j| \leq \sqrt{3}^{{n \choose 2}}
        \end{align*}
        However it alludes me on how to get $\sqrt{2}$, this approach fails immediately when you try a square since there are non-cyclic quadrilaterals on 4 vertices. The vandermonde determinant approach is honestly shocking in that regard.
    \end{enumerate}
\end{document}