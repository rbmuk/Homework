\documentclass[12pt]{article}
\usepackage[margin=1in]{geometry}
\usepackage{setspace}
\onehalfspacing{}
\usepackage[dvipsnames,table,xcdraw]{xcolor} % colors

% Start of preamble
%==========================================================================================%
% Required to support mathematical unicode
\usepackage[warnunknown, fasterrors, mathletters]{ucs}
\usepackage[utf8x]{inputenc}

% Standard mathematical typesetting packages
\usepackage{amsmath,amssymb,amscd,amsthm,amsxtra}
\usepackage{mathtools,mathrsfs,xparse}

% Symbol and utility packages
\usepackage{cancel, textcomp}
\usepackage[mathscr]{euscript}
\usepackage[nointegrals]{wasysym}
\usepackage{apacite}

% Extras
\usepackage{physics}  % Lots of useful shortcuts and macros
\usepackage{tikz-cd}  % For drawing commutative diagrams easily
\usepackage{microtype}  % Minature font tweaks
%\usepackage{pgfplots} % plots

\usepackage{enumitem}
\usepackage{titling}

\usepackage{graphicx}

%\usepackage{quiver}

% Fancy theorems due to @intuitively on discord
\usepackage{mdframed}
\newmdtheoremenv[
backgroundcolor=NavyBlue!30,
linewidth=2pt,
linecolor=NavyBlue,
topline=false,
bottomline=false,
rightline=false,
innertopmargin=10pt,
innerbottommargin=10pt,
innerrightmargin=10pt,
innerleftmargin=10pt,
skipabove=\baselineskip,
skipbelow=\baselineskip]{mytheorem}{Theorem}

\newenvironment{theorem}{\begin{mytheorem}}{\end{mytheorem}}

\newtheorem{corollary}{Corollary}
\newtheorem{lemma}{Lemma}

\newtheoremstyle{definitionstyle}
{\topsep}%
{\topsep}%
{}%
{}%
{\bfseries}%
{.}%
{.5em}%
{}%
\theoremstyle{definitionstyle}
\newmdtheoremenv[
backgroundcolor=Violet!30,
linewidth=2pt,
linecolor=Violet,
topline=false,
bottomline=false,
rightline=false,
innertopmargin=10pt,
innerbottommargin=10pt,
innerrightmargin=10pt,
innerleftmargin=10pt,
skipabove=\baselineskip,
skipbelow=\baselineskip,
]{mydef}{Definition}
\newenvironment{definition}{\begin{mydef}}{\end{mydef}}

\newtheorem*{remark}{Remark}

\newtheorem*{example}{Example}
\newtheorem*{claim}{Claim}

% Common shortcuts
\def\mbb#1{\mathbb{#1}}
\def\mfk#1{\mathfrak{#1}}

\def\bN{\mbb{N}}
\def\C{\mbb{C}}
\def\R{\mbb{R}}
\def\bQ{\mbb{Q}}
\def\bZ{\mbb{Z}}
\def\cph{\varphi}
\renewcommand{\th}{\theta}
\def\ve{\varepsilon}
\newcommand{\mg}[1]{\| #1 \|}

% Often helpful macros
\newcommand{\floor}[1]{\left\lfloor#1\right\rfloor}
\newcommand{\ceil}[1]{\left\lceil#1\right\rceil}
\renewcommand{\qed}{\hfill\qedsymbol}
\renewcommand{\ip}[1]{\langle#1\rangle}
\newcommand{\seq}[2]{\qty(#1_#2)_{#2=1}^{\infty}}

\newcommand{\SET}[1]{\Set{\mskip-\medmuskip #1 \mskip-\medmuskip}}

% End of preamble
%==========================================================================================%

% Start of commands specific to this file
%==========================================================================================%

\usepackage{braket}
\newcommand{\Z}{\mbb Z}
\newcommand{\gen}[1]{\left\langle #1 \right\rangle}
\newcommand{\nsg}{\trianglelefteq}
\newcommand{\F}{\mbb F}
\newcommand{\Aut}{\mathrm{Aut}}
\newcommand{\sepdeg}[1]{[#1]_{\mathrm{sep}}}
\newcommand{\Q}{\mbb Q}
\newcommand{\Gal}{\mathrm{Gal}\qty}
\newcommand{\id}{\mathrm{id}}
\newcommand{\Hom}{\mathrm{Hom}_R}
\newcommand{\1}{\mathds 1}
\newcommand{\N}{\mathbb N}
\renewcommand{\P}{\mathbb P \qty}
\newcommand{\E}{\mathbb E \qty}
\newcommand{\Var}{\mathrm{Var}}
\everymath{\displaystyle}
\newcommand{\argmax}{\mathrm{argmax}}

%==========================================================================================%
% End of commands specific to this file

\title{Math HW5}
\date{\today}
\author{Rohan Mukherjee}

\begin{document}
    \maketitle
    \begin{enumerate}
        \item I claim that if $B$ is an invertible matrix, then $BA$ and $A$ have the same kernel. This is because $BAx = 0 \iff Ax = 0$, the first direction following since $B$ is invertible we must have $Ax = 0$, and the other direction is clear. Also, since the rank is the dimension of the image, equivalently the dimension of the rowspace or column space, we have $\rank(A) = \rank(A^T)$. So, $\rank(AB) = \rank(B^TA^T) = \rank(A^T) \rank(A)$ by above. So $\rank(B^TAB) = \rank(A)$ for invertible $B$. By the structure of a diagonal matrix, the rank of a diagonal matrix is just the number of zeros in the diagonal. So indeed, the theorem follows.

        \item We prove the result by induction. The base case of $n=2$ is as follows. Place the first vector $v_1$. The second vector has to have negative inner product with $v_1$, so in particular it has angle at least $90$ degrees with $v_1$. Then place the third vector, which has angle at least $90$ degrees with both $v_1$ and $v_2$. If somehow the fourth vector had angle greater than 90 degres, if we order the vectors counter-clockwise, we would get that a circle has angle greater than 360 degrees, which can't be.

        Now assume that we had $n+2$ vectors in $\R^n$ $\SET{v_1, \ldots, v_{n+2}}$ with $\mg{v_i} = 1$ and $\ip{v_i, v_j} < 0$ for $i \neq j$. In particular, $\ip{v_1, v_i} < 0$ for $i > 1$. Project $v_2, \ldots, v_{n+2}$ onto the $n-1$ dimensional subspace $\SET{\ip{x,v_1} = 0}$, i.e. take $v_i' = v_i - \ip{v_i, v_1}v_1$. Consider:
        \begin{align*}            \ip{v_i',v_j'} &= \ip{v_i - \ip{v_i, v_1}v_1, v_j - \ip{v_j, v_1}v_1} 
            \\&= \ip{v_i, v_j} - \ip{v_i, v_1}\ip{v_j, v_1} - \ip{v_j, v_1}\ip{v_i, v_1} + \ip{v_i, v_1}\ip{v_j, v_1} 
            \\&= \ip{v_i, v_j} - \ip{v_i, v_1}\ip{v_j, v_1} < 0
        \end{align*}
        Since, crucially, $\ip{v_1, v_i}$ and $\ip{v_j, v_1}$ are both negative, their product is positive, which can only make $\ip{v_i', v_j'}$ smaller. By induction, this setup isn't possible, which completes the proof.

        \item Since $(AB)_{ii} = \sum_{j=1}^n a_{ij}b_{ji}$ by expanding, we know that $\Tr(AB) = \sum_{i,j} a_{ij}b_{ji}$. Switching the order of summation and renaming the new $i$ with $j$, we get that it also equals $\sum_{j,i} b_{ij} a_{ji} = \Tr(BA)$. Taking $B = A_2 \cdots A_k$ in the question proves the result.

        \item Let $Ax = \lambda x$, and write $\lambda = a + bi$ and $x = y + zi$. Then $Ay + iAz = ay - bz + i(az+by)$. Matching real and imaginary parts, the $k$th row of this equation is just:
        \begin{align*}
            &\sum_j A_{kj}y_j = ay_k - bz_k \\
            &\sum_j A_{kj}z_j = az_k + by_k
        \end{align*}
        The only number we care about bounding is $b$, so we multiply the top equation by $-z_k$ and the bottom by $y_k$, add them together and get:
        \begin{align*}
            \sum_j A_{kj}(-y_jz_k + y_kz_j) = b(y_k^2+z_k^2)
        \end{align*}
        Summing over $k$ yields:
        \begin{align*}
            \sum_{k,j} A_{kj}(-y_jz_k + y_kz_j) = b\sum_k (y_k^2+z_k^2)
        \end{align*}
        Sending $(k,j) \to (j,k)$, and adding the two equations up gives:
        \begin{align*}
            \sum_{k,j} (A_{kj} - A_{jk})(-y_jz_k + y_kz_j) = 2b\sum_k (y_k^2+z_k^2)
        \end{align*}
        Firstly, 
        \begin{align*}
            \qty(\sum_{k,j} (A_{kj}-A_{jk})(-y_jz_k+y_kz_j))^2 &= 4 \qty(\sum_{k<j} (A_{kj}-A_{jk})(-y_jz_k + y_kz_j))^2
        \end{align*}
        By Cauchy-Schwarz, we have that:
        \begin{align*}
            \qty(\sum_{k<j} (A_{kj}-A_{jk})(-y_jz_k + y_kz_j))^2 \leq \qty(\sum_{k<j} (A_{kj}-A_{jk})^2)\qty(\sum_{k<j} (-y_jz_k + y_kz_j)^2)
        \end{align*}
        Each term in the first sum is bounded by $\max_{k,j} |A_{kj}-A_{jk}|$, and so:
        \begin{align*}
            b^2 \qty(\sum_k (y_k^2+z_k^2))^2 \leq {n \choose 2}\max_{k,j} |A_{kj}-A_{jk}|^2\qty(\sum_{k<j} (-y_jz_k + y_kz_j)^2)
        \end{align*}
        Adding back a factor of 2 (Recall that the diagonal terms are 0):
        \begin{align*}
            2b^2 \qty(\sum_k (y_k^2+z_k^2))^2 \leq {n \choose 2}\max_{k,j} |A_{kj}-A_{jk}|^2\sum_{k,j} (-y_jz_k + y_kz_j)^2
        \end{align*}
        Now we prove Langrange's identity. Notice that, for vectors $a, b \in \R^n$, we have that, by the circular property of the trace we proved before:
        \begin{align*}
            \mg{ab^T - ba^T}_F^2 &= \Tr((ab^T - ba^T)^T(ab^T - ba^T)) 
            \\&= \Tr(ba^Tab^T - ba^Tba^T - ab^Tab^T + ab^Tba^T)
            \\&= \mg{a}^2 \Tr(bb^T) - 2\ip{a,b}\ip{b,a} + \mg{b}^2\Tr(aa^T)
            \\&= 2\mg{a}^2\mg{b}^2 - 2\ip{a,b}^2
        \end{align*}
        Writing this out gives:
        \begin{align*}
            \qty(\sum_i a_i^2) \qty(\sum_i b_i^2) - \qty(\sum_i a_ib_i)^2 = \frac 12 \sum_{i,j} (a_ib_j-a_jb_i)^2
        \end{align*}
        In particular,
        \begin{align*}
            \sum_{k,j} (-y_jz_k + y_kz_j)^2 \leq 2\qty(\sum_k y_k^2)\qty(\sum_k z_k^2) \leq \qty(\sum_k (y_k^2+z_k^2))^2
        \end{align*}
        Putting our findings together yields:
        \begin{align*}
            2b^2 \qty(\sum_k (y_k^2+z_k^2))^2 \leq {n \choose 2}\max_{k,j} |A_{kj}-A_{jk}|^2 \qty(\sum_k (y_k^2+z_k^2))^2
        \end{align*}
        Thus,
        \begin{align*}
            |b| \leq \sqrt{\frac{n(n-1)}{8}} \max_{k,j} |A_{kj}-A_{jk}|. 
        \end{align*}
    \end{enumerate}
\end{document}