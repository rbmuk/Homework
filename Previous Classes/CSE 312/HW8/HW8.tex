\documentclass[12pt]{article}
\usepackage[margin=1in]{geometry}

% Start of preamble
%==========================================================================================%
% Required to support mathematical unicode
\usepackage[warnunknown, fasterrors, mathletters]{ucs}
\usepackage[utf8x]{inputenc}

\usepackage[dvipsnames,table,xcdraw]{xcolor} % colors
\usepackage{hyperref} % links
\hypersetup{
	colorlinks=true,
	linkcolor=blue,
	filecolor=magenta,      
	urlcolor=cyan,
	pdfpagemode=FullScreen
}

% Standard mathematical typesetting packages
\usepackage{amsmath,amssymb,amscd,amsthm,amsxtra, pxfonts}
\usepackage{mathtools,mathrsfs,dsfont,xparse}

% Symbol and utility packages
\usepackage{cancel, textcomp}
\usepackage[mathscr]{euscript}
\usepackage[nointegrals]{wasysym}
\usepackage{apacite}

% Extras
\usepackage{physics}  % Lots of useful shortcuts and macros
\usepackage{tikz-cd}  % For drawing commutative diagrams easily
\usepackage{microtype}  % Minature font tweaks
%\usepackage{pgfplots} % plots

\usepackage{enumitem}
\usepackage{titling}

\usepackage{graphicx}

% Fancy theorems due to @intuitively on discord
\usepackage{mdframed}
\newmdtheoremenv[
backgroundcolor=NavyBlue!30,
linewidth=2pt,
linecolor=NavyBlue,
topline=false,
bottomline=false,
rightline=false,
innertopmargin=10pt,
innerbottommargin=10pt,
innerrightmargin=10pt,
innerleftmargin=10pt,
skipabove=\baselineskip,
skipbelow=\baselineskip
]{mytheorem}{Theorem}

\newenvironment{theorem}{\begin{mytheorem}}{\end{mytheorem}}

\newtheorem{corollary}{Corollary}
\newtheorem{lemma}{Lemma}

\newtheoremstyle{definitionstyle}
{\topsep}%
{\topsep}%
{}%
{}%
{\bfseries}%
{.}%
{.5em}%
{}%
\theoremstyle{definitionstyle}
\newmdtheoremenv[
backgroundcolor=Violet!30,
linewidth=2pt,
linecolor=Violet,
topline=false,
bottomline=false,
rightline=false,
innertopmargin=10pt,
innerbottommargin=10pt,
innerrightmargin=10pt,
innerleftmargin=10pt,
skipabove=\baselineskip,
skipbelow=\baselineskip,
]{mydef}{Definition}
\newenvironment{definition}{\begin{mydef}}{\end{mydef}}

\newtheorem*{remark}{Remark}

\newtheorem*{example}{Example}

% Common shortcuts
\def\mbb#1{\mathbb{#1}}
\def\mfk#1{\mathfrak{#1}}

\def\bN{\mbb{N}}
\def \C{\mbb{C}}
\def \R{\mbb{R}}
\def\bQ{\mbb{Q}}
\def\bZ{\mbb{Z}}
\def \cph{\varphi}
\renewcommand{\th}{\theta}
\def \ve{\varepsilon}
\newcommand{\mg}[1]{\| #1 \|}

% Often helpful macros
\newcommand{\floor}[1]{\left\lfloor#1\right\rfloor}
\newcommand{\ceil}[1]{\left\lceil#1\right\rceil}
\renewcommand{\qed}{\hfill\qedsymbol}
\renewcommand{\ip}[2]{\langle #1, #2 \rangle}
\newcommand{\seq}[2]{\qty(#1_#2)_{#2=1}^{\infty}}

% Sets
\DeclarePairedDelimiterX\set[1]\lbrace\rbrace{\def\given{\;\delimsize\vert\;}#1}

% End of preamble
%==========================================================================================%

% Start of commands specific to this file
%==========================================================================================%

\renewcommand{\P}{\mathbb{P}\qty}
\newcommand{\E}{\mathbb{E}\qty}
\newcommand{\Var}{\mathrm{Var}\qty}
\renewcommand{\L}{\mathcal{L}}
\newcommand{\semi}{; \;}

%==========================================================================================%
% End of commands specific to this file

\title{CS 312 HW8}
\date{\today}
\author{Rohan Mukherjee}

\begin{document}
	\maketitle
	\begin{enumerate}[leftmargin=\labelsep]
		\item \begin{enumerate}
			\item We recall that 
			\begin{align*}
				\mathcal{L}(x_1, \ldots, x_n \semi \theta_B, \theta_C) = \prod_{i=1}^n \P(x_i \semi \theta_B, \theta_C)
			\end{align*}
			If $x_i$ is a Bulbasaur, 
			\begin{align*}
				\P(x_i \semi \theta_B, \theta_C) = \theta_B
			\end{align*}
			And similarly, if $x_i$ is a Charmander, $\P(x_i \semi \theta_B, \theta_C) = \theta_C$, and finally if $x_i$ is a Squirtle, $\P(x_I \semi \theta_B, \theta_C) = 1 - \theta_B - \theta_C$. Now, since precisely $n_B$ of the $x_i$'s are equal to Bulbasaur, $n_C$ are equal to Charmander, and $n_S$ are equal to Squirtle, we can simplify this product to
			\begin{align*}
				\mathcal{L}(x_1, \ldots, x_n \semi \theta_B, \theta_C) = \theta_B^{n_B} \cdot \theta_C^{n_C} \cdot (1-\theta_B-\theta_C)^{n_S}
			\end{align*}
			Since multiplication is commutative.
			
			\item The log of my previous answer is just
			\begin{align*}
				\log(\mathcal{L}(x_1, \ldots, x_n \semi \theta_B, \theta_C)) = n_B\log(\theta_B) + n_C\log(\theta_C) + n_S\log(1-\theta_B-\theta_C)
			\end{align*}
			\item Taking partials gives
			\begin{align*}
				\pdv{\log(\L(-))}{\theta_B} = \frac{n_B}{\theta_B} - \frac{n_S}{1-\theta_B-\theta_C}
			\end{align*}
			And similarly,
			\begin{align*}
				\pdv{\log(\L(-))}{\theta_C} = \frac{n_C}{\theta_C} - \frac{n_S}{1-\theta_B-\theta_C}
			\end{align*}
			We want each of these to be equal to 0 to find the critical point. Setting the first to 0 gives
			\begin{align*}
				&\quad \quad \quad \frac{n_S}{1-\theta_B-\theta_C} = \frac{n_B}{\theta_B} \\
				&\iff n_S\theta_B = n_B(1-\theta_B-\theta_C) \\
				&\iff (n_S+n_B)\theta_B + n_B\theta_C = n_B
			\end{align*}
			Doing the exact same thing in the other variable gives \begin{align*}
				&\quad \quad \quad n_S\theta_C = n_C(1-\theta_B-\theta_C) \\
				&\iff n_C\theta_B + (n_S+n_C)\theta_C = n_C
			\end{align*}
			
			\item Plugging this into wolfram gives
			\begin{align*}
				\theta_B = \frac{n_B}{n_B+n_C+n_S} \text{ and } \theta_C = \frac{n_C}{n_B+n_C+n_S}
			\end{align*}
			Exactly what you might expect.
		\end{enumerate}
		
		\newpage
		\item \begin{enumerate}
			\item The graph is a flipped absolute value shifted up a little bit and stretched. If theta is large the graph is flatter and wider, and if theta is small the graph is taller. It also is continuous at $x = \pm \theta$ with value 0, and zero at $|x| \geq \theta$ (i.e., the support of $f$ is just $[-\theta, \theta]$).
			\item We recall from part c that the likelihood is only nonzero when all samples have absolute value strictly smaller than $\theta$, i.e. $\max\set{|x_1|, \ldots, |x_n|} < \theta$. If this is not the case, i.e. if there is an $x_i$ with $|x_i| \geq \theta$, then the likelihood is just 0. So, in the case where $\max\set{|x_1|, \ldots, |x_n|} < \theta$, the likelihood equals
			\begin{align*}
				\L(x_1, \ldots, x_n \semi \theta) = \prod_{i=1}^n f_X(x_i) = \prod_{i=1}^n \qty(-\frac{|x_i|}{\theta^2} + \frac{1}{\theta})
			\end{align*}
			\item The likelihood in this case would be
			\begin{align*}
				f_X(x_1) \cdot \prod_{i=2}^n f_X(x)
			\end{align*}
			If $|x_1| > \theta$, then the density gives 0, so the product (= likelihood) is 0. If $|x_1| = 1$, the density would give
			\begin{align*}
				-\frac{\theta}{\theta^2}+\frac{1}{\theta} = 0
			\end{align*}
			So again the product would be 0, and hence the likelihood is 0.
			
			\item Under the assumption that we have done two draws, with $x_1 = -x_2$, if $|x_1| \geq \theta$ (equivalently, $|x_2| \geq \theta$), then we know that the likelihood is just equivalently 0, so we could take for example $\theta = 1$ to be a maximizer, since constant functions are maximized (and minimized) everywhere. In the more interesting case that $|x_1| < \theta$,
			\begin{align*}
				\L(x_1, -x_1 \semi \theta) = \prod_{i=1}^2 \qty(-\frac{|x_i|}{\theta^2} + \frac1\theta) = \qty(-\frac{|x_1|}{\theta^2} + \frac1\theta)^2
			\end{align*}
			Taking the log of this quantity gives
			\begin{align*}
				\log(\L(-)) = 2\log(\frac{-|x_1|}{\theta^2} + \frac1\theta) = 2\log(\frac{-|x_1|+\theta}{\theta^2}) = 2\qty(\log(-|x_1|+\theta)-2\log(\theta))
			\end{align*}
			Taking derivatives now gives
			\begin{align*}
				\dv{\log(\L(-))}{\theta} = 2\qty(\frac{1}{-|x_1|+\theta} - \frac{2}{\theta})
			\end{align*}
			Setting this to zero gives
			\begin{align*}
				&\quad \quad \quad \frac{2}{\theta} = \frac{1}{-|x_1| + \theta} \\
				&\iff 2\theta - 2|x_1| = \theta \\
				&\iff \theta = 2|x_1|
			\end{align*}
			
		\end{enumerate}
		
		\newpage
		\item \begin{enumerate}
			\item We recall that
			\begin{align*}
				f_{Y \mid X}(y \mid x) = \frac{f_{X,Y}(x, y)}{f_X(x)}
			\end{align*}
			We must find $f_X(x)$. We recall that this equals
			\begin{align*}
				f_X(x) = \int_{-\infty}^\infty f_{X,Y}(x, y)dy
			\end{align*}
			For $x \not \in [0, 3]$, the density is always zero, so $f_X(x) = 0$ there. For $x \in [0, 3]$, this integrand is only not zero when $x \leq y \leq 3$, so we get
			\begin{align*}
				f_X(x) = \int_x^3 \frac{4}{27}x^2dy = \frac{4}{27} x^2 \cdot y\eval_x^3 = \frac{4}{27} x^2 (3-x)
			\end{align*}
			We conclude that
			\begin{align*}
				f_{Y \mid X}(y \mid x) = \frac{\frac{4}{27}x^2}{\frac{4}{27} x^2 (3-x)} = \frac{1}{3-x}
			\end{align*}
		
			\item For $x \not \in [0, 3]$, this conditional probability isn't even defined, so I will say it is 0. So assume that $x \in [0, 3]$.
			\begin{align*}
				\E[Y \mid X = x] = \int_{-\infty}^\infty yf_{Y \mid X}(y \mid x)dy
			\end{align*}
			Once again the integrand is only not zero when $x \leq y \leq 3$, so this integral simplifies to 
			\begin{align*}
				\int_{x}^3 y \frac{1}{3-x} dy = \frac{1}{3-x} \cdot \frac{y^2}2\eval_x^3 = \frac12 \cdot \frac{1}{3-x}(9-x^2) = \frac12(3+x)
			\end{align*}
		
			\item By the law of total expectation,
			\begin{align*}
				\E[Y] = \int_{-\infty}^\infty \E[Y \mid X = x]f_X(x)dx
			\end{align*}
			The support of $X$ is just $[0, 3]$, so this integral simplifies to
			\begin{align*}
				\int_{0}^3 \E[Y \mid X = x]f_X(x)dx
			\end{align*}
			Now we may use the results from part (a) and (b) to conclude that this integral equals
			\begin{align*}
				\int_0^3 \frac12(3+x) \frac{4}{27}x^2(3-x)dx &= \frac{4}{54}\int_0^3  x^2(9-x^2)dx = \frac{4}{54} \int_0^3 9x^2 - x^4dx \\&= \frac{4}{54} \qty(3x^3 - \frac15 x^5) \eval_0^3 = \frac{4}{54}\qty(3 \cdot 27 - \frac{1}{5}243) = \frac{12}{5}
			\end{align*}
		\end{enumerate}
		
		\newpage
		\item Since $X_i \sim \mathrm{Uniform}(0, \theta)$, $\E[X_i] = \theta/2$ from the zoo. Hence,
		\begin{align*}
			\E[\hat{\theta}] = \E[\frac2n\sum_{i=1}^n X_i] = \frac2n \sum_{i=1}^n \E[X_i] = \frac2n \cdot n \cdot \theta / 2 = \theta
		\end{align*}
		So yes, this estimator is unbiased.
	\end{enumerate}
\end{document}
