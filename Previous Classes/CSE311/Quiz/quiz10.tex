\documentclass[12pt]{article}
\usepackage[margin=1in]{geometry}

% Start of preamble
%==========================================================================================%
% Required to support mathematical unicode
\usepackage[warnunknown, fasterrors, mathletters]{ucs}
\usepackage[utf8x]{inputenc}

% Always typeset math in display style
%\everymath{\displaystyle}

% GROUPOIDS FONT!
\usepackage{eulervm}
\usepackage{charter}

% Standard mathematical typesetting packages
\usepackage{amsthm, amsmath, amssymb}
\usepackage{mathtools}  % Extension to amsmath

% Symbol and utility packages
\usepackage{cancel, textcomp}
\usepackage[mathscr]{euscript}
\usepackage[nointegrals]{wasysym}

% Extras
\usepackage{physics}  % Lots of useful shortcuts and macros
\usepackage{tikz-cd}  % For drawing commutative diagrams easily
\usepackage{color}  % Add some color to life
\usepackage{microtype}  % Minature font tweaks
%\usepackage{pgfplots} % plots

\usepackage{enumitem}
\usepackage{titling}

\usepackage{graphicx}

% Common shortcuts
\def\mbb#1{\mathbb{#1}}
\def\mfk#1{\mathfrak{#1}}

\def\bN{\mbb{N}}
\def\bC{\mbb{C}}
\def\bR{\mbb{R}}
\def\bQ{\mbb{Q}}
\def\bZ{\mbb{Z}}

% Sometimes helpful macros
\newcommand{\floor}[1]{\left\lfloor#1\right\rfloor}
\newcommand{\ceil}[1]{\left\lceil#1\right\rceil}
\DeclarePairedDelimiterX\set[1]\lbrace\rbrace{\def\given{\;\delimsize\vert\;}#1}

% Some standard theorem definitions
\newtheorem{theorem}{Theorem}[section]
\newtheorem{corollary}{Corollary}[theorem]
\newtheorem{lemma}[theorem]{Lemma}

\theoremstyle{definition}
\newtheorem{definition}{Definition}[section]

\theoremstyle{remark}
\newtheorem*{remark}{Remark}

% End of preamble
%==========================================================================================%

% Start of commands specific to this file
%==========================================================================================%

\newcommand{\R}{\mathbb{R}}
\renewcommand{\ip}[2]{\langle #1, #2 \rangle}
\newcommand{\mg}[1]{\| #1 \|}
\newcommand{\linf}[1]{\max_{1\leq i \leq #1}}
\newcommand{\ve}{\varepsilon}
\renewcommand{\qed}{\hfill\qedsymbol}
\newcommand{\seq}[2]{\qty(#1_#2)_{#2=1}^{\infty}}
\newcommand\setItemnumber[1]{\setcounter{enumi}{\numexpr#1-1\relax}}
\newcommand{\justif}[1]{&\quad &\text{(#1)}}
\newcommand{\ra}{\rightarrow}


%==========================================================================================%
% End of commands specific to this file

\title{CSE 311 Template}
\date{\today}
\author{Rohan Mukherjee}

\begin{document}
	\maketitle
	\begin{enumerate}[leftmargin=\labelsep]
		\item Suppose for the sake of contradiction that $\Sigma$ is regular. Then there is some DFA $M$ that recognizes $L$.
		
		let $S$ be $\set{0^n1^n \given n \geq 0}$. 
		
		Because the DFA is finite, and $S$ is infinite, there are two strings $x \neq y \in S$ so that $x, y$ go to the same state when read by $M$. So we see that $x = 0^a1^a$ for some $a \geq 0$, and $y = 0^b1^b$ for some $b \geq 0$. WLOG $a > b$. 
		
		Consider the string $z = 0^a$. We notice that $xz = 0^a1^a0^a$, and that $xy = 0^b1^b0^a$, so $xz$ is of the form in $\Sigma$, while $xy$ is not of the form in $\Sigma$.
		
		Since $xz \in \Sigma$ while $yz \not \in \Sigma$, $M$ does not recognize $L$. But that's a contradiction! So $\Sigma$ must be an irregular language.
		
		\setItemnumber{6}
		\item Let $y \in f(A \cap B)$ be an arbitrary element in the image. By definition of the image, we see that there exists an $x \in A \cap B$ so that $f(x) = y$. Then clearly $x \in A$ and $x \in B$, which means that $f(x) \in f(A)$ by definition and at the same time $f(x) \in f(B)$. By the definition of intersection, $f(x) = y \in f(A) \cap f(B)$. As $y$ was an arbitrary element of $f(A \cap B)$, we see that $f(A \cap B) \subseteq f(A) \cap f(B)$.
		
		\item For a Husky Tree $T$, let $P(T) \coloneq $ ``if $T$ has a purple root, then it has an even number of leaves $\land$ if $T$ has a gold root, then it has an odd number of leaves."
		
		\textbf{Base case:} If $T$ is a single gold node, then it clearly has an odd number of leaves (it has only 1 leaf, which is itself). As $T$ is does not have a purple root, we see the other part of $P$ holds vacuously, so $P(T)$ holds.
		
		\textbf{Inductive Hypothesis:} Let $T_1, T_2$ be arbitrary Husky Trees, and suppose $P(T_1)$ and $P(T_2)$.
		
		\textbf{Inductive Step:} There are three cases (without loss of generality): both $T_1$ and $T_2$ have gold roots, both $T_1$ and $T_2$ have purple roots, or $T_1$ has a gold root and $T_2$ has a purple root.
		
		In the first case, by the recursive rule the new husky tree has a purple root with $T_1$ and $T_2$ as it's children. By our inductive hypothesis, $T_1$ and $T_2$ both have an odd number of leaves, and as odd $+$ odd = even, the new tree has an even number of leaves. As this new tree does not have a gold root, the second part of $P$ holds vacuously, so we see $P(\mathrm{new tree})$ holds.
		
		In the second case, by the recursive rule the new tree would be one with a purple root and $T_1$, $T_2$ as its children. As in this case $T_1$ and $T_2$ have purple nodes, and as $P(T_1)$ and $P(T_2)$ hold true, we see that $T_1$ and $T_2$ both have an even number of leaves. The number of leaves of the new tree is equal to the sum of the number of leaves of $T_1$ and $T_2$, which would be even $+$ even, which is of course even. As this new tree does not have a gold root, the second part of $P$ holds vacuously, So $P(\mathrm{new tree})$ holds in this case.
		
		Finally, in the last case $T_1$ has a gold root and $T_2$ has a purple root. The new tree would have a gold root and $T_1$ and $T_2$ as it's children. By the inductive hypothesis, $T_1$ has an odd number of leaves and $T_2$ has an even number of leaves. So the new tree would have odd $+$ even = odd number of leaves. As the new tree does not have a purple root, the second part of $P$ holds vacuously, which shows $P(\mathrm{new tree})$. 
		
		Therefore, $P(T)$ holds for every husky tree $T$ by structural induction.
		
		\item For a positive integer $n$, let $P(n) \coloneq \sum_{k=1}^n 4k-3 = n(2n-1)$. We show $P(n)$ for all $n \geq 1$ by induction.
		
		\textbf{Base case:} $\sum_{k=1}^1 4k-3 = 4\cdot 1 - 3 = 1 = 1(2 \cdot 1 - 1)$, $P(1)$ holds.
		
		\textbf{Inductive Hypothesis:} Suppose $P(w)$ for an arbitrary $w \geq 1$. 
		
		\textbf{Inductive Step:} We notice that
		\begin{align*}
			\sum_{k=1}^{w+1} 4k-3 &= \sum_{k=1}^w 4k-3 + 4(w+1)-3 \\
			&= w(2w-1) + 4w+4 - 3 \justif{Inductive Hypothesis} \\
			&= 2w^2 - w + 4w + 1 \\
			&= 2w^2 + 3w + 1 \\
			&= (w+1)(2w+1) \\
			&= (w+1)(2(w+1) - 1)
		\end{align*}
		Which shows $P(w+1)$. So we conclude that $P(n)$ holds for all $n \geq 1$.
	\end{enumerate}
\end{document}
