\documentclass[12pt]{article}
\usepackage[margin=1in]{geometry}

% Start of preamble
%==========================================================================================%
% Required to support mathematical unicode
\usepackage[warnunknown, fasterrors, mathletters]{ucs}
\usepackage[utf8x]{inputenc}

% Always typeset math in display style
%\everymath{\displaystyle}

% GROUPOIDS FONT!
\usepackage{eulervm}
\usepackage{charter}

% Standard mathematical typesetting packages
\usepackage{amsthm, amsmath, amssymb}
\usepackage{mathtools}  % Extension to amsmath

% Symbol and utility packages
\usepackage{cancel, textcomp}
\usepackage[mathscr]{euscript}
\usepackage[nointegrals]{wasysym}

% Extras
\usepackage{physics}  % Lots of useful shortcuts and macros
\usepackage{tikz-cd}  % For drawing commutative diagrams easily
\usepackage{color}  % Add some color to life
\usepackage{microtype}  % Minature font tweaks
%\usepackage{pgfplots} % plots

\usepackage{enumitem}
\usepackage{titling}

\usepackage{graphicx}

% Common shortcuts
\def\mbb#1{\mathbb{#1}}
\def\mfk#1{\mathfrak{#1}}

\def\bN{\mbb{N}}
\def\bC{\mbb{C}}
\def\bR{\mbb{R}}
\def\bQ{\mbb{Q}}
\def\bZ{\mbb{Z}}

% Sometimes helpful macros
\newcommand{\floor}[1]{\left\lfloor#1\right\rfloor}
\newcommand{\ceil}[1]{\left\lceil#1\right\rceil}
\DeclarePairedDelimiterX\set[1]\lbrace\rbrace{\def\given{\;\delimsize\vert\;}#1}

% Some standard theorem definitions
\newtheorem{theorem}{Theorem}[section]
\newtheorem{corollary}{Corollary}[theorem]
\newtheorem{lemma}[theorem]{Lemma}

\theoremstyle{definition}
\newtheorem{definition}{Definition}[section]

\theoremstyle{remark}
\newtheorem*{remark}{Remark}

% End of preamble
%==========================================================================================%

% Start of commands specific to this file
%==========================================================================================%

\newcommand{\R}{\mathbb{R}}
\renewcommand{\ip}[2]{\langle #1, #2 \rangle}
\newcommand{\mg}[1]{\| #1 \|}
\newcommand{\linf}[1]{\max_{1\leq i \leq #1}}
\newcommand{\ve}{\varepsilon}
\renewcommand{\qed}{\hfill\qedsymbol}
\newcommand{\seq}[2]{\qty(#1_#2)_{#2=1}^{\infty}}
\newcommand\setItemnumber[1]{\setcounter{enumi}{\numexpr#1-1\relax}}
\newcommand{\justif}[1]{&\quad &\text{(#1)}}
\newcommand{\ra}{\rightarrow}


%==========================================================================================%
% End of commands specific to this file

\title{CSE section 4}
\date{\today}
\author{Rohan Mukherjee}

\begin{document}
	\maketitle
	\begin{enumerate}[leftmargin=\labelsep]
		\setItemnumber{2}
		\item 
		\begin{enumerate}
			\item Let the domain of discourse be integers. $Greater(x, y) = x > y$, and $Even(x) = \exists k x = 2k$. The statement is now $\exists x (Greater(x^2, 10) \land Even(3x))$.
		\begin{alignat*}{2}
			Greater(4^2, 10) \justif{Definition of Greater} \\
			Even(3\cdot 4) \justif{Definition of  Even} \\
			\exists x (Greater(x^2, 10) \land Even(3x)) \justif{Intro $\land$}
		\end{alignat*}
		\item Let the domain of discourse be integers. Let $Prime(x) = x$ is prime, our statement is now $\forall n \exists p(Prime(p) \land Greater(p, n))$
		\begin{alignat*}{2}
			\text{Let $a$ be arbitrary} \\
			Prime(p) \justif{Definition of prime} \\
			Greater(p) \justif{Definition of greater} \\ 
			\forall n \exists p (Prime(p) \land Greater(p, n)) \justif{Intro $\forall$}
		\end{alignat*}
		\end{enumerate}
		\item 
		\begin{enumerate}
			\item 3 elements.
		\item Infinitely many elements.
		\end{enumerate}
		\item Given any $(x, y) \in (A \cap B)\cross C$, we know that $x \in (A \cap B)$, and that $y \in C$. Clearly then $y \in C \cup D$, and also that $x \in A$, because $x \in A$ and $x \in B$. So we have figured out that $(x, y) \in A \cross (C \cup D)$.
		\item 
		$A \cap (A \cup B) = \set{x \given x \in A \land (x \in A \lor x \in B)} = \set{x \given x \in A}$, as $p \land (p \lor q) \equiv p$. (Also, the first part is just writing out what the set operations mean).
	\end{enumerate}
\end{document}
