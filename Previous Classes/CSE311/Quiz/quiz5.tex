\documentclass[12pt]{article}
\usepackage[margin=1in]{geometry}

% Start of preamble
%==========================================================================================%
% Required to support mathematical unicode
\usepackage[warnunknown, fasterrors, mathletters]{ucs}
\usepackage[utf8x]{inputenc}

% Always typeset math in display style
%\everymath{\displaystyle}

% GROUPOIDS FONT!
\usepackage{eulervm}
\usepackage{charter}

% Standard mathematical typesetting packages
\usepackage{amsthm, amsmath, amssymb}
\usepackage{mathtools}  % Extension to amsmath

% Symbol and utility packages
\usepackage{cancel, textcomp}
\usepackage[mathscr]{euscript}
\usepackage[nointegrals]{wasysym}

% Extras
\usepackage{physics}  % Lots of useful shortcuts and macros
\usepackage{tikz-cd}  % For drawing commutative diagrams easily
\usepackage{color}  % Add some color to life
\usepackage{microtype}  % Minature font tweaks
%\usepackage{pgfplots} % plots

\usepackage{enumitem}
\usepackage{titling}

\usepackage{graphicx}

% Common shortcuts
\def\mbb#1{\mathbb{#1}}
\def\mfk#1{\mathfrak{#1}}

\def\bN{\mbb{N}}
\def\bC{\mbb{C}}
\def\bR{\mbb{R}}
\def\bQ{\mbb{Q}}
\def\bZ{\mbb{Z}}

% Sometimes helpful macros
\newcommand{\floor}[1]{\left\lfloor#1\right\rfloor}
\newcommand{\ceil}[1]{\left\lceil#1\right\rceil}
\DeclarePairedDelimiterX\set[1]\lbrace\rbrace{\def\given{\;\delimsize\vert\;}#1}

% Some standard theorem definitions
\newtheorem{theorem}{Theorem}[section]
\newtheorem{corollary}{Corollary}[theorem]
\newtheorem{lemma}[theorem]{Lemma}

\theoremstyle{definition}
\newtheorem{definition}{Definition}[section]

\theoremstyle{remark}
\newtheorem*{remark}{Remark}

% End of preamble
%==========================================================================================%

% Start of commands specific to this file
%==========================================================================================%

\newcommand{\R}{\mathbb{R}}
\renewcommand{\ip}[2]{\langle #1, #2 \rangle}
\newcommand{\mg}[1]{\| #1 \|}
\newcommand{\linf}[1]{\max_{1\leq i \leq #1}}
\newcommand{\ve}{\varepsilon}
\renewcommand{\qed}{\hfill\qedsymbol}
\newcommand{\seq}[2]{\qty(#1_#2)_{#2=1}^{\infty}}
\newcommand\setItemnumber[1]{\setcounter{enumi}{\numexpr#1-1\relax}}
\newcommand{\justif}[1]{&\quad &\text{(#1)}}
\newcommand{\ra}{\rightarrow}


%==========================================================================================%
% End of commands specific to this file

\title{CSE 311 Quiz 5}
\date{\today}
\author{Rohan Mukherjee}

\begin{document}
	\maketitle
	\begin{enumerate}
		\item[5. a)] If $a = 0$, we see that $0 \mid b$, which forces $b$ to be 0, as the only multiple of 0 is 0 itself. Clearly then $a = b$. The exact same scenario plays out if $b = 0$, so we may exclude the case where $a = 0$ or $b = 0$. Therefore, given $a, b \in \bZ \setminus 0$, if $a \mid b$, then there is some $k \in \bZ$ so that $b = ak$. Similarly, as $b \mid a$, there is some $l \in \bZ$ so that $a = bl$. Plugging the second equation into the first one, we see that $b = blk$, and as $b \neq 0$, we see can divide both sides by $b$ to get that $1 = lk$. The only way this is possible is if $l, k \in \set{\pm 1}$ (either -1 $\cdot$ -1, or $1 \cdot 1$). Plugging in the value for $l$, we see that $a = b$ or $a = -b$, so we are done.
		
		\item[6. a)] Let $P(n) = \sum_{k=0}^{n} k = \frac{n(n+1)}{2}$. Clearly $\sum_{k=0}^{0} k = 0 = 0(0+1)/2$, so the base case is finished. Now suppose for some arbitrary $l \in \bN$ that $P(l)$ holds. Then 
		\begin{align*}
			\sum_{k=0}^{l+1} k = \qty(\sum_{k=0}^l k) + (l+1) = \frac{l(l+1)}{2} + (l+1)
		\end{align*}
		Where for the last equality we have used the inductive hypothesis. Simplifying the RHS, we get that
		\begin{align*}
			\frac{l(l+1)}{2} + (l+1) = \frac{l(l+1)}{2} + \frac{2l+2}{2} &= \frac{l^2+l+2l+2}{2} \\
			&= \frac{l^2+3l+2}{2} \\
			&= \frac{(l+1)(l+2)}{2}
		\end{align*}
		Which shows that $P(l+1)$ holds. By the principal of mathematical induction, $P(n)$ must be true for all $n \in \bN$, and we are done.
	\end{enumerate}
\end{document}