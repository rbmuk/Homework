\documentclass[12pt]{article}
\usepackage[margin=1in]{geometry}

% Start of preamble
%==========================================================================================%
% Required to support mathematical unicode
\usepackage[warnunknown, fasterrors, mathletters]{ucs}
\usepackage[utf8x]{inputenc}

% Always typeset math in display style
%\everymath{\displaystyle}

% GROUPOIDS FONT!
\usepackage{eulervm}
\usepackage{charter}

% Standard mathematical typesetting packages
\usepackage{amsthm, amsmath, amssymb}
\usepackage{mathtools}  % Extension to amsmath

% Symbol and utility packages
\usepackage{cancel, textcomp}
\usepackage[mathscr]{euscript}
\usepackage[nointegrals]{wasysym}

% Extras
\usepackage{physics}  % Lots of useful shortcuts and macros
\usepackage{tikz-cd}  % For drawing commutative diagrams easily
\usepackage{color}  % Add some color to life
\usepackage{microtype}  % Minature font tweaks
%\usepackage{pgfplots} % plots

\usepackage{enumitem}
\usepackage{titling}

\usepackage{graphicx}

% Common shortcuts
\def\mbb#1{\mathbb{#1}}
\def\mfk#1{\mathfrak{#1}}

\def\bN{\mbb{N}}
\def\bC{\mbb{C}}
\def\bR{\mbb{R}}
\def\bQ{\mbb{Q}}
\def\bZ{\mbb{Z}}

% Sometimes helpful macros
\newcommand{\floor}[1]{\left\lfloor#1\right\rfloor}
\newcommand{\ceil}[1]{\left\lceil#1\right\rceil}
\DeclarePairedDelimiterX\set[1]\lbrace\rbrace{\def\given{\;\delimsize\vert\;}#1}

% Some standard theorem definitions
\newtheorem{theorem}{Theorem}[section]
\newtheorem{corollary}{Corollary}[theorem]
\newtheorem{lemma}[theorem]{Lemma}

\theoremstyle{definition}
\newtheorem{definition}{Definition}[section]

\theoremstyle{remark}
\newtheorem*{remark}{Remark}

% End of preamble
%==========================================================================================%

% Start of commands specific to this file
%==========================================================================================%

\newcommand{\R}{\mathbb{R}}
\renewcommand{\ip}[2]{\langle #1, #2 \rangle}
\newcommand{\mg}[1]{\| #1 \|}
\newcommand{\linf}[1]{\max_{1\leq i \leq #1}}
\newcommand{\ve}{\varepsilon}
\renewcommand{\qed}{\hfill\qedsymbol}
\newcommand{\seq}[2]{\qty(#1_#2)_{#2=1}^{\infty}}
\newcommand\setItemnumber[1]{\setcounter{enumi}{\numexpr#1-1\relax}}
\newcommand{\justif}[1]{&\quad &\text{(#1)}}
\newcommand{\ra}{\rightarrow}


%==========================================================================================%
% End of commands specific to this file

\title{CSE 311 Quiz 7}
\date{\today}
\author{Rohan Mukherjee}

\begin{document}
	\maketitle
	\begin{enumerate}[leftmargin=\labelsep]
		\setItemnumber{0}
		\item Let $P(n) = f(n) = n$. I prove $P(n) \; \forall n \in \bN$ by strong induction.
		
		\textbf{Base cases: } $f(0)=0$, and $f(1)=1$, so we may conclude $P(0), \; P(1)$.
		
		\textbf{Inductive Hypothesis: } Suppose $P(0) \land \cdots \land P(k)$ for an arbitrary $k \geq 1$.
		
		\textbf{Inductive Step: } We see that as $k+1 \geq 2$, we have to use the last definition for $f(k+1)$. So we see that $f(k+1)=2f(k)-f(k-1) = 2k-(k-1)$ by the inductive hypothesis. We conclude that $f(k+1)=k+1$ by algebra. As this is what $P(k+1)$ asserts, we see that $P(n)$ holds for all integer $n \geq 1$.
		
		\item Let $P(n) = 6n+6 < 2^n$. We prove $P(n) \forall n \geq 6$ by induction.
		
		\textbf{Base case: } $6 \cdot 6 +  6 = 42 < 2^6 = 64$ so we see that $P(6)$ holds.
		
		\textbf{Inductive Hypothesis: } Suppose $P(k)$ for an arbitrary $k \geq 6$.
		
		\textbf{Inductive Step: } We notice that $6(k+1)+6=6k+6+6<2^k+6$ by the inductive hypothesis. We also note that $6 < 2^6 \leq 2^k$, because $k \geq 6$, and $2^z$ is an increasing function. So $6(k+1)+6 < 2^k+6 \leq 2^k + 2^k = 2^{k+1}$, which is what $P(k+1)$ asserts. We conclude that $P(n)$ holds for all integer $n \geq 6$ by induction.
		
		\item[3 b) ] Let $P(x) = leaves(x) \geq size(x)/2 + 1/2$. We prove $P(x)$ for all trees $x$ by structural induction.
		\textbf{Base case: } $size(\bullet)/2 + 1/2 = 1/2 + 1/2 = 1 \leq leaves(\bullet) = 1$, so we see that $P(\bullet)$ is true.
		
		\textbf{Inductive Hypothesis: } Suppose $L, R$ are  trees, and suppose $P(L), P(R)$. We notice that 
		\begin{align*}
			leaves((L, R, \bullet)) &= leaves(L) + leaves(R) \\
			&\geq size(L)/2 + 1/2 + size(R)/2 + 1/2 \justif{(I.H.)}\\
			&= 1/2(size(L) + size(R) + 1) + 1/2 \\
			&= 1/2size(Tree(\bullet, L, R)) + 1/2
		\end{align*}
		Which is precisely what $P(Tree(\bullet, L, R))$ states. So we may conclude that $P(x)$ holds for all trees $x$ by structural induction.
	\end{enumerate}
\end{document}
