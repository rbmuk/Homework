\documentclass[12pt]{article}
\usepackage[margin=1in]{geometry}
\usepackage{setspace}
\onehalfspacing{}
\usepackage[dvipsnames,table,xcdraw]{xcolor} % colors

% Start of preamble
%==========================================================================================%
% Required to support mathematical unicode
\usepackage[warnunknown, fasterrors, mathletters]{ucs}
\usepackage[utf8x]{inputenc}
\usepackage{R:/sty/quiver}


% Standard mathematical typesetting packages
\usepackage{amsmath,amssymb,amscd,amsthm,amsxtra, pxfonts}
\usepackage{mathtools,mathrsfs,xparse}

% Symbol and utility packages
\usepackage{cancel, textcomp}
\usepackage[mathscr]{euscript}
\usepackage[nointegrals]{wasysym}
\usepackage{apacite}

% Extras
\usepackage{physics}  % Lots of useful shortcuts and macros
\usepackage{tikz-cd}  % For drawing commutative diagrams easily
\usepackage{microtype}  % Minature font tweaks
%\usepackage{pgfplots} % plots

\usepackage{enumitem}
\usepackage{titling}

\usepackage{graphicx}

%\usepackage{quiver}

% Fancy theorems due to @intuitively on discord
\usepackage{mdframed}
\newmdtheoremenv[
backgroundcolor=NavyBlue!30,
linewidth=2pt,
linecolor=NavyBlue,
topline=false,
bottomline=false,
rightline=false,
innertopmargin=10pt,
innerbottommargin=10pt,
innerrightmargin=10pt,
innerleftmargin=10pt,
font=\normalfont, % Add this line to remove italics
skipabove=\baselineskip,
skipbelow=\baselineskip]{mytheorem}{Theorem}

\newenvironment{theorem}{\begin{mytheorem}}{\end{mytheorem}}

\newtheorem{corollary}{Corollary}
\newtheorem{lemma}{Lemma}

\newtheoremstyle{definitionstyle}
{\topsep}%
{\topsep}%
{}%
{}%
{\bfseries}%
{.}%
{.5em}%
{}%
\theoremstyle{definitionstyle}
\newmdtheoremenv[
backgroundcolor=Violet!30,
linewidth=2pt,
linecolor=Violet,
topline=false,
bottomline=false,
rightline=false,
innertopmargin=10pt,
innerbottommargin=10pt,
innerrightmargin=10pt,
innerleftmargin=10pt,
skipabove=\baselineskip,
skipbelow=\baselineskip,
]{mydef}{Definition}
\newenvironment{definition}{\begin{mydef}}{\end{mydef}}

\newtheorem*{remark}{Remark}

\newtheorem*{example}{Example}
\newtheorem*{claim}{Claim}

% Common shortcuts
\def\mbb#1{\mathbb{#1}}
\def\mfk#1{\mathfrak{#1}}

\def\bN{\mbb{N}}
\def\C{\mbb{C}}
\def\R{\mbb{R}}
\def\bQ{\mbb{Q}}
\def\bZ{\mbb{Z}}
\def\cph{\varphi}
\renewcommand{\th}{\theta}
\def\ve{\varepsilon}
\newcommand{\mg}[1]{\left\| #1 \right\|}

% Often helpful macros
\newcommand{\floor}[1]{\left\lfloor#1\right\rfloor}
\newcommand{\ceil}[1]{\left\lceil#1\right\rceil}
\renewcommand{\qed}{\hfill\qedsymbol}
\renewcommand{\ip}[1]{\langle#1\rangle}
\newcommand{\seq}[2]{\qty(#1_#2)_{#2=1}^{\infty}}

\newcommand{\SET}[1]{\Set{\mskip-\medmuskip #1 \mskip-\medmuskip}}

% End of preamble
%==========================================================================================%

% Start of commands specific to this file
%==========================================================================================%

\usepackage{braket}
\newcommand{\Z}{\mbb Z}
\newcommand{\gen}[1]{\left\langle #1 \right\rangle}
\newcommand{\nsg}{\trianglelefteq}
\newcommand{\F}{\mbb F}
\newcommand{\Aut}{\mathrm{Aut}}
\newcommand{\sepdeg}[1]{[#1]_{\mathrm{sep}}}
\newcommand{\Q}{\mbb Q}
\newcommand{\Gal}{\mathrm{Gal}\qty}
\newcommand{\id}{\mathrm{id}}
\newcommand{\Hom}{\mathrm{Hom}_R}

%==========================================================================================%
% End of commands specific to this file

\title{Math Template}
\date{\today}
\author{Rohan Mukherjee}

\begin{document}
    \maketitle
    We pose the following question: Of all subsets $U \subset H = \SET{\pm 1}^n$ with $|U| = 2^{n-1}$ containing no antipodal points, which maximizes
    \begin{align*}
        \mg{\sum_{x \in U} x}?
    \end{align*}
    We answer this question with the following theorem:
    \begin{theorem}
        The best $U$ as above is just $U = \SET{x \in H : x_1 = 1}$.
    \end{theorem}
    The proof is as follows. Recall the structure theorem for the $m \times n$ matrices:
    \begin{theorem}[Structure Theorem]
        Let $A$ be the matrix maximizing 
        \begin{align*}
            \beta(A) = \sum_{x \in D} \mg{Ax}_\infty
        \end{align*}
        Define $W_i = \SET{x \in H \; | \; \mg{Ax}_\infty = |a_i^Tx|}$ to be the vertices row $a_i^T$ is useful to and $V_i = \SET{x \in W \; | \; a_i^Tx = |a_i^Tx|}$ as the positive half of $W_i$. Then,
        \begin{align*}
            a_i = \sum_{x \in V_i} x
        \end{align*}
    \end{theorem}
    And our answer to the question for the $1 \times n$ case:
    \begin{theorem}
        The optimal $1 \times n$ matrix is just $(1, 0, \ldots, 0)$. 
    \end{theorem}

    We now prove Theorem 1.
    \begin{proof}[Proof of Theorem 1.]
        I claim that finding the optimal $1 \times n$ matrix is equivalent to the above problem. With $u = \sum_{x \in U} x$, I claim that we can reduce our search space to only those $U$ such that $u^Tx \geq 0$ for every $x \in U$. This follows because if there is an $x \in U$ with $u^Tx < 0$, then $x$ is pointing away from the average direction of the rest of the points in $U$, so swapping $x$ with $-x$ will make the sum become larger (This needs more justification). Now,
        \begin{align*}
            \beta(u) = \sum_{x \in \SET{\pm 1}^n} |u^Tx| = \sum_{x \in U} u^Tx - \sum_{x \in -U} u^Tx = 2\sum_{x \in U} u^Tx = \sum_{x \in U} \sum_{y \in U} y^Tx = \mg{\sum_{x \in U} x}^2
        \end{align*}

        By the structure theorem, finding the optimal matrix $A \in \R^{m \times n}$ is equivalent to finding first the optimal partition of $H$ into $m$ parts of even size, and then finding the optimal positive half for each partition. For the $1 \times n$ case, the optimal partition is just all of $H$, there is no choice. Then Theorem 3 tells us that the optimal positive half is just the positive half of $H$ associated to $(1, \ldots, 0)$, which is seen to be $\SET{x \in H : x_1 = 1}$. This completes the proof.
    \end{proof}
\end{document}