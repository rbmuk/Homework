\documentclass[12pt]{article}
\usepackage[margin=1in]{geometry}
\usepackage{setspace}
\onehalfspacing{}

% Start of preamble
%==========================================================================================%
% Required to support mathematical unicode
\usepackage[warnunknown, fasterrors, mathletters]{ucs}
\usepackage[utf8x]{inputenc}

%\usepackage[dvipsnames,table,xcdraw]{xcolor} % colors

\usepackage{float}

% Standard mathematical typesetting packages
\usepackage{amsmath,amssymb,amscd,amsthm,amsxtra, pxfonts}
\usepackage{mathtools,mathrsfs,xparse}

% Symbol and utility packages
\usepackage{cancel, textcomp}
\usepackage[mathscr]{euscript}
\usepackage[nointegrals]{wasysym}
\usepackage{apacite}

% Extras
\usepackage{physics}  % Lots of useful shortcuts and macros
\usepackage{tikz-cd}  % For drawing commutative diagrams easily
\usepackage{microtype}  % Minature font tweaks
%\usepackage{pgfplots} % plots

\usepackage{enumitem}
\usepackage{titling}

\usepackage{graphicx}

%\usepackage{quiver}

% Fancy theorems due to @intuitively on discord
\usepackage{mdframed}
\newmdtheoremenv[
backgroundcolor=blue!30,
linewidth=2pt,
linecolor=blue,
topline=false,
bottomline=false,
rightline=false,
innertopmargin=10pt,
innerbottommargin=10pt,
innerrightmargin=10pt,
innerleftmargin=10pt,
skipabove=\baselineskip,
skipbelow=\baselineskip]{mytheorem}{Theorem}

\newenvironment{theorem}{\begin{mytheorem}}{\end{mytheorem}}

\newtheorem{corollary}{Corollary}
\newtheorem{lemma}{Lemma}

\newtheoremstyle{definitionstyle}
{\topsep}%
{\topsep}%
{}%
{}%
{\bfseries}%
{.}%
{.5em}%
{}%
\theoremstyle{definitionstyle}
\newmdtheoremenv[
backgroundcolor=Violet!30,
linewidth=2pt,
linecolor=Violet,
topline=false,
bottomline=false,
rightline=false,
innertopmargin=10pt,
innerbottommargin=10pt,
innerrightmargin=10pt,
innerleftmargin=10pt,
skipabove=\baselineskip,
skipbelow=\baselineskip,
]{mydef}{Definition}
\newenvironment{definition}{\begin{mydef}}{\end{mydef}}

\newtheorem*{remark}{Remark}

\newtheorem*{example}{Example}
\newtheorem*{claim}{Claim}

% Common shortcuts
\def\mbb#1{\mathbb{#1}}
\def\mfk#1{\mathfrak{#1}}

\def\bN{\mbb{N}}
\def\C{\mbb{C}}
\def\R{\mbb{R}}
\def\bQ{\mbb{Q}}
\def\bZ{\mbb{Z}}
\def\cph{\varphi}
\renewcommand{\th}{\theta}
\def\ve{\varepsilon}
\newcommand{\mg}[1]{\| #1 \|}

% Often helpful macros
\newcommand{\floor}[1]{\left\lfloor#1\right\rfloor}
\newcommand{\ceil}[1]{\left\lceil#1\right\rceil}
\renewcommand{\qed}{\hfill\qedsymbol}
\renewcommand{\ip}[1]{\langle#1\rangle}
\newcommand{\seq}[2]{\qty(#1_#2)_{#2=1}^{\infty}}

\newcommand{\SET}[1]{\Set{\mskip-\medmuskip #1 \mskip-\medmuskip}}

% End of preamble
%==========================================================================================%

% Start of commands specific to this file
%==========================================================================================%

\usepackage{braket}
\newcommand{\Z}{\mbb Z}
\newcommand{\gen}[1]{\left\langle #1 \right\rangle}
\newcommand{\nsg}{\trianglelefteq}
\newcommand{\F}{\mbb F}
\newcommand{\Aut}{\mathrm{Aut}}
\newcommand{\sepdeg}[1]{[#1]_{\mathrm{sep}}}
\newcommand{\Q}{\mbb Q}
\newcommand{\Gal}{\mathrm{Gal}\qty}
\newcommand{\id}{\mathrm{id}}
\newcommand{\Hom}{\mathrm{Hom}_R}
\renewcommand{\P}{\mathbb P \qty}
\newcommand{\E}{\mathbb E}

%==========================================================================================%
% End of commands specific to this file

\title{Hypercube Vertices}
\date{\today}
\author{Rohan Mukherjee}

\begin{document}
    \maketitle
    Our main result today is to show that in the $1 \times n$ case, picking vertices of the hypercube is the "worst possible" in the sense that the beta value tends to 0. We make the following result precise in the following sense:

    \begin{theorem}[Hypercube Vertices Scale Poorly]
        Let $a \in \SET{-1, 1}^n$ be any hypercube vertex. Then,
        \begin{align*}
            \sqrt{n}\beta(a) = \frac{1}{2^n} \sum_{x \in \SET{-1, 1}^n} |a^tx| \ll n^2(0.91)^n.
        \end{align*}
    \end{theorem}
    We claim that we can assume without loss of generality that $a = \mathbf{1}$. Indeed, If $a \in \SET{-1, 1}^n$, then we can write
    \begin{align*}
        a^tx = \mathrm{diag}(a) \mathbf{1}^tx = x^t\mathrm{diag}(a)\mathbf{1}
    \end{align*}
    Since $x$ will run through all $\pm 1$ combinations, multiplying $x^t$ on the right by a $\pm 1$ diagonal matrix will just permute those vectors among themselves. Thus, we can assume that $a = \mathbf{1}$.

    Notice that in this case, $a^tx = $ \# of $1$s in $x$ - \# of $-1$s in $x$. We divide the $x$'s into classes based on the number of $1$s in $x$. Let $S_{n-k}$ be the set of $x$'s with $n-k$ $1$s and $k$ $-1$s. Clearly, $|S_{n-k}| = {n \choose k}$. We see then that (after some simple approximations)
    \begin{align*}
        \sqrt{n}\beta(\mathbf{1}) &= \frac{1}{2^n} \sum_{x \in \SET{-1, 1}^n} |a^tx| = \frac{1}{2^n} \sum_{k=0}^n {n \choose k} |n-2k| = \frac{1}{2^n} \sum_{k=0}^{n/2} {n \choose k} (n-2k) \\
        &= \sum_{k=0}^{n/2} \frac{n^k}{k!} (n-2k) = n^2 \frac{n^{n/4}}{(n/4)!} = n^2\frac{(4e)^{n/4}}{2^n} = n^2(0.91)^n.
    \end{align*}

    We give a second proof. Let $S_n = X_1 + \cdots + X_n$ be the sum of $n$ i.i.d. Radamacher random variables. By Hoeffding's Inequality,
    \begin{align*}
        \P(S_n \neq 0) = \P(|S_n| \geq \frac 12) \leq e^{-4/n}
    \end{align*}
    Recall that if $a$ has norm 1 and $x \in \SET{\pm 1}^n$, then by Cauchy-Schwartz $|a^Tx| \leq \sqrt{n}$. Putting these together tells us that,
    \begin{align*}
        \beta\qty(\frac{\mathbf 1}{\sqrt{n}}) = \E_{x}\qty[\qty|\qty(\frac{\mathbf 1}{\sqrt{n}})^T x|] \leq \sqrt{n}e^{-4/n}  \to 0
    \end{align*}
\end{document}