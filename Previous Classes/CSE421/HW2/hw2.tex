\documentclass[12pt]{article}
\usepackage[margin=1in]{geometry}
\usepackage{setspace}
\onehalfspacing

% Start of preamble
%==========================================================================================%
% Required to support mathematical unicode
\usepackage[warnunknown, fasterrors, mathletters]{ucs}
\usepackage[utf8x]{inputenc}

\usepackage[dvipsnames,table,xcdraw]{xcolor}
\usepackage{hyperref} 
\hypersetup{
colorlinks=true,
linkcolor=blue,
filecolor=magenta,
urlcolor=cyan,
pdfpagemode=FullScreen
}

\usepackage{quiver}

% Standard mathematical typesetting packages
\usepackage{amsmath,amssymb,amscd,amsthm,amsxtra, pxfonts}
\usepackage{mathtools,mathrsfs,dsfont,xparse}

% Symbol and utility packages
\usepackage{cancel, textcomp}
\usepackage[mathscr]{euscript}
\usepackage[nointegrals]{wasysym}
\usepackage{apacite}

% Extras
\usepackage{physics}  
\usepackage{tikz-cd} 
\usepackage{microtype}
\usepackage{enumitem}
\usepackage{titling}
\usepackage{graphicx}

% Fancy theorems due to @intuitively on discord
\usepackage{mdframed}
\newmdtheoremenv[
backgroundcolor=NavyBlue!30,
linewidth=2pt,
linecolor=NavyBlue,
topline=false,
bottomline=false,
rightline=false,
innertopmargin=10pt,
innerbottommargin=10pt,
innerrightmargin=10pt,
innerleftmargin=10pt,
skipabove=\baselineskip,
skipbelow=\baselineskip
]{mytheorem}{Theorem}

\newenvironment{theorem}{\begin{mytheorem}}{\end{mytheorem}}

\newtheorem{corollary}{Corollary}
\newtheorem{lemma}{Lemma}

\newtheoremstyle{definitionstyle}
{\topsep}%
{\topsep}%
{}%
{}%
{\bfseries}%
{.}%
{.5em}%
{}%
\theoremstyle{definitionstyle}
\newmdtheoremenv[
backgroundcolor=Violet!30,
linewidth=2pt,
linecolor=Violet,
topline=false,
bottomline=false,
rightline=false,
innertopmargin=10pt,
innerbottommargin=10pt,
innerrightmargin=10pt,
innerleftmargin=10pt,
skipabove=\baselineskip,
skipbelow=\baselineskip,
]{mydef}{Definition}
\newenvironment{definition}{\begin{mydef}}{\end{mydef}}

\newtheorem*{remark}{Remark}

\newtheorem*{example}{Example}

% Common shortcuts
\def\mbb#1{\mathbb{#1}}
\def\mfk#1{\mathfrak{#1}}

\def\bN{\mbb{N}}
\def \C{\mbb{C}}
\def \R{\mbb{R}}
\def\bQ{\mbb{Q}}
\def\bZ{\mbb{Z}}
\def \cph{\varphi}
\renewcommand{\th}{\theta}
\def \ve{\varepsilon}
\newcommand{\mg}[1]{\| #1 \|}

% Often helpful macros
\newcommand{\floor}[1]{\left\lfloor#1\right\rfloor}
\newcommand{\ceil}[1]{\left\lceil#1\right\rceil}
\renewcommand{\qed}{\hfill\qedsymbol}
\renewcommand{\P}{\mathbb P\qty}
\newcommand{\E}{\mathbb{E}\qty}
\newcommand{\Cov}{\mathrm{Cov}\qty}
\newcommand{\Var}{\mathrm{Var}\qty}

% Sets
\usepackage{braket}

\graphicspath{{/}}
\usepackage{float}

\newcommand{\SET}[1]{\Set{\mskip-\medmuskip #1 \mskip-\medmuskip}}
\usepackage{algorithm2e, algpseudocode}

% End of preamble
%==========================================================================================%

% Start of commands specific to this file
%==========================================================================================%

%==========================================================================================%
% End of commands specific to this file

\title{CSE 421 HW2}
\date{\today}
\author{Rohan Mukherjee}

\begin{document}
    \maketitle
    \begin{enumerate}[leftmargin=\labelsep]
        \item We first prove the base case. The only two possible trees with vertices labeled 1 and 0 with 2 vertices are, where the sum of the vertices is even, are:
        \[\begin{tikzcd}
            0 & 0 \\
            1 & 1
            \arrow[no head, from=1-1, to=1-2]
            \arrow[no head, from=2-1, to=2-2]
        \end{tikzcd}\]
        In the first case, we can take $F$ to be the empty set, and in the second we can take $F = E$.

        Suppose the statement is true for all trees with $n$ vertices, and let $T$ be an arbitrary tree of $n+1$ vertices. Find a leaf $v$ of the tree. If the label of $v$ is 0, removing $v$ from our tree will result in another tree, who's label-sum is also even since we removed 0 from it. Thus we can find a subset of the edges $F$ so that every vertex with label 0 is adjacent to an even number of edges and every vertex with label 1 an odd number of edges. Now, the vertex $v$ with label 0 has 0 edges adjacent to it, which is an even number, so we are done by taking the above $F$.

        If $v$ instead had label 1, we have two cases. Since $v$ has degree precisely 1 let $v'$ be its parent (the node that it is adjacent to). Flip $v'$'s label, so that if $v'$ was labeled 1 it would now be labeled 0 and vice versa. Now remove $v$ from the tree, and let the remaining tree $T' = T \setminus v$. We see that $T'$ still has an even label sum, since if $v'$ was 1 we would flip it to 0, removing 1, then remove $v$, removing 1 again and still have an even number, and similarly if $v'$ was 0 then we would remove $v$, removing 1, and change $v'$ to 1, adding 1, still resulting in an even number. Now let $F$ be the subset where every vertex with label 1 has an odd number of adjacent edges and every vertex with label 0 has an even number. By adding the edge connecting $v$ with $v'$, we first see that $v$ satisfies the property we need. By our construction, we flipped the parity of $v'$, so we get the opposite parity of adjacent edges needed. After this we added precisely 1 to this parity, flipping it back, so we have the right parity for $v'$ as well, which completes the proof.

        \newpage
        \item We first prove the following lemmata:
        \begin{lemma}
            If $G$'s edges can be partitioned into disjoint edge-cycles then every vertex in $G$ has even degree.
        \end{lemma}

        \begin{proof}
            Suppose that $G = (V, E)$'s edges can be partitioned into disjoint-edge cycles, say as 
            \begin{align*}
                E = \coprod_{i=1}^k E_i.
            \end{align*} Clearly, every vertex in a cycle has degree precisely equal to 2. After removing the edges in $E_i$ from $E$, the vertices that the edges in $E_i$ touch go down by precisely 2, and the other vertices are unchanged. Since the $E_i$ are disjoint, and cover all of $E$, after removing all the $E_i$, all vertices have degree 0. Since removing each $E_i$ reduces the degree of the vertices in the cycle by 2, we see that every vertex in $G$ has even degree.
        \end{proof}
        and,
        \begin{lemma}
            If every vertex in $G$ has degree $\geq 2$, then $G$ contains a cycle.
        \end{lemma}
        \begin{proof}
            Suppose that $G$ has at least one edge, and thus let $v_1$ be a vertex of degree $> 0$. We can construct a cycle as follows: let $v_2$ be an edge adjacent to $v_1$, and since $\deg v_2 \geq 2$, we can find $v_3$ that is adjacent to $v_2$ that is not $v_1$. Now, since $v_3$ has degree $\geq 2$, it either has an edge trailing back to one of $v_1, v_2$, or there is a $v_4$ that is not $v_1$ or $v_2$ that is adjacent to $v_3$. We can continue this process, and since there are only finitely many vertices, we must eventually reach a vertex that we have already visited, which yields a cycle.

            \[\begin{tikzcd}
                {v_1} & {v_2} & {v_3} & {v_4} & \cdots & {v_k}
                \arrow[no head, from=1-4, to=1-5]
                \arrow[no head, from=1-3, to=1-4]
                \arrow[no head, from=1-2, to=1-3]
                \arrow[no head, from=1-1, to=1-2]
                \arrow[no head, from=1-5, to=1-6]
                \arrow[curve={height=-30pt}, no head, from=1-6, to=1-1]
            \end{tikzcd}\]
        \end{proof}

        The above two lemmata pave the way for a concise polynomial-time algorithm to solve our problem:

        \begin{algorithm}[H]
            \SetAlgoLined
            \SetKwFunction{EdgePartition}{EdgePartition}

            \SetKwProg{Fn}{Function}{:}{}

            \Fn{\EdgePartition{$G$}}{
                \If{$G$ has a vertex with odd degree} {
                    Output ``Impossible''\;
                    \Return\;
                }
                \If{every vertex in $G$ has degree 0} {
                    \Return\;
                }
                Initialize $E \leftarrow \emptyset$\;
                Let $v$ be a vertex in $G$ with degree $\geq 2$\;
                \While{true} {
                    \For{each edge $(v, u)$} {
                        \If{$u$ is adjacent to a seen vertex $w$} {
                            Add $(u, w)$ to $E$\;
                            Remove all vertices of $E$ added before $w$\;
                            \textbf{break}\;
                        }
                        Let $w$ be a vertex adjacent to $v$\;
                        Mark $w$ as seen\;
                        Add $(v, w)$ to $E$\;
                        $v \leftarrow w$\;
                    }
                }
                Remove all edges in $E$ from $G$\;
                \Return $E \; \coprod$ \EdgePartition{$G$}\;
            }
        \end{algorithm}
        
        The above algorithm works by finding a cycle (if one exists), removing it from the graph, and calling itself recursively. The algorithm is correct first beacuse clearly each of the $E_i$ returned are disjoint, and second beacuse since in a cycle each vertex has precisely degree 2, so when we remove the edges from $G$, we remove precisely 2 from the degree of each vertex in the cycle, so the degree of each vertex in the remaining graph is still even. This means that unless $G$ is empty after removing the final cycle, we can find another one, hence we have certainly partitioned the edges into disjoint sets. 

        Let $T(m)$ be the running time of the above algorithm on a graph with $n$ vertices and $m$ edges. The first check can be completed in $O(n)$ time, and if it doesn't return, can save a vertex of positive degree. The second check can be completed in $O(n)$ time. The while loop termiantes in at most $n$ steps, since the longest cycle in a graph with $n$ vertices is precisely $n$ (since each vertex can only be visited once). The if statement can be completed in $O(n)$ time (in the worst case the $w$ is apoproximately the $(n-3)$rd element seen). Since each cycle has at least 3 edges, we see that
        \begin{align*}
            T(m) \leq T(m-3) + O(n) \implies T(m) = O(nm).
        \end{align*}



        \newpage
        \item We use a slight variation on BFS. 
        
        \begin{algorithm}[H]
            \SetAlgoLined
            \SetKwFunction{BFS}{BFS}

            \SetKwProg{Fn}{Function}{:}{}

            \Fn{\BFS{$discovered$, $s$}}{
                Initialize an empty queue $Q$\;
                $Q$.enqueue($s$)\;
                $discovered[s] \leftarrow$ true\;
                Initialize $total \leftarrow 0$\;
                \While{$Q$ is not empty}{
                    $u \leftarrow Q$.dequeue()\;
                    $total \leftarrow total + c_u$\;
                    \For{each neighbor $v$ of $u$}{
                        \If{$v$ is undiscovered and $v$ is not dead}{
                            $discovered[v] \leftarrow$ true\;
                            $Q$.enqueue($v$)\;
                        }
                    }
                }
                \Return $dist$, $prev$\;
            }

            \SetKwFunction{LCV}{LCV}
            \Fn{\LCV{$G$}}{
                Initialize an empty dictionary $discovered$\;
                Initialize $lcv \leftarrow 0$\;
                \For{each vertex $v$ in $G$}{
                    \If{$v$ is undiscovered and $v$ is not dead}{
                        $lcv \leftarrow \max(lcv, \BFS{discovered, v})$\;
                    }
                }
                \Return $lcv$\;
            }
        \end{algorithm}

        Since BFS will travel to a node iff there is a path from one node to another, we can see that the above algorithm gives us a collectible value since it only adds the value of each node once, and it never visits dead nodes. This variation of BFS will also output the aggregate sum of the nodes that are reachable from the starting node that don't cross dead vertices (since in the for loop we exclude dead vertices). The function LCV then outputs the maximum of these values. It is thus clear that the LCV is $\geq$ the output of the above algorithm. The LCV is less beacuse it is equivalent to the sum of some list $v_1, \ldots, v_k$ whose sum is equal to the LCV. Running BFS on, say, $v_1$ would either output this sum if $v_1, \ldots, v_k$ are all the elements in the connected component (excluding the dead vertices), or something higher, which completes the proof of correctness. Finally, the above algorithm is just a BFS, possibly excluding some dead vertices, so the running time can only go down--in any case it is certainly bounded by $O(n+m)$.

        \newpage
        \item We want to minimize the product of the weights: given a path $s=v_1, \ldots, v_n=t$, the weight of this path is just
        \begin{align*}
            \prod_{i=1}^{n-1} w_{(v_i, v_{i+1})}
        \end{align*}
        Since $\log(x)$ is an increasing function, this is fully equivalent to minimizing the sum of the logs of the weights:
        \begin{align*}
            \sum_{i=1}^{n-1} \log(w_{(v_i, v_{i+1})})
        \end{align*}
        Our algorithm will thus have 3 steps:

        (1) Collapse each weight-1 edge. We must define the subroutine $\text{collapse}(v,w)$ that takes in two vertices, with the assumption that there is a weight-0 edge between them, removes one of the vertices (say, the one with lower lexicorigraphical ordering) $v$, and collapses the edge into a vertex $vw$ where all the old edges going to $v$ now point to $w$ with the same weight, where if $w$ was already connected to one of those vertices pointing to $v$ then we break ties by picking the smaller-weight edge. For example,
        \[\begin{tikzcd}
            b &&& b \\
            c & v & w & c & w \\
            a &&& a
            \arrow["0", no head, from=2-2, to=2-3]
            \arrow["0", no head, from=2-1, to=2-2]
            \arrow["0"', no head, from=2-2, to=1-1]
            \arrow["1"', no head, from=2-2, to=3-1]
            \arrow["0", no head, from=3-4, to=2-5]
            \arrow["0", no head, from=2-4, to=2-5]
            \arrow["0", no head, from=1-4, to=2-5]
            \arrow["0"', no head, from=3-1, to=2-3]
        \end{tikzcd}\]

        We define the subroutine $\text{collapse}(G)$ by calling collapse on each of its edges. This can be accomplished using an adjacency matrix in $O(n+m)$ time, since given a weight 0 edge $(v,w)$, all $m_v$ of $v$'s edges will have to be transferred to $w$, and breaking ties can be accomplished in $O(1)$ time since we can just access entry $(v,w)$ in the adjacency matrix and set it to the minimum of the it's current weight and the new proposed weight. Doing this on all $n$ vertices yields a running time of $\sum_{i=1}^n m_i = m$, and since we need to access each vertex at least once to collapse its edges, we see that the running time is $O(n+m)$. 
        
        We claim that the collapsing procedure preserves the length of the shortest path. We shall show this by showing that given a path of the original vertices, there is a path of the new vertices with the same weight, and vice versa. Suppose that the path is $v_1, \ldots, v_n$. If there is a weight 1 edge between $v_i$ and $v_{i+1}$, then the collapse procedure will collapse (WLOG) $v_i$ into $v_{i+1}$, whence we have a new path $v_1, \ldots, v_{i-1}, v_{i+1}, \ldots, v_n$ of the same length since we just divided by 1. Given a path of the collapsed vertices $w_1, \ldots, w_k$, we can unravel this path by remembering which vertices we collapsed into which, and lift this to a path $v_1, \ldots, v_n$ (where $n \geq k$ possibly), of the same length, just multiplied by a high power of 1.

        (2) After collapsing edges, we know we will be left with just weight-2 edges, so we can replace the weight of each edge with the log of it's weight. This can clearly be done in $O(n+m)$ time, and ensures that the BFS level of each vertex is the log of the shortest path to it.

        (3) Finally, we can run BFS on the graph starting at $s$, and return $2^{\text{BFSLevel}(t)}$ as the answer. Since BFS will return the shortest path, by our logic in previous two steps we can see that we will return the shortest path in the original graph with the original weights and weight function. Since BFS runs in $O(n+m)$ time, we see that our entire procedure runs in $O(n+m)$ time, and we are done.


        
    \end{enumerate}
\end{document}