\documentclass[12pt]{article}
\usepackage[margin=1in]{geometry}
\usepackage{setspace}
\onehalfspacing
\setlength{\parindent}{4em}

% Start of preamble
%==========================================================================================%
% Required to support mathematical unicode
\usepackage[warnunknown, fasterrors, mathletters]{ucs}
\usepackage[utf8x]{inputenc}

\usepackage[dvipsnames,table,xcdraw]{xcolor}
\usepackage{hyperref} 
\hypersetup{
	colorlinks=true,
	linkcolor=blue,
	filecolor=magenta,      
	urlcolor=cyan,
	pdfpagemode=FullScreen
}

% Standard mathematical typesetting packages
\usepackage{amsmath,amssymb,amscd,amsthm,amsxtra, pxfonts}
\usepackage{mathtools,mathrsfs,dsfont,xparse}

% Symbol and utility packages
\usepackage{cancel, textcomp}
\usepackage[mathscr]{euscript}
\usepackage[nointegrals]{wasysym}
\usepackage{apacite}

% Extras
\usepackage{physics}  
\usepackage{tikz-cd} 
\usepackage{microtype}
\usepackage{enumitem}
\usepackage{titling}
\usepackage{graphicx}

% Fancy theorems due to @intuitively on discord
\usepackage{mdframed}
\newmdtheoremenv[
backgroundcolor=NavyBlue!30,
linewidth=2pt,
linecolor=NavyBlue,
topline=false,
bottomline=false,
rightline=false,
innertopmargin=10pt,
innerbottommargin=10pt,
innerrightmargin=10pt,
innerleftmargin=10pt,
skipabove=\baselineskip,
skipbelow=\baselineskip
]{mytheorem}{Theorem}

\newenvironment{theorem}{\begin{mytheorem}}{\end{mytheorem}}

\newtheorem{corollary}{Corollary}
\newtheorem{lemma}{Lemma}

\newtheoremstyle{definitionstyle}
{\topsep}%
{\topsep}%
{}%
{}%
{\bfseries}%
{.}%
{.5em}%
{}%
\theoremstyle{definitionstyle}
\newmdtheoremenv[
backgroundcolor=Violet!30,
linewidth=2pt,
linecolor=Violet,
topline=false,
bottomline=false,
rightline=false,
innertopmargin=10pt,
innerbottommargin=10pt,
innerrightmargin=10pt,
innerleftmargin=10pt,
skipabove=\baselineskip,
skipbelow=\baselineskip,
]{mydef}{Definition}
\newenvironment{definition}{\begin{mydef}}{\end{mydef}}

\newtheorem*{remark}{Remark}

\newtheorem*{example}{Example}

% Common shortcuts
\def\mbb#1{\mathbb{#1}}
\def\mfk#1{\mathfrak{#1}}

\def\bN{\mbb{N}}
\def \C{\mbb{C}}
\def \R{\mbb{R}}
\def\bQ{\mbb{Q}}
\def\bZ{\mbb{Z}}
\def \cph{\varphi}
\renewcommand{\th}{\theta}
\def \ve{\varepsilon}
\newcommand{\mg}[1]{\| #1 \|}

% Often helpful macros
\newcommand{\floor}[1]{\left\lfloor#1\right\rfloor}
\newcommand{\ceil}[1]{\left\lceil#1\right\rceil}
\renewcommand{\qed}{\hfill\qedsymbol}
\renewcommand{\P}{\mathrm{Pr}\qty}
\newcommand{\E}{\mathbb{E}\qty}

% Sets
\usepackage{braket}

% End of preamble
%==========================================================================================%

% Start of commands specific to this file
%==========================================================================================%

%==========================================================================================%
% End of commands specific to this file

\title{CSE 421 HW1}
\date{\today}
\author{Rohan Mukherjee}

\begin{document}
	\maketitle
	\begin{enumerate}[leftmargin=\labelsep]
		\item \begin{enumerate}
			\item We give the following counter-example:
			\[\begin{tikzcd}
				{c_1: a_1>a_2>a_3} && {a_1: c_3 > c_2 > c_1 } \\
				{c_2: a_3 > a_1 > a_2 } && {a_2: c_2 > c_1 > c_3 } \\
				{c_3: a_2 > a_3 > a_1 } && {a_3: c_1 > c_3 > c_2 }
				\arrow[no head, from=1-1, to=2-3]
				\arrow[no head, from=2-1, to=1-3]
				\arrow[no head, from=3-1, to=3-3]
			\end{tikzcd}\]
			We see that each company is paired with their second favorite applicant and each applicant their second favorite company, and this this is indeed a perfect matching. We need only check stability, and we can do so by just checking it on the applicants. Since there is only company $a_1$ likes more than its current pair, we only need to check that that company does not also prefer $a_1$. Indeed, $c_3$'s least favorite applicant is $a_1$ and that is below its current pair of $a_3$. The same pattern holds for each of the other $a_i$'s, completing the counter-example.
			
			\item We give the (very complicated) counter-example:
			\[\begin{tikzcd}
				{c_1:a_1>a_4>a_3>a_2} & {a_1:c_4>c_3>c_2>c_1} \\
				{c_2:a_1>a_4>a_3>a_2} & {a_2:c_4>c_3>c_2>c_1} \\
				{c_3:a_1>a_4>a_3>a_2} & {a_3:c_4>c_3>c_2>c_1} \\
				{c_4:a_4>a_1>a_2>a_3} & {a_4:c_1>c_2>c_3>c_4}
			\end{tikzcd}\]
			If $c_1$ gets its favorite, $c_4$ must also get it's favorite because it cannot have either $a_2$ or $a_3$ or else it would deviate with $a_1$. Then if we have $c_2 \leftrightarrow a_2$ and $c_3 \leftrightarrow a_3$ or vice versa, both will want to deviate with $a_1$.
			
			If $c_2$ gets its favorite, once again $c_4$ must be with it's favorite otherwise $a_1$ will prefer $c_4$. Then in any arrangement of $c_1$ and $c_3$, they both prefer $a_4$ over their current match and $a_4$ prefers them over their current match. The case $c_3$ is exactly the same and is thus skipped.
			
			If $c_4$ gets its favorite, $c_1$ must get its favorite otherwise $c_1$ and $a_4$ will deviate. Then in any assignment of $c_2$ and $c_3$, they both prefer $a_4$ over their current matches and $a_4$ prefers them as well.
			
			Thus there is no stable matching where a company gets its favorite in this instance.
		\end{enumerate}
		\item Suppose instead that at least two companies $c_1, c_2$ get their last choice (where there are $n$ companies $c_i$ $1 \leq i \leq n$ and $n$ applicants $a_i$ $1 \leq i \leq n$). Our first observation is the last choice of $c_1$ is distinct from $c_2$ (since we have a perfect matching), and WLOG we shall call them $a_1$ and $a_2$ respectively. We observe that for $c_1$ to be rejected by it's first $n-1$ choices, those $n-1$ choices must currently have their final company currently married with them. This is because (1) for an applicant to replace / reject a company, it needs to either have gotten a new company with a better priority, or already have a company with better priority, in any case it will have a company. Next we observe (2) that it must be the case that $a_1$ is free, by our last observation we have used up the rest of the $n-1$ companies. Then the algorithm will assign $c_1$ to $a_1$ and then immediately terminate, since no other company is free. In particular, it cannot be the case that the second company was rejected by $a_1$, otherwise $a_1$ by observation (1) would not be free, a contradiction. Thus, $n-1$ of the companies make at most $n-1$ offerings, and the last company, the only one who potentially gets it's last choice, makes $n$ offerings, giving us a total of
		\begin{align*}
			(n-1)(n-1) + n = (n-1)(n-1) + n-1 + 1 = n(n-1) + 1
		\end{align*}
		
		\item We proceed by induction on $k$. For $k = 0$, we have $2^0 = 1$ numbers, say $x$. The condition $\sum_i x_i = 1$ just says that $x = 1$. In this case, $\sum_i x_i^2 = \sum_i 1^2 = 1 \geq \frac{1}{2^0}$, completing the base case. In general, assume that for every list of $2^k$ real numbers $z_1, \ldots, z_{2^k}$ satisfying $\sum_{i=1}^{2^k} z_i = 1$ we have that 
		\begin{align*}
			\sum_i z_i^2 \geq \frac{1}{2^k}
		\end{align*}
		And let $x_1, \ldots, x_{2^{k+1}}$ be a list of $2^{k+1}$ numbers. Consider the list of $2^k$ numbers $y_i = x_{2i-1} + x_{2i}$ for $1 \leq i \leq 2^k$ (i.e, $y_1 = x_1 + x_2, y_2 = x_3 + x_4$ and so on). Clearly,
		\begin{align*}
			\sum_{i=1}^{2^k} y_i = \sum_{i=1}^{2^{k+1}} x_i = 1
		\end{align*}
		Thus we see that
		\begin{align*}
			\frac{1}{2^k} \leq \sum_{i=1}^{2^k} y_i^2 \leq \sum_{i=1}^{2^k} (2x_{2i-1}^2 + 2x_{2i}^2) = 2\sum_{i=1}^{2^{k+1}} x_i^2
		\end{align*}
		Thus,
		\begin{align*}
			\sum_{i} x_i^2 \geq \frac{1}{2^{k+1}}
		\end{align*}
		which completes the proof.
		
		\item Our final answer is
		\begin{align*}
			a \ll h \ll c \ll i \ll g \ll b \ll j \ll e \ll d \ll f
		\end{align*}
		Meaning:
		\begin{align*}
			2^{2\sqrt{\log n}} &\ll n^{\frac{1}{\log\log(n)}} \ll \frac{n (\log\log(n))^{99}}{\log(n)^{99}} \ll 2^{\log n-\log \log n} 
			\\&\ll \log(n!) \ll 2^{\log(n^2)} \ll (4^2)^{\log n} \ll 4^{2^{\log n}} \ll n!^2 \ll n^{n \log n}
		\end{align*}
	\end{enumerate}
\end{document}
