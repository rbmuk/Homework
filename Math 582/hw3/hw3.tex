\documentclass[12pt]{article}
\usepackage[margin=1in]{geometry}
\usepackage{setspace}
\onehalfspacing{}
\usepackage[dvipsnames,table,xcdraw]{xcolor} % colors

% Start of preamble
%==========================================================================================%
% Required to support mathematical unicode
\usepackage[warnunknown, fasterrors, mathletters]{ucs}
\usepackage[utf8x]{inputenc}

% Standard mathematical typesetting packages
\usepackage{amsmath,amssymb,amscd,amsthm,amsxtra, pxfonts}
\usepackage{mathtools,mathrsfs,xparse}

% Symbol and utility packages
\usepackage{cancel, textcomp}
\usepackage[mathscr]{euscript}
\usepackage[nointegrals]{wasysym}
\usepackage{apacite}

% Extras
\usepackage{physics}  % Lots of useful shortcuts and macros
\usepackage{tikz-cd}  % For drawing commutative diagrams easily
\usepackage{microtype}  % Minature font tweaks
%\usepackage{pgfplots} % plots

\usepackage{enumitem}
\usepackage{titling}

\usepackage{graphicx}

%\usepackage{quiver}

% Fancy theorems due to @intuitively on discord
\usepackage{mdframed}
\newmdtheoremenv[
backgroundcolor=NavyBlue!30,
linewidth=2pt,
linecolor=NavyBlue,
topline=false,
bottomline=false,
rightline=false,
innertopmargin=10pt,
innerbottommargin=10pt,
innerrightmargin=10pt,
innerleftmargin=10pt,
skipabove=\baselineskip,
skipbelow=\baselineskip]{mytheorem}{Theorem}

\newenvironment{theorem}{\begin{mytheorem}}{\end{mytheorem}}

\newtheorem{corollary}{Corollary}
\newtheorem{lemma}{Lemma}

\newtheoremstyle{definitionstyle}
{\topsep}%
{\topsep}%
{}%
{}%
{\bfseries}%
{.}%
{.5em}%
{}%
\theoremstyle{definitionstyle}
\newmdtheoremenv[
backgroundcolor=Violet!30,
linewidth=2pt,
linecolor=Violet,
topline=false,
bottomline=false,
rightline=false,
innertopmargin=10pt,
innerbottommargin=10pt,
innerrightmargin=10pt,
innerleftmargin=10pt,
skipabove=\baselineskip,
skipbelow=\baselineskip,
]{mydef}{Definition}
\newenvironment{definition}{\begin{mydef}}{\end{mydef}}

\newtheorem*{remark}{Remark}

\newtheorem*{example}{Example}
\newtheorem*{claim}{Claim}

% Common shortcuts
\def\mbb#1{\mathbb{#1}}
\def\mfk#1{\mathfrak{#1}}

\def\bN{\mbb{N}}
\def\C{\mbb{C}}
\def\R{\mbb{R}}
\def\bQ{\mbb{Q}}
\def\bZ{\mbb{Z}}
\def\cph{\varphi}
\renewcommand{\th}{\theta}
\def\ve{\varepsilon}
\newcommand{\mg}[1]{\| #1 \|}

% Often helpful macros
\newcommand{\floor}[1]{\left\lfloor#1\right\rfloor}
\newcommand{\ceil}[1]{\left\lceil#1\right\rceil}
\renewcommand{\qed}{\hfill\qedsymbol}
\renewcommand{\ip}[1]{\langle#1\rangle}
\newcommand{\seq}[2]{\qty(#1_#2)_{#2=1}^{\infty}}

\newcommand{\SET}[1]{\Set{\mskip-\medmuskip #1 \mskip-\medmuskip}}

% End of preamble
%==========================================================================================%

% Start of commands specific to this file
%==========================================================================================%

\usepackage{braket}
\newcommand{\Z}{\mbb Z}
\newcommand{\gen}[1]{\left\langle #1 \right\rangle}
\newcommand{\nsg}{\trianglelefteq}
\newcommand{\F}{\mbb F}
\newcommand{\Aut}{\mathrm{Aut}}
\newcommand{\sepdeg}[1]{[#1]_{\mathrm{sep}}}
\newcommand{\Q}{\mbb Q}
\newcommand{\Gal}{\mathrm{Gal}\qty}
\newcommand{\id}{\mathrm{id}}
\newcommand{\Hom}{\mathrm{Hom}_R}
\newcommand{\1}{\mathds 1}
\newcommand{\N}{\mathbb N}
\renewcommand{\P}{\mathbb P \qty}
\newcommand{\E}{\mathbb E \qty}
\newcommand{\Var}{\mathrm{Var}}
\everymath{\displaystyle}
\newcommand{\argmax}{\mathrm{argmax}}

%==========================================================================================%
% End of commands specific to this file

\title{Math 582 HW3}
\date{\today}
\author{Rohan Mukherjee}

\begin{document}
    \maketitle
    \begin{enumerate}
        \item Look at the $n$-simplex $\SET{e_1, \ldots, e_n}$ in $n$-dimensional space. This convex hull is an $n-1$ dimensional shape, which we can draw the $n-1$ dimensional sphere around. In this new affine subspace, call the vectors $\SET{v_1, \ldots, v_n}$. For the function $\sum_i \ip{v_i, x}^2 = x^T \sum_i v_iv_i^T x$. This projects $x$ onto the $v_i$'s, takes a sum, and then calculates an inner product. Since the shape is symmetrical, and by rotational symmetry of the sphere, any rotation of the $v_i$'s will yield the same value of the function for all $x$'s. Again by symmetry, the function on $f(v_i)$ for each $i$ will be the same. Finally, it is proportional to $|x|^2$, if we call the common value of $f(v_i) = C$, then $f(x_i) = C|x_i|^2$. This is a tight frame by lots of symmetry. I am unfortunately not sure how to get the exact coordinates of these $v_i$, but many symmetries shows it is a tight frame.

        I didn't realize that this was only for 2-dimensions inintially, but now I have. Let's use the vectors $(\cos(2\pi k/n), \sin(2 \pi k/n))$ for $k = 1, \ldots, n$. Then the function becomes $\sum x_1^2 \cos^2(2\pi k /n) + x_2^2 \sin(2\pi k/n) + x_1x_2 \cos(2\pi k/n) \sin(2 \pi k / n)$.

        Now, notice that:
        \begin{align*}
            \sum_{i=1}^n \cos^2(2\pi k/n) &= \frac n2 + \frac 1 2 \sum_{i=1}^n \cos(4\pi i/n)
        \end{align*}
        Let $\zeta = e^{4\pi i/n}$. For odd $n$, $\zeta$ generates the group of $n$th roots of unity, so the sum will certainly be 0 (by the factorization $x^n - 1 = (x-1)(x^{n-1} + \cdots + 1)$). For even $n$, $\zeta, \ldots, \zeta^n = 1$ goes twice through the $n/2$ roots of unity, so by applying this fact about the sum of the odd ones, we get that this eventually equals 0 too. In any case, the sum on the rightmost side is just 0, so $\sum_{i=1}^n \cos^2(2\pi k/n) = n/2$. Similarly, since $\cos^2(x) + \sin^2(x) = 1$, we get that $\sum_{i=1}^n \sin^2(2\pi k/n) = n/2$. Finally, $\sum_{i=1}^n \cos(2\pi k/n) \sin(2\pi k/n) = \frac 12\sum_{i=1}^n \sin(4\pi k/n) = 0$ by the same reason. So the function is just $n|x|^2/2$.

        \item If $\mg{\cdot}_1$ and $\mg{\cdot}_2$ have the same unit ball, this means that $\mg{\cdot}_1 = 1$ iff $\mg{\cdot}_2 = 1$. For arbitrary $x \in \R^n$, we know that $\mg{x/\mg{x}_1}_1 = 1$, so $\mg{x/\mg{x}_1}_2 = 1$, which says that $\mg{x}_1 = \mg{x}_2$ for all $x \in \R^n$. 

        With the assumption that $\mg{x} = 0$ iff $x = 0$ (I believe you would need some more conditions on $B$ to ensure this is true always. Like, you could take just the set $\SET{0}$ or a line through the origin and this definition would break down as the norm would be infinite). Then, $\mg{\eta x} = \inf \SET{\lambda > 0 \mid \eta x \in \lambda B} = \SET{\lambda > 0 \mid x \in \lambda / \eta B} = \SET{\eta \lambda \mid x \in \lambda B} = \eta \mg{x}$. Since $B$ is convex, $\mg{x}B \subset B$, so $x \in B$. On the other hand, if $x \in B$, then $\inf \SET{\lambda > 0 \mid x \in \lambda B} \leq 1$ (since 1 works), which shows the other inclusion.

        I now claim that $B = \SET{x \mid \mg{x} \leq 1}$. Clearly, $\SET{x \mid \mg{x} < 1} \subset B$, since if $\mg{x} < 1$, then $\mg{x} < 1-\eta$ for some small $\eta$, and then $x \in (1-\eta)B \subset B$. Since the interior of $\SET{x \mid \mg{x} \leq 1}$ is just $\SET{x \mid \mg{x} < 1}$, (this follows since all norms are equivalent to $\ell^2$), and we shown that the second set is contained in $B$, we know that the closure of it is contained in $B$ (assuming that $B$ is closed, which we can just replace it with if not), so $B$ contains its closure too. The reverse inclusion is clear.
        
        Finally, let $x,y \in \R^n$ and consider $u = x/\mg{x}$ and $v = y/\mg{y}$. Then since $B$ is convex, and $u,v \in B$, we know that:
        \begin{align*}
            \frac{\mg{x} u}{\mg{x} + \mg{y}} + \frac{\mg{y} v}{\mg{x} + \mg{y}} \in B
        \end{align*}
        This says that $\mg{\frac{\mg{x} u}{\mg{x} + \mg{y}} + \frac{\mg{y} v}{\mg{x} + \mg{y}}} \leq 1$, so $\mg{x+y} \leq \mg{x} + \mg{y}$, which shows that it is a norm.

        \item Plugging in $y = x/\mg{x}$, we get that $\sup_{\mg{y} = 1} \ip{x,y} \geq \ip{x, x/\mg{x}} = \mg{x}$. On the other hand, if $\mg{x} = 1$, by the simple inequality $2ab \leq a^2+b^2$,
        \begin{align*}
            \ip{x,y} = \sum_i x_iy_i \leq \frac 12 \sum_i x_i^2 + y_i^2 = 1
        \end{align*}
        And otherwise, $\ip{x,y} = \mg{x} \ip{x/\mg{x}, y} \leq \mg{x}$ by above. This completes the proof.

        \item We need only show that $B^\circ$ is centrally symmetric and convex. It is clearly centrally symmetric, since $\sup_{x \in B} |\ip{x,y}| \leq 1$ is invariant under $y \mapsto -y$. Let $x, y \in B^\circ$ and $t \in [0,1]$. Then $\sup_{z \in B} |\ip{tx+(1-t)y, z}| = \sup_{z \in B} |t\ip{x,y} + (1-t)\ip{y,z}| \leq t \sup_{z \in B} |\ip{x,z}| + (1-t) \sup_{z \in B} |\ip{y,z}| \leq 1$. So $tx + (1-t)y \in B^\circ$, showing that $B^\circ$ is centrally symmetric and convex, so it induces a norm.

        \item Let $y = \mg{y}_{B^\circ} \zeta$ for $\zeta \in B^\circ$ (taking $\zeta = y/\mg{y}_{B^\circ}$, we by a previous question that $\zeta \in B^\circ$). Then by definition of $B^\circ$, we have $\sup_{a \in B} |\ip{a, \zeta}| \leq 1$. Then again $x/\mg{x}_B \in B$ by the previous question again, so $|\ip{x/\mg{x}_B, \zeta}| \leq 1$, which shows after multiplying both sides by $\mg{x}_B \mg{y}_{B^\circ}$ that $|\ip{x,y}| \leq \mg{x}_B \mg{y}_{B^\circ}$.
    \end{enumerate}
\end{document}