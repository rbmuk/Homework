\documentclass[12pt]{article}
\usepackage[margin=1in]{geometry}
\usepackage{setspace}
\onehalfspacing

% Start of preamble
%==========================================================================================%

% Required to support mathematical unicode
\usepackage[warnunknown, fasterrors, mathletters]{ucs}
\usepackage[utf8x]{inputenc}

\usepackage[dvipsnames,table,xcdraw]{xcolor}

% Standard mathematical typesetting packages
\usepackage{amsmath,amssymb,amscd,amsthm,amsxtra}
\usepackage{mathtools,mathrsfs,xparse,newtxtext,newtxmath}

% Symbol and utility packages
\usepackage{cancel, textcomp}
\usepackage[mathscr]{euscript}
\usepackage[nointegrals]{wasysym}
\usepackage{apacite}

% Extras
\usepackage{physics}
\usepackage{tikz-cd}
\usepackage{microtype}
\usepackage{enumitem}
\usepackage{titling}
\usepackage{graphicx}

\usepackage{listings}
\usepackage{xcolor}

\lstset{
    basicstyle=\ttfamily\small,
    keywordstyle=\color{blue},
    commentstyle=\color{green},
    stringstyle=\color{red},
    numbers=left,
    numberstyle=\tiny\color{gray},
    breaklines=true,
    frame=single,
    language=Python
}

% Common shortcuts
\def\mbb#1{\mathbb{#1}}
\def\mfk#1{\mathfrak{#1}}

\def\C{\mbb{C}}
\def\R{\mbb{R}}
\def\Z{\mbb{Z}}
\def\cph{\varphi}
\renewcommand{\th}{\theta}
\def\ve{\varepsilon}
\newcommand{\mg}[1]{\| #1 \|}

% Often helpful macros
\newcommand{\floor}[1]{\left\lfloor#1\right\rfloor}
\newcommand{\ceil}[1]{\left\lceil#1\right\rceil}
\renewcommand{\qed}{\hfill\qedsymbol}
\renewcommand{\P}{\mathbb P\qty}
\newcommand{\E}{\mathbb{E}\qty}
\newcommand{\Cov}{\mathrm{Cov}\qty}
\newcommand{\Var}{\mathrm{Var}\qty}

% Sets
\usepackage{braket}

\graphicspath{{/}}
\usepackage{float}

\newcommand{\SET}[1]{\Set{\mskip-\medmuskip #1 \mskip-\medmuskip}}

% End of preamble
%==========================================================================================%

\title{CSE Template}
\date{\today}
\author{Rohan Mukherjee}

\begin{document}
    \maketitle
    \subsection*{Problem 1.}
    Let $G = (A,B,E)$ be a bipartite graph with parts $A$ and $B$. Since $G$ is $d$-regular, $|A| \cdot d = |B| \cdot d$, both sides count the total number of edges leaving $A$ and $B$. Thus $|A| = |B|$. For a subset $S$ of $A$, we consider the quantity
    \begin{align*}
        d |S| = \sum_{a \in S} \deg a
    \end{align*}
    Since $G$ is d-regular, the degree of each vertex in $B$ is $d$. Thus, in this sum, we count each neighbor of $S$ at the very most $d$ times. So, $d|S| \leq d|N(S)|$ meaning that $|S| \leq |N(S)|$ for each subset $S$ of $A$. Hall's marriage theorem states that a perfect matching exists. Call this $E_1$, and remove all those edges. Repeat by induction to get $d$ edge-disjoint subsets of $E$: $E_1, \ldots, E_d$ (which we color $d$ different colors). Now suppose that every cycle in $G$ has length at least $L$, to be chosen later. Independently choose a random edge $X_i$ of each cycle $C_i$ to be ``corrupted'' to the $(d+1)$st color. Then define $A_{C_i \to e, C_j \to e'} = \SET{X_i = e, X_j = e'}$ only for when $d(e,e') = 1$ and $e \in C_i, e' \in C_j$. Clearly, by our definition of the $X_i$,
    \begin{align*}
        P(A_{C_i \to e, C_j \to e'}) = P(X_i = e) P(X_j = e') = \frac{1}{|C_i||C_j|}
    \end{align*}
    We must now look deeply into dependencies. This random event is fully measurable with respect to $X_i$ and $X_j$. So it is not independent only on events that contain one of those variables.
    
    We prove the following lemma: in a 2-regular graph, $e$ is in $\leq 1$ cycle. This is because a 2-regular graph is a union of disjoint cycles. Thus there are at most ${d \choose 2}$ 2-colored cycles containing $e$, each cycle is found in a union of two of the $E_i$, which is a 2-regular graph. So, if a different event contains $C_i$, then we have $|C_i|$ choices for $e$, then at most $2d$ choices for $e'$, and from our lemma above, there are at most ${d \choose 2} \leq d^2$ 2-colored cycles containing $e'$, yielding a total number of $2|C_i| d^3$. The same is true for $|C_j|$. We now apply (Asymmetric) LLL. Take $x_{C_i \to e, C_j \to e'} = 2/|C_i||C_j|$. Then fixing $A$, we know that:
    \begin{align*}
        \sum_{A' \sim A} x_j &= \sum_{A_{C_i \to f, C_k \to f'}} \frac{1}{|C_i||C_k|} + \sum_{A_{C_k \to f, C_j \to f'}} \frac{1}{|C_k||C_j|} 
    \end{align*}
    We have:
    \begin{align*}
        \sum_{A_{C_i \to f, C_k \to f'}} \frac{1}{|C_i||C_k|} &\leq \sum_{A_{C_i \to f, C_k \to f'}}  \frac{1}{|C_i| L} \leq \frac{2d^3}{L}
    \end{align*}
    This is because there are around $|C_i|$ choices for $f$, and then $2d$ choices for $f'$, with ${d \choose 2} \leq d^2$ cycles containing $f'$. The same is true for $|C_j|$, so we get:
    \begin{align*}
        \sum_{A' \sim A} x_j &\leq \frac{4d^3}{L}
    \end{align*}
    To satisfy the conditions of LLL, we need:
    \begin{align*}
        \frac{2}{|C_i||C_j|} \prod_{A' \sim A} (1-x_j) &\geq \frac{1}{|C_i||C_j|}
    \end{align*}
    Using $1-x \approx e^{-x}$, we have:
    \begin{align*}
        \prod_{A' \sim A} (1-x_j) &= \exp(-\sum_{A' \sim A} x_j) \geq \exp(-\frac{4d^3}{L})
    \end{align*}
    It is this sufficient to take $\exp(-\frac{4d^3}{L}) \geq \frac{1}{2}$, or $L = O(d^3)$. 
\end{document}