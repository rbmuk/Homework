\documentclass[12pt]{article}
\usepackage[margin=1in]{geometry}
\usepackage{setspace}
\onehalfspacing{}
\usepackage[dvipsnames,table,xcdraw]{xcolor} % colors

% Start of preamble
%==========================================================================================%
% Required to support mathematical unicode
\usepackage[warnunknown, fasterrors, mathletters]{ucs}
\usepackage[utf8x]{inputenc}

% Standard mathematical typesetting packages
\usepackage{amsmath,amssymb,amscd,amsthm,amsxtra, pxfonts}
\usepackage{mathtools,mathrsfs,xparse}

% Symbol and utility packages
\usepackage{cancel, textcomp}
\usepackage[mathscr]{euscript}
\usepackage[nointegrals]{wasysym}
\usepackage{apacite}

% Extras
\usepackage{physics}  % Lots of useful shortcuts and macros
\usepackage{tikz-cd}  % For drawing commutative diagrams easily
\usepackage{microtype}  % Minature font tweaks
%\usepackage{pgfplots} % plots

\usepackage{enumitem}
\usepackage{titling}

\usepackage{graphicx}

%\usepackage{quiver}

% Fancy theorems due to @intuitively on discord
\usepackage{mdframed}
\newmdtheoremenv[
backgroundcolor=NavyBlue!30,
linewidth=2pt,
linecolor=NavyBlue,
topline=false,
bottomline=false,
rightline=false,
innertopmargin=10pt,
innerbottommargin=10pt,
innerrightmargin=10pt,
innerleftmargin=10pt,
skipabove=\baselineskip,
skipbelow=\baselineskip]{mytheorem}{Theorem}

\newenvironment{theorem}{\begin{mytheorem}}{\end{mytheorem}}

\newtheorem{corollary}{Corollary}
\newtheorem{lemma}{Lemma}

\newtheoremstyle{definitionstyle}
{\topsep}%
{\topsep}%
{}%
{}%
{\bfseries}%
{.}%
{.5em}%
{}%
\theoremstyle{definitionstyle}
\newmdtheoremenv[
backgroundcolor=Violet!30,
linewidth=2pt,
linecolor=Violet,
topline=false,
bottomline=false,
rightline=false,
innertopmargin=10pt,
innerbottommargin=10pt,
innerrightmargin=10pt,
innerleftmargin=10pt,
skipabove=\baselineskip,
skipbelow=\baselineskip,
]{mydef}{Definition}
\newenvironment{definition}{\begin{mydef}}{\end{mydef}}

\newtheorem*{remark}{Remark}

\newtheorem*{example}{Example}
\newtheorem*{claim}{Claim}

% Common shortcuts
\def\mbb#1{\mathbb{#1}}
\def\mfk#1{\mathfrak{#1}}

\def\bN{\mbb{N}}
\def\C{\mbb{C}}
\def\R{\mbb{R}}
\def\bQ{\mbb{Q}}
\def\bZ{\mbb{Z}}
\def\cph{\varphi}
\renewcommand{\th}{\theta}
\def\ve{\varepsilon}
\newcommand{\mg}[1]{\| #1 \|}

% Often helpful macros
\newcommand{\floor}[1]{\left\lfloor#1\right\rfloor}
\newcommand{\ceil}[1]{\left\lceil#1\right\rceil}
\renewcommand{\qed}{\hfill\qedsymbol}
\renewcommand{\ip}[1]{\langle#1\rangle}
\newcommand{\seq}[2]{\qty(#1_#2)_{#2=1}^{\infty}}

\newcommand{\SET}[1]{\Set{\mskip-\medmuskip #1 \mskip-\medmuskip}}

% End of preamble
%==========================================================================================%

% Start of commands specific to this file
%==========================================================================================%

\usepackage{braket}
\newcommand{\Z}{\mbb Z}
\newcommand{\gen}[1]{\left\langle #1 \right\rangle}
\newcommand{\nsg}{\trianglelefteq}
\newcommand{\F}{\mbb F}
\newcommand{\Aut}{\mathrm{Aut}}
\newcommand{\sepdeg}[1]{[#1]_{\mathrm{sep}}}
\newcommand{\Q}{\mbb Q}
\newcommand{\Gal}{\mathrm{Gal}\qty}
\newcommand{\id}{\mathrm{id}}
\newcommand{\Hom}{\mathrm{Hom}_R}
\newcommand{\1}{\mathds 1}
\newcommand{\N}{\mathbb N}
\renewcommand{\P}{\mathbb P \qty}
\newcommand{\E}{\mathbb E \qty}
\newcommand{\Var}{\mathrm{Var}}
\newcommand{\argmax}{\mathrm{argmax}}

%==========================================================================================%
% End of commands specific to this file

\title{Math 522 Hw1}
\date{\today}
\author{Rohan Mukherjee}

\begin{document}
    \maketitle
    \begin{enumerate}
        \item Let \begin{align*}
            \xi_n = \begin{cases}
                -1 & \text{w.p. } 1-\frac{1}{n^2} \\
                n^2 - 1 & \text{w.p. } \frac{1}{n^2}
            \end{cases}
        \end{align*}
        be independent. Then clearly $\E[\xi_n] = 0$. With $X_n = \xi_1 + \cdots + \xi_n$, we know that $X_n$ is a martingale. I claim that $\P(\limsup_{n \to \infty} X_n = -\infty) \geq 1/2$. This is because $\limsup_{n \to \infty} X_n = -\infty$ occurs certainly whenever all the $\xi_n$ are $-1$. Thus, by independence:
        \begin{align*}
            \P(\limsup_{n \to \infty} X_n = -\infty) \geq \P(\bigcap_{n=1}^\infty \SET{\xi_n = -1}) = \prod_{n=1}^\infty \qty(1-\frac{1}{n^2}) = \frac{1}{2}
        \end{align*}
        Now the event $\SET{\limsup X_n = -\infty}$ is tail measurable, since if we throw out the first $m$ many terms, the limsup will still be $-\infty$. Thus, by Kolmogorov 0-1 law, we know that $\P(\limsup X_n = -\infty) = 0$ or $1$. But we have shown that $\P(\limsup X_n = -\infty) \geq 1/2$, so we must have $\P(\limsup X_n = -\infty) = 1$.

        \item We know that $X_n = \prod_{m \leq n} Y_n$ is a non-negative martingale, thus by Martingale convergence theorem, $X_n \to X$ a.s. with $X$ finite a.s., as $\E[X] \leq \E[X_0] = 1$. In particular, $X_n$ is Cauchy in the following sense:
        \begin{align*}
            \P(|X_{n+1} - X_n| > \eta) \to 0
        \end{align*}
        This is beacuse $\eta \leq |X_{n+1}-X_n|$ implies $\eta \leq |X_n-X| + |X_{n+1} - X|$ which in turn implies $\eta/2 \leq |X_n-X|$ or $\eta/2 \leq |X_{n+1} - X|$. Since $X_n \to X$ almost surely, it converges in probability, so it is easy to see that those last 2 events converge to 0 as $n \to \infty$. Thus the first event must also converge to 0. Since $\P(Y_m = 1) < 1$, and $\E[Y_m] = 1$, we can find some $\delta > 0$ so that $\P(Y_m > 1-\delta) > 0$. This is because otherwise $Y_m \geq 1$ almost surely, and $\P(Y_m > 1) > 0$, so $Y_m - 1$ is a non-negative r.v. with positive expectation, a contradiction. 

        \item First, we prove that if $y_n > -1$ for all $n$ and $\sum |y_n| < \infty$, then $\prod_{m=1}^\infty (1+y_m)$ exists. By Taylor's theorem for remainders, $\log(1+x) = 0 + \frac{1}{1+\zeta} x$ for some $\zeta \in (x, 0)$ for $x < 0$ and $(0, x)$ for $x > 0$. Since $|y_n| \to 0$ since the series converges, we know that $|y_n| < 1/2$ eventually, and since $y_n > -1$, this means that $y_n > -1/2$ eventually. For these values, we then have $|\log(1+y)| \leq 2|y|$ by the calculation above. Thus, for $n$ sufficiently large:
        \begin{align*}
            \qty|\sum_{m \geq n} \log(1+y_m)| \leq 2\sum_{m \geq n} |y_m| \to 0
        \end{align*}
        So $\sum \log(1+y_m)$ converges as it is cauchy, so $\exp(\sum \log(1+y_m)) = \prod (1+y_m)$ converges.
        
        Now define a new r.v. 
        \begin{align*}
            Z_n = \frac{X_n}{\prod_{m \leq n-1} (1+Y_m)}
        \end{align*}
        where the empty product has value 1. Then:
        \begin{align*}
            \E(Z_{n+1} \mid \mathcal F_n) = \frac{1}{\prod_{m \leq n} (1+Y_m)} \E(X_{n+1} \mid \mathcal F_n) \leq \frac{1}{\prod_{m \leq n} (1+Y_m)} (1+Y_n)X_n = \frac{X_n}{\prod_{m \leq n-1} (1+Y_m)} = Z_n
        \end{align*}
        so $Z_n$ is a non-negative supermartingale, thus it converges almost surely. Since $\sum Y_n \leq \infty$ a.s., the product $\prod_{m \leq n} (1+Y_m)$ converges a.s. so $Z_n \cdot \prod_{m \leq n} (1+Y_m) = X_n$ converges a.s. which completes the proof.

        \item Let $X^1_n$, $X^2_n$ be supermartingales adapted to $\mathcal F_n$ and let $N = \inf\SET{m : X^1_m \geq X^2_m}$. Then $N$ is a stopping time, and let:
        \begin{align*}
            Y_n = X^1_n 1_{N > n} + X^2_n 1_{N \leq n} \\
            Z_n = X^1_n 1_{N \geq n} + X^2_n 1_{N < n}
        \end{align*}
        We claim that $Y_n, Z_n$ are supermartingales. First, we show that $Y_n \leq Z_n$ everywhere. This is because when $N > n$, $Y_n = X^1_n$ and $Z_n = X^1_n$, when $N < n$, $Y_n = X^2_n$ and $Z_n = X^2_n$, and when $N = n$, $X^1_n \geq X^2_n$ by definition of $N$, and $Y_n = X^2_n$ while $Z_n = X^1_n$, so $Y_n \leq Z_n$ in all cases. 

        Now,
        \begin{align*}
            \E(Y_{n+1} \mid \mathcal F_n) \leq \E(Z_{n+1} \mid \mathcal F_n) = \E(X^1_{n+1} 1_{N \geq n+1} + X^2_{n+1} 1_{N < n+1} \mid \mathcal F_n)
        \end{align*}
        From here, $1_{N \geq n+1}$ and $1_{N < n+1}$ are $\mathcal F_n$ measurable, since the first event is the complement of $N \leq n$ and the second is equal to $N \leq n$. Thus we can take them out of the conditional expectation, getting:
        \begin{align*}
            \E(X^1_{n+1}  \mid \mathcal F_n)1_{N \geq n+1} + \E(X^2_{n+1} \mid \mathcal F_n)1_{N < n+1} \leq X^1_n 1_{N > n} + X^2_n 1_{N \leq n} = Y_n \leq Z_n
        \end{align*}
        where we used that $X^1_n$ and $X^2_n$ are supermartingales in the middle step, and just rewrote the indices of the indicator functions. By then re-using that $Y_n \leq Z_n$, we get that both $Y_n, Z_n$ are supermartingales when all the equations are put together.

        \item We can obviously write:
        \begin{align*}
            Z^{2j-1}_n &= 1_{N_1 > n} + (X_n/a)1_{N_1 \leq n < N_2} \\
            &+ (b/a)1_{N_2 \leq n < N_3} + (b/a) (X_n/a) 1_{N_3 \leq n < N_4} \\ 
            &\vdots \\
            &+ (b/a)^{j-1} 1_{N_{2j-2} \leq n < N_{2j-1}} + (b/a)^{j-1} (X_n/a) 1_{N_{2j-1} \leq n}
        \end{align*}
        and
        \begin{align*}
            Z^{2j}_n &= 1_{N_1 > n} + (X_n/a)1_{N_1 \leq n < N_2} \\
            &+ (b/a)1_{N_2 \leq n < N_3} + (b/a) (X_n/a) 1_{N_3 \leq n < N_4} \\ 
            &\vdots \\
            &+ (b/a)^{j-1} 1_{N_{2j-2} \leq n < N_{2j-1}} + (b/a)^{j-1} (X_n/a) 1_{N_{2j-1} \leq n < N_{2j}} \\
            &+ (b/a)^{j} 1_{N_{2j} \leq n}
        \end{align*}
        We prove that $Z^j_n$ is a supermartingale by induction. First, $Z^1_n = 1_{N_1 > n} + (X_n/a)1_{N_1 \leq n}$. Let $X^1_n$ be the supermartingale (constant) 1, and $X^2_n = X_n/a$. Then the stopping time $N$ that is $1 \geq X_n/a$ is just the first time $X_n \leq a$, which is precisely $N_1$. Thus by switching principle, $Z^1_n$ is a supermartingale. Then clearly, $Z^{2j}_n = 1_{N_{2j} > n} Z^{2j-1}_n + (b/a)^j 1_{N_{2j} \leq n}$. 

        We now investigate the first time when $Z^{2j-1}_n \geq (b/a)^j$. If $n < N_{2j-1}$, then precisely one term of the form $(b/a)^\ell$ or $(b/a)^{\ell-1} X/a$ is non-zero for $\ell \leq j-1$. Then the condition would become either $(b/a)^\ell \geq (b/a)^j$, obviously impossible, or $(b/a)^{\ell-1} X_n/a \geq (b/a)^j$, which is also impossible since in this case $N_{2\ell-1} \leq n < N_{2\ell}$, and the condition would become $X_n \geq a (b/a)^j (a/b)^{\ell-1} \geq b$, but this would mean that we would already be over $b$, which contradicts the definition of $N_{2\ell}$ since we should be in between an upcrossing. Thus the only way this can happen is if $N_{2j-1} \leq n$. The term would becomme $(b/a)^{j-1} X_n/a \geq (b/a)^j$, which is equivalent to $X_n \geq b$, which is precisely how $N_{2j}$ is defiend. Thus $N_{2j}$ is the stopping time in the switching theorem with $X^1_n = Z^{2j-1}_n$ and $X^2_n = (b/a)^j$, so $Z^{2j}_n$ is a supermartingale as well. 

        Similarly, $Z^{2j+1}_n = 1_{N_{2j+1} > n} Z^{2j}_n + (b/a)^{j} (X/a) 1_{N_{2j+1} \leq n}$, and the same logic applies and the only way $Z^{2j}_n \geq (b/a)^j (X/a)$ is when $N_{2j} \leq n$ and in this case the only term that shows up is $(b/a)^j$, so the condition becomes equivalent to $(b/a)^j \geq (b/a)^j (X/a)$, which is thus the first time after $N_{2j}$ that $X_n \leq a$, which is how $N_{2j+1}$ is defined. Thus $Z^{2j+1}_n$ is a supermartingale as well.

        From martingale convergence theorem, we know that:
        \begin{align*}
            \E(Y_{n \land N_{2k}}) \leq \E(Y_0)
        \end{align*}
        We claim that $Y_0 = X_0 / a \land 1$. This is because if $X_0 \leq a$ then $N_1$ is 0, and $Y_0$ will be $X_0/a$, otherwise $X_0 > a$ and $X_0 = 1$. In conclusion,
        \begin{align*}
            \qty(\frac{b}{a})^{2k} \P(N_{2k} \leq n) = \E(Y_{n \land N_{2k}} 1_{N_{2k} \leq n}) \leq \E(Y_{n \land N_{2k}}) \leq \E(X_0/a \land 1)
        \end{align*}
        Sending $n \to \infty$ shows that:
        \begin{align*}
            \P(N_{2k} < \infty) \leq \qty(\frac{a}{b})^{2k} \E(1 \land X_0/a)
        \end{align*}
        Since $N_{2k} < \infty$ iff the number of upcrossings is at least $k$, we conclude Dubins that:
        \begin{align*}
            \P(U \geq k) \leq \qty(\frac{a}{b})^{2k} \E(1 \land X_0/a)
        \end{align*}
        which completes the proof. $\hfill \blacksquare$
    \end{enumerate}
\end{document}